
\chapter{Interludio 1: Triqui }

\label{cap:inter1:triqui} \index{Triqui}

\section{Motivación}

Con el fin de poner en práctica los conceptos de los capítulos anteriores
vamos a desarrollar un sencillo juego de triqui para dos jugadores.
La idea es seguir un desarrollo iterativo que toma un pequeño programa
y lo convierte, poco a poco, en un juego de triqui con toda la funcionalidad
esperada.

El código fuente tiene diez versiones, comenzando desde \texttt{triqui0.py}
hasta \texttt{triqui9.py}. Los diez programas puede descargarse de:

\url{}

Para comprender el capítulo hay que ejecutar cada versión del programa
a medida que se avanza en la lectura. En las secciones a continuación
se discuten los fragmentos del programa a medida que se van agregando,
cada fragmento tiene el nombre del archivo en que se introdujo como
comentario inicial.

\section{Estructuras de datos}

El código fuente de un programa extenso puede ser difícil de comprender.
Esta es la razón por la que lo primero que se debe explicar de un
software son sus estructuras de datos: las variables, listas, matrices
que usa para representar la información.

En nuestro triqui, el tablero de juego es una matriz de 3 filas y
3 columnas que se almacena en una lista de listas usando la representación
usual del capítulo \ref{cap:listas} en la que cada fila de la matriz
es una sublista.

Todos los elementos de la matriz serán caracteres de texto con la
convención que sigue. Si el elemento de la matriz es:
\begin{description}
\item [{' ' (esto es el caracter espacio):}] nadie ha jugado en esa casilla. 
\item [{'O': }] el primer jugador jugó en esa casilla. 
\item [{'X':}] el segundo jugador jugó en esa casilla. 
\end{description}

\section{Inicio}

Con la convención anterior, nuestro triqui empieza humildemente:

\inputencoding{latin9}\begin{lstlisting}
# triqui0.py
def crear():
    m =  [ [' ',' ',' '],
           [' ',' ',' '],
           [' ',' ',' '] ]
    return m


def imprimir(tablero):
    for i in range(3):
        print("|"),
        for j in range(3):
            print(tablero[i][j]),
        print("|")
        
triqui = crear()
imprimir(triqui)
\end{lstlisting}
\inputencoding{utf8}
Ahora podemos agregar un ciclo para jugar, sencillo, con un solo jugador:\pagebreak{}

\inputencoding{latin9}\begin{lstlisting}
# triqui1.py
while True:
    print("Juegue jugador O")
    f = int(input("fila? "))
    c = int(input("columna? "))
    triqui[f][c] = "O"
    imprimir(triqui)
\end{lstlisting}
\inputencoding{utf8}
Agregar al segundo jugador es casi idéntico al fragmento anterior
y está en el archivo triqui2.py.

\section{¿Quien gana?}

Para verificar si un jugador gana la partida, vamos a empezar por
las diagonales, implementando un ciclo para la verificar si alguien
gana en la diagonal principal:

\inputencoding{latin9}\begin{lstlisting}
# triqui3.py
def ganaDiagonal1(jugador,tablero):
    for i in range(3):
        if tablero[i][i]!=jugador:
            return False
    return True
\end{lstlisting}
\inputencoding{utf8}
La idea es que si encuentra algo diferente del símbolo del jugador
('X' ó 'O'), retorna False. Sino, retorna True. La otra diagonal requiere
mas trabajo, usamos el hecho de que \texttt{tablero{[}i{]}{[}2-i{]}}
va dando los elementos de la segunda diagonal para i de 0 a 2. ¡Verifiquelo!

\inputencoding{latin9}\begin{lstlisting}
# triqui3.py
def ganaDiagonal2(jugador,tablero):
    for i in range(3):
        if tablero[i][2-i]!=jugador:
            return False
    return True
\end{lstlisting}
\inputencoding{utf8}
Falta llamar las funciones en el ciclo del juego, y si alguien gana,
terminamos el juego con la sentencia \texttt{break}. Por ejemplo,
para el primer jugador:

\inputencoding{latin9}\begin{lstlisting}
# triqui3.py
print("Juegue jugador O")
    f = int(input("fila? "))
    c = int(input("columna? "))
    triqui[f][c] = "O"
    imprimir(triqui)
    if ganaDiagonal1("O",triqui) or ganaDiagonal2("O",triqui):
        print("Gana el jugador O!!!!")
        break
\end{lstlisting}
\inputencoding{utf8}
Agregar las funciones para verificar si alguien gana por alguna fila
es sencillo. Seguimos la misma estructura de las diagonales, creando
una función \texttt{ganaFila}, que verifica si el jugador gana en
una de las filas del tablero.

\inputencoding{latin9}\begin{lstlisting}
# triqui4.py
def ganaFila(fila,jugador,tablero):
    """Chequea si el jugador gana en la fila dada"""
    for i in range(3):
        if tablero[fila][i]!=jugador:
            return False
    return True
\end{lstlisting}
\inputencoding{utf8}
La función anterior debe ser llamada para todas las filas:

\inputencoding{latin9}\begin{lstlisting}
# triqui4.py

def ganaHorizontal(jugador,tablero):
    for i in range(3):
        if ganaFila(i,jugador,tablero):
            return True
    return False
\end{lstlisting}
\inputencoding{utf8}
Las funciones para chequear las columnas son muy parecidas. Para llamarlas
modificamos el ciclo del juego. Por ejemplo, para el jugador 'X':

\inputencoding{latin9}\begin{lstlisting}
# triqui4.py
while True:
    print("Juegue jugador X")
    f = int(input("fila? "))
    c = int(input("columna? "))
    triqui[f][c] = "X"
    imprimir(triqui)
    if ganaDiagonal1("X",triqui) or ganaDiagonal2("X",triqui) or \
       ganaHorizontal("X",triqui) or ganaVertical("X",triqui):
        print("Gana el jugador X!!!!")
        break
\end{lstlisting}
\inputencoding{utf8}
\section{Reestructurando el código}

\index{reestructuración}

Casi siempre que se está desarrollando un programa y uno encuentra
que está copiando y pegando código para hacer pequeños cambios vale
la pena analizar si se pueden crear funciones para evitar problemas
futuros. Una función correcta, que se llama desde varias partes del
programa es más fácil de mantener que una serie de porciones de código
parecidas, pero con cambios, que se han copiado, pegado y modificado.

En el triqui podemos observar que el código dentro del ciclo para
el jugador 'O' y el 'X' es muy parecido. Así que lo podemos poner
en una función que tenga como parámetro el símbolo del jugador:

\inputencoding{latin9}\begin{lstlisting}
# triqui5.py
def jugar(jugador,tablero):
    print("Juegue jugador ", jugador)
    f = int(input("fila? "))
    c = int(input("columna? "))
    tablero[f][c] = jugador
    imprimir(triqui)
    diag = ganaDiagonal1(jugador,tablero) or \
           ganaDiagonal2(jugador,tablero)
    linea = ganaHorizontal(jugador,tablero) or \ 
            ganaVertical(jugador,tablero)
    return  diag or linea
\end{lstlisting}
\inputencoding{utf8}
Con este cambio nuestro ciclo de juego es más pequeño, y el programa
es más fácil de mantener:

\inputencoding{latin9}\begin{lstlisting}
# triqui5.py
while True:
    if jugar("O",triqui):
        print("Gana el jugador O !!!!")
        break
    if jugar("X",triqui):
        print("Gana el jugador X !!!!")
        break
\end{lstlisting}
\inputencoding{utf8}
\section{Validaciones}

\index{validación}

Los usuarios pueden cometer errores, por esta razón los programas
deben revisar todos los datos que generan para ver si cumplen las
condiciones para operar. El código que revisa una condición o restricción
de este tipo se llama validación.

En el triqui podemos agregar validación al juego. Tanto \texttt{f}
como \texttt{c},los valores que el usuario digita para jugar en una
fila y columna deben ser enteros en el intervalo {[}0,2{]} para que
podamos representarlos en la matriz de 3 filas y 3 columnas. Además,
la casilla \texttt{tablero{[}f{]}{[}c{]}} debe estar vacía para que
una jugada nueva pueda hacerse allí.

Estas validaciones pueden ponerse en un ciclo que le pida al jugador
digitar los valores para f y c varias veces, hasta que sean correctos:

\inputencoding{latin9}\begin{lstlisting}
# triqui6.py

def valido(n):
    return 0<=n<=2
    
def jugar(jugador,tablero):
    while True:     
        print("Juegue jugador ", jugador)
        f = int(input("fila? "))
        c = int(input("columna? "))
        if type(f)==type(c)==type(1) and valido(f) 
           and valido(c) and tablero[f][c]==' ':
            tablero[f][c] = jugador
            break      

    imprimir(tablero)
    diag = ganaDiagonal1(jugador,tablero) or \
           ganaDiagonal2(jugador,tablero)
    linea = ganaHorizontal(jugador,tablero) or \
            ganaVertical(jugador,tablero)
    return  diag or linea
\end{lstlisting}
\inputencoding{utf8}
\section{Empates}

Ahora agregamos una función para chequear si hay empate entre los
jugadores. Esto sucede si el tablero está lleno, o sea que no hay
ninguna casilla vacía (con el carácter ' '):

\inputencoding{latin9}\begin{lstlisting}
# triqui7.py

def empate(tablero):
    for i in range(3):
        for j in range(3):
            if tablero[i][j]==' ':
                return False
    return True
\end{lstlisting}
\inputencoding{utf8}
Llamamos a empate después de cada jugador: 

\inputencoding{latin9}\begin{lstlisting}
# triqui8.py
while True:
    if jugar("O",triqui):
        print("Gana el jugador O !!!!")
        break
    if empate(triqui):
        print("Empate !!!")
        break
    if jugar("X",triqui):
        print("Gana el jugador X !!!!")
        break
    if empate(triqui):
        print("Empate !!!")
        break
\end{lstlisting}
\inputencoding{utf8}
Y también agregamos un mensaje de retroalimentación para el jugador
cuando no ha escogido una casilla válida:

\inputencoding{latin9}\begin{lstlisting}
# triqui8.py
def jugar(jugador,tablero):
    while True:     
        print("Juegue jugador ", jugador)
        f = int(input("fila? "))
        c = int(input("columna? "))
        if type(f)==type(c)==type(1) and valido(f) 
           and valido(c) and tablero[f][c]==' ':
            tablero[f][c] = jugador
            break
        else:
            print("Posici�n inv�lida!")

    imprimir(triqui)
    diag = ganaDiagonal1(jugador,tablero) or \
           ganaDiagonal2(jugador,tablero)
    linea = ganaHorizontal(jugador,tablero) or \
            ganaVertical(jugador,tablero)
    return  diag or linea
\end{lstlisting}
\inputencoding{utf8}
\section{Reestructurando más}

Podemos crear una función \texttt{gana}, fructífera, que nos permita
que jugar sea mas pequeña:

\inputencoding{latin9}\begin{lstlisting}
# triqui9.py
def gana(jugador,tablero):
    """ Analiza si el jugador gana la partida """
    diag = ganaDiagonal1(jugador,tablero) or \
           ganaDiagonal2(jugador,tablero)
    linea = ganaHorizontal(jugador,tablero) or \
            ganaVertical(jugador,tablero)
    return  diag or linea
\end{lstlisting}
\inputencoding{utf8}
Con este cambio también ganamos algo: la verificación de quien gana
el juego puede hacerse con una sola función en otro programa, por
ejemplo uno con una interfaz gráfica de usuario, como el del capítulo
\ref{triqui-kivy}.

Ahora, \texttt{gana} se llama en el ciclo principal. Y todo esto se
puede poner en la parte principal del programa:

\inputencoding{latin9}\begin{lstlisting}
# triqui9.py
if __name__ == '__main__':
    triqui = crear()
    imprimir(triqui)

    while True:
        jugar("O",triqui)
        if gana("O",triqui):
            print("Gana el jugador O !!!!")
            break
        if empate(triqui):
            print("Empate !!!")
            break
        jugar("X",triqui)
        if gana("X",triqui):
            print("Gana el jugador X !!!!")
            break
        if empate(triqui):
            print("Empate !!!")
            break
\end{lstlisting}
\inputencoding{utf8}
Así terminamos con triqui9.py, un programa con 12 funciones y un ciclo
de juego que tiene en total 124 líneas de código, ¡pero empezó como
una simple impresión de una matriz vacía!

\section{Resumen}
\begin{description}
\item [{triqui0.py:}] crea e imprime el tablero vacío 
\item [{triqui1.py:}] permite que un solo jugador juegue por siempre 
\item [{triqui2.py:}] permite que dos jugadores jueguen por siempre 
\item [{triqui3.py:}] revisa si algún jugador gana en las diagonales 
\item [{triqui4.py:}] revisa si algún jugador gana en filas o columnas 
\item [{triqui5.py:}] evita la duplicación de código creando una función
'jugar' 
\item [{triqui6.py:}] introduce un ciclo en 'jugar' para validar las jugadas 
\item [{triqui7.py:}] introduce la verificación de empates en el juego 
\item [{triqui8.py:}] llama a empate 2 veces en lugar de 1 y añade un mensaje
al jugador 
\item [{triqui9.py:}] crea una función 'gana' para que 'jugar' sea mas
pequeña y pone el ciclo del juego en la parte principal. 
\end{description}

\section{Glosario}
\begin{description}
\item [{validación:}] análisis de los datos que genera un usuario humano
para que estén dentro de los límites de operación del software.
\item [{reestructuración:}] reescritura de porciones del código para mejorar
la calidad del programa. Por ejemplo, se puede mejorar la legibilidad
del código, también se puede eliminar la redundancia.

\index{validación} \index{reestructuración}
\end{description}

\section{Ejercicios}
\begin{enumerate}
\item Modifique el triqui para que el computador juegue automáticamente,
seleccionando una casilla vacía al azar.
\item Modifique el triqui para que tenga un menú de entrada en el que se
pueda escoger entre dos modalidades: 1 jugador contra el computador,
2 jugadores entre si por turnos y 3, salir del programa. Ahora, cada
vez que termine una partida el flujo de ejecución del programa debe
volver al menú.
\end{enumerate}

