% !TEX TS-program = xelatex

\documentclass[10pt,letterpaper,usenames,dvipsnames]{book}
\renewcommand{\rmdefault}{ppl}
\renewcommand{\sfdefault}{kurier}
\renewcommand{\ttdefault}{cmtt}
\renewcommand{\familydefault}{\rmdefault}
\usepackage[T1]{fontenc}
\setcounter{secnumdepth}{3}
\setcounter{tocdepth}{1}
\synctex=-1
\usepackage{color}

\usepackage{calc}
\usepackage{textcomp}
\usepackage{url}
\usepackage{makeidx}
\makeindex
\usepackage{graphicx}
\usepackage{setspace}
\onehalfspacing
\usepackage[
	unicode=true,
	bookmarks=true,
	bookmarksnumbered=true,
	bookmarksopen=true,
	bookmarksopenlevel=0,
	breaklinks=false,
	pdfborder={0 0 0},
	pdfborderstyle={},
	backref=false,
	colorlinks=true]{hyperref}
	\hypersetup{
		pdftitle={Introducción a la programación con Python},
		pdfauthor={Andrés Becerra Sandoval},
		pdfsubject={programación de computadores},
		urlcolor=blue,
		linkcolor=black
	}

\label{MY_VARS}
\def\AutorPrefijo{Prof.}
\def\AutorNombre{Leandro E.}
\def\AutorApellido{Colombo Viña}
\def\AutorNombreCompleto{\AutorPrefijo\ \AutorNombre\ \AutorApellido}
\def\AutorURL{http://leo.bitson.com.ar/}
\def\AutorNombreLink{\href{\AutorURL}{\AutorNombreCompleto}}
\def\AutorCargo{Profesor en Disciplinas Industriales - Técnico Superior en Informática Aplicada}
\def\Title{Aprenda a pensar como un programador con Python 3}
\def\PDFSubject{Adaptación de la traducción de la 2da edición}
\def\Fecha{febrero de 2018}
\def\Lugar{Ciudad Autónoma de Buenos Aires}
\def\BitsonCoop{Cooperativa de trabajo bitson ltda.}
\def\BitsonURL{https://bitson.group}
\def\BitsonMail{info@bitson.group}

\makeatletter

%%%%%%%%%%%%%%%%%%%%%%%%%%%%%% LyX specific LaTeX commands.
\pdfpageheight\paperheight
\pdfpagewidth\paperwidth

\providecommand{\LyX}{\texorpdfstring%
  {L\kern-.1667em\lower.25em\hbox{Y}\kern-.125emX\@}
  {LyX}}
%% Because html converters don't know tabularnewline
\providecommand{\tabularnewline}{\\}

%%%%%%%%%%%%%%%%%%%%%%%%%%%%%% User specified LaTeX commands.
% LyX source for the spanish traslation of the textbook ``How to think like a computer scientist''
% Copyright (c)  2001,2002  Allen B. Downey.
% Traslation to spanish completed by
% Andrés Becerra Sandoval
% abecerra@cic.javerianacali.edu.co


% Permission is granted to copy, distribute and/or modify this
% document under the terms of the GNU Free Documentation License,
% Version 1.1  or any later version published by the Free Software
% Foundation; with the Invariant Sections being "Contributor List",
% with no Front-Cover Texts, and with no Back-Cover Texts. A copy of
% the license is included in the section entitled "GNU Free
% Documentation License".

% This distribution includes a file named fdl.tex that contains the text
% of the GNU Free Documentation License.  If it is missing, you can obtain
% it from www.gnu.org or by writing to the Free Software Foundation,
% Inc., 59 Temple Place - Suite 330, Boston, MA 02111-1307, USA.
%

%
%\usepackage[text={5in,7in},twoside,letterpaper]{geometry}
%
%\setlength{\marginparsep}{0in}
%\setlength{\marginparwidth}{0in}
%\setlength{\marginparpush}{0in}
%\setlength{\hoffset}{-5pt}

\usepackage{geometry}
	\geometry{
		twoside,
		letterpaper,
		footskip=35pt,
	}

\newcommand{\introprog}{\Title}
%\addto\captionsspanish{\renewcommand{\indexname}{Índice analítico}}

%\normalsize

\title{\introprog}
\author{\AutorNombreCompleto}
\date{\Fecha}


\newcommand{\beforefig}{\vspace{1.3\parskip}}
\newcommand{\afterfig}{\vspace{-0.2\parskip}}

\newcommand{\beforeverb}{\vspace{0.6\parskip}}
\newcommand{\afterverb}{\vspace{0.6\parskip}}

\newcommand{\adjustpage}[1]{\enlargethispage{#1\baselineskip}}
\newcommand{\clearemptydoublepage}{\newpage{\pagestyle{empty}\cleardoublepage}}
\newcommand{\blankpage}{\pagestyle{empty}\vspace*{1in}\newpage}


\label{HEADER_AND_FOOTER}
\usepackage{fancyhdr}         %
\pagestyle{fancy}
\fancyhf{} % clear all header and footer fields
\fancyhead[LE]{\leftmark} % Left even (left-side)
\fancyhead[CE]{}
\fancyhead[RE]{\thepage}
\fancyhead[LO]{\thepage} % Left odd (right-side)
\fancyhead[CO]{} 
\fancyhead[RO]{\rightmark}

\fancyfoot[LE]{\includegraphics[scale=0.05]{illustrations/mascota_azul_bitson-alpha.png}} % Left even (left-side)
\fancyfoot[CE]{}
\fancyfoot[RE]{}
\fancyfoot[LO]{} % Left odd (right-side)
\fancyfoot[CO]{}
\fancyfoot[RO]{}
\renewcommand{\headrulewidth}{0.4pt}
%\renewcommand{\footrulewidth}{0.4pt}

\fancypagestyle{plain}{%
	\fancyhf{} % clear all header and footer fields
	%		\fancyfoot[LE]{\rightmark} % Left even (left-side)
	%		\fancyfoot[CE]{}
	%		\fancyfoot[RE]{\thepage}
	%		\fancyfoot[LO]{\PDFTitle} % Left odd (right-side)
	%		\fancyfoot[CO]{}
	%		\fancyfoot[RO]{\thepage}
	\renewcommand{\headrulewidth}{0pt}
	\renewcommand{\footrulewidth}{0pt}
}
\fancyhfoffset[]{0.5cm} % Extendemos el ancho del encabezado y pie de página.

\renewcommand{\chaptermark}[1]{\markboth{#1}{}}
\renewcommand{\sectionmark}[1]{\markright{\thesection\ #1}{}}


% The following lines add a little extra space to the column
% in which the Section numbers appear in the table of contents

\makeatletter
\renewcommand{\l@section}{\@dottedtocline{1}{1.5em}{3.0em}}
\makeatother
\setcounter{tocdepth}{1}

\usepackage{listings}
%\usepackage[usenames,dvipsnames]{color}
\usepackage{textcomp}

\definecolor{purple}{RGB}{153,0,153} 
\definecolor{green}{RGB}{0,153,0} 

\lstdefinestyle{python-idle-code}{
  language=Python,                   
  basicstyle=\normalsize\ttfamily,  
  % Color settings to match IDLE style
  keywordstyle=\color{Orange},       % core keywords,
  morekeywords={assert},
  morekeywords=[2]{print,None,True,False,len,abs,id,type,input,raise,open,range,str,int,float,self,super},
  keywordstyle={[2]\color{purple}}, % built-ins
  %identifierstyle=\color{blue}, (in IDLE only for defining functions and classes is blue)
  stringstyle=\color{green},
  commentstyle=\color{red},
  morecomment=[s][\color{green}]{"""}{"""},
  upquote=true,                      % requires textcomp
  showspaces=false,
  showstringspaces=false  
}

\lstset{style=python-idle-code}

\makeatother

%\usepackage{listings}
%\renewcommand{\pythoncodename}{Listado de c�digo}

\label{MY_PACKAGES}
\input{lecv}
\input{mintedopt}

\begin{document}
\frontmatter

% First page ---------------------------------------------------------------------------------
\thispagestyle{empty} 

\vfill{}
\begin{center}
\textbf{\huge{}\introprog}
\par\end{center}{\huge \par}

\begin{center}
\vfill{}
\par\end{center}

\begin{center}
\textbf{\Large{}\AutorNombreCompleto}{\huge{} }
\par\end{center}{\huge \par}

\vfill{}
\begin{flushright}
{\small{}Traducción y Adaptación del libro }\\
{\small{}``How to think like a computer scientist, learning with Python", }\\
{\small{} escrito por: }\\
{\small{} Allen Downey}\\
{\small{} Jeffrey Elkner}\\
{\small{} Chris Meyers}\\
{\small{} traducido por: }\\
{\small{} Andrés Becerra Sandoval }\\
{\small{} } 
\par\end{flushright}

\vfill{}
\begin{center}
\includegraphics[scale=0.3]{illustrations/logotipo-alpha.png} 
\par\end{center}

\begin{center}
{\Large{}\BitsonCoop} 
\par\end{center}

\vfill{}

% end of First page ---------------------------------------------------------------------------------

\newpage{}
\newpage{}

% version -------------------------------------------------------------------------------------------

\thispagestyle{empty} \vfill{}
\includegraphics[scale=0.3]{illustrations/logotipo-alpha.png} \\

\parindent0pt {\tiny{}\ }{\tiny \par}

%{\scriptsize{}Rector: Carlos Andrés Perez.}\\
%{\scriptsize{} Vicerrector Académico: }\\
%{\scriptsize{} }{\scriptsize \par}
%
%{\scriptsize{}Facultad de Ingeniería}\\
%{\scriptsize{} Decano Académico: Jorge Antonio Silva Ph.D }\\
%{\scriptsize{}Directorr del departamento de Tecnologías de la información
%y la comunicación: Diego Duque}\\
%{\scriptsize \par}

{\scriptsize{}Titulo: \introprog }\\
{\scriptsize{}Autor: \AutorNombreCompleto}
{\scriptsize{}Titulo original: How to think like a computer scientist, learning with Python}\\
{\scriptsize{}Autores: Allen Downey, Jeffrey Elkner, Chris Meyers }\\
{\scriptsize{}Traducción y adaptación original: Andrés Becerra Sandoval }\\
%{\scriptsize{} Colección: Libro}\\
{\scriptsize \par}

{\scriptsize{}ISBN: N/A}\\
{\scriptsize \par}

%{\scriptsize{}Coordinador Editorial: }\\
%{\scriptsize{}Email: @usc.edu.co}{\scriptsize \par}
%
%{\scriptsize{}© Derechos Reservados}\\
%{\scriptsize{}© Editorial USC}\\
%{\scriptsize \par}

{\scriptsize{}Correspondencia, suscripciones y solicitudes de canje:}\\
{\scriptsize{} Rivadavia 755, 3°15 }\\
{\scriptsize{} Ciudad Autónoma de Buenos Aires, República Argentina}\\
{\scriptsize{} \BitsonCoop}\\
{\scriptsize{} \url{\BitsonURL}}\\
{\scriptsize{} \texttt{\BitsonMail}}\\
%{\scriptsize{} Teléfonos: (57-2) Exts. - Fax }\\
%{\scriptsize{} Email: andres.becerra00@usc.edu.co }\\
{\scriptsize \par}

{\scriptsize{}Formato 17 x 25 cms}\\
{\scriptsize{} % Diseño e Impresión: \\}{\scriptsize \par}

%{\scriptsize{}Diseño de Carátula:, basada en una imagen de Ken Manheimer}\\
%{\scriptsize{}\url{ http://myriadicity.net}}{\scriptsize \par}

{\scriptsize{} Impresión: \Fecha}

% end of version -------------------------------------------------------------------------------------------

\newpage{}

\thispagestyle{empty} \vspace{0.25in}

Se concede permiso para copiar, distribuir, y/o modificar este documento
bajo los términos de la GNU Free Documentation License, Versión 1.1
o cualquier versión posterior publicada por la Free Software Foundation;
manteniendo sin variaciones las secciones ``Prólogo,'' ``Prefacio,''
y ``Lista de contribuidores,'' sin texto de cubierta, y sin texto
de contracubierta. Una copia de la licencia está incluida en el apéndice
titulado ``GNU Free Documentation License'' y una traducción de
ésta al español en el apéndice titulado ``Licencia de Documentación
Libre de GNU''.

La GNU Free Documentation License también está disponible a través
de \url{www.gnu.org} o escribiendo a la Free Software Foundation,
Inc., 59 Temple Place, Suite 330, Boston, MA 02111-1307, USA.

La forma original de este libro es código fuente LyX %\LyX{}\ 
y compilarlo
tiene el efecto de generar un libro de texto en una representación
independiente del dispositivo que puede ser convertida a otros formatos
e imprimirse.

El código fuente \LaTeX{}, xfig para este libro y mas información
sobre este proyecto se encuentra en los sitios web:

Repositorio: \url{https://github.com/lecovi/thinkcs-py_es}

Sitio Web del Libro Original: \url{http://greenteapress.com/wp/think-python-2e/}

Este libro ha sido preparado utilizando \LaTeX{}\ y las figuras se han realizado con xfig.
Todos estos son programas de código abierto, gratuito.

\vspace{0.25in}

\newpage{}

\thispagestyle{empty}

\noindent\fbox{\begin{minipage}[t]{1\columnwidth - 2\fboxsep - 2\fboxrule}%
Downey, Allen \\
\introprog / Allen Downey, Jeffrey Elkner, Chris Meyers; traducido
y adaptado por Andrés Becerra Sandoval. – Santiago de Cali: Universidad
Santiago de Cali, Editorial USC, 2017. \\
345 p. ; 26 cm. \\
\\
Incluye referencias bibliográficas e índice.\\
\\
ISBN \textit{N/A}\\
\\
%1. Programación (computadores electrónicos) – Metodología 2. Python
%(lenguaje de programación para computadores) I. Meyer, Chris II. Universidad
%Santiago de Cali III. How to think like a computer scientist: learning
%with python IV. Tít. \\
%\\
%
%SCDD 005.1 \hfill{} USC%
\end{minipage}}


\normalsize

\tableofcontents{}\cleardoublepage{}

\input{foreword.tex}\cleardoublepage{}

\input{preface.tex}\cleardoublepage{}

\input{contrib.tex}\cleardoublepage{}

\input{traduccion.tex}\cleardoublepage{}

\mainmatter\cleardoublepage{}

%\input{solucionProblemas.tex}\cleardoublepage{}

\input{elCaminoHaciaElPrograma.tex}\cleardoublepage{}

\input{variablesExpresionesSentencias.tex}\cleardoublepage{}

\input{funciones.tex}\cleardoublepage{}

\input{condicionalesYRecursion.tex}\cleardoublepage{}

\input{funcionesFructiferas.tex}\cleardoublepage{}

\input{iteracion.tex}\cleardoublepage{}

\input{cadenas.tex}\cleardoublepage{}

\input{listas.tex}\cleardoublepage{}

%\input{interludioTriqui.tex}\cleardoublepage{}

\input{tuplas.tex}\cleardoublepage{}

\input{diccionarios.tex}\cleardoublepage{}

\input{archivosYExcepciones.tex}\cleardoublepage{}

\input{clasesYObjetos.tex}\cleardoublepage{}

\input{clasesYFunciones.tex}\cleardoublepage{}

\input{clasesYMetodos.tex}\cleardoublepage{}

%\input{interludioFraccion.tex}\cleardoublepage{}

\input{conjuntosDeObjetos.tex}\cleardoublepage{}

\input{herencia.tex}\cleardoublepage{}

%\input{interludioTriquiGUI.tex}\cleardoublepage{}


\appendix
\input{depuracion.tex}\cleardoublepage{}

\input{lecturasRecomendadas.tex}\cleardoublepage{}

\input{fdl.tex}\cleardoublepage{}

\input{gfdles.tex}\cleardoublepage{}

\printindex{}
\end{document}
