
\chapter{Clases y objetos}

\index{clase} \index{objeto}

\section{Tipos compuestos definidos por el usuario}

\label{point} \index{tipo de datos compuestos} \index{tipo de datos!compuesto}
\index{tipo de datos definido por el usuario} \index{tipo de datos!definido por el usuario}
\index{constructor}

Una vez utilizados algunos de los tipos internos de Python, estamos
listos para crear un tipo definido por el usuario: el \texttt{Punto}.

Piense en el concepto de un punto matemático. En dos dimensiones,
un punto tiene dos números (coordenadas) que se tratan colectivamente
como un solo objeto. En notación matemática, los puntos suelen escribirse
entre paréntesis con una coma separando las coordenadas. Por ejemplo,
$(0,0)$ representa el origen, y $(x,y)$ representa el punto $x$
unidades a la derecha e $y$ unidades hacia arriba desde el origen.

Una forma natural de representar un punto en Python es con dos valores
en punto flotante. La cuestión es, entonces, cómo agrupar esos dos
valores en un objeto compuesto. La solución rápida y burda es utilizar
una lista o tupla, y para algunas aplicaciones esa podría ser la mejor
opción.

\index{coma flotante}

Una alternativa es que el usuario defina un nuevo tipo de dato compuesto,
también llamado una \textbf{clase}. Esta aproximación exige un poco
más de esfuerzo, pero tiene algunas ventajas que pronto se harán evidentes.

Una definición de clase se parece a esto:
\begin{lstlisting}
class Punto:
  pass
\end{lstlisting}

Las definiciones de clase pueden aparecer en cualquier lugar de un
programa, pero normalmente están al principio (tras las sentencias
\texttt{import}). Las reglas sintácticas de la definición de clases
son las mismas que para las otras sentencias compuestas. (ver la Sección~\ref{alternative execution}).

Esta definición crea una nueva clase llamada \texttt{Punto}. La sentencia
\textbf{pass} no tiene efectos; sólo es necesaria porque una sentencia
compuesta debe tener algo en su cuerpo.

Al crear la clase \texttt{Punto} hemos creado un nuevo tipo, que también
se llama \texttt{Punto}. Los miembros de este tipo se llaman \textbf{instancias}
del tipo u \textbf{objetos}. La creación de una nueva instancia se
llama \textbf{instanciación}. Para instanciar un objeto \texttt{Punto}
ejecutamos una función que se llama \texttt{Punto}:

\index{instancia!objeto} \index{instancia de un objeto} \index{instanciación}
\begin{lstlisting}
limpio = Punto()
\end{lstlisting}

A la variable \texttt{limpio} se le asigna una referencia a un nuevo
objeto \texttt{Punto}. A una función como \texttt{Punto} que crea
un objeto nuevo se le llama \textbf{constructor}.

\section{Atributos}

\index{atributos}

Podemos añadir nuevos datos a una instancia utilizando la notación
de punto:
\begin{lstlisting}
>>> limpio.x = 3.0
>>> limpio.y = 4.0
\end{lstlisting}

Esta sintaxis es similar a la sintaxis para seleccionar una variable
de un módulo, como \texttt{math.pi} o \texttt{string.uppercase}. En
este caso, sin embargo, estamos seleccionando un dato de una instancia.
Estos datos con nombre se denominan \textbf{atributos}.

El diagrama de estados que sigue muestra el resultado de esas asignaciones:

\beforefig \centerline{\includegraphics{illustrations/point}}
\afterfig

La variable \texttt{limpio} apunta a un objeto Punto, que contiene
dos atributos. Cada atributo apunta a un número en punto flotante.

Podemos leer el valor de un atributo utilizando la misma sintaxis:

\begin{lstlisting}
>>> print(limpio.y)
4.0
>>> x = limpio.x
>>> print(x)
3.0
\end{lstlisting}

La expresión \texttt{limpio.x} significa, ``ve al objeto al que apunta
\texttt{limpio} y toma el valor de \texttt{x}''. En este caso, asignamos
ese valor a una variable llamada \texttt{x}. No hay conflicto entre
la variable \texttt{x} y el atributo \texttt{x}. El propósito de la
notación punto es identificar de forma inequívoca a qué variable se
refiere el programador.

Se puede usar la notación punto como parte de cualquier expresión.
Así, las sentencias que siguen son correctas:
\begin{lstlisting}
print('(' + str(limpio.x) + ', ' + str(limpio.y) + ')')
distanciaAlCuadrado = limpio.x * limpio.x + 
                      limpio.y * limpio.y
\end{lstlisting}

La primera línea presenta \texttt{(3.0, 4.0)}; la segunda línea calcula
el valor 25.0.

Usted puede estar tentado a imprimir el propio valor de \texttt{limpio}:
\begin{lstlisting}
>>> print(limpio)
<__main__.Point instance at 80f8e70>
\end{lstlisting}

El resultado indica que \texttt{limpio} es una instancia de la clase
\texttt{Punto} que se definió en \texttt{\_\_main\_\_}. \texttt{80f8e70}
es el identificador único de este objeto, escrito en hexadecimal.
Probablemente ésta no es la manera más clara de mostrar un objeto
\texttt{Punto}. En breve veremos cómo cambiar esto.

\index{imprimir!objeto}

\section{Instancias como parámetro}

\index{instancia} \index{parámetro}

Se puede pasar una instancia como parámetro de la forma habitual.
Por ejemplo:
\begin{lstlisting}
def imprimePunto(p):
  print('(' + str(p.x) + ', ' + str(p.y) + ')')
\end{lstlisting}

\texttt{imprimePunto} acepta un punto como argumento y lo muestra
en el formato estándar de la matemática. Si llamas a \texttt{imprimePunto(limpio)},
el resultado es \texttt{(3.0, 4.0)}.

\section{Mismidad}

\index{mismidad}

El significado de la palabra ``mismo'' parece totalmente claro hasta
que uno se detiene a pensarlo un poco y se da cuenta de que hay algo
más de lo que se supone comúnmente.

\index{ambigüedad} \index{lenguaje natural} \index{lenguaje}

Por ejemplo, si alguien dice ``Pepe y yo tenemos la misma moto'',
lo que quiere decir es que su moto y la de Pepe son de la misma marca
y modelo, pero que son dos motos distintas. Si dice ``Pepe y yo tenemos
la misma madre'', quiere decir que su madre y la de Pepe son la misma
persona\footnote{No todas las lenguas tienen el mismo problema. Por ejemplo, el alemán
tiene palabras diferentes para los diferentes tipos de identidad.
``Misma moto'' en este contexto sería ``gleiche Motorrad'' y ``misma
madre'' sería ``selbe Mutter''.}. Así que la idea de ``identidad'' es diferente según el contexto.

Cuando uno habla de objetos, hay una ambigüedad parecida. Por ejemplo,
si dos \texttt{Puntos} son el mismo, ¿significa que contienen los
mismos datos (coordenadas) o que son de verdad el mismo objeto?

Para averiguar si dos referencias se refieren al mismo objeto, se
utiliza el operador \texttt{==}. Por ejemplo:
\begin{lstlisting}
>>> p1 = Punto()
>>> p1.x = 3
>>> p1.y = 4
>>> p2 = Punto()
>>> p2.x = 3
>>> p2.y = 4
>>> p1 == p2
False
\end{lstlisting}

Aunque \texttt{p1} y \texttt{p2} contienen las mismas coordenadas,
no son el mismo objeto. Si asignamos \texttt{p1} a \texttt{p2}, las
dos variables son alias del mismo objeto:
\begin{lstlisting}
>>> p2 = p1
>>> p1 == p2
True
\end{lstlisting}

Este tipo de igualdad se llama \textbf{igualdad superficial}, porque
sólo compara las referencias, pero no el contenido de los objetos.

\index{igualdad} \index{identidad} \index{igualdad superficial}
\index{igualdad profunda}

Para comparar los contenidos de los objetos (\textbf{igualdad profunda})
podemos escribir una función llamada \texttt{mismoPunto}:

\begin{lstlisting}
def mismoPunto(p1, p2) :
  return (p1.x == p2.x) and (p1.y == p2.y)
\end{lstlisting}
 Si ahora creamos dos objetos diferentes que contienen los mismos
datos podremos usar \texttt{mismoPunto} para averiguar si representan
el mismo punto:

\begin{lstlisting}
>>> p1 = Punto()
>>> p1.x = 3
>>> p1.y = 4
>>> p2 = Punto()
>>> p2.x = 3
>>> p2.y = 4
>>> mismoPunto(p1, p2)
True
\end{lstlisting}
 Por supuesto, si las dos variables apuntan al mismo objeto \texttt{mismoPunto}
devuelve verdadero.

\section{Rectángulos}

\label{embedded} \index{rectángulo}

Digamos que queremos una clase que represente un rectángulo. La pregunta
es, ¿qué información tenemos que proporcionar para definir un rectángulo?
Para simplificar las cosas, supongamos que el rectángulo está orientado
vertical u horizontalmente, nunca en diagonal.

Tenemos varias posibilidades: podemos señalar el centro del rectángulo
(dos coordenadas) y su tamaño (ancho y altura); o podemos señalar
una de las esquinas y el tamaño; o podemos señalar dos esquinas opuestas.
Un modo convencional es señalar la esquina superior izquierda del
rectángulo y el tamaño.

De nuevo, definiremos una nueva clase:
\begin{lstlisting}
class Rectangulo:	
  pass
\end{lstlisting}

Y la instanciaremos:
\begin{lstlisting}
caja = Rectangulo()
caja.ancho = 100.0
caja.altura = 200.0
\end{lstlisting}

Este código crea un nuevo objeto \texttt{Rectangulo} con dos atributos
flotantes. ¡Para señalar la esquina superior izquierda podemos incrustar
un objeto dentro de otro!
\begin{lstlisting}
caja.esquina = Punto()
caja.esquina.x = 0.0;
caja.esquina.y = 0.0;
\end{lstlisting}

El operador punto compone. La expresión \texttt{caja.esquina.x} significa
``ve al objeto al que se refiere \texttt{caja} y selecciona el atributo
llamado \texttt{esquina}; entonces ve a ese objeto y selecciona el
atributo llamado x''.

La figura muestra el estado de este objeto:

\beforefig \centerline{\includegraphics{illustrations/rectangle}}
\afterfig

\section{Instancias como valores de retorno}

\index{instancia} \index{valor de retorno}

Las funciones pueden devolver instancias. Por ejemplo, \texttt{encuentraCentro}
acepta un \texttt{Rectangulo} como argumento y devuelve un \texttt{Punto}
que contiene las coordenadas del centro del \texttt{Rectangulo}:
\begin{lstlisting}
def encuentraCentro(caja):
  p = Punto()
  p.x = caja.esquina.x + caja.ancho/2.0
  p.y = caja.esquina.y + caja.altura/2.0
  return p
\end{lstlisting}

Para llamar a esta función, se pasa \texttt{caja} como argumento y
se asigna el resultado a una variable:
\begin{lstlisting}
>>> centro = encuentraCentro(caja)
>>> imprimePunto(centro)
(50.0, 100.0)
\end{lstlisting}

\section{Los objetos son mutables}

\index{objeto!mutable} \index{objeto mutable}

Podemos cambiar el estado de un objeto efectuando una asignación sobre
uno de sus atributos. Por ejemplo, para cambiar el tamaño de un rectángulo
sin cambiar su posición, podemos cambiar los valores de \texttt{ancho}
y \texttt{altura}:
\begin{lstlisting}
caja.ancho = caja.ancho + 50
caja.altura = caja.altura + 100
\end{lstlisting}

Podemos encapsular este código en un método y generalizarlo para agrandar
el rectángulo en cualquier cantidad:

\index{encapsulamiento} \index{generalización}
\begin{lstlisting}
def agrandarRect(caja, dancho, daltura) :
  caja.ancho = caja.ancho + dancho
  caja.altura = caja.altura + daltura
\end{lstlisting}

Las variables \texttt{dancho} y \texttt{daltura} indican cuánto debe
agrandarse el rectángulo en cada dirección. Invocar este método tiene
el efecto de modificar el \texttt{Rectangulo} que se pasa como argumento.

Por ejemplo, podemos crear un nuevo \texttt{Rectangulo} llamado \texttt{b}
y pasárselo a la función \texttt{agrandarRect}:
\begin{lstlisting}
>>> b = Rectangulo()
>>> b.ancho = 100.0
>>> b.altura = 200.0
>>> b.esquina = Punto()
>>> b.esquina.x = 0.0;
>>> b.esquina.y = 0.0;
>>> agrandarRect(b, 50, 100)
\end{lstlisting}

Mientras \texttt{agrandarRect} se está ejecutando, el parámetro \texttt{caja}
es un alias de \texttt{b}. Cualquier cambio que se haga a \texttt{caja}
afectará también a \texttt{b}.

\section{Copiado}

\index{uso de alias} \index{copiado} \index{módulo copy} \index{módulo!copy}

El uso de un alias puede hacer que un programa sea difícil de leer,
porque los cambios hechos en un lugar pueden tener efectos inesperados
en otro lugar. Es difícil estar al tanto de todas las variables que
pueden apuntar a un objeto dado.

Copiar un objeto es, muchas veces, una alternativa a la creación de
un alias. El módulo \texttt{copy} contiene una función llamada \texttt{copy}
que puede duplicar cualquier objeto:
\begin{lstlisting}
>>> import copy
>>> p1 = Punto()
>>> p1.x = 3
>>> p1.y = 4
>>> p2 = copy.copy(p1)
>>> p1 == p2
False
>>> mismoPunto(p1, p2)
True
\end{lstlisting}

Una vez que hemos importado el módulo \texttt{copy}, podemos usar
el método \texttt{copy} para hacer un nuevo \texttt{Punto}. \texttt{p1}
y \texttt{p2} no son el mismo punto, pero contienen los mismos datos.

Para copiar un objeto simple como un \texttt{Punto}, que no contiene
objetos incrustados, \texttt{copy} es suficiente. Esto se llama \textbf{copiado
superficial}.

Para algo como un \texttt{Rectangulo}, que contiene una referencia
a un \texttt{Punto}, \texttt{copy} no lo hace del todo bien. Copia
la referencia al objeto \texttt{Punto}, de modo que tanto el \texttt{Rectangulo}
viejo como el nuevo apuntan a un único \texttt{Punto}.

Si creamos una caja, \texttt{b1}, de la forma habitual y entonces
hacemos una copia, \texttt{b2}, usando \texttt{copy}, el diagrama
de estados resultante se ve así:

\beforefig \centerline{\includegraphics{illustrations/rectangle2}}
\afterfig

Es casi seguro que esto no es lo que queremos. En este caso, la invocación
de \texttt{agrandaRect} sobre uno de los \texttt{Rectangulo}s no afectaría
al otro, ¡pero la invocación de \texttt{mueveRect} sobre cualquiera
afectaría a ambos! Este comportamiento es confuso y propicia los errores.

Afortunadamente, el módulo \texttt{copy} contiene un método llamado
\texttt{deepcopy} que copia no sólo el objeto, sino también cualesquiera
objetos incrustados en él. No lo sorprenderá saber que esta operación
se llama \textbf{copia profunda} (deep copy).

\begin{lstlisting}
>>> b2 = copy.deepcopy(b1)
\end{lstlisting}
 Ahora \texttt{b1} y \texttt{b2} son objetos totalmente independientes.

Podemos usar \texttt{deepcopy} para reescribir \texttt{agrandaRect}
de modo que en lugar de modificar un \texttt{Rectangulo} existente,
cree un nuevo \texttt{Rectangulo} que tiene la misma localización
que el viejo pero nuevas dimensiones:

\begin{lstlisting}
def agrandaRect(caja, dancho, daltura) :
  import copy
  nuevaCaja = copy.deepcopy(caja)
  nuevaCaja.ancho = nuevaCaja.ancho + dancho
  nuevaCaja.altura = nuevaCaja.altura + daltura
  return nuevaCaja
\end{lstlisting}
 

\section{Glosario}
\begin{description}
\item [{Clase:}] tipo compuesto definido por el usuario. También se puede
pensar en una clase como una plantilla para los objetos que son instancias
de la misma.
\item [{Instanciar:}] Crear una instancia de una clase.
\item [{Instancia:}] objeto que pertenece a una clase.
\item [{Objeto:}] tipo de dato compuesto que suele usarse para representar
una cosa o concepto del mundo real.
\item [{Constructor:}] método usado para crear nuevos objetos.
\item [{Atributo:}] uno de los elementos de datos con nombre que constituyen
una instancia.
\item [{Igualdad superficial:}] igualdad de referencias, o dos referencias
que apuntan al mismo objeto.
\item [{Igualdad profunda:}] igualdad de valores, o dos referencias que
apuntan a objetos que tienen el mismo valor.
\item [{Copia superficial:}] copiar el contenido de un objeto, incluyendo
cualquier referencia a objetos incrustados; implementada por la función
\texttt{copy} del módulo \texttt{copy}.
\item [{Copia profunda:}] copiar el contenido de un objeto así como cualesquiera
objetos incrustados, y los incrustados en estos, y así sucesivamente.
Está implementada en la función \texttt{deepcopy} del módulo \texttt{copy}.

\index{clase} \index{instanciar} \index{instancia} \index{objeto}
\index{constructor} \index{atributo} \index{igualdad superficial}
\index{igualdad profunda} \index{copia superficial} \index{copia profunda}
\end{description}

\section{Ejercicios}
\begin{enumerate}
\item Cree e imprima un objeto \texttt{Punto} y luego use \texttt{id} para
imprimir el identificador único del objeto. Traduzca el número hexadecimal
a decimal y asegúrese de que coincidan.
\item Reescriba la función \texttt{distancia} de la Sección~\ref{program development}
de forma que acepte dos \texttt{Puntos} como parámetros en lugar de
cuatro números.
\item Escriba una función llamada \texttt{mueveRect} que tome un \texttt{Rectangulo}
y dos parámetros llamados \texttt{dx} y \texttt{dy}. Tiene que cambiar
la posición del rectángulo añadiendo en la \texttt{esquina}: \texttt{dx}
a la coordenada \texttt{x} y \texttt{dy} a la coordenada \texttt{y}.
\item Reescriba \texttt{mueveRect} de modo que cree y devuelva un nuevo
\texttt{Rectangulo} en lugar de modificar el viejo.
\item Investigue la documentación del módulo Turtle de Python. Escriba una
función que dibuje un rectángulo con la tortuga.
\item Escriba un programa que dibuje dos tortugas haciendo un movimiento
circular permanente.Pista: Piense en como intercalar los movimientos
de las dos tortugas.
\end{enumerate}

