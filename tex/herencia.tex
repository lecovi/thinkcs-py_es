
\chapter{Herencia}

\section{Definición}

\index{herencia} \index{programación orientada a objetos} \index{clase madre}
\index{clase hija} \index{subclase}

La característica más asociada con la programación orientada a objetos
quizás sea la \textbf{herencia}. Ésta es la capacidad de definir una
nueva clase que constituye la versión modificada de una clase preexistente.

La principal ventaja de la herencia consiste en que se pueden agregar
nuevos métodos a una clase sin modificarla. El nombre ``herencia''
se usa porque la nueva clase hereda todos los métodos de la clase
existente. Extendiendo esta metáfora, la clase preexistente se denomina
la clase \textbf{madre}. La nueva clase puede llamarse clase \textbf{hija}
o, ``subclase.''

\index{diseño orientado a objetos}

La herencia es muy poderosa. Algunos programas complicados se pueden
escribir de una manera más sencilla y compacta a través del uso de
la herencia. Además, facilita la reutilización de código, ya que se
puede especializar el comportamiento de una clase madre sin modificarla.
En algunos casos, las relaciones entre las clases reflejan la estructura
de las entidades del mundo real que se presentan en un problema, y
esto hace que los programas sean más comprensibles.

Por otra parte, la herencia puede dificultar la lectura de los programas.
Cuando un método se llama, puede que no sea muy claro donde está definido.
El código relevante puede estar disperso entre varios módulos. Además,
muchas soluciones que se pueden escribir con el uso de herencia, también
se pueden escribir sin ella, y de una manera elegante (algunas veces
más elegante). Si la estructura natural de las entidades que intervienen
en un problema no se presta a pensar en términos de herencia, este
estilo de programación puede causar más perjuicios que beneficios.

En este capítulo demostraremos el uso de la herencia como parte de
un programa que implementa el juego de cartas la solterona. Una de
las metas es escribir una base de código que pueda reutilizarse para
implementar otros juegos de cartas.

\section{Una mano de cartas}

Para casi todos los juegos de cartas vamos a requerir representar
una mano de cartas. Hay una clara semejanza entre un mazo y una mano
de cartas, ambos son conjuntos de cartas y requieren operaciones como
agregar o eliminar cartas. También, podríamos barajar mazos y manos
de cartas.

Por otro lado, una mano se diferencia de un mazo; dependiendo del
juego que estemos considerando podríamos ejecutar algunas operaciones
sobre las manos que no tienen sentido en un mazo. Por ejemplo, en
el póker, podríamos clasificar manos como un par, una terna, 2 pares,
escalera, o podríamos comparar un par de manos. En el bridge, podríamos
calcular el puntaje para una mano con el objetivo de hacer una apuesta.

Esta situación sugiere el uso de la herencia. Si una \texttt{Mano}
es una subclase de \texttt{Mazo}, tendrá todos los métodos definidos
en ésta y se podrán agregar nuevos métodos.

\index{clase madre} \index{clase!madre}

En la definición de una clase hija, el nombre de la clase madre se
encierra entre paréntesis:

\begin{pythoncode}
class Mano(Mazo):
  pass
\end{pythoncode}
 Esta sentencia indica que la nueva clase \texttt{Mano} hereda de
la clase \texttt{Mazo}.

El constructor de \texttt{Mano} inicializa los atributos \texttt{nombre}
y \texttt{cartas}. La cadena \texttt{nombre} identifica cada mano,
probablemente con el nombre del jugador que la sostiene. Por defecto,
el parámetro nombre es una cadena vacía. \texttt{cartas} es la lista
de cartas en la mano, inicializada a una lista vacía:

\begin{pythoncode}
class Mano(Mazo):
  def __init__(self, nombre=""):
    self.cartas = []
    self.nombre = nombre
\end{pythoncode}
 Para la gran mayoría de juegos de cartas es necesario agregar y remover
cartas del mazo. Como \texttt{Mano} hereda \texttt{eliminarCarta}
de \texttt{Mazo}, ya tenemos la mitad del trabajo hecho, solo tenemos
que escribir \texttt{agregarCarta}:
\begin{pythoncode}
class Mano(Mazo):
  ...
  def agregarCarta(self,carta) :
    self.cartas.append(carta)
\end{pythoncode}

Recuerde que la elipsis (...) indica que estamos omitiendo los otros
métodos. El método \texttt{append} de las listas permite agregar la
nueva carta.

\section{Repartiendo cartas}

\index{repartiendo cartas}

Ahora que tenemos una clase \texttt{Mano} deseamos transferir cartas
del \texttt{Mazo} a las manos. No es fácil decidir que el método que
implemente esta transferencia se incluya en la clase \texttt{Mano}
o en la clase \texttt{Mazo}, pero como opera sobre un solo mazo y
(probablemente) sobre varias manos, es más natural incluirlo en \texttt{Mazo}.

El método \texttt{repartir} debería ser muy general, ya que diferentes
juegos tienen diversas reglas. Podríamos transferir todo el mazo de
una vez, o repartir carta a carta alternando por cada mano, como se
acostumbra en los casinos.

El método \texttt{repartir} toma dos parámetros, una lista (o tupla)
de manos y el número total de cartas que se deben entregar. Si no
hay suficientes cartas en el mazo para repartir, entrega todas las
cartas y se detiene:

\begin{pythoncode}
class Mazo:
  ...
  def repartir(self, manos, nCartas=999):
    nManos = len(manos)
    for i in range(nCartas):
      if self.estaVacia(): 
            break    # se detiene si no hay cartas
      # toma la carta en el tope
      carta = self.eliminarCarta() 
      # siguiente turno!
      mano = manos[i % nManos] 
      # agrega la carta a la mano
      mano.agregarCarta(carta) 
\end{pythoncode}
 El segundo parámetro, \texttt{nCartas}, es opcional; y por defecto
es un número entero grande que garantice que todas las cartas del
mazo se repartan.

\index{variable de ciclo} \index{variable!de ciclo}

La variable de ciclo \texttt{i} va de 0 a \texttt{nCartas-1}. En cada
iteración remueve una carta al tope del mazo usando el método \texttt{pop},
que elimina y retorna el último elemento de la lista.

\index{operador residuo} \index{operador!residuo}

El operador residuo (\texttt{\%}) nos permite repartir cartas de manera
circular (una carta a una mano distinta en cada iteración). Cuando
\texttt{i} es igual al número de manos en la lista, la expresión \texttt{i
\% nManos} \textit{da la vuelta} retornando la primera posición de
la lista (la posición 0).

\section{Imprimiendo una mano}

\index{imprimiendo!manos de cartas}

Para imprimir el contenido de una mano podemos aprovechar los métodos
\texttt{\_\_str\_\_} e \texttt{imprimirMazo} heredados de \texttt{Mazo}.
Por ejemplo:

\begin{pyconcode}
>>> mazo = Mazo()
>>> mazo.barajar()
>>> mano = Mano("Rafael")
>>> mazo.repartir([mano], 5)
>>> print(mano)
La Mano Rafael contiene
2 de Picas
 3 de Picas
  4 de Picas
   As de Corazones
    9 de tréboles
\end{pyconcode}
 No es una gran mano, pero se puede mejorar.

Aunque es conveniente heredar los métodos existentes, hay un dato
adicional que un objeto \texttt{Mano} puede incluir cuando se imprime,
para lograr esto implementamos \texttt{\_\_str\_\_} sobrecargando
el que está definido en la clase \texttt{Mazo}:

\begin{pythoncode}
class Mano(Mazo)
  ...
  def __str__(self):
    s = "Mano " + self.nombre
    if self.estaVacia():
      s = s + " esta vacia\n"
    else:
      s = s + " contiene\n"
    return s + Mazo.__str__(self)
\end{pythoncode}
 Inicialmente, \texttt{s} es una cadena que identifica la mano. Si
está vacía, se añade la cadena \texttt{esta vacia}. Si esto no es
así se añade la cadena \texttt{contiene} y la representación textual
de la clase \texttt{Mazo}, que se obtiene aplicando el método \texttt{\_\_str\_\_}
a \texttt{self}.

Parece extraño aplicar el método \texttt{\_\_str\_\_} de la clase
\texttt{Mazo} a \texttt{self} que se refiere a la \texttt{Mano} actual.
Para disipar cualquier duda, recuerde que \texttt{Mano} es una clase
de \texttt{Mazo}. Los objetos \texttt{Mano} pueden hacer todo lo que
los objetos \texttt{Mazo} hacen, así que esto es legal.

\index{subclase} \index{clase padre} \index{clase!padre}

En general, siempre se puede usar una instancia de una subclase en
vez de una instancia de la clase padre.

\section{La clase \texttt{JuegoDeCartas}}

La clase \texttt{JuegoDeCartas} se encarga de algunas operaciones
básicas comunes a todos los juegos, tales como crear el mazo y barajarlo:

\begin{pythoncode}
class JuegoDeCartas:
  def __init__(self):
    self.mazo = Mazo()
    self.mazo.barajar()
\end{pythoncode}
 En este ejemplo vemos que la inicialización realiza algo más importante
que asignar valores iniciales a los atributos.

Para implementar un juego específico podemos heredar de \texttt{JuegoDeCartas}
y agregar las características del juego particular que estemos desarrollando.

A manera de ejemplo, escribiremos una simulación del juego La Solterona.

El objetivo de La Solterona es deshacerse de todas las cartas. Cada
jugador hace esto emparejando cartas por figura y valor. Por ejemplo
el 4 de Treboles se empareja con el 4 de Picas, por que son cartas
negras. La J de Corazones se empareja con la J de Diamantes, porque
son cartas rojas.

Para empezar, la reina de Treboles se elimina del mazo, de forma que
la reina de Picas no tenga pareja. Las otras 51 cartas se reparten
equitativamente entre los jugadores. Después de repartir cada jugador
busca parejas en su mano y las descarta.

Cuando ningún jugador pueda descartar más se empieza a jugar por turnos.
Cada jugador escoge una carta de su vecino a la izquierda (sin mirarla).
Si la carta escogida se empareja con alguna carta en la mano del jugador,
el par se elimina. Si esto no es así, la carta debe agregarse a la
mano del jugador que escoge. Poco a poco, se realizarán todos los
emparejamientos posibles, dejando únicamente a la reina de Picas en
la mano del perdedor.

En nuestra simulación del juego, el computador juega todas las manos.
Desafortunadamente, algunos matices del juego real se pierden en la
simulación por computador. En un juego real, el jugador con la Solterona
(la reina de Picas) intenta deshacerse de ella de diferentes formas,
ya sea desplegándola de una manera prominente, o ocultándola de alguna
manera. El programa simplemente escogerá una carta de algún vecino
aleatoriamente.

\section{La clase \texttt{ManoSolterona}}

\index{clase!ManoSolterona}

Una mano para jugar a la Solterona requiere algunas capacidades que
no están presentes en la clase \texttt{Mano}. Vamos a definir una
nueva clase \texttt{ManoSolterona}, que hereda de \texttt{Mano} y
provee un método adicional llamado \texttt{quitarPareja}:

\begin{pythoncode}
class ManoSolterona(Mano):
  def quitarPareja(self):
    conteo = 0
    cartasOriginales = self.cartas[:]
    for carta in cartasOriginales:
      m = Carta(3 - carta.figura, carta.valor)
      if pareja in self.cartas:
        self.cartas.remove(carta)
        self.cartas.remove(m)
        print("Mano %s: %s se empareja con %s" % \
              (self.name,carta,m))
        cont = cont + 1
    return cont
\end{pythoncode}
 Empezamos haciendo una copia de la lista de cartas, de forma que
podamos recorrerla y simultáneamente eliminar cartas. Como \texttt{self.cartas}
se modifica en el ciclo, no vamos a utilizarlo para controlar el recorrido.
¡Python puede confundirse totalmente si empieza a recorrer una lista
que está cambiando!

\index{recorrido}

Para cada carta en la mano, averiguamos si se empareja con una carta
escogida de la mano de otra persona. Para esto, tienen que tener el
mismo valor y la otra figura del mismo color. La expresión \texttt{3
- carta.figura} convierte un trébol (figura 0) en una Pica (figura
3) y a un Diamante (figura 1) en un Corazón (figura 2). Usted puede
comprobar que las operaciones inversas también funcionan. Si hay una
carta que se empareje, las dos se eliminan.

El siguiente ejemplo demuestra cómo usar \texttt{quitarPareja}:

\begin{pyconcode}
>>> juego = JuegoDeCartas()
>>> mano = ManoSolterona("frank")
>>> juego.mazo.repartir([mano], 13)
>>> print(mano)
Mano frank contiene
As de Picas
 2 de Diamantes
  7 de Picas
   8 de Treboles
    6 de Corazones
     8 de Picas
      7 de Treboles
       Reina de Treboles
        7 de Diamantes
         5 de Treboles
          Jota de Diamantes
           10 de Diamantes
            10 de Corazones

>>> mano.quitarPareja()
Mano frank: 7 de Picas se empareja con 7 de Treboles
Mano frank: 8 de Picas se empareja con 8 de Treboles
Mano frank: 10 de Diamantes se empareja con 10 de Corazones
>>> print(mano)
Mano frank contiene
Ace de Picas
 2 de Diamantes
  6 de Corazones
   Reina de Treboles
    7 de Diamantes
     5 de Treboles
      Jota de Diamantes
\end{pyconcode}
 Tenga en cuenta que no hay método \texttt{\_\_init\_\_} en la clase
\texttt{ManoSolterna}. Lo heredamos de \texttt{Mano}.

\section{La clase \texttt{JuegoSolterona}}

\index{clase!JuegoSolterona}

Ahora podemos dedicarnos a desarrollar el juego. \texttt{JuegoSolterona}
es una subclase de \texttt{JuegoDeCartas} con un nuevo método llamado
\texttt{jugar} que recibe una lista de jugadores como parámetro.

Como \texttt{\_\_init\_\_} se hereda de \texttt{JuegoDeCartas}, tendremos
la garantía de que un objeto de tipo \texttt{JuegoSolterona} contendrá
un mazo barajado:

\pagebreak

\begin{pythoncode}
class JuegoSolterona(JuegoDeCartas):
  def jugar(self, nombres):
    # elimina la Reina de Treboles
    self.mazo.eliminarCarta(Carta(0,12))

    # prepara una mano para cada jugador
    self.manos = []
    for nombre in nombres :
      self.manos.append(ManoJuegoSolterona(nombre))

    # reparte las  cartas
    self.mazo.repartir(self.cartas)
    print("---------- Cartas repartidas!")
    self.imprimirManos()

    # quitar parejas iniciales
    parejas = self.eliminarParejas()
    print("---------- Parejas descartadas, comienza el juego")
    self.imprimirManos()

    # jugar hasta que las 50 cartas sean descartadas
    turno = 0
    numManos = len(self.manos)
    while parejas < 25:
      parejas = parejas+self.jugarUnTurno(turno)
      turno = (turno + 1) % numManos

    print("-------- Juego terminado")
    self.printManos()
\end{pythoncode}

\pagebreak

 Algunos de las etapas del juego se han separado en métodos. \texttt{eliminarParejas}
recorre la lista de manos invocando \texttt{eliminarPareja} en cada
una de ellas:
\begin{pythoncode}
class JuegoSolterona(JuegoDeCartas):
  ...
  def eliminarParejas(self):
    cont = 0
    for mano in self.manos:
      cont = cont + mano.eliminarParejas()
    return cont
\end{pythoncode}

\texttt{cont} es un acumulador que lleva cuenta del número de parejas
que se encontraron en cada mano.

\index{acumulador}

Cuando el número total de parejas encontradas llega a ser veinticinco,
se han quitado cincuenta cartas de las manos, y esto implica que la
única carta que resta es la reina de Picas y que el juego ha terminado.

La variable \texttt{turno} lleva la pista de cual es el jugador que
tiene el turno para jugar. Empieza en 0 y se incrementa de uno en
uno; cuando alcanza el valor \texttt{numManos}, el operador residuo
lo reinicia en 0.

El método \texttt{jugarUnTurno} toma un parámetro que indica el jugador
que tiene el turno. El valor de retorno es el número de parejas encontradas
durante este turno:

\begin{pythoncode}
class JuegoSolterona(JuegoDeCartas):
  ...
  def jugarUnTurno(self, i):
    if self.manos[i].estaVacia():
      return 0
    vecino = self.encontrarVecino(i)
    cartaEscogida = self.manos[vecino].eliminarCarta()
    self.manos[i].agregarCarta(cartaEscogida)
    print("Mano", self.manos[i].nombre, 
          "escogio", cartaEscogida)
    cont = self.manos[i].eliminarParejas()
    self.manos[i].barajar()
    return cont
\end{pythoncode}
 Si la mano de un jugador está vacía, este jugador está fuera del
juego, así que no hace ninguna acción y retorna 0.

Si esto no es así, un turno consiste en encontrar el primer jugador
en la izquierda que tenga cartas, tomar una de él, y buscar por parejas.
Antes de retornar, las cartas en la mano se barajan para que la elección
del siguiente jugador sea al azar.

El método \texttt{encontrarVecino} comienza con el jugador a la izquierda
y continua buscando de manera circular hasta que encuentra un jugador
que tenga cartas:
\begin{pythoncode}
class JuegoSolterona(JuegoDeCartas):
  ...
  def encontrarVecino(self, i):
    numManos = len(self.manos)
    for siguiente in range(1,numManos):
      vecino = (i + siguiente) % numManos
      if not self.manos[vecino].estaVacia():
        return vecino
\end{pythoncode}

Si \texttt{encontrarVecino} diera toda la vuelta sin encontrar cartas,
retornaría \texttt{None} y causaría un error en algún otro lugar del
programa. Afortunadamente, usted puede comprobar que esto nunca va
a pasar (siempre y cuando el fin del juego se detecte correctamente).

Hemos omitido el método \texttt{imprimaManos}. Usted puede escribirlo.

La siguiente salida es de un juego en el que solo las primeras 15
cartas mas altas (con valor 10 y superior) se repartieron a tres jugadores.
Con este pequeño mazo, el juego termina después de siete parejas encontradas,
en vez de veinticinco.
\begin{pyconcode}
>>> import cartas
>>> juego = cartas.JuegoSolterona()
>>> juego.jugar(["Allen","Jeff","Chris"])
---------- Las cartas se han repartido
Mano Allen contiene
Rey de Corazones
 Jota de Treboles
  Reina de Picas
   Rey de Picas
    10 de Diamantes

Mano Jeff contiene
Reina de Corazones
 Jota de Picas
  Jota de Corazones
   Rey de Diamantes
    Reina de Diamantes

Mano Chris contiene
Jota de Diamantes
 Rey de Treboles
  10 de Picas
   10 de Corazones
    10 de Treboles

Mano Jeff: Reina de Corazones se empareja con Reina de 
Diamantes
Mano Chris: 10 de Picas se empareja con 10 de Treboles
----------  Parejas eliminadas, comienza el juego
Mano Allen contiene
Rey de Corazones
 Jota de Treboles
  Reina de Picas
   Rey de Picas
    10 de Diamantes

Mano Jeff contiene
Jota de Picas
 Jota de Corazones
  Rey de Diamantes

Mano Chris contiene
Jota de Diamantes
 Rey de Treboles
  10 de Corazones

Mano Allen escogio Rey de Diamantes
Mano Allen: Rey de Corazones se empareja con Rey de 
Diamantes
Mano Jeff escogio 10 de Corazones
Mano Chris escogio Jota de Treboles
Mano Allen escogio Jota de Corazones
Mano Jeff escogio Jota de Diamantes
Mano Chris escogio Reina de Picas
Mano Allen escogio Jota de Diamantes
Mano Allen: Jota de Corazones se empareja con Jota de 
Diamantes
Mano Jeff escogio Rey de Treboles
Mano Chris escogio Rey de Picas
Mano Allen escogio 10 de Corazones
Mano Allen: 10 de Diamantes se empareja con 10 de Corazones
Mano Jeff escogio Reina de Picas
Mano Chris escogio Jota de Picas
Mano Chris: Jota de Treboles se empareja con Jota de Picas
Mano Jeff escogio Rey de Picas
Mano Jeff: Rey de Treboles se empareja con Rey de Picas
---------- Game is Over
Mano Allen esta vacia

Mano Jeff contiene
Reina de Picas

Mano Chris esta vacia
\end{pyconcode}
\begin{verbatim}

\end{verbatim}
Así que Jeff pierde.

\section{Glosario}
\begin{description}
\item [{Herencia:}] es la capacidad de definir una clase, modificando una
clase definida previamente.
\item [{Clase madre:}] esta es la clase de la que una clase hereda.
\item [{Clase hija:}] nueva clase creada por medio de herencia, también
recibe el nombre de ``subclase.''

\index{herencia} \index{clase madre} \index{clase madre} \index{subclase}
\end{description}

\section{Ejercicios}
\begin{enumerate}
\item Escriba \texttt{imprimaManos} que recorre la lista \texttt{self.manos}
e imprime cada mano.
\item Con base en la clase \texttt{secuenciaADN}, escriba una clase más
general que se denomine \texttt{biosecuencia} que maneje el alfabeto
de la biosecuencia como un atributo.
\item Agregue un método para chequear que la secuencia está bien formada,
esto es, cada elemento pertenece al alfabeto.
\end{enumerate}

