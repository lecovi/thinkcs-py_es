
\chapter{Conjuntos de objetos}

\section{Composición}

\index{composición} \index{estructura anidada}

En este momento usted ya ha visto varios ejemplos de composición.
Uno de los primeros fue una invocación de un método como parte de
una expresión. Otro ejemplo es la estructura anidada de sentencias;
por ejemplo, se puede colocar una sentencia \texttt{if} dentro de
un ciclo \texttt{while}, dentro de otra sentencia \texttt{if}.

Después de observar esto y haber aprendido sobre listas y objetos
no debería sorprenderse al saber que se pueden crear listas de objetos.
También pueden crearse objetos que contengan listas (como atributos),
listas que contengan listas, objetos que contengan objetos, y así
sucesivamente.

En este capítulo y el siguiente, mostraremos algunos ejemplos de estas
combinaciones, usando objetos \texttt{Carta}.

\section{Objeto \texttt{Carta} }

\index{Carta} \index{clase!Carta}

Si usted no tiene familiaridad con juegos de cartas este es un buen
momento para conseguir una baraja, de lo contrario este capítulo no
tendrá mucho sentido. Hay cincuenta y dos cartas en una baraja, cada
una pertenece a una de las cuatro figuras y uno de los trece valores.
Las figuras son Picas, Corazones, Diamantes y Tréboles. Los valores
son As, 2, 3, 4, 5, 6, 7, 8, 9, 10, J, Q, K. Dependiendo del juego,
el valor del As puede ser más alto que el de un rey o más bajo que
2.

\index{valor} \index{figura}

Si deseamos definir un nuevo objeto para representar una carta del
naipe, parece obvio que los atributos deberían ser \texttt{valor}
y \texttt{figura}. No es tan obvio que tipo de dato asignar a estos
atributos. Una posibilidad consiste en usar cadenas de texto con palabras
como \texttt{Picas} para las figuras y \texttt{Reina} para los valores.
Un problema de esta implementación es que no sería tan fácil comparar
cartas para ver cuál tiene un valor mayor o una figura mayor.

\index{codificar} \index{encriptar} \index{correspondencia}

Una alternativa consiste en usar enteros para \textbf{codificar} los
valores y las figuras. Por ``codificar'', no estamos haciendo alusión
a encriptar o traducir a un código secreto. Lo que un científico de
la computación considera ``codificar'' es ``definir una correspondencia
entre una secuencia de números y los objetos que deseamos representar''.
Por ejemplo:

\beforefig %
\begin{tabular}{lcl}
Picas  & $\mapsto$  & 3 \tabularnewline
Corazones  & $\mapsto$  & 2 \tabularnewline
Diamantes  & $\mapsto$  & 1 \tabularnewline
Tréboles  & $\mapsto$  & 0 \tabularnewline
\end{tabular}\afterfig

Una característica notable de esta correspondencia es que las figuras
aparecen en orden decreciente de valor así como los enteros van disminuyendo.
De esta forma, podemos comparar figuras mediante los enteros que las
representan. Una correspondencia para los valores es bastante sencilla;
cada número se relaciona con el entero correspondiente, y para las
cartas que se representan con letras tenemos lo siguiente:

\beforefig %
\begin{tabular}{lcl}
A  & $\mapsto$  & 1 \tabularnewline
J  & $\mapsto$  & 11 \tabularnewline
Q  & $\mapsto$  & 12 \tabularnewline
K  & $\mapsto$  & 13 \tabularnewline
\end{tabular}\afterfig

La razón para usar notación matemática en estas correspondencias es
que ellas no hacen parte del programa en Python. Son parte del diseño,
pero nunca aparecen explícitamente en el código fuente. La definición
de la clase \texttt{Carta} luce así:\inputencoding{latin9}
\begin{lstlisting}
class Carta:
  def __init__(self, figura=0, valor=0):
    self.figura = figura
    self.valor = valor
\end{lstlisting}
\inputencoding{utf8} Como de costumbre, proporcionamos un método de inicialización que
toma un parámetro opcional para cada atributo.

\index{constructor}

Para crear un objeto que represente el 3 de tréboles, usamos este
comando:\inputencoding{latin9}
\begin{lstlisting}
tresTreboles = Carta(0, 3)
\end{lstlisting}
\inputencoding{utf8}
El primer argumento, \texttt{0}, representa la figura (tréboles).

\section{Atributos de clase y el método \texttt{\_\_str\_\_}}

\index{atributo de clase} \index{atributo!de clase}

Para imprimir objetos \texttt{Carta} en una forma que la gente pueda
leer fácilmente, queremos establecer una correspondencia entre códigos
enteros y palabras. Una forma natural de hacerlo es con listas de
cadenas de texto. Asignamos estas listas a \textbf{atributos de clase}
al principio de la clase:\inputencoding{latin9}
\begin{lstlisting}
class Carta:
  listaFiguras = ["Treboles", "Diamantes", "Corazones", 
  "Picas"]
  listaValores = ["narf", "As", "2", "3", "4", "5", "6", 
  "7","8", "9", "10", "Jota", "Reina", "Rey"]

  # se omite el metodo init

  def __str__(self):
    return self.listaFiguras[self.valor] + " de " + 
            self.listaValores[self.figura]
\end{lstlisting}
\inputencoding{utf8}
Un atributo de clase se define afuera de los métodos y puede ser accedido
desde cualquiera de ellos.

Dentro de \texttt{\_\_str\_\_}, podemos usar a \texttt{listaFiguras}
y \texttt{listaValores} para establecer una correspondencia entre
los valores numéricos de \texttt{figura}, \texttt{valor} y los nombres
de las cartas. La expresión \verb+self.listaFiguras[self.figura]+
significa ``use el atributo \texttt{figura} del objeto \texttt{self}
como índice dentro del atributo de clase \texttt{listaFiguras}, esto
seleccionará la cadena de texto apropiada''.

La razón para el \texttt{``narf''} en el primer elemento de \texttt{listaValores}
consiste en ocupar el elemento cero de la lista que no va a ser usado
en el programa. Los valores válidos son de 1 a 13. Este elemento desperdiciado
no es necesario, podríamos haber empezado a contar desde 0, pero es
menos confuso codificar 2 como 2, 3 como 3 ... y 13 como 13.

Con los métodos que tenemos hasta aquí, podemos crear e imprimir cartas:\inputencoding{latin9}
\begin{lstlisting}
>>> c1 = Carta(1, 11)
>>> print(c1)
Jota de Diamantes
\end{lstlisting}
\inputencoding{utf8} Los atributos de clase como \texttt{listaFiguras} se comparten por
todos los objetos \texttt{Carta}. La ventaja de esto es que podemos
usar cualquier objeto \texttt{Carta} para acceder a ellos:

\inputencoding{latin9}\begin{lstlisting}
>>> c2 = Carta(1, 3)
>>> print(c2)
3 de Diamantes
>>> print(c2.listaFiguras[1])
Diamantes
\end{lstlisting}
\inputencoding{utf8} La desventaja es que si modificamos un atributo de clase, afecta
a todas las otras instancias de la clase. Por ejemplo, si decidimos
que ``Jota de Diamantes'' debería llamarse ``Caballero de Rombos
rojos,'' podríamos ejecutar:

\index{instancia!objeto} \index{objeto instancia}

\inputencoding{latin9}\begin{lstlisting}
>>> c1.listaFiguras[1] = "Caballero de Rombos rojos"
>>> print(c1)
Caballero de Rombos rojos
\end{lstlisting}
\inputencoding{utf8} El problema es que {\em todos} los Diamantes ahora son Rombos
rojos:

\inputencoding{latin9}\begin{lstlisting}
>>> print(c2)
3 de Rombos rojos
\end{lstlisting}
\inputencoding{utf8} Usualmente no es una buena idea modificar los atributos de clase.

\section{Comparando cartas}

\label{comparecard} \index{operador!condicional} \index{operador condicional}

Para los tipos primitivos contamos con los operadores (\texttt{<},
\texttt{>}, \texttt{==}, etc.) que determinan cuándo un valor es mayor,
menor, mayor o igual, menor o igual, o igual al otro. Para los tipos
definidos por el programador podemos sobrecargar el comportamiento
de los operadores predefinidos proporcionando un método llamado \texttt{\_\_cmp\_\_}.
Por convención, \texttt{\_\_cmp\_\_} toma dos parámetros, \texttt{self}
y \texttt{otro}, y retorna 1 si el primer objeto es más grande, -1
si el segundo es más grande y 0 si son iguales entre si.

\index{sobrecargar} \index{sobrecarga de operadores} \index{orden}
\index{orden total} \index{orden parcial}

Algunos tipos tienen un orden total, lo que quiere decir que cualquier
pareja de elementos se puede comparar para decidir cuál de ellos es
mayor. Por ejemplo, los números enteros y los de punto flotante tienen
un orden total. Algunos conjuntos no tienen relación de orden, lo
que quiere decir que no hay una manera sensata de determinar que un
elemento es mayor que otro. Por ejemplo, las frutas no tienen una
relación de orden, y esta es la razón por la que no se pueden comparar
manzanas con naranjas.

El conjunto de cartas tiene un orden parcial, lo que quiere decir
que algunas veces se pueden comparar elementos, y otras veces no.
Por ejemplo, el 3 de Picas es mayor que el 2 de picas, y el 3 de Diamantes
es mayor que el 3 de Picas. Pero, ¿que es más alto, el 3 de Picas
o el 2 de Diamantes? Uno tiene un valor más alto, pero el otro tiene
una figura más alta.

\index{comparable}

Para lograr comparar las cartas, hay que tomar una decisión sobre
la importancia del valor y de la figura. Para ser honestos, esta decisión
es arbitraria. Así que tomaremos la opción de determinar qué figura
es más importante, basándonos en que un mazo de cartas nuevo viene
con las Picas (en orden), luego los Diamantes, y así sucesivamente.

Con esta decisión \texttt{\_\_cmp\_\_} queda así:\inputencoding{latin9}
\begin{lstlisting}
def __cmp__(self, otro):
  # chequea las figuras
  if self.figura > otro.figura: return 1
  if self.figura < otro.figura: return -1
  # Si tienen la misma figura... 
  if self.valor > otro.valor: return 1
  if self.valor < otro.valor: return -1
  # si los valores son iguales... hay un empate
  return 0
\end{lstlisting}
\inputencoding{utf8}
Con este orden los Ases valen menos que los Dos.

\section{Mazos}

\index{lista!de objetos} \index{objeto!lista de} \index{mazo}

Ahora que tenemos objetos para representar \texttt{Carta}s, el siguiente
paso lógico consiste en definir una clase para representar un \texttt{Mazo}.
Por supuesto, un mazo (o baraja) está compuesto por cartas, así que
cada instancia de \texttt{Mazo} contendrá como atributo una lista
de cartas.

\index{método de inicialización} \index{método!de inicialización}

La siguiente es la definición de la clase \texttt{Mazo}. El método
de inicialización crea el atributo \texttt{cartas} y genera el conjunto
usual de cincuenta y dos cartas:

\index{composición} \index{ciclo!anidado}

\inputencoding{latin9}\begin{lstlisting}
class Mazo:
  def __init__(self):
    self.cartas = []
    for figura in range(4):
      for valor in range(1, 14):
        self.cartas.append(Carta(figura, valor))
\end{lstlisting}
\inputencoding{utf8} La forma más sencilla de llenar el mazo consiste en usar un ciclo
anidado. El ciclo exterior enumera las figuras de 0 a 3. El ciclo
interno enumera los valores de 1 a 13. Como el ciclo exterior itera
cuatro veces y el interno itera trece veces, el número total de iteraciones
es cincuenta y dos ($4\times13$). Cada iteración crea una nueva instancia
de \texttt{Carta} y la pega a la lista \texttt{cartas}.

El método \texttt{append} acepta secuencias mutables como las listas
y no acepta tuplas.

\index{método append} \index{método de lista} \index{método!de lista}

\section{Imprimiendo el mazo}

\label{printdeck} \index{imprimir!objeto mazo}

Como de costumbre, cuando definimos un nuevo tipo de objeto, deseamos
tener un método que imprima su contenido. Para imprimir un \texttt{Mazo},
recorremos la lista e imprimimos cada objeto \texttt{Carta}:\inputencoding{latin9}
\begin{lstlisting}
class Mazo:
  ...
  def imprimirMazo(self):
    for carta in self.cartas:
      print(carta)
\end{lstlisting}
\inputencoding{utf8}
En este ejemplo y en los que siguen, los puntos suspensivos indican
que hemos omitido los otros métodos de la clase.

Otra alternativa a \texttt{imprimirMazo} puede ser escribir un método
\texttt{\_\_str\_\_} para la clase \texttt{Mazo}. La ventaja de \texttt{\_\_str\_\_}
radica en su mayor flexibilidad. Además de imprimir el contenido del
objeto, genera una representación de él en una cadena de texto que
puede manipularse en otros lugares del programa, incluso antes de
imprimirse.

A continuación hay una versión de \texttt{\_\_str\_\_} que retorna
una representación de un \texttt{Mazo}. Para añadir un estilo de cascada,
cada carta se imprime un espacio mas hacia la derecha que la anterior:\inputencoding{latin9}
\begin{lstlisting}
class Mazo:
  ...
  def __str__(self):
    s = ""
    for i in range(len(self.cartas)):
      s = s + " "*i + str(self.cartas[i]) + "\n"
    return s
\end{lstlisting}
\inputencoding{utf8}
Este ejemplo demuestra varios puntos. Primero, en vez de recorrer
los elementos de la lista \texttt{self.cartas}, estamos usando a \texttt{i}
como variable de ciclo que lleva la posición de cada elemento en la
lista de cartas.

Segundo, estamos usando el operador multiplicación aplicado a un número
y una cadena, de forma que la expresión \verb+" "*i+ produce un número
de espacios igual al valor actual de \texttt{i}.

Tercero, en vez de usar el comando \texttt{print} para realizar la
impresión, utilizamos la función \texttt{str}. Pasar un objeto como
argumento a \texttt{str} es equivalente a invocar el método \texttt{\_\_str\_\_}
sobre el objeto.

\index{acumulador}

Finalmente, estamos usando a la variable \texttt{s} como \textbf{acumulador}.
Inicialmente \texttt{s} es la cadena vacía. En cada iteración del
ciclo se genera una nueva cadena y se concatena con el valor viejo
de \texttt{s} para obtener el nuevo valor. Cuando el ciclo finaliza,
\texttt{s} contiene la representación completa del \texttt{Mazo},
que se despliega (parcialmente) así:

\inputencoding{latin9}\begin{lstlisting}
>>> mazo = Mazo()
>>> print(mazo)
As de Picas
 2 de Picas
  3 de Picas
   4 de Picas
    5 de Picas
     6 de Picas
      7 de Picas
       8 de Picas
        9 de Picas
         10 de Picas
          J de Picas
           Reina de Picas
            Rey de Picas
             As de Diamantes
\end{lstlisting}
\inputencoding{utf8} Aunque el resultado se despliega en 52 líneas, es una sola cadena
que contiene caracteres nueva linea \verb+(\n)+.

\section{Barajando el mazo}

\index{barajar}

Si un mazo se baraja completamente, cualquier carta tiene la misma
probabilidad de aparecer en cualquier posición, y cualquier posición
tiene la misma probabilidad de contener a cualquier carta.

\index{random} \index{randrange}

Para barajar el mazo, usamos la función \texttt{randrange} que pertenece
al módulo del sistema \texttt{random}. \texttt{randrange} recibe dos
parámetros enteros \texttt{a} y \texttt{b}, y se encarga de escoger
al azar un valor perteneciente al rango \texttt{a <= x < b}. Como
el límite superior es estrictamente menor que \texttt{b}, podemos
usar el número de elementos de una lista como el segundo parámetro
y siempre obtendremos un índice válido como resultado. Por ejemplo,
esta expresión escoge al azar el índice de una carta en un mazo:

\inputencoding{latin9}\begin{lstlisting}
random.randrange(0, len(self.cartas))
\end{lstlisting}
\inputencoding{utf8} Una manera sencilla de barajar el mazo consiste en recorrer todas
las cartas intercambiando cada una con otra carta escogida al azar.
Es posible que la carta se intercambie consigo misma, pero esto no
causa ningún problema. De hecho, si prohibiéramos esto, el orden de
las cartas no sería tan aleatorio:

\inputencoding{latin9}\begin{lstlisting}
class Mazo:
  ...
  def barajar(self):
    import random
    nCartas = len(self.cartas)
    for i in range(nCartas):
      j = random.randrange(i, nCartas)
      self.cartas[i], self.cartas[j] = self.cartas[j], \
                                       self.cartas[i]
\end{lstlisting}
\inputencoding{utf8} En vez de asumir que hay 52 cartas en el mazo, obtenemos el número
de ellas a través de la función len y lo almacenamos en la variable
\texttt{nCartas}.

\index{intercambiar} \index{asignación de tuplas} \index{asignación!de tuplas}

Para cada carta en el mazo, escogemos, aleatoriamente, una carta de
las que no han sido barajadas todavía. Intercambiamos la carta actual
(con índice \texttt{i}) con la seleccionada (con índice \texttt{j}).
Para intercambiar las cartas usamos asignación de tuplas, como en
la sección ~\ref{tuple assignment}:\inputencoding{latin9}
\begin{lstlisting}
self.cartas[i], self.cartas[j] = self.cartas[j], \
                                 self.cartas[i]
\end{lstlisting}
\inputencoding{utf8}
\section{Eliminando y entregando cartas}

\index{eliminando cartas}

Otro método que sería útil para la clase \texttt{Mazo} es \texttt{eliminarCarta},
que toma una carta como parámetro, la remueve y retorna True si la
encontró en el mazo o False si no estaba:

\inputencoding{latin9}\begin{lstlisting}
class Mazo:
  ...
  def eliminarCarta(self, carta):
    if carta in self.cartas:
      self.cartas.remove(carta)
      return True
    else: 
      return True
\end{lstlisting}
\inputencoding{utf8} El operador \texttt{in} retorna True si el primer operando se encuentra
dentro del segundo, que debe ser una secuencia. Si el primer operando
es un objeto, Python usa el método \texttt{\_\_cmp\_\_} para determinar
la igualdad de elementos en la lista. Como la función \texttt{\_\_cmp\_\_}
en la clase \texttt{Carta} detecta igualdad profunda, el método \texttt{eliminarCarta}
detecta igualdad profunda.

\index{operador in} \index{operador!in}

Para entregar cartas necesitamos eliminar y retornar la primera carta
del mazo. El método \texttt{pop} de las listas proporciona esta funcionalidad:\inputencoding{latin9}
\begin{lstlisting}
class Mazo:
  ...
  def entregarCarta(self):
    return self.cards.pop()
\end{lstlisting}
\inputencoding{utf8} En realidad, \texttt{pop} elimina la {\em última} carta de la
lista, así que realmente estamos entregando cartas por la parte inferior,
y esto no causa ningún inconveniente.

\index{función booleana} \index{función!booleana}

Una operación más que podemos requerir es la función booleana \texttt{estaVacio},
que retorna True si el mazo está vacío:

\inputencoding{latin9}\begin{lstlisting}
class Mazo:
  ...
  def estaVacio(self):
    return (len(self.cartas) == 0)
\end{lstlisting}
\inputencoding{utf8}
\section{Glosario}
\begin{description}
\item [{Codificar:}] representar un conjunto de valores usando otro conjunto
de valores estableciendo una correspondencia entre ellos.
\item [{Atributo de clase:}] variable de una clase que está fuera de todos
los métodos. Puede ser accedida desde todos los métodos y se comparte
por todas las instancias de la clase.
\item [{Acumulador:}] variable que se usa para acumular una serie de valores
en un ciclo. Por ejemplo, concatenar varios valores en una cadena
o sumarlos.

\index{codificar} \index{atributo de clase} \index{atributo!de clase}
\index{acumulador}
\end{description}

\section{Ejercicios}
\begin{enumerate}
\item Modifique \texttt{\_\_cmp\_\_} para que los Ases tengan mayor puntaje
que los reyes.
\item Reescriba el intercambio que se hace en \texttt{barajar} sin usar
asignación de tuplas.
\item Escriba una clase \texttt{secuenciaADN} que permita representar una
secuencia de ADN con un método \texttt{\_\_init\_\_} adecuado.
\item Agregue cuatro métodos a la clase para averiguar la cantidad de cada
nucleótido en la secuencia, cuantas A, G, C, T.
\end{enumerate}

