
\chapter{Tuplas}

\label{tuplechap} \index{tupla}

\section{Mutabilidad y tuplas}

\index{tupla} \index{tipo de dato!tupla} \index{tipo de dato!inmutable}

Hasta aquí, usted ha visto dos tipos de datos compuestos: cadenas,
que están compuestas de caracteres; y listas, compuestas de elementos
de cualquier tipo. Una de las diferencias que notamos es que los elementos
de una lista pueden modificarse, pero los caracteres en una cadena
no. En otras palabras, las cadenas son \textbf{inmutables} y las listas
son \textbf{mutables}.

\index{mutable} \index{inmutable}

Hay otro tipo de dato en Python denominado \textbf{tupla}, que se
parece a una lista, con la excepción de que es inmutable. Sintácticamente,
una tupla es una lista de valores separados por comas:\inputencoding{latin9}
\begin{lstlisting}
>>> tupla = 'a', 'b', 'c', 'd', 'e'
\end{lstlisting}
\inputencoding{utf8}
Aunque no es necesario, se pueden encerrar entre paréntesis:\inputencoding{latin9}
\begin{lstlisting}
>>> tupla = ('a', 'b', 'c', 'd', 'e')
\end{lstlisting}
\inputencoding{utf8}
Para crear una tupla con un único elemento, tenemos que incluir la
coma final:\inputencoding{latin9}
\begin{lstlisting}
>>> t1 = ('a',)
>>> type(t1)
<type 'tuple'>
\end{lstlisting}
\inputencoding{utf8}
Sin la coma, Python creería que \texttt{('a')} es una cadena en paréntesis:\inputencoding{latin9}
\begin{lstlisting}
>>> t2 = ('a')
>>> type(t2)
<type 'string'>
\end{lstlisting}
\inputencoding{utf8}
Las operaciones sobre tuplas son las mismas que vimos con las listas.
El operador corchete selecciona un elemento de la tupla.\inputencoding{latin9}
\begin{lstlisting}
>>> tupla = ('a', 'b', 'c', 'd', 'e')
>>> tupla[0]
'a'
\end{lstlisting}
\inputencoding{utf8}
Y el operador segmento selecciona un rango de elementos:\inputencoding{latin9}
\begin{lstlisting}
>>> tupla[1:3]
('b', 'c')
\end{lstlisting}
\inputencoding{utf8}
Pero si intentamos modificar un elemento de la tupla obtenemos un
error:

\index{error en tiempo de ejecución}\inputencoding{latin9}
\begin{lstlisting}
>>> tupla[0] = 'A'
TypeError: object doesn't support item assignment
\end{lstlisting}
\inputencoding{utf8}
Aunque no podemos modificar los elementos, sí podemos modificar toda
la tupla:\inputencoding{latin9}
\begin{lstlisting}
>>> tupla = ('A',) + tupla[1:]
>>> tupla
('A', 'b', 'c', 'd', 'e')
>>> tupla = (1,2,3)
>>> tupla
\end{lstlisting}
\inputencoding{utf8}
\section{Asignación de tuplas}

\label{tuple assignment} \index{asignación de tuplas} \index{Asignación!de tuplas}

De vez en cuando necesitamos intercambiar los valores de dos variables.
Con el operador de asignación normal tenemos que usar una variable
temporal. Por ejemplo, para intercambiar \texttt{a} y \texttt{b}:\inputencoding{latin9}
\begin{lstlisting}
>>> temp = a
>>> a = b
>>> b = temp
\end{lstlisting}
\inputencoding{utf8}
Si tenemos que intercambiar variables muchas veces, el código tiende
a ser engorroso. Python proporciona una forma de \textbf{asignación
de tuplas} que resuelve este problema:\inputencoding{latin9}
\begin{lstlisting}
>>> a, b = b, a
\end{lstlisting}
\inputencoding{utf8}
El lado izquierdo es una tupla de variables; el derecho es una tupla
de valores. Cada valor se asigna a su respectiva variable en el orden
en que se presenta. Las expresiones en el lado derecho se evalúan
antes de cualquier asignación. Esto hace a la asignación de tuplas
una herramienta bastante versátil.

Naturalmente, el número de variables a la izquierda y el número de
valores a la derecha deben coincidir.\inputencoding{latin9}
\begin{lstlisting}
>>> a, b, c, d = 1, 2, 3
ValueError: unpack tuple of wrong size
\end{lstlisting}
\inputencoding{utf8}
\section{Tuplas como valores de retorno}

\index{tupla} \index{valor!tupla} \index{valor de retorno!tupla}
\index{función!tupla como valor de retorno}

Las funciones pueden tener tuplas como valores de retorno. Por ejemplo,
podríamos escribir una función que intercambie sus dos parámetros:\inputencoding{latin9}
\begin{lstlisting}
def intercambiar(x, y):
  return y, x
\end{lstlisting}
\inputencoding{utf8}
Así podemos asignar el valor de retorno a una tupla con dos variables:\inputencoding{latin9}
\begin{lstlisting}
a, b = intercambiar(a, b)
\end{lstlisting}
\inputencoding{utf8}
En este caso, escribir una función \texttt{intercambio} no es muy
provechoso. De hecho, hay un peligro al tratar de encapsular \texttt{intercambio},
que consiste en el siguiente error:\inputencoding{latin9}
\begin{lstlisting}
def intercambio(x, y):      # version incorrecta
  x, y = y, x
\end{lstlisting}
\inputencoding{utf8}
Si llamamos a esta función así:\inputencoding{latin9}
\begin{lstlisting}
intercambio(a, b)
\end{lstlisting}
\inputencoding{utf8}
entonces \texttt{a} y \texttt{x} son dos alias para el mismo valor.
Cambiar \texttt{x} dentro de \texttt{intercambio} hace que \texttt{x}
se refiera a un valor diferente, pero no tiene efecto en la \texttt{a}
dentro de \texttt{\_\_main\_\_}. Igualmente, cambiar \texttt{y} no
tiene efecto en \texttt{b}.

Esta función se ejecuta sin errores, pero no hace lo que se pretende.
Es un ejemplo de error semántico.

\index{error semántico}

\section{Números aleatorios}

\index{número aleatorio} \index{número!aleatorio}

La gran mayoría de los programas hacen lo mismo cada vez que se ejecutan,
esto es, son \textbf{determinísticos}. El determinismo generalmente
es una buena propiedad, ya que usualmente esperamos que los cálculos
produzcan el mismo resultado. Sin embargo, para algunas aplicaciones
necesitamos que el computador sea impredecible. Los juegos son un
ejemplo inmediato, pero hay más.

Lograr que un programa sea verdaderamente no determinístico no es
una tarea fácil, pero hay formas de que parezca no determinístico.
Una de ellas es generar números aleatorios y usarlos para determinar
la salida de un programa. Python tiene una función primitiva que genera
números \textbf{pseudoaleatorios}, que, aunque no sean aleatorios
desde el punto de vista matemático, sirven para nuestros propósitos.

El módulo \texttt{random} contiene una función llamada \texttt{random}
que retorna un número flotante entre 0.0 y 1.0. Cada vez que se llama
a \texttt{random}, se obtiene el siguiente número de una serie muy
larga. Para ver un ejemplo ejecute el siguiente ciclo:\inputencoding{latin9}
\begin{lstlisting}
import random

for i in range(10):
  x = random.random()
  print(x)
\end{lstlisting}
\inputencoding{utf8}
Para generar un número aleatorio entre 0.0 y un límite superior como
\texttt{sup}, multiplique \texttt{x} por \texttt{sup}.

\section{Lista de números aleatorios}

Vamos a escribir una función llamada \texttt{listaAleatoria} que cree
una lista de números aleatorios. Recibirá un parámetro entero que
especifique el número de elementos a generar. Primero, genera una
lista de \texttt{n} ceros. Luego cada vez que itera en un ciclo for,
reemplaza uno de los ceros por un número aleatorio. El valor de retorno
es una referencia a la lista construida:\inputencoding{latin9}
\begin{lstlisting}
def listaAleatoria(n):
  s = [0] * n
  for i in range(n):
    s[i] = random.random()
  return s
\end{lstlisting}
\inputencoding{utf8}
La probaremos con ocho elementos. Para depurar es una buena idea empezar
con pocos datos:\inputencoding{latin9}
\begin{lstlisting}
>>> listaAleatoria(8)
0.15156642489
0.498048560109
0.810894847068
0.360371157682
0.275119183077
0.328578797631
0.759199803101
0.800367163582
\end{lstlisting}
\inputencoding{utf8}
Los números generados por \texttt{random} deben distribuirse uniformemente,
lo que significa que cada valor es igualmente probable.

Si dividimos el rango de valores posibles en ``regiones'' del mismo
tamaño y contamos el número de veces que un valor aleatorio cae en
cada región, deberíamos obtener un resultado aproximado en todas las
regiones.

Podemos probar esta hipótesis escribiendo un programa que divida el
rango en regiones y cuente el número de valores que caen en cada una.

\section{Conteo}

\index{conteo}

Un enfoque que funciona en problemas como éste es dividir el problema
en subproblemas que se puedan resolver con un patrón computacional
que ya sepamos.

En este caso, necesitamos recorrer una lista de números y contar el
número de veces que un valor cae en un rango dado. Esto parece familiar.
En la Sección~\ref{counter}, escribimos un programa que recorría
una cadena y contaba el números de veces que aparecía una letra determinada.

Entonces podemos copiar el programa viejo para adaptarlo posteriormente
a nuestro problema actual. El original es:\inputencoding{latin9}
\begin{lstlisting}
cont = 0
for c in fruta:
  if c == 'a':
    cont = cont + 1
print(cont)
\end{lstlisting}
\inputencoding{utf8}
El primer paso es reemplazar \texttt{fruta} con \texttt{lista} y \texttt{c}
por \texttt{num}. Esto no cambia el programa, sólo lo hace más legible.

El segundo paso es cambiar la prueba. No queremos buscar letras. Queremos
ver si \texttt{num} está entre dos valores dados \texttt{inf} y \texttt{sup}.

\inputencoding{latin9}\begin{lstlisting}
cont = 0
for num in lista:
  if inf < num < sup:
    cont = cont + 1
print(cont)
\end{lstlisting}
\inputencoding{utf8}
El último paso consiste en encapsular este código en una función denominada
\texttt{enRegion}. Los parámetros son la lista y los valores \texttt{inf}
y \texttt{sup}.\inputencoding{latin9}
\begin{lstlisting}
def enRegion(lista, inf, sup):
  cont = 0
  for num in lista:
    if inf < num < sup:
      cont = cont + 1
  return cont
\end{lstlisting}
\inputencoding{utf8}
Copiando y modificando un programa existente fuimos capaces de escribir
esta función rápidamente y ahorrarnos un buen tiempo de depuración.
Este plan de desarrollo se denomina \textbf{concordancia de patrones}.
Si se encuentra trabajando en un problema que ya ha resuelto antes,
reutilice la solución.

\section{Muchas regiones}

\label{muchasregiones}

Como el número de regiones aumenta, \texttt{enRegion} es un poco engorroso.
Con dos no esta tan mal:\inputencoding{latin9}
\begin{lstlisting}
inf = enRegion(a, 0.0, 0.5)
sup = enRegion(a, 0.5, 1)
\end{lstlisting}
\inputencoding{utf8} Pero con cuatro:\inputencoding{latin9}
\begin{lstlisting}
Region1 = enRegion(a, 0.0, 0.25)
Region2 = enRegion(a, 0.25, 0.5)
Region3 = enRegion(a, 0.5, 0.75)
Region4 = enRegion(a, 0.75, 1.0)
\end{lstlisting}
\inputencoding{utf8}
Hay dos problemas. Uno es que siempre tenemos que crear nuevos nombres
de variables para cada resultado. El otro es que tenemos que calcular
el rango de cada región.

Primero resolveremos el segundo problema. Si el número de regiones
está dado por la variable \texttt{numRegiones}, entonces, el ancho
de cada región está dado por la expresión \texttt{1.0/numRegiones}.

Usaremos un ciclo para calcular el rango de cada región. La variable
de ciclo \texttt{i} cuenta de 1 a \texttt{numRegiones-1}:\inputencoding{latin9}
\begin{lstlisting}
ancho = 1.0 / numRegiones
for i in range(numRegiones):
  inf = i * ancho
  sup = inf + ancho
  print(inf, " hasta ", sup)
\end{lstlisting}
\inputencoding{utf8}
Para calcular el extremo inferior de cada región, multiplicamos la
variable de ciclo por el ancho. El extremo superior está a un \texttt{ancho}
de región de distancia.

Con \texttt{numRegiones = 8}, la salida es:
\begin{verbatim}
0.0 hasta 0.125
0.125 hasta 0.25
0.25 hasta 0.375
0.375 hasta 0.5
0.5 hasta 0.625
0.625 hasta 0.75
0.75 hasta 0.875
0.875 hasta 1.0
\end{verbatim}
Usted puede confirmar que cada región tiene el mismo ancho, que no
se solapan y que cubren el rango completo de 0.0 a 1.0.

Ahora regresemos al primer problema. Necesitamos una manera de almacenar
ocho enteros, usando una variable para indicarlos uno a uno. Usted
debe estar pensando ``¡una lista!''

Tenemos que crear la lista de regiones fuera del ciclo, porque esto
sólo debe ocurrir una vez. Dentro del ciclo, llamaremos a \texttt{enRegion}
repetidamente y actualizaremos el \texttt{i}ésimo elemento de la lista:\inputencoding{latin9}
\begin{lstlisting}
numRegiones = 8
Regiones = [0] * numRegiones
ancho = 1.0 / numRegiones
for i in range(numRegiones):
  inf = i * ancho
  sup = inf + ancho
  Regiones[i] = enRegion(lista, inf, sup)
print(Regiones)
\end{lstlisting}
\inputencoding{utf8}
Con una lista de 1000 valores, este código produce la siguiente lista
de conteo:
\begin{verbatim}
[138, 124, 128, 118, 130, 117, 114, 131]
\end{verbatim}
Todos estos valores están muy cerca a 125, que es lo que esperamos.
Al menos, están lo suficientemente cerca como para creer que el generador
de números pseudoaleatorios está funcionando bien.

\section{Una solución en una sola pasada}

\label{histograma} \index{histograma}

Aunque funciona, este programa no es tan eficiente como debería. Cada
vez que llama a \texttt{enRegion}, recorre la lista entera. A medida
que el número de regiones incrementa, va a hacer muchos recorridos.

Sería mejor hacer una sola pasada a través de la lista y calcular
para cada región el índice de la región en la que cae. Así podemos
incrementar el contador apropiado.

En la sección anterior tomamos un índice \texttt{i} y lo multiplicamos
por el \texttt{ancho} para encontrar el extremo inferior de una región.
Ahora vamos a encontrar el índice de la región en la que cae.

Como este problema es el inverso del anterior, podemos intentar dividir
por \texttt{ancho} en vez de multiplicar. ¡Esto funciona!

Como \texttt{ancho = 1.0 / numRegiones}, dividir por \texttt{ancho}
es lo mismo que multiplicar por \texttt{numRegiones}. Si multiplicamos
un número en el rango 0.0 a 1.0 por \texttt{numRegiones}, obtenemos
un número en el rango de 0.0 a \texttt{numRegiones}. Si redondeamos
ese número al entero más cercano por debajo, obtenemos lo que queremos,
un índice de región:\inputencoding{latin9}
\begin{lstlisting}
numRegiones = 8
Regiones = [0] * numRegiones
for i in lista:
  ind = int(i * numRegiones)
  Regiones[ind] = Regiones[ind] + 1
\end{lstlisting}
\inputencoding{utf8}
Usamos la función \texttt{int} para pasar de número de punto flotante
a entero.

¿Es posible que este programa produzca un índice que esté fuera de
rango (por ser negativo o mayor que \texttt{len(Regiones)-1})?

Una lista como \texttt{Regiones} que almacena los conteos del número
de valores que hay en cada rango se denomina \textbf{histograma}.

\section{Glosario}
\begin{description}
\item [{Tipo inmutable:}] es un tipo de dato en el que los elementos no
pueden ser modificados. Las asignaciones a elementos o segmentos de
tipos inmutables causan errores. Las cadenas y las tuplas son inmutables.
\item [{Tipo mutable:}] tipo de dato en el que los elementos pueden ser
modificados. Todos los tipos mutables son compuestos. Las listas y
los diccionarios son mutables.
\item [{Tupla:}] tipo de dato secuencial similar a la lista, pero inmutable.
Las tuplas se pueden usar donde se requiera un tipo inmutable, por
ejemplo como llaves de un diccionario.
\item [{Asignación de tuplas:}] una asignación a todos los elementos
de una tupla en una sola sentencia. La asignación ocurre en paralelo
y no secuencialmente. Es útil para intercambiar valores de variables.
\item [{Determinístico:}] programa que hace lo mismo cada vez que se llama.
\item [{Pseudoaleatoria:}] secuencia de números que parece aleatoria, pero
en realidad es el resultado de un cómputo determinístico, bien diseñado.
\item [{Histograma:}] lista de enteros en la que cada elemento cuenta el
número de veces que algo sucede.
\item [{Correspondencia de patrones:}] plan de desarrollo de programas
que implica identificar un patrón computacional familiar y copiar
la solución de un problema similar.

\index{tipo mutable} \index{tipo inmutable} \index{tupla} \index{asignación de tupla}
\index{asignación!de tupla} \index{determinístico} \index{pseudoaleatorio}
\index{histograma} \index{correspondencia de patrones}
\end{description}

\section{Ejercicios}

Para cada función, agregue chequeo de tipos y pruebas unitarias.
\begin{enumerate}
\item Escriba una función mas\_frecuentes que tome una cadena e imprima
las letras en orden descendente por frecuencia. Ejecútela con textos
de diferentes lenguajes y observe como varían las frecuencias de letras.
Compare sus resultados con las tablas en:

\url{http://en.wikipedia.org/wiki/Letter_frequencies}

Solución: \url{http://thinkpython.com/code/most_frequent.py}
\item Escriba un programa que lea una lista de palabras de un archivo e
imprima todos los conjuntos de palabras que son anagramas.

Este es un ejemplo de la salida del programa con palabras en inglés:
\begin{verbatim}
['deltas', 'desalt', 'lasted', 'salted', 'slated', 'staled'] 
['retainers', 'ternaries'] 
['generating', 'greatening']
['resmelts', 'smelters', 'termless']
\end{verbatim}
Pista: cree un diccionario que asocie cada conjunto de letras a una
lista de palabras que puedan ser formadas con esas letras. ¿Como se
puede representar el conjunto de letras de forma que pueda ser usado
como llave? Modifique el programa que obtuvo para que imprima en orden
descendente por tamaño los conjuntos de anagramas. En el juego Scrabble,
un 'bingo' se da cuando se juegan las 7 fichas, junto con otra letra
en el tablero, para formar una palabra de 8 letras. ¿Que conjunto
de 8 letras forma los bingos mas posibles?

Solución: \url{http://thinkpython.com/code/anagram_sets.py}
\item Dos palabras forma un 'par de metatesis' si se puede transformar una
en otra intercambiando dos letras. Por ejemplo, 'conversación' y 'conservación'.
Escriba un programa que encuentre todos los pares de metatesis en
el diccionario. Pista: no pruebe todos los pares.

Solución: \url{http://thinkpython.com/code/metathesis.py}

Crédito: el ejercicio está inspirado por un ejemplo de \url{http://puzzlers.org}
\item ¿Cual es la palabra mas larga que sigue siendo válida a medida que
se remueven una a una sus letras? Por ejemplo, en inglés, 'sprite'
sin la 'r' es 'spite', que sin la 'e', es 'spit', que sin la 's' es
'pit', que sin la 'p' es 'it' que sin la 't' es 'i'. Las letras se
pueden remover de cualquier posición, pero no se pueden reordenar.

Escriba un programa que encuentre todas las palabras que pueden reducirse
de esta forma y que encuentre la mas larga.

Pistas:

Escriba una función que tome una palabra y calcule todas las palabras
que pueden formarse al removerle una letra. Estas son las palabras
'hijas'. Recursivamente, una palabra es reducible si alguno de sus
hijas es reducible. El caso base lo da la cadena vacía.

Solución: \url{http://thinkpython.com/code/reducible.py} 
\end{enumerate}

