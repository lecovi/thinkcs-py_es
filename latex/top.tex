%% LyX 2.2.3 created this file.  For more info, see http://www.lyx.org/.
%% Do not edit unless you really know what you are doing.
\documentclass[10pt,spanish,letter]{book}
\usepackage[utf8x]{inputenc}
\setcounter{secnumdepth}{3}
\setcounter{tocdepth}{3}
\setlength{\parskip}{1.7ex}
\setlength{\parindent}{0pt}
\usepackage{makeidx}
\makeindex

\makeatletter
%%%%%%%%%%%%%%%%%%%%%%%%%%%%%% User specified LaTeX commands.
% LaTeX source for the spanish traslation of the textbook ``How to think like a computer scientist''
% Copyright (c)  2001,2002  Allen B. Downey.
% Traslation completed by
% Andrés Becerra Sandoval
% abecerra@cic.puj.edu.co


% Permission is granted to copy, distribute and/or modify this
% document under the terms of the GNU Free Documentation License,
% Version 1.1  or any later version published by the Free Software
% Foundation; with the Invariant Sections being "Contributor List",
% with no Front-Cover Texts, and with no Back-Cover Texts. A copy of
% the license is included in the section entitled "GNU Free
% Documentation License".

% This distribution includes a file named fdl.tex that contains the text
% of the GNU Free Documentation License.  If it is missing, you can obtain
% it from www.gnu.org or by writing to the Free Software Foundation,
% Inc., 59 Temple Place - Suite 330, Boston, MA 02111-1307, USA.
%

\usepackage[spanish]{babel}
\usepackage{fancyhdr}
\usepackage{graphicx}
\usepackage{ucs}
\usepackage{hyperref}
%\pssilent

%%%%%%%%%%%%%%%%%%%%%%%%%%%%%%%%%%%%%%
%         INCLUDE ONLY 
%\includeonly{chap20}
%FIX
%\setcounter{chapter}{20}

%%%%%%%%%%%%%%%%%%%%%%%%%%%%%%%%%%%%%%

\sloppy
\setlength{\topmargin}{-0.375in}
\setlength{\oddsidemargin}{0.0in}
\setlength{\evensidemargin}{0.0in}

% Uncomment these to center on 8.5 x 11
\setlength{\topmargin}{0.625in}
\setlength{\oddsidemargin}{0.875in}
\setlength{\evensidemargin}{0.875in}

\setlength{\headsep}{3ex}
\setlength{\textheight}{7.2in}



\renewcommand{\baselinestretch}{1.02}

% see LaTeX Companion page 62
\setlength{\topsep}{-0.0\parskip}
\setlength{\partopsep}{-0.5\parskip}
\setlength{\itemindent}{0.0in}
\setlength{\listparindent}{0.0in}

% see LaTeX Companion page 26
% these are copied from /usr/local/teTeX/share/texmf/tex/latex/base/book.cls
% all I changed is afterskip


\renewcommand{\section}{\@startsection 
    {section} {1} {0mm}%
    {-3.5ex \@plus -1ex \@minus -.2ex}%
    {0.7ex \@plus.2ex}%
    {\normalfont\Large\bfseries}}
\renewcommand{\subsection}{\@startsection {subsection}{2}{0mm}%
    {-3.25ex\@plus -1ex \@minus -.2ex}%
    {0.3ex \@plus .2ex}%
    {\normalfont\large\bfseries}}
\renewcommand{\subsubsection}{\@startsection {subsubsection}{3}{0mm}%
    {-3.25ex\@plus -1ex \@minus -.2ex}%
    {0.3ex \@plus .2ex}%
    {\normalfont\normalsize\bfseries}}


\newcommand{\beforefig}{\vspace{1.3\parskip}}
\newcommand{\afterfig}{\vspace{-0.2\parskip}}

\newcommand{\beforeverb}{\vspace{0.6\parskip}}
\newcommand{\afterverb}{\vspace{0.6\parskip}}

\newcommand{\adjustpage}[1]{\enlargethispage{#1\baselineskip}}
\newcommand{\clearemptydoublepage}{\newpage{\pagestyle{empty}\cleardoublepage}}
\newcommand{\blankpage}{\pagestyle{empty}\vspace*{1in}\newpage}



\renewcommand{\chaptermark}[1]{\markboth{#1}{}}
\renewcommand{\sectionmark}[1]{\markright{\thesection\ #1}{}}

\lhead[\fancyplain{}{\bfseries\thepage}]%
      {\fancyplain{}{\bfseries\rightmark}}
\rhead[\fancyplain{}{\bfseries\leftmark}]%
      {\fancyplain{}{\bfseries\thepage}}
\cfoot{}

% turn off the rule under the header
%\setlength{\headrulewidth}{0pt}

% the following is a brute-force way to prevent the headers
% from getting transformed into all-caps
\renewcommand{\MakeUppercase}{}




\makeatother

\usepackage{babel}
\addto\shorthandsspanish{\spanishdeactivate{~<>}}

\begin{document}
\frontmatter

%-half title--------------------------------------------------\thispagestyle{empty}

\begin{flushright}
\vspace*{2.5in}
 {\huge{}Como Pensar como un Científico de la Computación} \vspace{0.25in}
 {\LARGE{}con Python} \vfill{}
\par\end{flushright}

%--verso------------------------------------------------------\clearemptydoublepage
%\pagebreak%\thispagestyle{empty}%\vspace*{6in}

%--title page--------------------------------------------------\pagebreak{}\thispagestyle{empty}

\begin{flushright}
\vspace*{2.5in}
 {\huge{}Como Pensar como un Científico de la Computación} \vspace{0.25in}
 {\LARGE{}con Python} \vspace{0.5in}
 
\par\end{flushright}

\author{Allen Downey, Jeffrey Elkner y Chris Meyers}
%\maketitle
\begin{flushright}
{\small{}Allen Downey}\\
{\small{} Jeffrey Elkner}\\
{\small{} Chris Meyers}\\
{\small{} }
\par\end{flushright}{\small \par}

\begin{flushright}
\vspace{0.25in}
\par\end{flushright}

\begin{flushright}
{\small{}Traducido y adaptado por}\\
{\small{} Andrés Becerra Sandoval }\\
{\small{} }
\par\end{flushright}{\small \par}

\begin{flushright}
\vspace{0.5in}
 {\Large{}Pontificia Universidad Javeriana} \\
 {\small{}Santiago de Cali, Colombia} %\includegraphics{illustrations/logo1.eps,width=1in}\vfill{}
\par\end{flushright}

%--copyright--------------------------------------------------\pagebreak{}\thispagestyle{empty}
Copyright \copyright ~2002 Allen Downey, Jeffrey Elkner, y Chris
Meyers. \vspace{0.25in}
 %Corregido por Shannon Turlington y Lisa Cutler.%Diseño de la cubierta por Rebecca Gimenez.

\vspace{0.25in}
\begin{flushleft}
Pontificia Universidad Javeriana \\
 Calle 18 No. 118-250 \\
 A.A. No. 26239 \\
 Cali, Colombia \\
 %Green Tea Press       \\%1 Grove St.           \\%P.O. Box 812901       \\%Wellesley, MA 02482   \\
\par\end{flushleft}

\vspace{0.25in}

Se concede permiso para copiar, distribuir, y/o modificar este documento
bajo los terminos de la GNU Free Documentation License, Versión 1.1
o cualquier versión posterior publicada por la Free Software Foundation;
manteniendo sin variaciones las secciones ``Prólogo,'' ``Prefacio,''
y ``Lista de contribuidores,'' sin texto de cubierta, y sin texto
de contracubierta. Una copia de la licencia está incluida en el apéndice
titulado ``GNU Free Documentation License'' y una traducción de
ésta al español en el apéndice titulado Licencia de Documentación
Libre de GNU

La GNU Free Documentation License está disponible a través de \texttt{www.gnu.org}
o escribiendo a la Free Software Foundation, Inc., 59 Temple Place,
Suite 330, Boston, MA 02111-1307, USA.

La forma original de este libro es código fuente \LaTeX{}\ y compilarlo
tiene el efecto de generar un libro de texto en una repesentacion
independiente del dispositivo que puede ser convertida a otros formatos
e imprimirse.

El código fuente \LaTeX{}\ para este libro y mas información sobre
este proyecto se encuentra en:
\begin{verbatim}
      http://www.thinkpython.com
\end{verbatim}
Este libro ha sido preparado utilizando \LaTeX{}\ y las figuras
se han realizado con xfig. Todos estos son programas de código abierto,
gratuito.

\vspace{0.25in}

Historia de la impresión:

%\begin{description}%\item[Abril 2002:] Primera edición.%\end{description}

%ISBN 0-9716775-0-6
\begin{verbatim}

\end{verbatim}
%-----------------------------------------------------------------

\chapter{Prólogo}

Por David Beazley

Como educador, investigador y autor de libros, estoy encantado de
ver la terminación de este texto. Python es un lenguaje de programación
divertido y extremadamente fácil de usar que ha ganado renombre constantemente
en los años recientes. Desarrollado hace diez años por Guido van Rossum,
la sintaxis simple de Python y su ``sabor'' se derivan, en gran
parte del ABC, un lenguaje de programación para enseñanza que se desarrolló
en los 1980s. Sin embargo, Python también fue creado para resolver
problemas reales y tiene una amplia gama de características que se
encuentran en lenguajes de programación como C++, Java, Modula-3,
y Scheme. Debido a esto, uno de las características notables de Python
es la atracción que ejerce sobre programadores profesionales, científicos,
investigadores, artistas y educadores.

A pesar de ésta atracción que ejerce en muchas comunidades diversas,
usted puede todavía preguntarse ``¿porque Python?'' o ``¿porque
enseñar programación con Python?'' Responder éstas preguntas no es
una tarea fácil— especialmente cuando la opinión popular está del
lado masoquista de usar alternativas como C++ y Java. Sin embargo,
pienso que la respuesta más directa es que la programación en Python
es simplemente más divertida y más productiva.

Cuando enseño cursos de informática, yo quiero cubrir conceptos importantes,
hacer el material interesante y enganchar a los estudiantes. Desafortunadamente,
hay una tendencia en la que los cursos de programación introductorios
dedican demasiada atención hacia la abstracción matemática y a hacer
que los estudiantes se frustren con problemas molestos relacionados
con la sintaxis, la compilación y la presencia de reglas arcanas en
los lenguajes. Aunque la abstracción y el formalismo son importantes
para los ingenieros de software y para los estudiantes de ciencias
de la computación, usar este enfoque hace la informática muy aburrida.
Cuando enseño un curso no quiero tener un grupo de estudiantes sin
inspiración. Quisiera verlos intentando resolver problemas interesantes,
explorando ideas diferentes, intentando enfoques no convencionales,
rompiendo reglas y aprendiendo de sus errores. En el proceso no quiero
perder la mitad del semestre tratando de resolver problemas sintácticos
oscuros, interpretando mensajes de error del compilador incomprensibles,
o descifrando cuál de las muchas maneras en que un programa puede
generar un error grave de memoria se está presentando.

Una de las razones del por qué me gusta Python es que proporciona
un equilibrio muy bueno entre lo práctico y lo conceptual. Puesto
que se interpreta Python, los principiantes pueden empezar a hacer
cosas interesantes casi de inmediato sin perderse en problemas de
compilación y enlace. Además, Python viene con una biblioteca grande
de módulos, que pueden ser usados en dominios que van desde programación
en la web hasta aplicaciones gráficas. Tener un foco práctico es una
gran manera de enganchar a los estudiantes y permite que emprendan
proyectos significativos. Sin embargo, Python también puede servir
como una excelente base para introducir conceptos importantes de la
informática. Puesto que Python soporta completamente procedimientos
y clases, los estudiantes pueden ser introducidos gradualmente a temas
como la abstracción procedimental, las estructuras de datos y la programación
orientada a objetos—lo que se puede aplicar después a cursos posteriores
en Java o C++. Python proporciona, incluso, varias características
de los lenguajes de programación funcionales y puede usarse para introducir
conceptos que se pueden explorar con más detalle en cursos con Scheme
y Lisp.

Leyendo, el prefacio de Jeffrey, estoy sorprendido por sus comentarios
de que Python le permita ver un ``más alto nivel de éxito y un nivel
bajo de frustración'' y que puede ``avanzar mas rápido con mejores
resultados.'' Aunque estos comentarios se refieren a sus cursos introductorios,
a veces uso Python por estas mismas razones en los cursos de informática
avanzada en la Universidad de Chicago. En estos cursos enfrento constantemente
la tarea desalentadora de cubrir un montón de material difícil durante
nueve semanas. Aunque es totalmente posible para mi infligir mucho
dolor y sufrimiento usando un lenguaje como C++, he encontrado a menudo
que este enfoque es improductivo—especialmente cuando el curso se
trata de un asunto sin relación directa con la ``programación.''
He encontrado que usar Python me permite enfocar el tema del curso
y dejar a los estudiantes desarrollar proyectos substanciales.

Aunque Python siga siendo un lenguaje joven y en desarrollo, creo
que tiene un futuro brillante en la educación. Este libro es un paso
importante en esa dirección.

\vspace{0.25in}
 
\begin{flushleft}
David Beazley \\
Universidad de Chicago, Autor de {\em Python Essential Reference} 
\par
\end{flushleft}

\clearemptydoublepage

\chapter{Prefacio}

Por Jeff Elkner

Este libro debe su existencia a la colaboración hecha posible por
Internet y el movimiento de software libre. Sus tres autores—un profesor
de colegio, un profesor de secundaria y un programador profesional—tienen
todavía que verse cara a cara, pero han podido trabajar juntos y han
sido ayudados por maravillosas personas, quienes han donado su tiempo
y energía para ayudar a hacer ver mejor este libro.

Nosotros pensamos que este libro es un testamento a los beneficios
y futuras posibilidades de esta clase de colaboración, el marco que
se ha puesto en marcha por Richard Stallman y el movimiento de software
libre.

\section*{Cómo y porqué vine a utilizar Python}

En 1999, el examen del College Board's Advanced Placement (AP) de
Informática se hizo en C++ por primera vez. Como en muchas escuelas
de Estados Unidos, la decisión para cambiar el lenguaje tenía un impacto
directo en el plan de estudios de informática en la escuela secundaria
de Yorktown en Arlington, Virginia, donde yo enseño. Hasta este punto,
Pascal era el lenguaje de instrucción en nuestros cursos del primer
año y del AP. Conservando la práctica usual de dar a los estudiantes
dos años de exposición al mismo lenguaje, tomamos la decisión de cambiar
a C++ en el curso del primer año durante el periodo escolar 1997-98
de modo que siguiéramos el cambio del College Board's para el curso
del AP el año siguiente.

Dos años después, estoy convencido de que C++ no era una buena opción
para introducir la informática a los estudiantes. Aunque es un lenguaje
de programación de gran alcance, también es extremadamente difícil
de aprender y de enseñar. Me encontré constantemente peleando con
la sintaxis difícil de C++ y sus múltiples maneras de hacer las cosas,
y estaba perdiendo muchos estudiantes, innecesariamente, como resultado.
Convencido de que tenía que haber una mejor opción para nuestras clases
de primer año, fui en busca de una alternativa a C++.

Necesitaba un lenguaje que pudiera correr en las máquinas en nuestro
laboratorio Linux, también en las plataformas de Windows y Macintosh,
que muchos de los estudiantes tienen en casa. Quería que fuese un
lenguaje de código abierto, para que los estudiantes lo pudieran usar
en casa sin pagar por una licencia. Quería un lenguaje usado por programadores
profesionales, y que tuviera una comunidad activa alrededor de él.
Tenía que soportar la programación procedimental y orientada a objetos.
Y más importante, tenía que ser fácil de aprender y de enseñar. Cuando
investigué las opciones con estas metas en mente, Python saltó como
el mejor candidato para la tarea.

Pedí a uno de los estudiantes más talentosos de Yorktown, Matt Ahrens,
que le diera a Python una oportunidad. En dos meses él no sólo aprendió
el lenguaje, sino que escribió una aplicación llamada pyTicket que
permitió a nuestro personal atender peticiones de soporte tecnológico
vía web. Sabia que Matt no podría terminar una aplicación de esa escala
en tan poco tiempo con C++, y esta observación, combinada con el gravamen
positivo de Matt sobre Python, sugirió que este lenguaje era la solución
que buscaba.

\section*{Encontrando un libro de texto}

Al decidir utilizar Python en mis dos clases de informática introductoria
para el año siguiente, el problema más acuciante era la carencia de
un libro.

El contenido libre vino al rescate. A principios del año, Richard
Stallman me presentó a Allen Downey. Los dos habíamos escrito a Richard
expresando interés en desarrollar un contenido gratis y educativo.
Allen ya había escrito un libro de texto para el primer año de informática,
{\em Como pensar como un científico de la computación}. Cuando
leí este libro, inmediatamente quise usarlo en mi clase. Era el texto
más claro y mas provechoso de introducción a la informática que había
visto. Acentúa los procesos del pensamiento implicados en la programación
más bien que las características de un lenguaje particular. Leerlo
me hizo sentir un mejor profesor inmediatamente. {\em Como pensar
como un científico de la computación con Java} no solo es un libro
excelente, sino que también había sido publicado bajo la licencia
publica GNU, lo que significa que podría ser utilizado libremente
y ser modificado para resolver otras necesidades. Una vez que decidí
utilizar Python, se me ocurrió que podía traducir la versión original
del libro de Allen (en Java) al nuevo lenguaje (Python). Aunque no
podía escribir un libro de texto solo, tener el libro de Allen me
facilitó la tarea, y al mismo tiempo demostró que el modelo cooperativo
usado en el desarrollo de software también podía funcionar para el
contenido educativo.

Trabajar en este libro, por los dos últimos años, ha sido una recompensa
para mis estudiantes y para mí; y mis estudiantes tuvieron una gran
participación en el proceso. Puesto que podía realizar cambios inmediatos,
siempre que alguien encontrara un error de deletreo o un paso difícil,
yo les animaba a que buscaran errores en el libro, dándoles un punto
cada vez que hicieran una sugerencia que resultara en un cambio en
el texto. Eso tenía la ventaja doble de animarles a que leyeran el
texto más cuidadosamente y de conseguir la corrección del texto por
sus lectores críticos más importantes, los estudiantes usándolo para
aprender informática.

Para la segunda parte del libro, enfocada en la programación orientada
a objetos, sabía que alguien con más experiencia en programación que
yo era necesario para hacer el trabajo correctamente. El libro estuvo
incompleto la mayoría del año hasta que la comunidad de software abierto
me proporcionó de nuevo los medios necesarios para su terminación.

Recibí un correo electrónico de Chris Meyers, expresando su interés
en el libro. Chris es un programador profesional que empezó enseñando
un curso de programación el año anterior, usando Python en el Lane
Community College en Eugene, Oregon. La perspectiva de enseñar el
curso llevó a Chris al libro, y él comenzó a ayudarme inmediatamente.
Antes del fin de año escolar él había creado un proyecto complementario
en nuestro Sitio Web \url{http://www.ibiblio.org/obp}, titulado {\em
Python for Fun} y estaba trabajando con algunos de mis estudiantes
más avanzados como profesor principal, guiándolos mas allá de donde
yo podía llevarlos.

\section*{Introduciendo la programación con Python}

El proceso de uso y traducción de {\em Como pensar como un científico
de la computación}, por los últimos dos años, ha confirmado la conveniencia
de Python para enseñar a estudiantes principiantes. Python simplifica
bastante los ejemplos de programación y hace que las ideas importantes
sean más fáciles de enseñar.

El primer ejemplo del texto ilustra este punto. Es el tradicional
``hola, mundo'', programa que en la versión C++ del libro se ve
así:
\begin{verbatim}
   #include <iostream.h>

   void main()
   {
     cout << "Hola, mundo." << endl;
   }
\end{verbatim}
en la versión Python es:
\begin{verbatim}
    print("Hola, Mundo!")
\end{verbatim}
Aunque este es un ejemplo trivial, las ventajas de Python salen a
la luz. El curso de Informática I, en Yorktown, no tiene prerrequisitos,
es por eso que muchos de los estudiantes, que ven este ejemplo, están
mirando a su primer programa. Algunos de ellos están un poco nerviosos,
porque han oído que la programación de computadores es difícil de
aprender. La versión C++ siempre me ha forzado a escoger entre dos
opciones que no me satisfacen: explicar el \texttt{\#include}, \texttt{void
main()}, y las sentencias \{, y \} y arriesgar a confundir o intimidar
a algunos de los estudiantes al principio, o decirles, ``No te preocupes
por todo eso ahora; lo retomaré más tarde,'' y tomar el mismo riesgo.
Los objetivos educativos en este momento del curso son introducir
a los estudiantes la idea de sentencia y permitirles escribir su primer
programa. Python tiene exactamente lo que necesito para lograr esto,
y nada más.

Comparar el texto explicativo de este programa en cada versión del
libro ilustra más de lo que esto significa para los estudiantes principiantes.
Hay trece párrafos de explicación de ``Hola, mundo!'' en la versión
C++; en la versión Python, solo hay dos. Aún mas importante, los 11
párrafos que faltan no hablan de ``grandes ideas'' en la programación
de computadores, sino de minucias de la sintaxis de C++. Encontré
la misma situación al repasar todo el libro. Párrafos enteros desaparecían
en la versión Python del texto, porque su sencilla sintaxis los hacía
innecesarios.

Usar un lenguaje de muy alto nivel, como Python, le permite a un profesor
posponer los detalles de bajo nivel de la máquina hasta que los estudiantes
tengan el bagaje que necesitan para entenderlos. Permite ``poner
cosas primero'' pedagógicamente. Unos de los mejores ejemplos de
esto es la manera en la que Python maneja las variables. En C++ una
variable es un nombre para un lugar que almacena una cosa. Las variables
tienen que ser declaradas con tipos, al menos parcialmente, porque
el tamaño del lugar al cual se refieren tiene que ser predeterminado.
Así, la idea de una variable se liga con el hardware de la máquina.
El concepto poderoso y fundamental de variable ya es difícil para
los estudiantes principiantes (de informática y álgebra). Bytes y
direcciones de memoria no ayudan para nada. En Python una variable
es un nombre que se refiere a una cosa. Este es un concepto más intuitivo
para los estudiantes principiantes y está más cerca del significado
de ``variable'' que aprendieron en los cursos de matemática del
colegio. Yo me demoré menos tiempo ayudándolos con el concepto de
variable y en su uso este año, que en el pasado.

Otro ejemplo de cómo Python ayuda en la enseñanza y aprendizaje de
la programación es su sintaxis para las funciones. Mis estudiantes
siempre han tenido una gran dificultad comprendiendo las funciones.
El problema principal se centra alrededor de la diferencia entre una
definición de función y un llamado de función, y la distinción relacionada
entre un parámetro y un argumento. Python viene al rescate con una
bella sintaxis. Una definición de función empieza con la palabra clave
\texttt{def}, y simplemente digo a mis estudiantes: ``cuando definas
una función, empieza con \texttt{def}, seguido del nombre de la función
que estás definiendo, cuando llames una función, simplemente llama
(digita) su nombre.'' Los parámetros van con las definiciones y los
argumentos van con los llamados. No hay tipos de retorno, tipos para
los parámetros, o pasos de parámetro por referencia y valor, y ahora
yo puedo enseñar funciones en la mitad de tiempo que antes, con una
mejor comprensión.

Usar Python ha mejorado la eficacia de nuestro programa de informática
para todos los estudiantes. Veo un nivel general de éxito más alto
y un nivel más bajo de frustración, de lo que ya había experimentado
durante los dos años que enseñé C++. Avanzo más rápido y con mejores
resultados. Más estudiantes terminan el curso con la habilidad de
crear programas significativos; esto genera una actitud positiva hacia
la experiencia de la programación.

\section*{Construyendo una comunidad}

He recibido correos electrónicos de todas partes del mundo, de personas
que están usando este libro para aprender o enseñar programación.
Una comunidad de usuarios ha comenzado a emerger, y muchas personas
han contribuido al proyecto mandando materiales a través del sitio
Web complementario: \\
 \\
\url{ http://www.thinkpython.com }\\
 \\

Con la publicación del libro, en forma impresa, espero que continúe
y se acelere el crecimiento de esta comunidad de usuarios.

La emergencia de esta comunidad y la posibilidad que sugiere para
otras experiencias de colaboración similar entre educadores han sido
las partes más excitantes de trabajar en este proyecto, para mí. Trabajando
juntos, nosotros podemos aumentar la calidad del material disponible
para nuestro uso y ahorrar tiempo valioso.

Yo les invito a formar parte de nuestra comunidad y espero escuchar
de ustedes. Por favor escriba a los autores a \texttt{\url{feedback@thinkpython.com}}.

\vspace{0.25in}
 
\begin{flushleft}
Jeffrey Elkner\\
 Escuela Secundaria Yortown\\
 Arlington, Virginia.\\
 
\par\end{flushleft}



\clearemptydoublepage
% LaTeX source for textbook ``How to think like a computer scientist''
% Copyright (c)  2001  Allen B. Downey, Jeffrey Elkner, and John Dewey.

% Permission is granted to copy, distribute and/or modify this
% document under the terms of the GNU Free Documentation License,
% Version 1.1  or any later version published by the Free Software
% Foundation; with the Invariant Sections being "Contributor List",
% with no Front-Cover Texts, and with no Back-Cover Texts. A copy of
% the license is included in the section entitled "GNU Free
% Documentation License".

% This distribution includes a file named fdl.tex that contains the text
% of the GNU Free Documentation License.  If it is missing, you can obtain
% it from www.gnu.org or by writing to the Free Software Foundation,
% Inc., 59 Temple Place - Suite 330, Boston, MA 02111-1307, USA.

\chapter{Lista de los colaboradores}

Este libro vino a la luz debido a una colaboración que no sería posible sin 
la licencia de documentación libre de la GNU (Free Documentation License).  
Quisiéramos agradecer a la Free Software Foundation por desarrollar esta 
licencia y, por supuesto, por ponerla a nuestra disposición.

Nosotros queremos agradecer a los mas de 100 juiciosos y reflexivos lectores 
que nos han enviado sugerencias y correcciones durante los años pasados.  
En el espíritu del  software libre, decidimos expresar nuestro agradecimiento 
en la forma de una lista de colaboradores.  Desafortunadamente, esta lista no 
está completa, pero estamos haciendo nuestro mejor esfuerzo para mantenerla 
actualizada.

Si tiene la oportunidad de leer la lista, tenga en cuenta que cada persona 
mencionada aquí le ha ahorrado a usted y a todos los lectores subsecuentes  la 
confusión debida a un error técnico o debido a una explicación confusa, solo 
por enviarnos una nota.

Después de tantas correcciones, todavía pueden haber errores en este libro.  
Si ve uno, esperamos que tome un minuto para contactarnos.  El correo 
electrónico es 
\href{mailto:feedback@thinkpython.com}{feedback@thinkpython.com}. Si hacemos 
un cambio debido a su sugerencias,usted aparecerá en la siguiente versión de la 
lista de colaboradores (a menos que usted pida ser omitido). Gracias!

\begin{itemize}
	\item Lloyd Hugh Allen remitió una corrección a la Sección 8.4.
	%He can be reached at: \texttt{lha2@columbia.edu}
	\item Yvon Boulianne corrigió un error semántico en el Capítulo 5.
	%She can be reached at: \texttt{mystic@monuniverse.net}
	\item Fred Bremmer hizo una corrección en la Sección 2.1.
	%He can be reached at:  \texttt{Fred.Bremmer@ubc.cu}
	\item Jonah Cohen escribió los guiones en Perl para convertir la fuente
	LaTeX, de este libro, a un maravilloso HTML.
	%His Web page is \texttt{jonah.ticalc.org}
	%and his email is \texttt{JonahCohen@aol.com}
	\item Michael Conlon remitió una corrección de gramática en el Capítulo 3
	una mejora de estilo en el Capítulo 2, e inició la discusión de 
	los aspectos técnicos de los intérpretes.
	%Michael can be reached at: \texttt{michael.conlon@sru.edu}
	\item Benoit Girard envió una corrección a un extraño error en la Sección 
	5.6.
	%Benoit can be reached at:
	%\texttt{benoit.girard@gouv.qc.ca}
	\item Courtney Gleason y Katherine Smith escribieron \texttt{horsebet.py}, 
	que
	se usaba como un caso de estudio en una versión anterior de este libro.  Su 
	programa se puede encontrar en su website.
	%Courtney can be reached at: {\tt
	%orion1558@aol.com}
	\item Lee Harr sometió más correcciones de las que tenemos espacio para 
	enumerar aquí, 
	y, por supuesto, debería ser listado como uno de los editores principales 
	del 
	texto.
	%He can be reached at: \texttt{missive@linuxfreemail.com}
	\item James Kaylin es un estudiante usando el texto. Él ha enviado numerosas
	correcciones.
	%James can be reached by email at: \texttt{Jamarf@aol.com}
	\item David Kershaw arregló la función errónea \texttt{imprimaDoble} en 
	la Sección 3.10.
	%He can be reached at: \texttt{david\_kershaw@merck.com}
	\item Eddie Lam ha enviado numerosas correcciones a los Capítulos 1, 2, y 
	3.  Él
	corrigió el Makefile para que creara un índice, la primera vez
	que se compilaba el documento, y  nos ayudó a instalar un sistema
	de control de versiones.
	%Eddie can be reached at:
	%\texttt{nautilus@yoyo.cc.monash.edu.au}
	\item Man-Yong Lee envió una corrección al código de ejemplo en la
	Sección 2.4.  
	%He can be reaced at: \texttt{yong@linuxkorea.co.kr}
	\item David Mayo notó que la palabra ``inconscientemente"
	debe cambiarse por  ``subconscientemente".
	%David can be reached at:\texttt{bdbear44@netscape.net}
	\item Chris McAloon envió varias correcciones a las Secciones 3.9 y 3.10.
	%He can be reached at: \texttt{cmcaloon@ou.edu}
	\item Matthew J. Moelter ha sido un contribuidor de mucho tiempo quien 
	remitió 
	numerosas correcciones y sugerencias al libro.  
	%He can be reached at:
	%\texttt{mmoelter@calpoly.edu}
	\item Simon Dicon Montford reportó una definición de función que faltaba y 
	varios errores en el Capítulo 3.  Él también encontró errores en la
	función \texttt{incrementar} del Capítulo 13.
	%He can be reached at: \texttt{dicon@bigfoot.com}
	\item John Ouzts corrigió la definición de ``valor de retorno'' en el
	Capítulo 3.
	%He can be reached at: \texttt{jouzts@bigfoot.com}
	\item Kevin Parks envió  sugerencias valiosas para mejorar la distribución 
	del libro.
	%He can be reached at: \texttt{cpsoct@lycos.com}
	\item David Pool envió la corrección de un error en el glosario del 
	Capítulo 1 y palabras de estímulo.
	%He can be reached at: \texttt{pooldavid@hotmail.com}
	\item Michael Schmitt envió una corrección al capítulo de archivos y 
	excepciones.
	%He can be reached at: \texttt{ipv6\_128@yahoo.com}
	\item Robin Shaw notó un error en la Sección 13.1, donde la función
	imprimirHora se usaba en un ejemplo sin estar definida.
	%Robin can be reached at: \texttt{randj@iowatelecom.net}
	\item Paul Sleigh encontró un error en el Capítulo 7 y otro en los
	guiones de Perl de Jonah Cohen que generan HTML a partir de LaTeX.
	%He can be reached at: \texttt{bat@atdot.dotat.org}
	%\item Christopher Smith is a computer science teacher at the Blake
	%School in Minnesota who teaches Python to his beginning students.
	%He can be reached at: \texttt{csmith@blakeschool.org or 
	%smiles@saysomething.com}
	\item Craig T. Snydal está probando el texto en un curso en  Drew
	University. El ha aportado varias sugerencias valiosas y correcciones.
	%and can be reached at: \texttt{csnydal@drew.edu}
	\item Ian Thomas y sus estudiantes están usando el texto en un curso
	de programación. Ellos son los primeros en probar los capítulos de la
	segunda mitad del libro y han enviado numerosas correcciones y 
	sugerencias.
	%Ian can be reached at: \texttt{ithomas@sd70.bc.ca}
	\item Keith Verheyden envió una corrección al Capítulo 3.
	%He can be reached at: \texttt{kverheyd@glam.ac.uk}
	\item Peter Winstanley descubrió un viejo error en nuestro Latín,
	en el capítulo 3.
	%He can be reached at:\texttt{Peter.Winstanley@scotland.gsi.gov.uk} 
	\item Chris Wrobel hizo correcciones al código en el capítulo sobre 
	archivos, E/S y excepciones. 
	%He can be reached at: \texttt{ferz980@yahoo.com}
	\item Moshe Zadka hizo contribuciones inestimables a este proyecto.  Además
	de escribir el primer bosquejo del capítulo sobre Diccionarios, también 
	proporcionó una dirección continua en los primeros años del libro.
	%He can be reached at: \texttt{moshez@math.huji.ac.il}
	\item Christoph Zwerschke envió varias correcciones y sugerencias 
	pedagógicas, y explicó la diferencia entre {\em gleich} y {\em selbe}.
	\item James Mayer nos envió una ciénaga entera de  errores tipográficos 
	y de deletreo, incluyendo dos en la lista de colaboradores
	% james.mayer@acm.org
	\item Hayden McAfee descubrió una inconsistencia potencialmente confusa 
	entre dos ejemplos.
	\item Angel Arnal hace parte de un equipo internacional de traductores que
	trabajan en la versión española del texto. Él también ha encontrado varios 
	errores en la versión inglesa.
\end{itemize}

\clearemptydoublepage

\chapter{Traducción al español}

Al comienzo de junio de 2007 tomé la iniciativa de traducir el texto
``How to think like a Computer Scientist, with Python'' al español.
Rápidamente me dí cuenta de que ya había un trabajo inicial de traducción
empezado por:
\begin{itemize}
\item Angel Arnal 
\item I Juanes 
\item Litza Amurrio 
\item Efrain Andia
\end{itemize}
Ellos habían traducido los capítulos 1,2,10,11, y 12, así como el
prefacio, la introducción y la lista de colaboradores. Tomé su valioso
trabajo como punto de partida, adapté los capítulos, traduje las secciones
faltantes del libro y añadí un primer capítulo adicional sobre solución
de problemas.

Aunque el libro traduce la primera edición del original, todo se ha
corregido para que sea compatible con Python 2.7, por ejemplo se usan
booleanos en vez de enteros en los condicionales y ciclos.

Para realizar este trabajo ha sido invaluable la colaboración de familiares,
colegas, amigos y estudiantes que han señalado errores, expresiones
confusas y han aportado toda clase de sugerencias constructivas. Mi
agradecimiento va para los traductores antes mencionados y para los
estudiantes de Biología que tomaron el curso de Informática en la
Pontificia Universidad Javeriana (Cali-Colombia), durante el semestre
2014-1:
\begin{itemize}
\item Estefanía Lopez 
\item Gisela Chaves 
\item Marlyn Zuluaga 
\item Francisco Sanchez 
\item María del Mar Lopez 
\item Diana Ramirez 
\item Guillermo Perez 
\item María Alejandra Gutierrez 
\item Sara Rodriguez 
\item Claudia Escobar
\item Yisveire Fontalvo
\end{itemize}
\vspace{0.25in}
 

Para la segunda edición todo el código fuente se cambió para ejecutarse
con Python 3 y se añadieron 2 capítulos interludios y un posludio
como capítulo final.
\begin{flushleft}
Andrés Becerra Sandoval \\
 Universidad Santiago de Cali \\
 andres.becerra00@usc.edu.co \\
\par\end{flushleft}

\clearemptydoublepage

% The following lines add a little extra space to the column% in which the Section numbers appear in the table of contents\makeatletter
%\global\long\def\{}
 \makeatother \setcounter{tocdepth}{1}

\tableofcontents{}\clearemptydoublepage

\mainmatter


\chapter{Preludio: Solución de problemas}

\index{solución de problemas}

\section{Solución de acertijos}

\index{acertijo}

Todos nos hemos topado con acertijos como el siguiente. Disponga los
dígitos del 1 al 9 en el recuadro siguiente, de manera que la suma
de cada fila, cada columna y las dos diagonales dé el mismo resultado:

\begin{center}
\begin{tabular}{|c|c|c|}
\hline 
 &  & \tabularnewline
\hline 
 &  & \tabularnewline
\hline 
 &  & \tabularnewline
\hline 
\end{tabular}
\par\end{center}

Este acertijo se denomina construcción de un \textbf{cuadrado mágico.
}Un acertijo, normalmente, es un enigma o adivinanza que se propone
como pasatiempo. Otros ejemplos de acertijo son un crucigrama, una
sopa de letras y un sudoku.

Los acertijos pueden tener varias soluciones, por ejemplo, la siguiente
es una solución propuesta al acertijo anterior:

\begin{center}
\begin{tabular}{|c|c|c|}
\hline 
1  & 2  & 3\tabularnewline
\hline 
4  & 5  & 6\tabularnewline
\hline 
7  & 8  & 9 \tabularnewline
\hline 
\end{tabular}
\par\end{center}

Usted puede notar que esta solución candidata no es correcta. Si tomamos
la suma por filas, obtenemos valores distintos:
\begin{itemize}
\item En la fila 1: 1+2+3=6 
\item En la fila 2: 4+5+6=15 
\item En la fila 3: 7+8+9=24
\end{itemize}
Si tomamos las columnas, tampoco obtenemos el mismo resultado en cada
suma:
\begin{itemize}
\item En la columna 1: 1+4+7=12 
\item En la columna 2: 2+5+8=15 
\item En la columna 3: 3+6+9=18
\end{itemize}
A pesar de que las diagonales sí suman lo mismo:
\begin{itemize}
\item En la diagonal 1: 1+5+9=15 
\item En la diagonal 2: 7+5+3=15
\end{itemize}
Tómese un par de minutos para resolver este acertijo, es decir, para
construir un cuadrado mágico y regrese a la lectura cuando obtenga
la solución.

Ahora, responda para sí mismo las siguientes preguntas:
\begin{itemize}
\item ¿Cuál es la solución que encontró? 
\item ¿Es correcta su solución? 
\item ¿Cómo le demuestra a alguien que su solución es correcta? 
\item ¿Cuál fue el proceso de solución que llevo a cabo en su mente? 
\item ¿Cómo le explicaría a alguien el proceso de solución que llevó a cabo? 
\item ¿Puede poner por escrito el proceso de solución que llevó a cabo? 
\end{itemize}
El reflexionar seriamente sobre estas preguntas es muy importante,
tenga la seguridad de que esta actividad será muy importante para
continuar con la lectura.

Vamos a ir contestando las preguntas desde una solución particular,
y desde un proceso de solución particular, el de los autores. Su solución
y su proceso de solución son igualmente valiosos, el nuestro solo
es otra alternativa; es más, puede que hayamos descubierto la misma:

\begin{center}
\begin{tabular}{|c|c|c|}
\hline 
4  & 9  & 2\tabularnewline
\hline 
3  & 5  & 7\tabularnewline
\hline 
8  & 1  & 6 \tabularnewline
\hline 
\end{tabular}
\par\end{center}

Esta solución es correcta, porque la suma por filas, columnas y de
las dos diagonales da el mismo valor, 15. Ahora, para demostrarle
a alguien este hecho podemos revisar las sumas por filas, columnas
y diagonales detalladamente.
\begin{itemize}
\item En la fila 1: 4+9+2=15\% 
\item En la fila 2: 3+5+7=15
\item En la fila 3: 8+1+6=15
\item En la columna 1: 4+3+8=15
\item En la columna 2: 9+5+1=15
\item En la columna 3: 2+7+6=15
\item En la diagonal 1: 4+5+6=15
\item En la diagonal 2: 8+5+2=15
\end{itemize}
El proceso de solución que llevamos a cabo fue el siguiente:
\begin{itemize}
\item Sospechábamos que el 5 debía estar en la casilla central, ya que es
el número medio de los 9: 1 2 3 4 \textbf{5} 6 7 8 9.
\item Observamos un patrón interesante de la primera solución propuesta:
las diagonales sumaban igual, 15:

\begin{center}
\begin{tabular}{|c|c|c|}
\hline 
\textbf{1}  & 2  & \textbf{3}\tabularnewline
\hline 
4  & \textbf{5}  & 6\tabularnewline
\hline 
\textbf{7}  & 8  & \textbf{9} \tabularnewline
\hline 
\end{tabular}
\par\end{center}
\item La observación anterior, 1+5+9=7+5+3, también permite deducir otro
hecho interesante. Como el 5 está en las dos sumas, podemos deducir
que 1+9=7+3, y esto es 10.
\item Una posible estrategia consiste en colocar parejas de números que
sumen 10, dejando al 5 ``emparedado'', por ejemplo, poner la pareja
6,4:

\begin{center}
\begin{tabular}{|c|c|c|}
\hline 
 & \textbf{6}  & \tabularnewline
\hline 
 & \textbf{5}  & \tabularnewline
\hline 
 & \textbf{4}  & \tabularnewline
\hline 
\end{tabular}
\par\end{center}
\item Para agilizar ésta estrategia es conveniente enumerar todas las parejas
de números entre 1 y 9 que suman 10, excluyendo al 5: (1,9),(2,8),(3,7),(4,6)
\item Ahora, podemos probar colocando éstas parejas en filas, columnas y
diagonales.
\item Un primer ensayo:

\begin{center}
\begin{tabular}{|c|c|c|}
\hline 
7  & \textbf{6}  & 2\tabularnewline
\hline 
 & \textbf{5}  & \tabularnewline
\hline 
8  & \textbf{4}  & 3 \tabularnewline
\hline 
\end{tabular}
\par\end{center}

Aquí encontramos que es imposible armar el cuadrado, pues no hay como
situar el 9 ni el 1. Esto sugiere que la pareja (6,4) no va en la
columna central, debemos cambiarla.
\item Después de varios ensayos, moviendo la pareja (6,4), y reacomodando
los otros números logramos llegar a la solución correcta.
\end{itemize}

\section{El método de solución}

\label{sec:metodosolucion} \index{método de solución de problemas}

Mas allá de la construcción de cuadrados mágicos, lo que el ejemplo
anterior pretende ilustrar es la importancia que tiene el usar un
método ordenado de solución de problemas, acertijos en este caso.

Observe que la solución anterior fue conseguida a través de varios
pasos sencillos, aunque el encadenamiento de todos éstos hasta producir
un resultado correcto pueda parecer algo complejo.

Lea cualquier historia de Sherlock Holmes y obtendrá la misma impresión.
Cada caso, tomado por el famoso detective, presenta un enigma difícil,
que al final es resuelto por un encadenamiento de averiguaciones,
deducciones, conjeturas y pruebas sencillas. Aunque la solución completa
de cada caso demanda un proceso complejo de razonamiento, cada paso
intermedio es sencillo y está guiado por preguntas sencillas y puntuales.

Las aventuras de Sherlock Holmes quizás constituyen una de las mejores
referencias bibliográficas en el proceso de solución de problemas.
Son como los capítulos de C.S.I\footnote{\url{http://www.cbs.com/primetime/csi/}},
puestos por escrito. Para Holmes, y quizás para Grissom, existen varios
principios que se deben seguir:
\begin{itemize}
\item Obtener las soluciones y quedarse satisfecho con ellas. El peor inconveniente
para consolidar un método de solución de problemas poderoso consiste
en que cuando nos topemos con una solución completa de un solo golpe
nos quedemos sin reflexionar cómo llegamos a ella. Así, nunca aprenderemos
estrategias valiosas para resolver el siguiente problema. Holmes decía
que nunca adivinaba, primero recolectaba la mayor información posible
sobre los problemas que tenía a mano, de forma que estos datos le
``sugirieran'' alguna solución.
\item Todo empieza por la observación, un proceso minucioso de recolección
de datos sobre el problema a mano. Ningún dato puede ignorarse de
entrada, hasta que uno tenga una comprensión profunda sobre el problema.
\item Hay que prepararse adecuadamente en una variedad de dominios. Para
Holmes esto iba desde tiro con pistola, esgrima, boxeo, análisis de
huellas, entre otros. Para cada problema que intentemos resolver habrá
un dominio en el cual debemos aprender lo más que podamos, esto facilitará
enormemente el proceso de solución.
\item Prestar atención a los detalles ``inusuales''. Por ejemplo, en nuestro
cuadrado mágico fue algo inusual que las diagonales de la primera
alternativa de solución, la más ingenua posible, sumaran lo mismo.
\item Para deducir más hechos, a partir de los datos recolectados y los
detalles inusuales, hay que usar todas las armas del razonamiento:
el poder de la deducción (derivar nuevos hechos a partir de los datos),
la inducción (generalizar a partir de casos), la refutación (el proceso
de probar la falsedad de alguna conjetura), el pensamiento analógico
(encontrando relaciones,metáforas, analogías) y, por último, pero
no menos importante, el uso del sentido común.
\item Después del análisis de datos, uno siempre debe proponer una alternativa
de solución, así sea simple e ingenua, y proceder intentando probarla
y \textbf{refutarla} al mismo tiempo. Vale la pena recalcar esto:
no importa que tan ingenua, sencilla e incompleta es una alternativa
de solución, con tal de que nos permita seguir indagando. Esto es
como la primera frase que se le dice a una chica (o chico, dado el
caso) que uno quiere conocer; no importa qué frase sea, no importa
que tan trivial sea, con tal de que permita iniciar una conversación.
\item El proceso de búsqueda de soluciones es como una conversación que
se inicia, aunque en este caso el interlocutor no es una persona,
sino el problema que tenemos a mano. Con una primera alternativa de
solución—no importa lo sencilla e incompleta–- podemos formularnos
una pregunta interesante: ¿resuelve esta alternativa el problema?
\item Lo importante de responder la pregunta anterior no es la obtención
de un \textbf{No} como respuesta; pues esto es lo que sucede la mayoría
de las veces. Lo importante viene cuando nos formulamos esta segunda
pregunta ¿Con lo que sabemos del problema hasta ahora \textbf{por
qué mi alternativa no es capaz de resolverlo?}
\item La respuesta a la pregunta anterior puede ser: todavía no sé lo suficiente
sobre el problema para entender por qué mi alternativa de solución
no lo resuelve; esto es una señal de alerta para recolectar más datos
y estudiar mas el dominio del problema.
\item Una respuesta más constructiva a la pregunta anterior puede ser: mi
alternativa de solución no resuelve el problema porque no considera
algunos hechos importantes, y no considera algunas restricciones que
debe cumplir una solución. Un ejemplo de este tipo de respuesta lo
da nuestro ensayo de colocar la pareja (6,4) emparedando al 5 en el
problema del cuadrado mágico:

\begin{center}
\begin{tabular}{|c|c|c|}
\hline 
7  & \textbf{6}  & 2\tabularnewline
\hline 
 & \textbf{5}  & \tabularnewline
\hline 
8  & \textbf{4}  & 3 \tabularnewline
\hline 
\end{tabular}
\par\end{center}

Cuando notamos que la pareja (6,4) no puede colocarse en la columna
central del cuadrado, intentamos otra alternativa de solución, colocando
estos números en filas o en las esquinas. Lo importante de este tipo
de respuesta es que nos va a permitir avanzar a {\em otra} alternativa
de solución, casi siempre más compleja y más cercana a la solución.
\item No poner obstáculos a la creatividad. Es muy difícil lograr esto porque
la mente humana siempre busca límites para respetar; así, hay que
realizar un esfuerzo consciente para eliminar todo límite o restricción
que nuestra mente va creando. Una estrategia interesante es el uso
y fortalecimiento del pensamiento lateral.
\item Perseverar. El motivo más común de fracaso en la solución de acertijos
y problemas es el abandono. No existe una receta mágica para resolver
problemas, lo único que uno puede hacer es seguir un método y perseverar,
perseverar sin importar cuantas alternativas de solución incorrectas
se hayan generado. Esta es la clave para el éxito. Holmes decía que
eran muchísimos más los casos que no había podido resolver, quizás
Grissom reconocería lo mismo. Lo importante entonces es perseverar
ante cada nuevo caso, esta es la única actitud razonable para enfrentar
y resolver problemas.
\end{itemize}

\section{Reflexión sobre este método de solución}

El proceso de solución de problemas que hemos presentado es muy sencillo.
Los grandes profesionales que tienen mucho éxito en su campo (científicos,
humanistas, ingenieros, médicos, empresarios, etc.) tienen métodos
de solución de problemas mucho más avanzados que el que hemos presentado
aquí. Con esto pretendemos ilustrar un punto: lo importante no es
el método propuesto, porque muy probablemente no le servirá para resolver
todos los problemas o acertijos que enfrente; lo importante es:
\begin{itemize}
\item Contar con un método de solución de problemas. Si no hay método, podemos
solucionar problemas, claro está, pero cada vez que lo hagamos será
por golpes de suerte o inspiración –que no siempre nos acompaña, desafortunadamente.
\item Desarrollar, a medida que transcurra el tiempo y adquiera más conocimientos
sobre su profesión y la vida, un método propio de solución de problemas.
Este desarrollo personal puede tomar como base el método expuesto
aquí, o algún otro que encuentre en la literatura o a través de la
interacción con otras personas.
\end{itemize}

\section{Acertijos propuestos}

Para que empiece inmediatamente a construir su método personal de
solución de problemas, tome cada uno de los siguientes acertijos,
siga el proceso de solución recomendado y documéntelo a manera de
entrenamiento. Si usted descubre estrategias generales que no han
sido consideradas aquí, compártalas con sus compañeros, profesores
y, mejor aún, con los autores del libro.
\begin{enumerate}
\item Considere un tablero de ajedrez de 4x4 y 4 damas del mismo color.
Su misión es colocar las 4 damas en el tablero sin que éstas se ataquen
entre sí. Recuerde que una dama ataca a otra ficha si se encuentran
en la misma fila, columna o diagonal.
\item Partiendo de la igualdad $a=b$, encuentre cuál es el problema en
el siguiente razonamiento:

\begin{eqnarray*}
a & = & b\\
a^{2} & = & ba\\
a^{2}-b^{2} & = & ba-b^{2}\\
(a-b)(a+b) & = & b(a-b)\\
a+b & = & b\\
a & = & 2b\\
\frac{a}{b} & = & 2\\
1 & = & 2
\end{eqnarray*}

Tenga en cuenta que en el último paso se volvió a usar el hecho de
que $a=b$, en la forma $\frac{a}{b}=1$.
\item Encuentre el menor número entero positivo que pueda descomponerse
como la suma de los cubos de dos números enteros positivos de dos
maneras distintas. Esto es, encontrar el mínimo $A$ tal que $A=b^{3}+c^{3}$
y $A=d^{3}+e^{3}$, con $A,b,c,d$ y $e$ números positivos, mayores
a cero, y distintos entre si.
\end{enumerate}

\section{Mas allá de los acertijos: problemas computacionales}

\index{problema computacional}

Un problema computacional es parecido a un acertijo; se presenta una
situación problemática y uno debe diseñar alguna solución. En los
problemas computacionales la solución consiste en una \textbf{descripción
general de procesos}; esto es, un problema computacional tiene como
solución la descripción de un conjunto de pasos que se podrían llevar
a cabo de manera general para lograr un objetivo.

Un ejemplo que ilustra esto es la multiplicación. Todos sabemos multiplicar
números de dos cifras, por ejemplo:

%\newpage
\[
\begin{array}{cccc}
\  & \  & 3 & 4\\
\  & \times & 2 & 1\\
\hline \  & \  & 3 & 4\\
+ & 6 & 8 & \ \\
\hline \  & 7 & 1 & 4
\end{array}
\]

Pero el problema computacional asociado a la multiplicación de números
de dos cifras consiste en hallar la descripción general de todos los
procesos posibles de multiplicación de parejas de números de dos cifras.
Este problema ha sido resuelto, desde hace varios milenios, por diferentes
civilizaciones humanas, siguiendo métodos alternativos. Un método
de solución moderno podría describirse así:

Tome los dos números de dos cifras, P y Q. Suponga que las cifras
de P son $p_{1}$ y $p_{2}$, esto es, $P=p_{1}p_{2}$. Igualmente,
suponga que $Q=q_{1}q_{2}$. La descripción \textbf{general} de todas
las multiplicaciones de dos cifras puede hacerse así:

\[
\begin{array}{cccc}
\  & \  & p_{1} & p_{2}\\
\  & \times & q_{1} & q_{2}\\
\hline \  & \  & q_{2}p_{1} & q_{2}p_{2}\\
+ & q_{1}p_{1} & q_{1}p_{2} & \ \\
\hline \  & q_{1}p_{1} & q_{2}p_{1}+q_{1}p_{2} & q_{2}p_{2}
\end{array}
\]
\begin{itemize}
\item Tome la cifra $q_{2}$ y multiplíquela por las cifras de $P$ (con
ayuda de una tabla de multiplicación). Ubique los resultados debajo
de cada cifra de $P$ correspondiente.
\item Tome la cifra $q_{1}$ y multiplíquela por las cifras de $P$ (con
ayuda de una tabla de multiplicación). Ubique los resultados debajo
de las cifras que se generaron en el paso anterior, aunque desplazadas
una columna hacia la izquierda.
\item Si en alguno de los pasos anteriores el resultado llega a 10 o se
pasa de 10, ubique las unidades únicamente y lleve un acarreo, en
decenas o centenas) para la columna de la izquierda.
\item Sume los dos resultados parciales, obteniendo el resultado final.
\end{itemize}
Usted puede estar quejándose en este momento, ¿para qué hay que complicar
tanto nuestro viejo y conocido proceso de multiplicación? Bueno, hay
varias razones para esto:
\begin{itemize}
\item Una descripción impersonal como ésta puede ser leída y ejecutada por
cualquier persona —o computador, como veremos mas adelante—.
\item Sólo creando descripciones generales de procesos se pueden analizar
para demostrar que funcionan correctamente.
\item Queríamos sorprenderlo, tomando algo tan conocido como la suma y dándole
una presentación que, quizás, nunca había visto. Este cambio de perspectiva
es una invitación a que abra su mente a pensar en descripciones generales
de procesos. 
\end{itemize}
Precisando un poco, un problema computacional es la descripción general
de una situación en la que se presentan unos datos de entrada y una
salida deseada que se quiere calcular. Por ejemplo, en el problema
computacional de la multiplicación de números de dos cifras, los datos
de entrada son los números a multiplicar; la salida es el producto
de los dos números. Existen más problemas computacionales como el
de ordenar un conjunto de números y el problema de encontrar una palabra
en un párrafo de texto. Como ejercicio defina para estos problemas
cuales son los datos de entrada y la salida deseada.

La solución de un problema computacional es una descripción general
del conjunto de pasos que se deben llevar a cabo con las entradas
del problema para producir los datos de salida deseados. Solucionar
problemas computacionales no es muy diferente de solucionar acertijos,
las dos actividades producen la misma clase de retos intelectuales,
y el método de solución de la sección \ref{sec:metodosolucion} es
aplicable en los dos casos. Lo único que hay que tener en cuenta es
que la solución de un problema es una \textbf{descripción general
o programa, como veremos más adelante}, que se refiere a las entradas
y salidas de una manera más técnica de lo que estamos acostumbrados.
Un ejemplo de esto lo constituyen los nombres $p_{1},p_{2},q_{1}$
y $q_{2}$ que usamos en la descripción general de la multiplicación
de números de dos cifras.

Aunque la solución de problemas es una actividad compleja, es muy
interesante, estimulante e intelectualmente gratificante; incluso
cuando no llegamos a solucionar los problemas completamente. En el
libro vamos a enfocarnos en la solución de problemas computacionales
por medio de programas, y, aunque solo vamos a explorar este tipo
de problemas, usted verá que las estrategias de solución, los conceptos
que aprenderá y la actitud de científico de la computación que adquirirá
serán valiosas herramientas para resolver todo tipo de problemas de
la vida real.

\section{Glosario}
\begin{description}
\item [{Acertijo:}] enigma o adivinanza que se propone como pasatiempo.
\item [{Solución de problemas:}] el proceso de formular un problema,
hallar la solución y expresar la solución.
\item [{Método de solución de problemas:}] un conjunto de pasos, estrategias
y técnicas organizados que permiten solucionar problemas de una manera
ordenada.
\item [{Problema:}] una situación o circunstancia en la que se dificulta
lograr un fin.
\item [{Problema computacional:}] una situación general con una especificación
de los datos de entrada y los datos de salida deseados.
\item [{Solución a un problema:}] conjunto de pasos y estrategias que
permiten lograr un fin determinado en una situación problemática,
cumpliendo ciertas restricciones.
\item [{Solución a un problema computacional:}] descripción general de
los pasos que toman cualquier entrada en un problema computacional
y la transforman en una salida deseada.
\item [{Restricción:}] una condición que tiene que cumplirse en un problema
dado.

\index{acertijo} \index{solución de problemas} \index{método de solución de problemas}
\index{método!de solución de problemas} \index{problema} \index{solución a un problema}
\index{problema!solución} \index{restricción} 
\end{description}

\section{Ejercicios}

Intente resolver los siguientes problemas computacionales, proponiendo
soluciones \textbf{generales e impersonales}:
\begin{enumerate}
\item Describa cómo ordenar tres números a, b y c. 
\item Describa cómo encontrar el menor elemento en un conjunto de números. 
\item Describa cómo encontrar una palabra dentro de un texto más largo. 
\end{enumerate}


\clearemptydoublepage  % solucion de problemas

\chapter{El camino hacia el programa}

El objetivo de este libro es el de enseñar al estudiante a pensar
como lo hacen los científicos informáticos. Esta manera de pensar
combina las mejores características de la matemática, la ingeniería
y las ciencias naturales. Como los matemáticos, los científicos informáticos
usan lenguajes formales para diseñar ideas (específicamente, cómputos).
Como los ingenieros, ellos diseñan cosas, construyendo sistemas mediante
el ensamble de componentes y evaluando las ventajas y desventajas
de cada una de las alternativas de construcción. Como los científicos,
ellos observan el comportamiento de sistemas complejos, forman hipótesis,
y prueban sus predicciones.

La habilidad más importante del científico informático es \textbf{la
solución de problemas}. La solución de problemas incluye poder formular
problemas, pensar en soluciones de manera creativa y expresar soluciones
clara y precisamente. Como se verá, el proceso de aprender a programar
es la oportunidad perfecta para desarrollar la habilidad de resolver
problemas. Por esa razón este capítulo se llama ``El camino hacia
el programa''.

A cierto nivel, usted aprenderá a programar, lo cual es una habilidad
muy útil por sí misma. A otro nivel, usted utilizará la programación
para obtener algún resultado. Ese resultado se verá más claramente
durante el proceso.

\section{El lenguaje de programación Python}

\index{lenguaje de programación} \index{lenguaje!programación}

El lenguaje de programación que aprenderá es Python. Python es un
ejemplo de \textbf{lenguaje de alto nivel}; otros ejemplos de lenguajes
de alto nivel son C, C++, Perl y Java.

Como se puede deducir de la nomenclatura ``lenguaje de alto nivel,''
también existen \textbf{lenguajes de bajo nivel}, que también se denominan
lenguajes de máquina o lenguajes ensambladores. A propósito, las computadoras
sólo ejecutan programas escritos en lenguajes de bajo nivel. Los programas
de alto nivel tienen que ser traducidos antes de ser ejecutados. Esta
traducción lleva tiempo, lo cual es una pequeña desventaja de los
lenguajes de alto nivel.

\index{portátil} \index{lenguaje de alto nivel} \index{lenguaje de bajo nivel}
\index{lenguaje!alto nivel} \index{lenguaje!bajo nivel}

Aun así, las ventajas son enormes. En primer lugar, la programación
en lenguajes de alto nivel es mucho más fácil; escribir programas
en un lenguaje de alto nivel toma menos tiempo ya que los programas
son más cortos, más fáciles de leer, y es más probable que queden
correctos. En segundo lugar, los lenguajes de alto nivel son \textbf{portables},
lo que significa que los programas escritos con estos pueden ser ejecutados
en tipos diferentes de computadoras sin modificación alguna o con
pocas modificaciones. Programas escritos en lenguajes de bajo nivel
sólo pueden ser ejecutados en un tipo de computadora y deben ser reescritos
para ser ejecutados en otra.

Debido a estas ventajas, casi todo programa se escribe en un lenguaje
de alto nivel. Los lenguajes de bajo nivel son sólo usados para unas
pocas aplicaciones especiales.

\index{compilar} \index{interpretar}

Hay dos tipos de programas que traducen lenguajes de alto nivel a
lenguajes de bajo nivel: \textbf{intérpretes} y \textbf{compiladores}.
Un intérprete lee un programa de alto nivel y lo ejecuta, lo que significa
que lleva a cabo lo que indica el programa. Traduce el programa poco
a poco, leyendo y ejecutando cada comando.

\vspace{0.1in}
 \centerline{\includegraphics[scale=0.7]{illustrations/interpret}}
\vspace{0.1in}

Un compilador lee el programa y lo traduce todo al mismo tiempo, antes
de ejecutar alguno de los programas. A menudo se compila un programa
como un paso aparte, y luego se ejecuta el código compilado. En este
caso, al programa de alto nivel se lo llama el \textbf{código fuente},
y al programa traducido es llamado el \textbf{código de objeto} o
el \textbf{código ejecutable}.

\vspace{0.1in}
 \centerline{\includegraphics[scale=0.7]{illustrations/compile}}
\vspace{0.1in}

A Python se lo considera un lenguaje interpretado, porque sus programas
son ejecutados por un intérprete. Existen dos maneras de usar el intérprete:
modo de comando y modo de guión. En modo de comando se escriben sentencias
en el lenguaje Python y el intérprete muestra el resultado.
\begin{verbatim}
$python
Python 3.6.1 (default, Apr 29 2017, 15:32:44) [GCC 5.4.0] on linux 
Type "help", "copyright", "credits" or "license" for more information. 
>>>1+1
2
\end{verbatim}
La primera línea de este ejemplo es el comando que pone en marcha
al intérprete de Python. Las dos líneas siguientes son mensajes del
intérprete. La tercera línea comienza con {\verb+>>>+}, lo que
indica que el intérprete está listo para recibir comandos. Escribimos
\texttt{1+1} y el intérprete contestó \texttt{2}.

Alternativamente, se puede escribir el programa en un archivo y usar
el intérprete para ejecutar el contenido de dicho archivo. El archivo,
en este caso, se denomina un \textbf{guión (script)}; por ejemplo,
en un editor de texto se puede crear un archivo \texttt{unomasuno.py}
que contenga esta línea:

\begin{pythoncode}
print(1 + 1)
\end{pythoncode}

Por acuerdo unánime, los archivos que contienen programas de Python
tienen nombres que terminan con \texttt{.py}. Para ejecutar el programa,
se le tiene que indicar el nombre del guión al intérprete.

\begin{pythoncode}
$ python unomasuno.py
2
\end{pythoncode}
\begin{verbatim}

\end{verbatim}
En otros entornos de desarrollo, los detalles de la ejecución de programas
diferirán. Además, la mayoría de programas son más interesantes que
el anterior.

La mayoría de ejemplos en este libro son ejecutados en la línea de
comandos. La línea de comandos es más conveniente para el desarrollo
de programas y para pruebas rápidas, porque las instrucciones de Python
se pueden pasar a la máquina para ser ejecutadas inmediatamente. Una
vez que el programa está completo, se lo puede archivar en un guión
para ejecutarlo o modificarlo en el futuro.

\section{¿Qué es un programa?}

Un programa es una secuencia de instrucciones que especifican cómo
ejecutar un cómputo. El cómputo puede ser matemático, cómo solucionar
un sistema de ecuaciones o determinar las raíces de un polinomio,
pero también puede ser simbólico, cómo buscar y reemplazar el texto
de un documento o (aunque parezca raro) compilar un programa.

\index{instrucción}

Las instrucciones (comandos, órdenes) tienen una apariencia diferente
en lenguajes de programación diferentes, pero existen algunas funciones
básicas que se presentan en casi todo lenguaje:
\begin{description}
\item [{Entrada:}] recibir datos del teclado, o de un archivo o de otro
aparato.
\item [{Salida:}] mostrar datos en el monitor o enviar datos a un archivo
u otro aparato.
\item [{Matemáticas:}] ejecutar operaciones básicas, como la adición y
la multiplicación.
\item [{Operación condicional:}] probar la veracidad de alguna condición
y ejecutar una secuencia de instrucciones apropiada.
\item [{Repetición}] ejecutar alguna acción repetidas veces, usualmente
con alguna variación.
\end{description}
Aunque sea difícil de creer, todos los programas en existencia son
formulados exclusivamente con tales instrucciones. Así, una manera
de describir la programación es: el proceso de romper una tarea en
tareas cada vez más pequeñas hasta que éstas sean lo suficientemente
sencillas como para ser ejecutadas con una secuencia de estas instrucciones
básicas.

Quizás esta descripción es un poco ambigua. No se preocupe. Explicaremos
ésto con más detalle en el tema de \textbf{algoritmos}.

\section{¿Qué es la depuración (debugging)?}

\index{depuración (debugging)} \index{error (bug)}

La programación es un proceso complejo, y a veces este proceso lleva
a \textbf{errores indefinidos}, también llamados \textbf{defectos}
o \textbf{errores de programación} (en inglés `bugs'), y el proceso
de buscarlos y corregirlos es llamado \textbf{depuración} (en inglés
`debugging').

Hay tres tipos de errores que pueden ocurrir en un programa. Es muy
útil distinguirlos para encontrarlos más rápido.

\subsection{Errores sintácticos}

\index{error sintáctico} \index{error!sintaxis}

Python sólo puede ejecutar un programa si está sintácticamente bien
escrito. Al contrario, es decir, si el programa tiene algún error
de sintaxis, el proceso falla y devuelve un mensaje de error. La palabra
\textbf{sintáctica} se refiere a la estructura de cualquier programa
y a las reglas de esa estructura. \index{sintáctica} Por ejemplo,
en español, la primera letra de toda oración debe ser mayúscula y
el fin de toda oración debe llevar un punto. esta oración tiene un
error sintáctico. Esta oración también

Para la mayoría de lectores, unos pocos errores no impiden la comprensión
de los grafitis en la calle que, a menudo, rompen muchas reglas de
sintaxis. Sin embargo Python no es así. Si hay aunque sea un error
sintáctico en el programa, Python mostrará un mensaje de error y abortará
su ejecución. Al principio usted pasará mucho tiempo buscando errores
sintácticos, pero con el tiempo no cometerá tantos errores y los encontrará
rápidamente.

\subsection{Errores en tiempo de ejecución}

\label{runtime} \index{error en tiempo de ejecución} \index{error!en tiempo de ejecución}
\index{excepción} \index{lenguaje seguro} \index{seguro!lenguaje}

El segundo tipo de error es el de tiempo de ejecución. Este error
aparece sólo cuando se ejecuta un programa. Estos errores también
se llaman \textbf{excepciones}, porque indican que algo excepcional
ha ocurrido.

Con los programas que vamos a escribir al principio, los errores de
tiempo de ejecución ocurrirán con poca frecuencia.

\subsection{Errores semánticos}

\index{semántica} \index{error semántico} \index{semántico!error}

El tercer tipo de error es el \textbf{semántico}. Si hay un error
de lógica en su programa, éste será ejecutado sin ningún mensaje de
error, pero el resultado no será el deseado. El programa ejecutará
la lógica que usted le dijo que ejecutara.

A veces ocurre que el programa escrito no es el que se tenía en mente.
El sentido o significado del programa no es correcto. Es difícil hallar
errores de lógica. Eso requiere trabajar al revés, comenzando a analizar
la salida para encontrar al problema.

\subsection{Depuración experimental}

Una de las técnicas más importantes que usted aprenderá es la depuración.
Aunque a veces es frustrante, la depuración es una de las partes de
la programación más estimulantes, interesantes e intelectualmente
exigentes.

La depuración es una actividad parecida a la tarea de un investigador:
se tienen que estudiar las pistas para inferir los procesos y eventos
que han generado los resultados encontrados.

La depuración también es una ciencia experimental. Una vez que se
tiene conciencia de un error, se modifica el programa y se intenta
nuevamente. Si la hipótesis fue la correcta se pueden predecir los
resultados de la modificación y estaremos más cerca a un programa
correcto. Si la hipótesis fue la errónea tendrá que idearse otra hipótesis.
Como dijo Sherlock Holmes: ``Cuando se ha descartado lo imposible,
lo que queda, no importa cuan inverosímil, debe ser la verdad'' (A.
Conan Doyle, {\em The Sign of Four})

\index{Holmes, Sherlock} \index{Doyle, Arthur Conan}

Para algunas personas, la programación y la depuración son lo mismo:
la programación es el proceso de depurar un programa gradualmente,
hasta que el programa tenga el resultado deseado. Esto quiere decir
que el programa debe ser, desde un principio, un programa que funcione,
aunque su función sea solo mínima. El programa es depurado mientras
crece y se desarrolla.

Por ejemplo, aunque el sistema operativo Linux contenga miles de líneas
de instrucciones, Linus Torvalds lo comenzó como un programa para
explorar el microprocesador Intel 80836. Según Larry Greenfield: ``Uno
de los proyectos tempranos de Linus fue un programa que intercambiaría
la impresión de AAAA con BBBB. Este programa se convirtió en Linux''
(de {\em The Linux Users' Guide} Versión Beta 1).

\index{Linux}

Otros capítulos tratarán más el tema de la depuración y otras técnicas
de programación.

\section{Lenguajes formales y lenguajes naturales}

\index{lenguaje formal} \index{lenguaje natural} \index{formal!lenguaje}
\index{natural!lenguaje}

Los \textbf{lenguajes naturales} son los hablados por seres humanos,
como el español, el inglés y el francés. Éstos no han sido diseñados
artificialmente (aunque se trate de imponer cierto orden en ellos),
pues se han desarrollado naturalmente.

Los \textbf{Lenguajes formales} son diseñados por humanos y tienen
aplicaciones específicas. La notación matemática, por ejemplo, es
un lenguaje formal, ya que se presta a la representación de las relaciones
entre números y símbolos. Los químicos utilizan un lenguaje formal
para representar la estructura química de las moléculas. Es necesario
notar que:
\begin{quote}
\textbf{Los lenguajes de programación son formales y han sido desarrollados
para expresar cómputos.} 
\end{quote}
Los lenguajes formales casi siempre tienen reglas sintácticas estrictas.
Por ejemplo, $3+3=6$ es una expresión matemática correcta, pero $3=+6\$$
no lo es. De la misma manera, $H_{2}O$ es una nomenclatura química
correcta, pero $_{2}Zz$ no lo es.

Existen dos clases de reglas sintácticas, en cuanto a unidades y estructura.
Las unidades son los elementos básicos de un lenguaje, como lo son
las palabras, los números y los elementos químicos. Por ejemplo, en
\texttt{3=+6\$}, \texttt{\$} no es una unidad matemática aceptada.
Similarmente, $_{2}Zz$ no es formal porque no hay ningún elemento
químico con la abreviación $Zz$.

La segunda clase de error sintáctico está relacionado con la estructura
de un elemento; mejor dicho, el orden de las unidades. La estructura
de la sentencia \texttt{3=+6\$} no es aceptada porque no se puede
escribir el símbolo de igualdad seguido de un símbolo más. Similarmente,
las fórmulas moleculares tienen que mostrar el número de subíndice
después del elemento, no antes.

Al leer una oración, sea en un lenguaje natural o una sentencia en
un lenguaje técnico, se debe discernir la estructura de la oración.
En un lenguaje natural este proceso, llamado \textbf{análisis sintáctico},
ocurre subconscientemente.

\index{analizar sintácticamente}

Por ejemplo cuando se escucha una oración simple como ``el otro zapato
se cayó'', se puede distinguir el sustantivo ``el otro zapato''
y el predicado ``se cayó''. Cuando se ha analizado la oración sintácticamente,
se puede deducir el significado, o la semántica, de la oración. Si
usted sabe lo que es un zapato y el significado de caer, comprenderá
el significado de la oración.

Aunque existen muchas cosas en común entre los lenguajes naturales
y los formales—por ejemplo las unidades, la estructura, la sintáctica
y la semántica— también existen muchas diferencias.

\index{Ambigüedad} \index{Redundancia} \index{Literalidad}
\begin{description}
\item [{Ambigüedad:}] los lenguajes naturales tienen muchísimas ambigüedades
que se superan usando claves contextuales e información adicional.
Los lenguajes formales son diseñados para estar completamente libres
de ambigüedades o, tanto como sea posible, lo que quiere decir que
cualquier sentencia tiene sólo un significado sin importar el contexto
en el que se encuentra.
\item [{Redundancia:}] para reducir la ambigüedad y los malentendidos,
los lenguajes naturales utilizan bastante redundancia. Como resultado
tienen una abundancia de posibilidades para expresarse. Los lenguajes
formales son menos redundantes y mas concisos.
\item [{Literalidad:}] los lenguajes naturales tienen muchas metáforas
y frases comunes. El significado de un dicho, por ejemplo: ``Estirar
la pata'', es diferente al significado de sus sustantivos y verbos.
En este ejemplo, la oración no tiene nada que ver con una pata y significa
'morirse'. En los lenguajes formales solo existe el significado literal.
\end{description}
Los que aprenden a hablar un lenguaje natural—es decir todo el mundo—muchas
veces tienen dificultad en adaptarse a los lenguajes formales. A veces
la diferencia entre los lenguajes formales y los naturales es comparable
a la diferencia entre la prosa y la poesía:

\index{Poesía} \index{Prosa}
\begin{description}
\item [{Poesía:}] se utiliza una palabra por su cualidad auditiva tanto
como por su significado. El poema, en su totalidad, produce un efecto
o reacción emocional. La ambigüedad no es sólo común, sino utilizada
a propósito.
\item [{Prosa:}] el significado literal de la palabra es más importante
y la estructura contribuye más al significado. La prosa se presta
más al análisis que la poesía, pero todavía contiene ambigüedad.
\item [{Programa:}] el significado de un programa es inequívoco y literal,
y es entendido en su totalidad analizando las unidades y la estructura.
\end{description}
He aquí unas sugerencias para la lectura de un programa (y de otros
lenguajes formales). Primero, recuerde que los lenguajes formales
son mucho más densos que los lenguajes naturales y, por consecuencia,
toma mas tiempo dominarlos. Además, la estructura es muy importante,
entonces no es una buena idea leerlo de pies a cabeza, de izquierda
a derecha. En lugar de ésto, aprenda a separar las diferentes partes
en su mente, identificar las unidades e interpretar la estructura.
Finalmente, ponga atención a los detalles. La fallas de puntuación
y la ortografía afectarán negativamente la ejecución de sus programas.

\section{El primer programa}

\label{hello} \label{hello world}

Tradicionalmente el primer programa en un lenguaje nuevo se llama
``Hola todo el mundo!'' (en inglés, Hello world!) porque sólo muestra
las palabras ``Hola todo el mundo'' . En el lenguaje Python es así:

\begin{pythoncode}
print("Hola todo el mundo!")
\end{pythoncode}

Este es un ejemplo de llamado a la función {\em print}, la cual
no imprime nada en papel, más bien muestra un valor. En este caso,
el resultado es mostrar en pantalla las palabras:

\begin{pythoncode}
Hola todo el mundo!
\end{pythoncode}

Las comillas señalan el comienzo y el final del valor; no aparecen
en el resultado.

\index{función print} \index{print!función}

Hay gente que evalúa la calidad de un lenguaje de programación por
la simplicidad del programa ``Hola todo el mundo!''. Si seguimos
ese criterio, Python cumple con esta meta.

\section{Glosario}
\begin{description}
\item [{Solución de problemas:}] el proceso de formular un problema,
hallar la solución y expresarla.
\item [{Lenguaje de alto nivel:}] un lenguaje como Python que es diseñado
para ser fácil de leer y escribir por la gente.
\item [{Lenguaje de bajo nivel:}] un lenguaje de programación que es diseñado
para ser fácil de ejecutar para una computadora; también se lo llama
``lenguaje de máquina'' o ``lenguaje ensamblador''.
\item [{Portabilidad:}] la cualidad de un programa que puede ser ejecutado
en más de un tipo de computadora.
\item [{Interpretar:}] ejecutar un programa escrito en un lenguaje de alto
nivel traduciéndolo línea por línea.
\item [{Compilar:}] traducir un programa escrito en un lenguaje de alto
nivel a un lenguaje de bajo nivel de una vez, en preparación para
la ejecución posterior.
\item [{Código fuente:}] un programa escrito en un lenguaje de alto nivel
antes de ser compilado.
\item [{Código objeto:}] la salida del compilador una vez que el programa
ha sido traducido.
\item [{Programa ejecutable:}] otro nombre para el código de objeto que
está listo para ser ejecutado.
\item [{Guión (script):}] un programa archivado (que va a ser interpretado).
\item [{Programa:}] un grupo de instrucciones que especifica un cómputo.
\item [{Algoritmo:}] un proceso general para resolver una clase completa
de problemas.
\item [{Error (bug):}] un error en un programa.
\item [{Depuración:}] el proceso de hallazgo y eliminación de los tres
tipos de errores de programación.
\item [{Sintaxis:}] la estructura de un programa.
\item [{Error sintáctico:}] un error estructural que hace que un programa
sea imposible de analizar sintácticamente (e imposible de interpretar).
\item [{Error en tiempo de ejecución:}] un error que no ocurre hasta
que el programa ha comenzado a ejecutar e impide que el programa continúe.
\item [{Excepción:}] otro nombre para un error en tiempo de ejecución.
\item [{Error semántico:}] un error en un programa que hace que ejecute
algo que no era lo deseado.
\item [{Semántica:}] el significado de un programa.
\item [{Lenguaje natural:}] cualquier lenguaje hablado que evolucionó
de forma natural.
\item [{Lenguaje formal:}] cualquier lenguaje diseñado que tiene un propósito
específico, como la representación de ideas matemáticas o programas
de computadoras; todos los lenguajes de programación son lenguajes
formales.
\item [{Unidad:}] uno de los elementos básicos de la estructura sintáctica
de un programa, análogo a una palabra en un lenguaje natural.
\item [{Análisis sintáctico:}] la revisión de un programa y el análisis
de su estructura sintáctica.
\item [{Función print:}] una función que causa que el intérprete de Python
muestre un valor en la pantalla.

\index{programa} \index{solución de problemas} \index{lenguaje de alto nivel}
\index{lenguaje de bajo nivel} \index{portabilidad} \index{interpretar}
\index{compilar} \index{código de fuente} \index{código de objeto}
\index{código ejecutable} \index{algoritmo} \index{error(bug)}
\index{depuración} \index{sintaxis} \index{semántica} \index{error sintáctico}
\index{error en tiempo de ejecución} \index{excepción} \index{error semántico}
\index{lenguaje formal} \index{lenguaje natural} \index{análisis sintáctico}
\index{unidad} \index{guión} \index{función print} \index{print!función}
\end{description}

\section{Ejercicios}

En los ejercicios 1,2,3 y 4 escriba una oración en español:\\
 \\
\begin{enumerate}
\item Con estructura válida pero compuesta de unidades irreconocibles.
\item Con unidades aceptables pero con estructura no válida.
\item Semánticamente comprensible pero sintácticamente incorrecta.
\item Sintácticamente correcta pero que contenga errores semánticos.
\item Inicie la terminal de Python. Escriba 1 + 2 y luego presione la tecla
Entrar. Python evalúa esta expresión, presenta el resultado, y enseguida
muestra otro intérprete. Considerando que el símbolo {*} es el signo
de multiplicación y el doble símbolo {*}{*} es el signo de potenciación,
realice dos ejercicios adicionales escribiendo diferentes expresiones
y reportando lo mostrado por el intérprete de Python.
\item ¿Qué sucede si utiliza el signo de división (/)? ¿Son los resultados
obtenidos los esperados? Explique.
\item Escriba 1 2 y presione la tecla Entrar. Python trata de evaluar esta
expresión, pero no puede, porque la expresión es sintácticamente incorrecta.
Así, Python responde con el siguiente mensaje de error:

\begin{verbatim}
 File "<stdin>", line 1

    1 2

      ^

SyntaxError: invalid syntax
\end{verbatim}
Muchas veces Python indica la ubicación del error de sintaxis, sin
embargo, no siempre es precisa, por lo que no proporciona suficiente
información sobre cuál es el problema. De esta manera, el mejor antídoto
es que usted aprenda la sintaxis de Python. En este caso, Python protesta
porque no encuentra signo de operación alguno entre los números.

Escriba una entrada que produzca un mensaje de error cuando se introduzca
en el intérprete de Python. Explique por qué no tiene una sintaxis
válida.
\item Escriba \verb+print('hola')+. Python ejecuta esta sentencia que muestra
las letras h-o-l-a. Nótese que las comillas simples en los extremos
de la cadena no son parte de la salida mostrada. Ahora escriba \verb+print('"hola"')+
y describa y explique el resultado.
\item Escriba \verb+print(queso)+ sin comillas. ¿Que sucede?
\item Escriba \verb+'Esta es una prueba...'+ en el intérprete de Python
y presione la tecla Entrar. Observe lo que pasa.
\item Ahora cree un guión de Python con el nombre prueba1.py que contenga
lo siguiente (asegúrese de guardar el archivo antes de intentar ejecutarlo):
'Esta es una prueba...'

¿Qué pasa cuando ejecuta este guión?
\item Ahora cambie el contenido del guión a: \verb+print('Esta es una prueba...')+
y ejecutelo de nuevo.

¿Qué pasó esta vez?

Cuando se escribe una expresión en el intérprete de Python, ésta es
evaluada y el resultado es mostrado en la línea siguiente. 'Esta es
una prueba...' es una expresión, que se evalúa a 'Esta es una prueba...'
(de la misma manera que 42 se evalúa a 42). Sin embargo, la evaluación
de expresiones en un guión no se envía a la salida del programa, por
lo que es necesario mostrarla explícitamente. 
\end{enumerate}


\clearemptydoublepage % el camino del programa

\chapter{Variables, expresiones y sentencias}

\section{Valores y tipos}

\index{valor} \index{tipo} \index{cadena}

Un \textbf{valor} es una de las cosas fundamentales—como una letra
o un número—que un programa manipula. Los valores que hemos visto
hasta ahorra son \texttt{2} (el resultado cuando añadimos \texttt{1
+ 1}, y {\verb+"Hola todo el Mundo!"+}.

Los valores pertenecen a diferentes \textbf{tipos}: \texttt{2} es
un entero, y {\verb+"Hola, Mundo!"+} es una \textbf{cadena}, llamada
así porque contiene una ``cadena'' de letras. Usted (y el intérprete)
pueden identificar cadenas porque están encerradas entre comillas.

La función de impresión también trabaja con enteros.

\begin{lstlisting}
>>> print(4)
4
\end{lstlisting}
 

Si no está seguro del tipo que un valor tiene, el intérprete le puede
decir.

\begin{lstlisting}
>>> type("Hola, Mundo!")
str
>>> type(17)
int
\end{lstlisting}
 

Sin despertar ninguna sorpresa, las cadenas pertenecen al tipo \texttt{\textbf{str}}\texttt{ing
(cadena)} y los enteros pertenecen al tipo \texttt{\textbf{int}}\texttt{eger}.
Menos obvio, los números con cifras decimales pertenecen a un tipo
llamado \texttt{float}, porque éstos se representan en un formato
denominado \textbf{punto flotante}.

\index{tipo} \index{cadena} \index{tipo!cadena} \index{int} \index{tipo!int}
\index{float} \index{tipo!float}

\begin{lstlisting}
>>> type(3.2)
float
\end{lstlisting}

¿Qué ocurre con valores como {\verb+"17"+} y {\verb+"3.2"+}?
Parecen números, pero están encerrados entre comillas como las cadenas.

\begin{lstlisting}
>>> type("17")
str
>>> type("3.2")
str
\end{lstlisting}

Ellos son cadenas.

Cuando usted digita un número grande, podría estar tentado a usar
comas para separar grupos de tres dígitos, como en \texttt{1,000,000}.
Esto no es un número entero legal en Python, pero esto si es legal:

\begin{lstlisting}
>>> print(1,000,000)
1 0 0
\end{lstlisting}

¡Bueno, eso no es lo que esperábamos!. Resulta que \texttt{1,000,000}
es una tupla, algo que encontraremos en el Capítulo \ref{tuplechap}.
De momento, recuerde no poner comas en sus números enteros.

\section{Variables}

\index{variable} \index{asignación} \index{sentencia!asignación}

Una de las características más poderosas en un lenguaje de programación
es la capacidad de manipular \textbf{variables}. Una variable es un
nombre que se refiere a un valor.

La \textbf{sentencia de asignación} crea nuevas variables y les da
valores:

\begin{lstlisting}
>>> mensaje = "¿Qué onda?"
>>> n = 17
>>> pi = 3.14159
\end{lstlisting}

Este ejemplo hace tres asignaciones: la primera asigna la cadena {\verb+"¿Qué Onda?"+}
a una nueva variable denominada \texttt{mensaje}, la segunda le asigna
el entero \texttt{17} a \texttt{n} y la tercera le asigna el número
de punto flotante \texttt{3.14159} a \texttt{pi}.

\index{diagrama de estados}

Una manera común de representar variables en el papel es escribir
el nombre de la variable con una flecha apuntando a su valor. Esta
clase de dibujo se denomina \textbf{diagrama de estados} porque muestra
el estado de cada una de las variables (piense en los valores como
el estado mental de las variables). Este diagrama muestra el resultado
de las sentencias de asignación anteriores:

\beforefig \centerline{\includegraphics{illustrations/state2}}
\afterfig

La función \texttt{print} también funciona con variables.

\begin{lstlisting}
>>> print(mensaje)
Que Onda?
>>> print(n)
17
>>> print(pi)
3.14159
\end{lstlisting}
 

En cada caso el resultado es el valor de la variable. Las variables
también tienen tipos; nuevamente, le podemos preguntar al intérprete
cuales son.

\begin{lstlisting}
>>> type(mensaje)
str
>>> type(n)
int
>>> type(pi)
float
\end{lstlisting}
 

El tipo de una variable es el mismo del valor al que se refiere.

\section{Nombres de variables y palabras reservadas}

\index{palabra reservada} \index{palabra!reservada}

Los programadores, generalmente, escogen nombres significativos para
sus variables —que especifiquen para qué se usa la variable.

Estos nombres pueden ser arbitrariamente largos. Pueden contener letras
y números, pero tienen que empezar con una letra. Aunque es legal
usar letras mayúsculas, por convención no lo hacemos. Si usted lo
hace, recuerde que la capitalización importa, \texttt{Pedro} y \texttt{pedro}
son variables diferentes.

El carácter subrayado (\texttt{\_}) puede aparecer en un nombre. A
menudo se usa en nombres con múltiples palabras, tales como \texttt{mi\_nombre}
ó \texttt{precio\_del\_café\_en\_china}.

\index{carácter subrayado}

Si usted le da un nombre ilegal a una variable obtendrá un error sintáctico:

\begin{lstlisting}
>>> 76trombones = "gran desfile"
SyntaxError: invalid syntax
>>> mas$ = 1000000
SyntaxError: invalid syntax
>>> class = "introducción a la programación"
SyntaxError: invalid syntax
\end{lstlisting}

\texttt{76trombones} es ilegal porque no empieza con una letra.\\
\texttt{mas\$} es ilegal porque contiene un carácter ilegal, el símbolo
\$.\\
¿Qué sucede con \texttt{class}?

Resulta que \texttt{class} es una de las \textbf{palabras reservadas
(keywords)} de Python. Las palabras reservadas definen las reglas
del lenguaje y su estructura, y no pueden ser usadas como nombres
de variables.

\index{palabra reservada}

Python tiene veintiocho palabras reservadas:

\begin{lstlisting}
and       continue  else      for       import    not       
assert    def       except    from      in        or        
break     del       exec      global    is        pass      
class     elif      finally   if        lambda    print     
raise     return    try       while
\end{lstlisting}
 

Usted puede mantener esta lista a mano. Si el intérprete se queja
por alguno de sus nombres de variables, y usted no sabe por qué, búsquelo
en esta lista.

\section{Sentencias}

Una sentencia es una instrucción que el intérprete de Python puede
ejecutar. Hemos visto una clase de sentencias: la asignación. Invocar
una función predefinida, como \texttt{print}, también es una sentencia.

Cuando usted digita una sentencia en la línea de comandos, Python
la ejecuta y despliega el resultado, si hay alguno. Las asignaciones
no producen un resultado.

Un guión usualmente contiene una secuencia de sentencias. Si hay más
de una, los resultados aparecen uno a uno a medida que las sentencias
se ejecutan.

Por ejemplo, el guión

\begin{lstlisting}
print(1)
x = 2
print(x)
\end{lstlisting}
 produce la salida:
\begin{verbatim}
1
2
\end{verbatim}
Observe nuevamente que la sentencia de asignación no produce salida.

\section{Evaluando expresiones}

Una expresión es una combinación de valores, variables y operadores.
Si usted digita una expresión en la línea de comandos, el intérprete
la \textbf{evalúa} y despliega su resultado:

\begin{lstlisting}
>>> 1 + 1
2
\end{lstlisting}
 

Un valor, por si mismo, se considera como una expresión, lo mismo
ocurre para las variables.

\begin{lstlisting}
>>> 17
17
>>> x
2
\end{lstlisting}
 

Aunque es un poco confuso, evaluar una expresión no es lo mismo que
imprimir o desplegar un valor.

\begin{lstlisting}
>>> mensaje = "Como le va, Doc?"
>>> mensaje
"Como le va, Doc?"
>>> print(mensaje)
Como le va, Doc?
\end{lstlisting}
 

Cuando Python muestra el valor de una expresión que ha evaluado, utiliza
el mismo formato que se usaría para entrar un valor. En el caso de
las cadenas, esto implica que se incluyen las comillas. Cuando se
usa la función print, el efecto es distinto como usted ya lo ha evidenciado.

En un guión, una expresión, por sí misma, es una sentencia legal,
pero no realiza nada. El guión:

\begin{lstlisting}
17
3.2
"Hola, Mundo!"
1 + 1
\end{lstlisting}
 

no produce ninguna salida. ¿Cómo cambiaría el guión de manera que
despliegue los valores de las cuatro expresiones?

\section{Operadores y operandos}

\index{operador} \index{operando} \index{expresión}

Los \textbf{operadores} son símbolos especiales que representan cómputos,
como la suma y la multiplicación. Los valores que el operador usa
se denominan \textbf{operandos}.

Los siguientes son expresiones válidas en Python, cuyo significado
es más o menos claro: 

\begin{lstlisting}
20+32       hora-1   hora*60+minuto   
minuto/60   5**2     (5+9)*(15-7)
\end{lstlisting}
\begin{verbatim}

\end{verbatim}
Los símbolos \texttt{+}, \texttt{-}, y \texttt{/}, y los paréntesis
para agrupar, significan en Python lo mismo que en la matemática.
El asterisco (\texttt{{*}}) es el símbolo para la multiplicación,
y \texttt{{*}{*}} es el símbolo para la exponenciación. Cuando el
nombre de una variable aparece en lugar de un operando, se reemplaza
por su valor antes de calcular la operación.

La suma, resta, multiplicación y exponenciación realizan lo que usted
esperaría, pero la división usando el operador // podría sorprenderlo.
La siguiente operación tiene un resultado inesperado:

\begin{lstlisting}
>>> minuto = 59
>>> minuto//60
0
\end{lstlisting}

El valor de \texttt{minuto} es 59, y 59 dividido por 60 es 0.98333,
no 0. La razón para esta discrepancia radica en que Python está realizando
\textbf{división entera}.

\index{división entera}

Cuando los dos operandos son enteros el resultado también debe ser
un entero; y, por convención, la división entera siempre redondea
{\em hacia abajo}, incluso en casos donde el siguiente entero está
muy cerca.

Recuerde que si divide con / obtendrá un numero flotante (los cuales
veremos con mayor detalle en el capítulo \ref{floatchap}) y si lo
hace con // obtendrá un número entero. 

\section{Orden de las operaciones}

\index{orden de las operaciones} \index{reglas de precedencia}

Cuando hay más de un operador en una expresión, el orden de evaluación
depende de las \textbf{reglas de precedencia}. Python sigue las mismas
reglas de precedencia a las que estamos acostumbrados para sus operadores
matemáticos. El acrónimo \textbf{PEMDAS} es útil para recordar el
orden de las operaciones:
\begin{itemize}
\item Los \textbf{P}aréntesis tienen la precedencia más alta y pueden usarse
para forzar la evaluación de una expresión de la manera que usted
desee. Ya que las expresiones en paréntesis se evalúan primero, \texttt{2
{*} (3-1)} es 4, y \texttt{(1+1){*}{*}(5-2)} es 8. Usted también puede
usar paréntesis para que una expresión quede más legible, como en
\texttt{(minuto {*} 100) / 60}, aunque esto no cambie el resultado.
\item La \textbf{E}xponenciación tiene la siguiente precedencia más alta,
así que \texttt{2{*}{*}1+1} es 3 y no 4, y \texttt{3{*}1{*}{*}3} es
3 y no 27.
\item La \textbf{M}ultiplicación y la \textbf{D}ivisión tienen la misma
precedencia, aunque es más alta que la de la \textbf{A}dición y la
\textbf{S}ubtracción, que también tienen la misma precedencia. Así
que \texttt{2{*}3-1} da 5 en lugar de 4, y \texttt{2//3-1} es \texttt{-1},
no \texttt{1} (recuerde que en división entera, \texttt{2//3=0}).
\item Los operadores con la misma precedencia se evalúan de izquierda a
derecha. Recordando que \texttt{minuto=59}, en la expresión \texttt{minuto{*}100/60};
la multiplicación se hace primero, resultando \texttt{5900/60}, lo
que a su vez da \texttt{98.33333333333333}.
\end{itemize}

\section{Operaciones sobre cadenas}

\index{operación sobre cadenas}

En general, usted no puede calcular operaciones matemáticas sobre
cadenas, incluso si las cadenas lucen como números. Las siguientes
operaciones son ilegales (asumiendo que \texttt{mensaje} tiene el
tipo \texttt{cadena}):

\begin{lstlisting}
 mensaje-1   "Hola"/123   mensaje*"Hola"   "15"+2
\end{lstlisting}

Sin embargo, el operador \texttt{+} funciona con cadenas, aunque no
calcula lo que usted esperaría. Para las cadenas, el operador \texttt{+}
representa la \textbf{concatenación}, que significa unir los dos operandos
enlazándolos en el orden en que aparecen. Por ejemplo:

\index{concatenación}

\begin{lstlisting}
fruta = "banano"
bienCocinada = " pan con nueces"
print(fruta + bienCocinada)
\end{lstlisting}

La salida de este programa es \texttt{banano pan con nueces}. El espacio
antes de la palabra \texttt{pan} es parte de la cadena y sirve para
producir el espacio entre las cadenas concatenadas.

El operador \texttt{{*}} también funciona con las cadenas; hace una
repetición. Por ejemplo, \texttt{'Fun'{*}3} es \texttt{'FunFunFun'.}
Uno de los operandos tiene que ser una cadena, el otro tiene que ser
un entero.

Estas interpretaciones de \texttt{+} y \texttt{{*}} tienen sentido
por la analogía con la suma y la multiplicación. Así como \texttt{4{*}3}
es equivalente a \texttt{4+4+4}, esperamos que \verb+"Fun"*3+ sea
lo mismo que {\verb/"Fun"+"Fun"+"Fun"/}, y lo es. Sin embargo,
las operaciones de concatenación y repetición sobre cadenas tienen
una diferencia significativa con las operaciones de suma y multiplicación.
¿Puede usted pensar en una propiedad que la suma y la multiplicación
tengan y que la concatenación y repetición no?

\section{Composición}

\index{composición}

Hasta aquí hemos considerado a los elementos de un programa—variables,
expresiones y sentencias—aisladamente, sin especificar cómo combinarlos.

Una de las características mas útiles de los lenguajes de programación
es su capacidad de tomar pequeños bloques para \textbf{componer} con
ellos. Por ejemplo, ya que sabemos cómo sumar números y cómo imprimirlos;
podemos hacer las dos cosas al mismo tiempo:

\begin{lstlisting}
>>> print(17 + 3)
20
\end{lstlisting}
 De hecho, la suma tiene que calcularse antes que la impresión, así
que las acciones no están ocurriendo realmente al mismo tiempo. El
punto es que cualquier expresión que tenga números, cadenas y variables
puede ser usada en una sentencia de impresión (\texttt{print}). Usted
ha visto un ejemplo de esto:

\begin{lstlisting}
print("Número de minutos desde media noche: ", hora*60+minuto)
\end{lstlisting}
 El caracter \textbackslash{} permite continuar en la siguiente línea.
Lo usaremos cuando las expresiones sean muy largas. Usted también
puede poner expresiones arbitrarias en el lado derecho de una sentencia
de asignación:

\begin{lstlisting}
porcentaje = (minuto * 100) / 60
\end{lstlisting}
 Esto no parece nada impresionante ahora, pero vamos a ver otros ejemplos
en los que la composición hace posible expresar cálculos complejos
organizada y concisamente.

Advertencia: hay restricciones sobre los lugares en los que se pueden
usar las expresiones. Por ejemplo, el lado izquierdo de una asignación
tiene que ser un nombre de {\em variable}, no una expresión. Así
que esto es ilegal: \texttt{minuto+1 = hora}.

\section{Comentarios}

\index{comentario}

A medida que los programas se hacen más grandes y complejos, se vuelven
más difíciles de leer. Los lenguajes formales son densos; y, a menudo,
es difícil mirar una sección de código y saber qué hace, o por qué
lo hace.

Por esta razón, es una muy buena idea añadir notas a sus programas
para explicar, en lenguaje natural, lo que hacen. Estas notas se denominan
\textbf{comentarios }y se marcan con el símbolo \texttt{\#}:

\begin{lstlisting}
# calcula el porcentaje de la hora que ha pasado
porcentaje = (minuto * 100) / 60
\end{lstlisting}
 En este caso, el comentario aparece en una línea completa. También
pueden ir comentarios al final de una línea:

\begin{lstlisting}
porcentaje = (minute * 100) // 60 # div entera
\end{lstlisting}

todo lo que sigue desde el \texttt{\#} hasta el fin de la línea se
ignora—no tiene efecto en el programa. El mensaje es para el programador
que escribe el programa o para algún programador que podría usar este
código en el futuro. En este caso, le recuerda al lector el sorprendente
comportamiento de la división entera en Python.

\section{Glosario}
\begin{description}
\item [{Valor:}] un número o una cadena (u otra cosa que se introduzca
más adelante) que puede ser almacenado en una variable o calculado
en una expresión.
\item [{Tipo:}] conjunto de valores. El tipo del valor determina cómo se
puede usar en expresiones. Hasta aquí, los tipos que usted ha visto
son enteros (tipo \texttt{int}), números de punto flotante (tipo \texttt{float})
y cadenas (tipo \texttt{string}).
\item [{Punto flotante:}] formato para representar números con parte decimal.
\item [{Variable:}] nombre que se refiere a un valor.
\item [{Sentencia:}] sección de código que representa un comando o acción.
Hasta aquí las sentencias que usted ha visto son la de asignación
y la de impresión.
\item [{Asignación:}] corresponde a la sentencia que pone un valor en una
variable.
\item [{Diagrama de estados:}] es la representación gráfica de un conjunto
de variables y los valores a los que se refieren.
\item [{Palabra reservada:}] es una palabra usada por el compilador para
analizar sintácticamente un programa; usted no puede usar palabras
reservadas como \texttt{if}, \texttt{def}, y \texttt{while} como nombres
de variables.
\item [{Operador:}] símbolo especial que representa un simple cálculo como
una suma, multiplicación o concatenación de cadenas.
\item [{Operando:}] uno de los valores sobre el cual actúa un operador.
\item [{Expresión:}] combinación de variables, operadores y valores que
representa un único valor de resultado.
\item [{Evaluar:}] simplificar una expresión ejecutando varias operaciones
a fin de retornar un valor único.
\item [{División entera:}] operación que divide un entero por otro y
retorna un entero. La división entera retorna el número de veces que
el denominador cabe en el numerador y descarta el residuo.
\item [{Reglas de precedencia:}] reglas que gobiernan el orden en que
las expresiones que tienen múltiples operadores y operandos se evalúan.
\item [{Concatenar:}] unir dos operandos en el orden en que aparecen.
\item [{Composición:}] es la capacidad de combinar simples expresiones
y sentencias dentro de sentencias y expresiones compuestas para representar
cálculos complejos concisamente.
\item [{Comentario:}] información que se incluye en un programa para otro
programador (o lector del código fuente) que no tiene efecto en la
ejecución.

\index{valor} \index{punto flotante} \index{variable} \index{tipo}
\index{palabra reservada} \index{sentencia} \index{asignación}
\index{comentario} \index{diagrama de estados} \index{expresión}
\index{operador} \index{operando} \index{división entera} \index{reglas de precedencia}
\index{precedencia} \index{concatenación} \index{composición}
\end{description}

\section{Ejercicios}
\begin{enumerate}
\item ¿Qué sucede cuando se usa la función print con una expresión?, por
ejemplo: 
\begin{lstlisting}
print(8+5)
\end{lstlisting}
\item ¿Qué sucede cuando se ejecuta esto?
\begin{lstlisting}
print(5.2, "esto", 4 - 2, "aquello", 5/2.0)
\end{lstlisting}
\item Tome la siguiente oración: Sólo trabajo y nada de juegos hacen de
Juan un niño aburrido. Almacene cada palabra en variables separadas,
después muestre la oración en una sola línea usando la función print.
\item Incluya paréntesis a la expresión 6 {*} 1 - 2 para cambiar su resultado
de 4 a -6.
\item Inserte una línea de comentario en un línea previa a una de código
funcional, y registre qué es lo que sucede cuando corre de nuevo el
programa.
%\item La diferencia entre la función input y la función raw\_input es que
%la función input evalúa la cadena introducida y la función raw\_input
%no lo hace. Escriba lo siguiente en el intérprete de Python, registre
%qué sucede y explique:
%
%\begin{lstlisting}
%>>> x = input()
%3.14
%>>> type(x)
%>>> x = raw_input()
%3.14
%>>> type(x)
%\end{lstlisting}
\item Escriba una expresión que calcule la nota definitiva de su curso de
programación.
\end{enumerate}


\clearemptydoublepage % variables, expresiones, sentencias

\chapter{Funciones }

\label{floatchap}

\section{Llamadas a funciones}

\label{functionchap} \index{llamada a función} \index{llamada!función}

Usted ya ha visto un ejemplo de una \textbf{llamada a función}:

\begin{pyconcode}
>>> type("32")
str
\end{pyconcode}

El nombre de la función es \texttt{type}, y despliega el tipo de un
valor o variable. El valor o variable, que se denomina el \textbf{argumento}
de la función, tiene que encerrarse entre paréntesis. Es usual decir
que una función ``toma'' un argumento y ``retorna'' un resultado.
El resultado se denomina el \textbf{valor de retorno}.

\index{argumento} \index{valor de retorno}

En lugar de imprimir el valor de retorno, podemos asignarlo a una
variable:

\begin{pyconcode}
>>> b = type("32")
>>> print(b)
<class 'string'>
\end{pyconcode}
 

Otro ejemplo es la función \texttt{id} que toma un valor o una variable
y retorna un entero que actúa como un identificador único:

\begin{pyconcode}
>>> id(3)
134882108
>>> b = 3
>>> id(b)
134882108
\end{pyconcode}
 Cada valor tiene un \texttt{id} que es un número único relacionado
con el lugar en la memoria en el que está almacenado. El \texttt{id}
de una variable es el \texttt{id} del valor al que la variable se
refiere.

\section{Conversión de tipos}

\index{conversión!tipo} \index{conversión}

Python proporciona una colección de funciones que convierten valores
de un tipo a otro. La función \texttt{int} toma cualquier valor y
lo convierte a un entero, si es posible, de lo contrario se queja:

\begin{pyconcode}
>>> int("32")
32
>>> int("Hola")
ValueError: invalid literal for int() with base 10: Hola
\end{pyconcode}
 \texttt{int} también puede convertir valores de punto flotante a
enteros, pero hay que tener en cuenta que va a eliminar la parte decimal:

\begin{pyconcode}
>>> int(3.99999)
3
>>> int(-2.3)
-2
\end{pyconcode}
 La función \texttt{float} convierte enteros y cadenas a números de
punto flotante:

\begin{pyconcode}
>>> float(32)
32.0
>>> float("3.14159")
3.14159
\end{pyconcode}
 Finalmente, la función \texttt{str} convierte al tipo cadena (\texttt{string}):

\begin{pyconcode}
>>> str(32)
'32'
>>> str(3.14149)
'3.14149'
\end{pyconcode}
 Puede parecer extraño el hecho de que Python distinga el valor entero
\texttt{1} del valor en punto flotante \texttt{1.0}. Pueden representar
el mismo número pero tienen diferentes tipos. La razón para esto es
que su representación interna en la memoria del computador es distinta.

\section{Coerción de tipos}

\index{coerción de tipos} \index{coerción!tipo} \index{división entera}
\index{división!entera}

Podemos sacar provecho de las reglas de conversión automática de tipos,
que se denominan \textbf{coerción de tipos}. Para los operadores matemáticos,
si algún operando es un número \texttt{flotante}, el otro se convierte
automáticamente a \texttt{flotante}:

\begin{pyconcode}
>>> minuto = 59
>>> minuto / 60.0
0.983333333333
\end{pyconcode}

Así que haciendo el denominador flotante, forzamos a Python a realizar
división en punto flotante.

\section{Funciones matemáticas}

\index{función matemática} \index{función!matemática}

En matemática usted probablemente ha visto funciones como el \texttt{seno}
y el \texttt{logaritmo}, y ha aprendido a evaluar expresiones como
\texttt{sen(pi/2)} y \texttt{log(1/x)}. Primero, se evalúa la expresión
entre paréntesis (el argumento). Por ejemplo, \texttt{pi/2} es aproximadamente
1.571, y \texttt{1/x} es 0.1 (si \texttt{x} tiene el valor 10.0).

Entonces, se evalúa la función, ya sea mirando el resultado en una
tabla o calculando varias operaciones. El \texttt{seno} de 1.571 es
1, y el \texttt{logaritmo} de 0.1 es -1 (asumiendo que \texttt{log}
indica el logaritmo en base 10).

Este proceso puede aplicarse repetidamente para evaluar expresiones
más complicadas como \texttt{log(1/sen(pi/2))}. Primero se evalúa
el argumento de la función más interna, luego la función, y se continúa
así.

Python tiene un módulo matemático que proporciona la mayoría de las
funciones matemáticas. Un módulo es un archivo que contiene una colección
de funciones relacionadas.

\index{módulo}

Antes de que podamos usar funciones de un módulo, tenemos que importarlas:

\begin{pyconcode}
>>> import math
\end{pyconcode}
 Para llamar a una de las funciones, tenemos que especificar el nombre
del módulo y el nombre de la función, separados por un punto. Este
formato se denomina \textbf{notación punto}.

\index{notación punto}

\begin{pyconcode}
>>> decibel = math.log10(17.0)
>>> angulo = 1.5
>>> altura = math.sin(angulo)
\end{pyconcode}
 La primera sentencia le asigna a \texttt{decibel} el logaritmo de
17, en base \texttt{10}. También hay una función llamada \texttt{log}
que usa la base logarítmica \texttt{e}.

La tercera sentencia encuentra el seno del valor de la variable \texttt{angulo}.
\texttt{sin} y las otras funciones trigonométricas (\texttt{cos},
\texttt{tan}, etc.) reciben sus argumentos en radianes. Para convertir
de grados a radianes hay que dividir por 360 y multiplicar por \texttt{2{*}pi}.
Por ejemplo, para encontrar el seno de 45 grados, primero calculamos
el ángulo en radianes y luego tomamos el seno:

\begin{pyconcode}
>>> grados = 45
>>> angulo = grados * 2 * math.pi / 360.0
>>> math.sin(angulo)
\end{pyconcode}
 La constante \texttt{pi} también hace parte del módulo matemático.
Si usted recuerda geometría puede verificar el resultado comparándolo
con la raíz cuadrada de 2 dividida por 2:

\begin{pyconcode}
>>> math.sqrt(2) / 2.0
0.707106781187
\end{pyconcode}
 

\section{Composición}

\index{composición} \index{función!composición}

Así como las funciones matemáticas, las funciones de Python pueden
componerse, de forma que una expresión sea parte de otra. Por ejemplo,
usted puede usar cualquier expresión como argumento a una función:

\begin{pyconcode}
>>> x = math.cos(angulo + math.pi/2)
\end{pyconcode}

Esta sentencia toma el valor de \texttt{pi}, lo divide por 2, y suma
este resultado al valor de \texttt{angulo}. Después, la suma se le
pasa como argumento a la función coseno (\texttt{cos}).

También se puede tomar el resultado de una función y pasarlo como
argumento a otra:
\begin{pyconcode}
>>> x = math.exp(math.log(10.0))
\end{pyconcode}

Esta sentencia halla el logaritmo en base \texttt{e} de 10 y luego
eleva \texttt{e} a dicho resultado. El resultado se asigna a \texttt{x}.

\section{Agregando nuevas funciones}

Hasta aquí solo hemos usado las funciones que vienen con Python, pero
también es posible agregar nuevas funciones. Crear nuevas funciones
para resolver nuestros problemas particulares es una de las capacidades
mas importantes de un lenguaje de programación de propósito general.

En el contexto de la programación, una \textbf{función} es una secuencia
de sentencias que ejecuta una operación deseada y tiene un nombre.
Esta operación se especifica en una \textbf{definición de función}.
Las funciones que hemos usado hasta ahora ya han sido definidas para
nosotros. Esto es bueno, porque nos permite usarlas sin preocuparnos
de los detalles de sus definiciones.

\index{función} \index{función definición} \index{definición!función}

La sintaxis para una definición de función es:

\begin{pythoncode}
def NOMBRE( LISTA DE PARAMETROS ):
  SENTENCIAS
\end{pythoncode}
 Usted puede inventar los nombres que desee para sus funciones con
tal de que no use una palabra reservada. La lista de parámetros especifica
que información, si es que la hay, se debe proporcionar a fin de usar
la nueva función.

Se puede incluir cualquier número de sentencias dentro de la función,
pero tienen que sangrarse o indentarse a partir de la margen izquierda.
La práctica estándar de Python es usar 4 espacios.

Las primeras funciones que vamos a escribir no tienen parámetros,
así que la sintaxis luce así:

\begin{pythoncode}
def nuevaLinea():
  print()
\end{pythoncode}
 Esta función se llama \texttt{nuevaLinea}. Los paréntesis vacíos
indican que no tiene parámetros. Contiene solamente una sentencia,
que produce como salida una línea vacía. Eso es lo que ocurre cuando
se usa el comando \texttt{print} sin argumentos.

La sintaxis para llamar la nueva función es la misma que para las
funciones predefinidas en Python:

\begin{pythoncode}
print("Primera Linea.")
nuevaLinea()
print("Segunda Linea.")
\end{pythoncode}
 La salida para este programa es:
\begin{verbatim}
Primera Linea.

Segunda Linea.
\end{verbatim}
Note el espacio extra entre las dos líneas. ¿Qué pasa si deseamos
más espacio entre las líneas? Podemos llamar la misma función repetidamente:

\begin{pythoncode}
print("Primera Linea.")
nuevaLinea()
nuevaLinea()
nuevaLinea()
print("Segunda Linea.")
\end{pythoncode}
 

O podemos escribir una nueva función llamada \texttt{tresLineas} que
imprima tres líneas:

\begin{pythoncode}
def tresLineas():
  nuevaLinea()
  nuevaLinea()
  nuevaLinea()

print("Primera Linea.")
tresLineas()
print("Segunda Linea.")
\end{pythoncode}
 Esta función contiene tres sentencias, y todas están sangradas por
dos espacios. Como la próxima sentencia, \texttt{}\lstinline!print("Primera Linea")!,
no está sangrada, Python la interpreta afuera de la función.

Hay que enfatizar dos hechos sobre este programa:
\begin{enumerate}
\item Usted puede llamar la misma función repetidamente. De hecho, es una
práctica muy común y útil.
\item Usted puede llamar una función dentro de otra función; en este caso
\texttt{tresLineas} llama a \texttt{nuevaLinea}.
\end{enumerate}
Hasta este punto, puede que no parezca claro porque hay que tomarse
la molestia de crear todas estas funciones. De hecho, hay muchas razones,
y este ejemplo muestra dos:
\begin{itemize}
\item Crear una nueva función le da a usted la oportunidad de nombrar un
grupo de sentencias. Las funciones pueden simplificar un programa
escondiendo un cálculo complejo detrás de un comando único que usa
palabras en lenguaje natural, en lugar de un código arcano.
\item Crear una nueva función puede recortar el tamaño de un programa eliminando
el código repetitivo. Por ejemplo, una forma más corta de imprimir
nueve líneas consecutivas consiste en llamar la función \texttt{tresLineas}
tres veces.
\end{itemize}

\pagebreak

\section{Definiciones y uso}

Uniendo los fragmentos de la sección 3.6, el programa completo luce
así:

\begin{pythoncode}
def nuevaLinea():
  print()

def tresLineas():
  nuevaLinea()
  nuevaLinea()
  nuevaLinea()

print("Primera Linea.")
tresLineas()
print("Segunda Linea.")
\end{pythoncode}
 Este programa contiene dos definiciones de funciones: \texttt{nuevaLinea}
y \texttt{tresLineas}. Las definiciones de funciones se ejecutan como
las otras sentencias, pero su efecto es crear nuevas funciones. Las
sentencias, dentro de la función, no se ejecutan hasta que la función
sea llamada, y la definición no genera salida.

Como usted puede imaginar, se tiene que crear una función antes de
ejecutarla. En otras palabras, la definición de función tiene que
ejecutarse antes de llamarla por primera vez.

\section{Flujo de ejecución}

\index{flujo de ejecución}

Con el objetivo de asegurar que una función se defina antes de su
primer uso usted tiene que saber el orden en el que las sentencias
se ejecutan, lo que denominamos \textbf{flujo de ejecución}.

La ejecución siempre empieza con la primera sentencia del programa.
Las sentencias se ejecutan una a una, desde arriba hacia abajo.

Las definiciones de funciones no alteran el flujo de ejecución del
programa, recuerde que las sentencias que están adentro de las funciones
no se ejecutan hasta que éstas sean llamadas. Aunque no es muy común,
usted puede definir una función adentro de otra. En este caso, la
definición interna no se ejecuta hasta que la otra función se llame.

Las llamadas a función son como un desvío en el flujo de ejecución.
En lugar de continuar con la siguiente sentencia, el flujo salta a
la primera línea de la función llamada, ejecuta todas las sentencias
internas, y regresa para continuar donde estaba previamente.

Esto suena sencillo, hasta que tenemos en cuenta que una función puede
llamar a otra. Mientras se está ejecutando una función, el programa
puede ejecutar las sentencias en otra función. Pero, mientras se está
ejecutando la nueva función, ¡el programa puede tener que ejecutar
\textit{otra} función!.

Afortunadamente, Python lleva la pista de donde está fielmente, así
que cada vez que una función termina, el programa continúa su ejecución
en el punto donde se la llamó. Cuando llega al fin del programa, la
ejecución termina.

¿Cual es la moraleja de esta sórdida historia? Cuando lea un programa,
no lo haga de arriba hacia abajo. En lugar de ésto, siga el flujo
de ejecución.

\section{Parámetros y argumentos}

\label{parameters} \index{parámetro} \index{función!parámetro}
\index{argumento} \index{función!argumento}

Algunas de las funciones primitivas que usted ha usado requieren argumentos,
los valores que controlan el trabajo de la función. Por ejemplo, si
usted quiere encontrar el seno de un número, tiene que indicar cual
es el número. Así que, \texttt{sin} toma un valor numérico como argumento.

Algunas funciones toman más de un argumento. Por ejemplo \texttt{pow}
(potencia) toma dos argumentos, la base y el exponente. Dentro de
una función, los valores que se pasan se asignan a variables llamadas
\textbf{parámetros}.

Aquí hay un ejemplo de una función definida por el programador que
toma un parámetro:

\begin{pythoncode}
def imprimaDoble(pedro):
  print(pedro, pedro)
\end{pythoncode}
 Esta función toma un argumento y lo asigna a un parámetro llamado
\texttt{pedro}. El valor del parámetro (en este momento no tenemos
idea de lo que será) se imprime dos veces, y después, se imprime una
línea vacía. El nombre \texttt{pedro} se escogió para sugerir que
el nombre que se le asigna a un parámetro queda a su libertad; pero,
en general, usted desea escoger algo mas ilustrativo que \texttt{pedro}.

La función \texttt{imprimaDoble} funciona para cualquier tipo que
pueda imprimirse:

\begin{pyconcode}
>>> imprimaDoble('Spam')
Spam Spam
>>> imprimaDoble(5)
5 5
>>> imprimaDoble(3.14159)
3.14159 3.14159
\end{pyconcode}
 En el primer llamado de función el argumento es una cadena. En el
segundo es un entero. En el tercero es un flotante (\texttt{float}).

Las mismas reglas de composición que se aplican a las funciones primitivas,
se aplican a las definidas por el programador, así que podemos usar
cualquier clase de expresión como un argumento para \texttt{imprimaDoble}:

\begin{pyconcode}
>>> imprimaDoble('Spam'*4)
SpamSpamSpamSpam SpamSpamSpamSpam
>>> imprimaDoble(math.cos(math.pi))
-1.0 -1.0
\end{pyconcode}
 Como de costumbre, la expresión se evalúa antes de que la función
se ejecute así que \texttt{imprimaDoble} retorna \texttt{SpamSpamSpamSpam
SpamSpamSpamSpam} en lugar de \texttt{'Spam'{*}4 'Spam'{*}4}.

También podemos usar una variable como argumento:

\begin{pyconcode}
>>> m = 'Oh, mundo cruel.'
>>> imprimaDoble(m)
Oh, mundo cruel. Oh, mundo cruel.
\end{pyconcode}
 Observe algo muy importante, el nombre de la variable que pasamos
como argumento (\texttt{m}) no tiene nada que ver con el nombre del
parámetro (\texttt{pedro}). No importa como se nombraba el valor originalmente
(en el lugar donde se hace el llamado); en la función \texttt{imprimaDoble},
la seguimos llamando de la misma manera \texttt{pedro}.

\section{Las variables y los parámetros son locales}

\index{variable local} \index{variable!local}

Cuando usted crea una \textbf{variable local} en una función, solamente
existe dentro de ella, y no se puede usar por fuera. Por ejemplo:

\begin{pythoncode}
def concatenarDoble(parte1, parte2):
  cat = parte1 + parte2
  imprimaDoble(cat)
\end{pythoncode}
 Esta función toma dos argumentos, los concatena, y luego imprime
el resultado dos veces. Podemos llamar a la función con dos cadenas:

\begin{pyconcode}
>>> cantar1 = "Pie Jesu domine, "
>>> cantar2 = "Dona eis requiem."
>>> concatenarDoble(cantar1, cantar2)
Pie Jesu domine, Dona eis requiem. Pie Jesu domine, Dona 
eis requiem.
\end{pyconcode}

Cuando \texttt{concatenarDoble} termina, la variable \texttt{cat}
se destruye. Si intentáramos imprimirla obtendríamos un error:

\begin{pyconcode}
>>> print(cat)
NameError: cat
\end{pyconcode}
 Los parámetros también son locales. Por ejemplo, afuera de la función
\texttt{imprimaDoble}, no existe algo como \texttt{pedro}. Si usted
intenta usarlo Python se quejará.

\section{Diagramas de pila}

\label{stackdiagram} \index{diagrama de pila} \index{marco de función}
\index{marco}

Para llevar pista de los lugares en que pueden usarse las variables
es útil dibujar un \textbf{diagrama de pila}. Como los diagramas de
estados, los diagramas de pila muestran el valor de cada variable
y además muestran a que función pertenece cada una.

Cada función se representa por un \textbf{marco}. Un marco es una
caja con el nombre de una función al lado y los parámetros y variables
adentro. El diagrama de pila para el ejemplo anterior luce así:

%\adjustpage{-4} 
\beforefig \centerline{\includegraphics{illustrations/stack}}
\afterfig

El orden de la pila muestra el flujo de ejecución. \texttt{imprimaDoble}
fue llamada por \texttt{concatenarDoble}, y \texttt{concatenarDoble}
fue llamada por \texttt{\_\_main\_\_}, que es un nombre especial para
la función más superior (la principal, que tiene todo programa). Cuando
usted crea una variable afuera de cualquier función, pertenece a \texttt{\_\_main\_\_}.

Cada parámetro se refiere al mismo valor que su argumento correspondiente.
Así que \texttt{parte1} tiene el mismo valor que \texttt{cantar1},
\texttt{parte2} tiene el mismo valor que \texttt{cantar2}, y \texttt{pedro}
tiene el mismo valor que \texttt{cat}.

Si hay un error durante una llamada de función, Python imprime el
nombre de ésta, el nombre de la función que la llamó, y así sucesivamente
hasta llegar a \texttt{\_\_main\_\_}.

Por ejemplo, si intentamos acceder a \texttt{cat} desde \texttt{imprimaDoble},
obtenemos un \texttt{error de nombre (NameError)}:
\begin{verbatim}
Traceback (innermost last):
  File "test.py", line 13, in __main__
    concatenarDoble(cantar1, cantar2)
  File "test.py", line 5, in concatenarDoble
    imprimaDoble(cat)
  File "test.py", line 9, in imprimaDoble
    print(cat)
NameError: cat
\end{verbatim}
Esta lista de funciones se denomina un \textbf{trazado inverso}. Nos
informa en qué archivo de programa ocurrió el error, en qué línea,
y qué funciones se estaban ejecutando en ese momento. También muestra
la línea de código que causó el error.

\index{trazado inverso}

Note la semejanza entre el trazado inverso y el diagrama de pila.
Esto no es una coincidencia.

\section{Funciones con resultados}

Usted ya puede haber notado que algunas de las funciones que estamos
usando, como las matemáticas, entregan resultados. Otras funciones,
como \texttt{nuevaLinea}, ejecutan una acción pero no entregan un
resultado. Esto genera algunas preguntas:
\begin{enumerate}
\item ¿Qué pasa si usted llama a una función y no hace nada con el resultado
(no lo asigna a una variable o no lo usa como parte de una expresión
mas grande)?
\item ¿Qué pasa si usted usa una función sin un resultado como parte de
una expresión, tal como \texttt{nuevaLinea() + 7}?
\item ¿Se pueden escribir funciones que entreguen resultados, o estamos
limitados a funciones tan simples como \texttt{nuevaLinea} y \texttt{imprimaDoble}?
\end{enumerate}
La respuesta a la tercera pregunta es afirmativa y lo lograremos en
el capítulo \ref{funcReturn}.

\section{Glosario}
\begin{description}
\item [{Llamada a función:}] sentencia que ejecuta una función. Consiste
en el nombre de la función seguido por una lista de argumentos encerrados
entre paréntesis.
\item [{Argumento:}] valor que se le da a una función cuando se la está
llamando. Este valor se le asigna al parámetro correspondiente en
la función.
\item [{Valor de retorno:}] es el resultado de una función. Si una llamada
a función se usa como una expresión, el valor de ésta es el valor
de retorno de la función.
\item [{Conversión de tipo:}] sentencia explícita que toma un valor de
un tipo y calcula el valor correspondiente de otro tipo.
\item [{Coerción de tipos:}] conversión de tipo que se hace automáticamente
de acuerdo a las reglas de coerción del lenguaje de programación.
\item [{Módulo:}] archivo que contiene una colección de funciones y clases
relacionadas.
\item [{Notación punto:}] sintaxis para llamar una función que se encuentra
en otro módulo, especificando el nombre módulo seguido por un punto
y el nombre de la función (sin dejar espacios intermedios).
\item [{Función:}] es la secuencia de sentencias que ejecuta alguna operación
útil y que tiene un nombre definido. Las funciones pueden tomar o
no tomar parámetros y pueden entregar o no entregar un resultado.
\item [{Definición de función:}] sentencia que crea una nueva función
especificando su nombre, parámetros y las sentencias que ejecuta.
\item [{Flujo de ejecución:}] orden en el que las sentencias se ejecutan
cuando un programa corre.
\item [{Parámetro:}] nombre usado dentro de una función para referirse
al valor que se pasa como argumento.
\item [{Variable local:}] variable definida dentro de una función. Una
variable local solo puede usarse dentro de su función.
\item [{Diagrama de pila:}] es la representación gráfica de una pila de
funciones, sus variables, y los valores a los que se refieren.
\item [{Marco:}] una caja en un diagrama de pila que representa un llamado
de función. Contiene las variables locales y los parámetros de la
función.
\item [{Trazado inverso:}] lista de las funciones que se estaban ejecutando
y que se imprime cuando ocurre un error en tiempo de ejecución.

\index{llamada a función} \index{valor de retorno} \index{argumento}
\index{coerción} \index{módulo} \index{notación punto} \index{función}
\index{definición de función} \index{flujo de ejecución} \index{parámetro}
\index{variable local} \index{diagrama de pila} \index{marco de función}
\index{marco} \index{trazado inverso}
\end{description}

\section{Ejercicios}
\begin{enumerate}
\item Con un editor de texto cree un guión de Python que se llame pruebame3.py.
Escriba en este archivo una función que se llame \verb+nueveLineas+
que use la función \verb+tresLineas+ para mostrar nueve líneas en
blanco. Enseguida agregue una función que se llame \verb+limpiaPantalla+
que muestre veinticinco líneas en blanco. La última instrucción en
su programa debe ser una llamada a \verb+limpiaPantalla+.
\item Mueva la última instrucción del archivo pruebame3.py al inicio del
programa, de forma tal que la llamada a la función \verb+limpiaPantalla+
esté antes que la definición de función. Ejecute el programa y registre
qué mensaje de error obtiene. ¿Puede establecer una regla sobre las
definiciones de funciones y las llamadas a función que describa la
posición relativa entre ellas en el programa?
\item Escriba una función que imprima la distancia que hay entre dos puntos
ubicados sobre el eje X de un plano cartesiano conociendo sus coordenadas
horizontales.
\item Escriba una función que imprima la distancia que hay entre dos puntos
ubicados sobre el eje Y de un plano cartesiano conociendo sus coordenadas
verticales.
\item Escriba una función que imprima la distancia que hay entre dos puntos
en un plano coordenado, recordando el teorema de Pitágoras.
\item Tome la solución del último ejercicio del capítulo anterior y conviértala
en una función que imprima la nota definitiva de su curso de programación.
\end{enumerate}


\clearemptydoublepage % funciones

\chapter{Condicionales y recursión}

\section{El operador residuo}

\index{operador residuo} \index{operador!residuo}

El \textbf{operador residuo} trabaja con enteros (y expresiones enteras)
calculando el residuo del primer operando cuando se divide por el
segundo. En Python este operador es un signo porcentaje (\texttt{\%}).
La sintaxis es la misma que para los otros operadores:

\inputencoding{latin9}\begin{lstlisting}
>>> cociente = 7 // 3
>>> print(cociente)
2
>>> residuo = 7 % 3
>>> print(residuo)
1
\end{lstlisting}
\inputencoding{utf8}
Así que 7 dividido por 3 da 2 con residuo 1.

El operador residuo resulta ser sorprendentemente útil. Por ejemplo,
usted puede chequear si un número es divisible por otro —si \texttt{x\%y}
es cero, entonces \texttt{x} es divisible por \texttt{y}.

Usted también puede extraer el dígito o dígitos más a la derecha de
un número. Por ejemplo, \texttt{x \% 10} entrega el dígito más a la
derecha de \texttt{x} (en base 10). Igualmente, \texttt{x \% 100}
entrega los dos últimos dígitos.

\section{Expresiones booleanas}

\index{expresión Booleana} \index{expresión!booleana} \index{operador lógico}
\index{operador!lógico}

El tipo que Python provee para almacenar valores de verdad (cierto
o falso) se denomina bool por el matemático británico George Bool.
Él creó el Álgebra Booleana, que es la base para la aritmética que
se usa en los computadores modernos.

Sólo hay dos valores booleanos: True (cierto) y False (falso). Las
mayúsculas importan, ya que true y false no son valores booleanos.

El operador \texttt{==} compara dos valores y produce una expresión
booleana:

\inputencoding{latin9}\begin{lstlisting}
>>> 5 == 5
True
>>> 5 == 6
False
\end{lstlisting}
\inputencoding{utf8} En la primera sentencia, los dos operandos son iguales, así que la
expresión evalúa a True (cierto); en la segunda sentencia, 5 no es
igual a 6, así que obtenemos False (falso).

El operador \texttt{==} es uno de los \textbf{operadores de comparación};
los otros son:\inputencoding{latin9}
\begin{lstlisting}
      x != y               # x no es igual y
      x > y                # x es mayor que y
      x < y                # x es menor que y
      x >= y               # x es mayor o igual a y
      x <= y               # x es menor o igual a y
\end{lstlisting}
\inputencoding{utf8}
Aunque estas operaciones probablemente son familiares para usted,
los símbolos en Python difieren de los matemáticos. Un error común
consiste en usar un solo signo igual (\texttt{=}) en lugar en un doble
signo igual (\texttt{==}). Recuerde que \texttt{=} es el operador
para la asignación y que \texttt{==} es el operador para comparación.
Tenga en cuenta que no existen los signos \texttt{=<} o \texttt{=>}.

\section{Operadores lógicos}

\index{operadores lógicos} \index{operador!lógicos}

Hay tres \textbf{operadores lógicos}: \texttt{and}, \texttt{or} y
\texttt{not}. La semántica (el significado) de ellos es similar a
su significado en inglés. Por ejemplo, \texttt{x>0 and x<10} es cierto,
sólo si \texttt{x} es mayor a cero {\em y} menor que 10.

\texttt{n\%2 == 0 or n\%3 == 0} es cierto si {\em alguna} de las
condiciones es cierta, esto es, si el número es divisible por 2 {\em
o} por 3.

Finalmente, el operador \texttt{not} niega una expresión booleana,
así que \texttt{not(x>y)} es cierta si \texttt{(x>y)} es falsa, esto
es, si \texttt{x} es menor o igual a \texttt{y}.

Formalmente, los operandos de los operadores lógicos deben ser expresiones
booleanas, pero Python no es muy formal. Cualquier número diferente
de cero se interpreta como ``cierto.''\inputencoding{latin9}
\begin{lstlisting}
>>>  x = 5
>>>  x and 1
1
>>>  y = 0
>>>  y and 1
0
\end{lstlisting}
\inputencoding{utf8}
En general, esto no se considera un buen estilo de programación. Si
usted desea comparar un valor con cero, procure codificarlo explícitamente.

\section{Ejecución condicional}

\label{alternative execution} \index{ramificación condicional}
\index{ejecución condicional}

A fin de escribir programas útiles, casi siempre necesitamos la capacidad
de chequear condiciones y cambiar el comportamiento del programa en
consecuencia. Las \textbf{sentencias condicionales} nos dan este poder.
La más simple es la sentencia \texttt{if}: \inputencoding{latin9}
\begin{lstlisting}
if x > 0:
  print("x es positivo")
\end{lstlisting}
\inputencoding{utf8}
La expresión después de la sentencia \texttt{if} se denomina la \textbf{condición}.
Si es cierta, la sentencia de abajo se ejecuta. Si no lo es, no pasa
nada.

\index{sentencia compuesta} \index{sentencia compuesta!cabecera}
\index{sentencia compuesta!cuerpo} \index{sentencia compuesta!bloque de sentencias}
\index{sentencia!compuesta}

Como otras sentencias compuestas, la sentencia \texttt{if} comprende
una cabecera y un bloque de sentencias:
\begin{verbatim}
CABECERA:
  PRIMERA SENTENCIA
  ...
  ULTIMA SENTENCIA
\end{verbatim}
La cabecera comienza en una nueva línea y termina con dos puntos seguidos
(:). Las sentencias sangradas o indentadas que vienen a continuación
se denominan el \textbf{bloque}. La primera sentencia sin sangrar
marca el fin del bloque. Un bloque de sentencias dentro de una sentencia
compuesta también se denomina el \textbf{cuerpo} de la sentencia.

\index{bloque} \index{sentencia!bloque} \index{cuerpo}

No hay límite en el número de sentencias que pueden aparecer en el
cuerpo de una sentencia, pero siempre tiene que haber, al menos, una.
Ocasionalmente, es útil tener un cuerpo sin sentencias (como un hueco
para código que aún no se ha escrito). En ese caso se puede usar la
sentencia \texttt{pass}, que no hace nada.

\index{sentencia pass} \index{sentencia!pass}

\section{Ejecución alternativa}

\label{alternative execution2}

Una segunda forma de sentencia \texttt{if} es la ejecución alternativa
en la que hay dos posibilidades y la condición determina cual de ellas
se ejecuta. La sintaxis luce así:\inputencoding{latin9}
\begin{lstlisting}
if x%2 == 0:
  print(x, "es par")
else:
  print(x, "es impar")
\end{lstlisting}
\inputencoding{utf8}
Si el residuo de dividir \texttt{x} por 2 es 0, entonces sabemos que
\texttt{x} es par, y el programa despliega un mensaje anunciando esto.
Si la condición es falsa, la segunda sentencia se ejecuta. Como la
condición, que es una expresión booleana, debe ser cierta o falsa,
exactamente una de las alternativas se va a ejecutar. Estas alternativas
se denominan \textbf{ramas}, porque, de hecho, son ramas en el flujo
de ejecución.

\index{rama}

Yéndonos ``por las ramas'', si usted necesita chequear la paridad
(si un número es par o impar) a menudo, se podría ``envolver'' el
código anterior en una función:\inputencoding{latin9}
\begin{lstlisting}
def imprimirParidad(x):
  if x%2 == 0:
    print(x, "es par")
  else:
    print(x, "es impar")
\end{lstlisting}
\inputencoding{utf8}
Para cualquier valor de \texttt{x}, \texttt{imprimirParidad} despliega
un mensaje apropiado. Cuando se llama la función, se le puede pasar
cualquier expresión entera como argumento.\inputencoding{latin9}
\begin{lstlisting}
>>> imprimirParidad(17)
>>> imprimirParidad(y+1)
\end{lstlisting}
\inputencoding{utf8}
\section{Condicionales encadenados}

\index{condicional encadenados} \index{condicional!encadenados}

Algunas veces hay más de dos posibilidades y necesitamos más de dos
ramas. Una forma de expresar un cálculo así es un \textbf{condicional
encadenado}:

\inputencoding{latin9}\begin{lstlisting}
if x < y:
  print(x, "es menor que", y)
elif x > y:
  print(x, "es mayor que", y)
else:
  print(x, "y", y, "son iguales")
\end{lstlisting}
\inputencoding{utf8}
\texttt{elif} es una abreviatura de ``else if.'' De nuevo, exactamente
una de las ramas se ejecutará. No hay límite en el número de sentencias
\texttt{elif}, pero la última rama tiene que ser una sentencia \texttt{else}:\inputencoding{latin9}
\begin{lstlisting}
if eleccion == 'A':
  funcionA()
elif eleccion == 'B':
  funcionB()
elif eleccion == 'C':
  funcionC()
else:
  print("Eleccion incorrecta.")
\end{lstlisting}
\inputencoding{utf8}
Cada condición se chequea en orden. Si la primera es falsa, se chequea
la siguiente, y así sucesivamente. Si una de ellas es cierta, se ejecuta
la rama correspondiente y la sentencia termina. Si hay más de una
condición cierta, sólo la primera rama que evalúa a cierto se ejecuta.

\section{Condicionales anidados}

Un condicional también se puede anidar dentro de otro. La tricotomía
anterior se puede escribir así:

\inputencoding{latin9}\begin{lstlisting}
if x == y:
  print(x, "y", y, "son iguales")
else:
  if x < y:
    print(x, "es menor que", y)
  else:
    print(x, "es mayor que", y)
\end{lstlisting}
\inputencoding{utf8} El condicional externo contiene dos ramas: la primera contiene una
sentencia de salida sencilla, la segunda contiene otra sentencia \texttt{if},
que tiene dos ramas propias. Esas dos ramas son sentencias de impresión,
aunque también podrían ser sentencias condicionales.

Aunque la indentación o sangrado de las sentencias sugiere la estructura,
los condicionales anidados rápidamente se hacen difíciles de leer.
En general, es una buena idea evitarlos cada vez que se pueda.

Los operadores lógicos proporcionan formas de simplificar las sentencias
condicionales anidadas. Por ejemplo, podemos reescribir el siguiente
código usando un solo condicional:\inputencoding{latin9}
\begin{lstlisting}
if 0 < x:
  if x < 10:
    print("x es un digito positivo.")
\end{lstlisting}
\inputencoding{utf8}
La sentencia \texttt{print} se ejecuta solamente si el flujo de ejecución
ha pasado las dos condiciones, así que podemos usar el operador \texttt{and}:\inputencoding{latin9}
\begin{lstlisting}
if 0 < x and x < 10:
  print("x es un digito positivo.")
\end{lstlisting}
\inputencoding{utf8}
Esta clase de condiciones es muy común, por esta razón Python proporciona
una sintaxis alternativa que es similar a la notación matemática:\inputencoding{latin9}
\begin{lstlisting}
if 0 < x < 10:
  print("x es un digito positivo")
\end{lstlisting}
\inputencoding{utf8}
Desde el punto de vista semántico ésta condición es la misma que la
expresión compuesta y que el condicional anidado.

\section{La sentencia \texttt{return} }

\index{sentencia return} \index{sentencia!return}

La sentencia \texttt{return} permite terminar la ejecución de una
función antes de llegar al final. Una razón para usarla es reaccionar
a una condición de error:

\inputencoding{latin9}\begin{lstlisting}
import math

def imprimirLogaritmo(x):
  if x <= 0:
    print("Numeros positivos solamente. Por favor")
    return

  result = math.log(x)
  print("El logaritmo de ",  x ," es ", result)
\end{lstlisting}
\inputencoding{utf8} La función \texttt{imprimirLogaritmo} toma un parámetro denominado
\texttt{x}. Lo primero que hace es chequear si \texttt{x} es menor
o igual a 0, caso en el que despliega un mensaje de error y luego
usa a \texttt{return} para salir de la función. El flujo de ejecución
inmediatamente retorna al punto donde se había llamado la función,
y las líneas restantes de la función no se ejecutan.

Recuerde que para usar una función del módulo matemático (math) hay
que importarlo previamente.

\section{Recursión}

\label{recursion} \index{recursión}

Hemos mencionado que es legal que una función llame a otra, y usted
ha visto varios ejemplos así. Hemos olvidado mencionar el hecho de
que una función también puede llamarse a sí misma. Al principio no
parece algo útil, pero resulta ser una de las capacidades más interesantes
y mágicas que un programa puede tener. Por ejemplo, observe la siguiente
función:

\pagebreak{}

\inputencoding{latin9}\begin{lstlisting}
def conteo(n):
  if n == 0:
    print("Despegue!")
  else:
    print(n)
    conteo(n-1)
\end{lstlisting}
\inputencoding{utf8}
\texttt{conteo} espera que el parámetro \texttt{n} sea un número entero
positivo. Si \texttt{n} es 0, despliega la cadena, ``Despegue!''.
Si no lo es, despliega \texttt{n} y luego llama a la función llamada
\texttt{conteo}—ella misma—pasando a \texttt{n-1} como argumento.

Analicemos lo que sucede si llamamos a esta función así:\inputencoding{latin9}
\begin{lstlisting}
>>> conteo(3)
\end{lstlisting}
\inputencoding{utf8}
La ejecución de \texttt{conteo} comienza con \texttt{n=3}, y como
\texttt{n} no es 0, despliega el valor 3, y se llama a sí misma ...
\begin{quote}
La ejecución de \texttt{conteo} comienza con \texttt{n=2}, y como
\texttt{n} no es 0, despliega el valor 2, y se llama a si misma ...

\begin{quote}
La ejecución de \texttt{conteo} comienza con \texttt{n=1}, y como
\texttt{n} no es 0, despliega el valor 1, y se llama a sí misma ...

\begin{quote}
La ejecución de \texttt{conteo} comienza con \texttt{n=0}, y como
\texttt{n} es 0, despliega la cadena ``Despegue!'' y retorna (finaliza). 
\end{quote}
El \texttt{conteo} que recibió \texttt{n=1} retorna. 
\end{quote}
El \texttt{conteo} que recibió \texttt{n=2} retorna. 
\end{quote}
El \texttt{conteo} que recibió \texttt{n=3} retorna.

Y el flujo de ejecución regresa a \texttt{\_\_main\_\_} (vaya viaje!).
Así que, la salida total luce así:
\begin{verbatim}
3
2
1
Despegue!
\end{verbatim}
Como otro ejemplo, utilizaremos nuevamente las funciones \texttt{nuevaLinea}
y \texttt{tresLineas}:

\inputencoding{latin9}\begin{lstlisting}
def nuevalinea():
  print()

def tresLineas():
  nuevaLinea()
  nuevaLinea()
  nuevaLinea()
\end{lstlisting}
\inputencoding{utf8}
Este trabajo no sería de mucha ayuda si quisiéramos desplegar 2 líneas
o 106. Una mejor alternativa sería:

\inputencoding{latin9}\begin{lstlisting}
def nLineas(n):
  if n > 0:
    print()
    nLineas(n-1)
\end{lstlisting}
\inputencoding{utf8} Esta función es similar a \texttt{conteo}; en tanto \texttt{n} sea
mayor a 0, despliega una nueva línea y luego se llama a sí misma para
desplegar \texttt{n-1} líneas adicionales. Así, el número total de
nuevas líneas es \texttt{1 + (n - 1)} que, si usted verifica con álgebra,
resulta ser \texttt{n}.

El proceso por el cual una función se llama a sí misma es la \textbf{recursión},
y se dice que estas funciones son recursivas.

\index{recursión} \index{función!recursiva}

\section{Diagramas de pila para funciones recursivas}

\index{diagrama de pila} \index{marco de función} \index{marco}

En la Sección~\ref{stackdiagram}, usamos un diagrama de pila para
representar el estado de un programa durante un llamado de función.
La misma clase de diagrama puede ayudarnos a interpretar una función
recursiva.

Cada vez que una función se llama, Python crea un nuevo marco de función
que contiene los parámetros y variables locales de ésta. Para una
función recursiva, puede existir más de un marco en la pila al mismo
tiempo.

Este es el diagrama de pila para \texttt{conteo} llamado con \texttt{n
= 3}:

\beforefig \centerline{\includegraphics{illustrations/stack2}}
\afterfig

Como siempre, el tope de la pila es el marco para \texttt{\_\_main\_\_}.
Está vacío porque no creamos ninguna variable en \texttt{\_\_main\_\_}
ni le pasamos parámetros.

Los cuatro marcos de \texttt{conteo} tienen diferentes valores para
el parámetro \texttt{n}. El fondo de la pila, donde \texttt{n=0},
se denomina el \textbf{caso base }. Como no hace una llamada recursiva,
no hay mas marcos.

\index{case base} \index{recursión!caso base}

\section{Recursión infinita}

\index{recursión infinita} \index{recursión!infinita} \index{error de tiempo de ejecución}
\index{error!de tiempo de ejecución} \index{trazado inverso}

Si una función recursiva nunca alcanza un caso base va a hacer llamados
recursivos por siempre y el programa nunca termina. Esto se conoce
como \textbf{recursión infinita}, y, generalmente, no se considera
una buena idea. Aquí hay un programa minimalista con recursión infinita:

\inputencoding{latin9}\begin{lstlisting}
def recurrir():
  recurrir()
\end{lstlisting}
\inputencoding{utf8} En la mayoría de ambientes de programación un programa con recursión
infinita no corre realmente para siempre. Python reporta un mensaje
de error cuando alcanza la máxima profundidad de recursión:
\begin{verbatim}
  File "<stdin>", line 2, in recurrir
  (98 repeticiones omitidas)
  File "<stdin>", line 2, in recurrir
RuntimeError: Maximum recursion depth exceeded
\end{verbatim}
Este trazado inverso es un poco más grande que el que vimos en el
capítulo anterior. Cuando se presenta el error, ¡hay más de 100 marcos
de \texttt{recurrir} en la pila!.

\section{Entrada por el teclado}

Los programas que hemos escrito son un poco toscos ya que no aceptan
entrada de un usuario. Sólo hacen la misma operación todo el tiempo.

Python proporciona funciones primitivas que obtienen entrada desde
el teclado. La más sencilla se llama \texttt{input}. Cuando esta
función se llama el programa se detiene y espera a que el usuario
digite algo. Cuando el usuario digita la tecla Enter o Intro, el programa
retoma la ejecución y \texttt{input} retorna lo que el usuario
digitó como una cadena (\texttt{string}):

\inputencoding{latin9}\begin{lstlisting}
>>> entrada = input()
Que esta esperando?
>>> print(entrada)
Que esta esperando?
\end{lstlisting}
\inputencoding{utf8} Antes de llamar a \texttt{input} es una muy buena idea desplegar
un mensaje diciéndole al usuario qué digitar. Este mensaje se denomina
indicador de entrada (\textbf{prompt} en inglés). Podemos dar un argumento
prompt a \texttt{input}:

\index{prompt}

\inputencoding{latin9}\begin{lstlisting}
>>> nombre = input("Cual es tu nombre? ")
Cual es tu nombre? Arturo, Rey de los Bretones!
>>> print(nombre)
Arturo, Rey de los Bretones!
\end{lstlisting}
\inputencoding{utf8}
Si esperamos que la respuesta sea un entero, podemos usar la función
\texttt{input}:

\inputencoding{latin9}\begin{lstlisting}
prompt = "�Cual es la velocidad de una golondrina sin carga?\n"
velocidad = input(prompt)
\end{lstlisting}
\inputencoding{utf8} Si el usuario digita una cadena de dígitos, éstos se convierten a
un entero que se asigna a \texttt{velocidad}. Desafortunadamente,
si el usuario digita una entrada que no representa un dígito, velocidad
almacenará una cadena de texto.

\inputencoding{latin9}\begin{lstlisting}
>>> prompt = "�Cual es la velocidad una golondrina sin carga?\n"
>>> velocidad = input(prompt)
�Cual es la velocidad una golondrina sin carga?
�Que quiere decir, una golondria Africana o Europea?
Traceback (most recent call last):
File "<stdin>", line 1, in <module>
ValueError: invalid literal for int() with base 10: '�Que quiere decir, una golondria Africana o Europea?'
\end{lstlisting}
\inputencoding{utf8}

Si la función \texttt{int} no puede convertir el valor ingresado, entonces lanzará un error.
%Para evitar este error, es una buena idea usar \texttt{input}
%para obtener una cadena y las funciones de conversión (int, float)
%para transformarla en otros tipos.

\section{Glosario}
\begin{description}
\item [{Operador residuo:}] operador que se denota con un signo porcentaje
(\texttt{\%}), y trabaja sobre enteros produciendo el residuo de un
número al dividirlo por otro.
\item [{Expresión booleana:}] expresión cierta o falsa.
\item [{Operador de comparación:}] uno de los operadores que compara
dos valores: \texttt{==}, \texttt{!=}, \texttt{>}, \texttt{<}, \texttt{>=},
y \texttt{<=}.
\item [{Operador lógico:}] uno de los operadores que combina expresiones
booleanas: \texttt{and}, \texttt{or}, y \texttt{not}.
\item [{Sentencia condicional:}] sentencia que controla el flujo de ejecución
dependiendo de alguna condición.
\item [{Condición:}] la expresión booleana en una sentencia condicional
que determina que rama se ejecuta.
\item [{Sentencia compuesta:}] es la sentencia que comprende una cabecera
y un cuerpo. La cabecera termina con dos puntos seguidos (:). El cuerpo
se sangra o indenta con respecto a la cabecera.
\item [{Bloque:}] grupo de sentencias consecutivas con la misma indentación.
\item [{Cuerpo:}] el bloque, en una sentencia compuesta, que va después
de la cabecera.
\item [{Anidamiento:}] situación en la que hay una estructura dentro de
otra, tal como una sentencia condicional dentro de una rama de otra
sentencia condicional.
\item [{Recursión:}] es el proceso de llamar la función que se está ejecutando
actualmente.
\item [{Caso base:}] corresponde a una rama de la sentencia condicional
dentro de una función recursiva, que no hace un llamado recursivo.
\item [{Recursión infinita:}] función que se llama a sí misma recursivamente
sin alcanzar nunca el caso base. En Python una recursión infinita
eventualmente causa un error en tiempo de ejecución.
\item [{Prompt (indicador de entrada):}] una pista visual que le indica
al usuario que digite alguna información.

\index{operador residuo} \index{expresión booleana} \index{expresión!booleana}
\index{sentencia condicional} \index{sentencia!condicional} \index{condición}
\index{sentencia compuesta} \index{rama} \index{cuerpo} \index{bloque}
\index{anidamiento} \index{recursión} \index{caso base} \index{recursión infinita}
\index{prompt}
\end{description}

\section{Ejercicios}
\begin{enumerate}
\item Evalúe la expresión \verb+7 % 0+. Explique lo que ocurre.
\item Envuelva el código que viene a continuación en una función llamada
\verb+comparar(x, y)+. Llame a la función comparar tres veces: una
en la que el primer argumento sea menor que el segundo, otra en la
que aquel sea mayor que éste, y una tercera en la que los argumentos
sean iguales. \inputencoding{latin9}
\begin{lstlisting}
 if x < y:
    print(x, "es menor que", y)
 elif x > y:
    print(x, "es mayor que", y)
 else:
    print(x, "y", y, "son iguales")
\end{lstlisting}
\inputencoding{utf8}\item Copie este programa en un archivo llamado tabladeverdad.py: \inputencoding{latin9}
\begin{lstlisting}
def tabladeverdad(expresion):
    print(" p      q      %s"  % expresion)
    longitud = len( " p      q      %s"  % expresion)
    print(longitud*"=")

    for p in True, False:
        for q in True, False:
            print("%-7s %-7s %-7s" % (p, q, eval(expresion)))
\end{lstlisting}
\inputencoding{utf8}
\begin{verbatim}
 
\end{verbatim}
Pruébelo con el llamado \verb+tabladeverdad("p or q")+. Ahora ejecútelo
con las siguientes expresiones: 
\begin{enumerate}
\item \texttt{\textquotedbl{}not(p or q)\textquotedbl{}} 
\item \texttt{\textquotedbl{}p and q\textquotedbl{}} 
\item \texttt{\textquotedbl{}not(p and q)\textquotedbl{}} 
\item \texttt{\textquotedbl{}not(p) or not(q)\textquotedbl{}} 
\item \texttt{\textquotedbl{}not(p) and not(q)\textquotedbl{} }
\end{enumerate}
¿Cuales de estas expresiones tienen el mismo valor de verdad (son
lógicamente equivalentes)?
\end{enumerate}


\clearemptydoublepage % condicionales y recursion

\chapter{Funciones fructíferas }

\label{funcReturn}

\section{Valores de retorno}

\index{valor de retorno}

Algunas de las funciones primitivas que hemos usado, como las matemáticas,
entregan resultados. El llamar a estas funciones genera un valor nuevo,
que usualmente asignamos a una variable o usamos como parte de una
expresión.
\begin{lstlisting}
e = math.exp(1.0)
altura = radio * math.sin(angulo)
\end{lstlisting}

Pero hasta ahora ninguna de las funciones que hemos escrito ha retornado
un valor.

En este capítulo vamos a escribir funciones que retornan valores,
los cuales denominamos \textbf{funciones fructíferas}, o provechosas\footnote{En algunos libros de programación, las \textit{funciones} que desarrollamos
en el capítulo anterior se denominan \textit{procedimientos} y las
que veremos en este capítulo sí se denominan \textit{funciones}, ya
que los lenguajes de programación usados para enseñar (como Pascal)
hacían la distinción. Muchos lenguajes de programación vigentes (incluido
Python y C) no diferencian sintácticamente entre procedimientos y
funciones, por eso usamos esta terminología.}. El primer ejemplo es \texttt{area}, que retorna el área de un círculo
dado su radio:
\begin{lstlisting}
import math

def area(radio):
  temp = math.pi * radio**2
  return temp
\end{lstlisting}

Ya nos habíamos topado con la sentencia \texttt{return} antes, pero,
en una función fructífera, la sentencia \texttt{return} incluye un
\textbf{valor de retorno}. Esta sentencia significa: ``Retorne inmediatamente
de esta función y use la siguiente expresión como un valor de retorno.''
La expresión proporcionada puede ser arbitrariamente compleja, así
que podríamos escribir esta función más concisamente:

\begin{lstlisting}
def area(radio):
  return math.pi * radio**2
\end{lstlisting}
 

Por otro lado, las \textbf{variables temporales}, como \texttt{temp},
a menudo permiten depurar los programas más fácilmente.

\index{variable temporal} \index{variable!temporal}

Algunas veces es muy útil tener múltiples sentencias return, ubicadas
en ramas distintas de un condicional:

\begin{lstlisting}
def valorAbsoluto(x):
  if x < 0:
    return -x
  else:
    return x
\end{lstlisting}

Ya que estas sentencias \texttt{return} están en un condicional alternativo,
sólo una será ejecutada. Tan pronto como esto suceda, la función termina
sin ejecutar las sentencias que siguen.

El código que aparece después de la sentencia \texttt{return}, o en
un lugar que el flujo de ejecución nunca puede alcanzar, se denomina
\textbf{código muerto}.

\index{código muerto}

En una función fructífera es una buena idea garantizar que toda ruta
posible de ejecución del programa llegue a una sentencia \texttt{return}.
Por ejemplo:

\begin{lstlisting}
def valorAbsoluto(x):
  if x < 0:
    return -x
  elif x > 0:
    return x
\end{lstlisting}
 Este programa no es correcto porque si \texttt{x} llega a ser 0,
ninguna condición es cierta y la función puede terminar sin alcanzar
una sentencia \texttt{return}. En este caso el valor de retorno que
Python entrega es un valor especial denominado \texttt{None}:

\index{None}
\begin{lstlisting}
>>> print(valorAbsoluto(0))
None
\end{lstlisting}

\section{Desarrollo de programas}

\label{program development} \index{andamiaje}

En este momento usted debería ser capaz de leer funciones completas
y deducir lo que hacen. También, si ha realizado los ejercicios, ya
ha escrito algunas funciones pequeñas. A medida en que usted escriba
funciones más grandes puede empezar a tener una dificultad mayor,
especialmente con los errores semánticos y de tiempo de ejecución.

Para desarrollar programas cada vez más complejos, vamos a sugerir
una técnica denominada \textbf{desarrollo incremental}. El objetivo
del desarrollo incremental es evitar largas sesiones de depuración
mediante la adición y prueba de una pequeña cantidad de código en
cada paso.

\index{desarrollo incremental} \index{desarrollo!incremental}

Como ejemplo, suponga que usted desea hallar la distancia entre dos
puntos dados por las coordenadas $(x_{1},y_{1})$ y $(x_{2},y_{2})$.
Por el teorema de Pitágoras, la distancia se calcula con:

\begin{equation}
distancia=\sqrt{(x_{2}-x_{1})^{2}+(y_{2}-y_{1})^{2}}
\end{equation}
El primer paso es considerar cómo luciría la función \texttt{distancia}
en Python. En otras palabras, ¿cuales son las entradas (parámetros)
y cual es la salida (valor de retorno)?

En este caso, los dos puntos son las entradas, que podemos representar
usando cuatro parámetros. El valor de retorno es la distancia, que
es un valor de punto flotante.

Ya podemos escribir un borrador de la función:

\begin{lstlisting}
def distancia(x1, y1, x2, y2):
  return 0.0
\end{lstlisting}
 Obviamente, esta versión de la función no calcula distancias; siempre
retorna cero. Pero es correcta sintácticamente y puede correr, lo
que implica que la podemos probar antes de que la hagamos más compleja.

Para probar la nueva función la llamamos con valores simples:

\begin{lstlisting}
>>> distancia(1, 2, 4, 6)
0.0
\end{lstlisting}
 Escogemos estos valores de forma que la distancia horizontal sea
3 y la vertical 4; de esta forma el resultado es 5 (la hipotenusa
de un triángulo con medidas 3-4-5). Cuando probamos una función es
fundamental conocer algunas respuestas correctas.

En este punto hemos confirmado que la función está bien sintácticamente,
y que podemos empezar a agregar líneas de código. Después de cada
cambio, probamos la función otra vez. Si hay un error, sabemos dónde
debe estar —en la última línea que agregamos.

Un primer paso lógico en este cómputo es encontrar las diferencias
$x_{2}-x_{1}$ y $y_{2}-y_{1}$. Almacenaremos estos valores en variables
temporales llamadas \texttt{dx} y \texttt{dy} y los imprimiremos.

\begin{lstlisting}
def distancia(x1, y1, x2, y2):
  dx = x2 - x1
  dy = y2 - y1
  print("dx es", dx)
  print("dy es", dy)
  return 0.0
\end{lstlisting}
 Si la función trabaja bien, las salidas deben ser 3 y 4. Si es así,
sabemos que la función está obteniendo los parámetros correctos y
calculando el primer paso correctamente. Si no ocurre ésto, entonces
hay unas pocas líneas para chequear.

Ahora calculamos la suma de los cuadrados de \texttt{dx} y \texttt{dy}:

\begin{lstlisting}
def distancia(x1, y1, x2, y2):
  dx = x2 - x1
  dy = y2 - y1
  discuadrado = dx**2 + dy**2
  print("discuadrado es: ", discuadrado)
  return 0.0
\end{lstlisting}
 Note que hemos eliminado las sentencias \texttt{print} que teníamos
en el paso anterior. Este código se denomina \textbf{andamiaje} porque
es útil para construir el programa pero no hace parte del producto
final.

De nuevo, corremos el programa y chequeamos la salida (que debe ser
25).

Finalmente, si importamos el módulo math, podemos usar la función
\texttt{sqrt} para calcular y retornar el resultado:

\begin{lstlisting}
def distancia(x1, y1, x2, y2):
  dx = x2 - x1
  dy = y2 - y1
  discuadrado = dx**2 + dy**2
  resultado = math.sqrt(discuadrado)
  return resultado
\end{lstlisting}
 Si esto funciona bien, usted ha terminado. Si no, se podría imprimir
el valor de \texttt{resultado} antes de la sentencia return.

Recapitulando, para empezar, usted debería agregar solamente una línea
o dos cada vez.

A medida que gane más experiencia podrá escribir y depurar trozos
mayores. De cualquier forma el proceso de desarrollo incremental puede
evitarle mucho tiempo de depuración.

Los aspectos claves del proceso son:
\begin{enumerate}
\item Empezar con un programa correcto y hacer pequeños cambios incrementales.
Si en cualquier punto hay un error, usted sabrá exactamente donde
está.
\item Use variables temporales para almacenar valores intermedios de manera
que se puedan imprimir y chequear.
\item Ya que el programa esté corriendo, usted puede remover parte del andamiaje
o consolidar múltiples sentencias en expresiones compuestas, pero
sólo si esto no dificulta la lectura del programa.
\end{enumerate}

\section{Composición}

\index{composición} \index{función!composición}

Como usted esperaría, se puede llamar una función fructífera desde
otra. Esta capacidad es la \textbf{composición}.

Como ejemplo vamos a escribir una función que toma dos puntos: el
centro de un círculo y un punto en el perímetro, y que calcule el
área total del círculo.

Asuma que el punto central está almacenado en las variables \texttt{xc}
y \texttt{yc}, y que el punto perimetral está en \texttt{xp} y \texttt{yp}.
El primer paso es encontrar el radio del círculo, que es la distancia
entre los dos puntos. Afortunadamente, hay una función, \texttt{distancia},
que hace eso:

\begin{lstlisting}
radio = distancia(xc, yc, xp, yp)
\end{lstlisting}

El segundo paso es encontrar el área de un círculo con dicho radio
y retornarla:

\begin{lstlisting}
resultado = area(radio)
return resultado
\end{lstlisting}
 Envolviendo todo en una función obtenemos:

\begin{lstlisting}
def area2(xc, yc, xp, yp):
  radio = distancia(xc, yc, xp, yp)
  resultado = area(radio)
  return resultado
\end{lstlisting}
Llamamos a esta función \texttt{area2} para distinguirla de la función
\texttt{area} definida previamente. Solo puede haber una función con
un nombre dado dentro de un módulo.

Las variables temporales \texttt{radio} y \texttt{area} son útiles
para desarrollar y depurar, pero una vez que el programa está funcionando
podemos hacer la función más concisa componiendo las llamadas a funciones:

\begin{lstlisting}
def area2(xc, yc, xp, yp):
  return area(distancia(xc, yc, xp, yp))
\end{lstlisting}

\section{Funciones booleanas}

\label{boolean} \index{función booleana} \index{función booleana}

Las funciones que pueden retornar un valor booleano son convenientes
para ocultar chequeos complicados adentro de funciones. Por ejemplo:

\begin{lstlisting}
def esDivisible(x, y):
  if x % y == 0:
    return True       #  es cierto
  else:
    return False      # es falso
\end{lstlisting}
 El nombre de esta función es \texttt{esDivisible}. Es muy usual nombrar
las funciones booleanas con palabras o frases que suenan como preguntas
de sí o no (que tienen como respuesta un sí o un no). \texttt{esDivisible}
retorna \texttt{True} ó \texttt{False} para indicar si x es divisible
exactamente por y.

Podemos hacerla más concisa tomando ventaja del hecho de que una condición
dentro de una sentencia \texttt{if} es una expresión booleana. Podemos
retornarla directamente, evitando completamente el \texttt{if}:

\begin{lstlisting}
def esDivisible(x, y):
  return x % y == 0
\end{lstlisting}
 Esta sesión muestra la nueva función en acción:

\begin{lstlisting}
>>>   esDivisible(6, 4)
False
>>>   esDivisible(6, 3)
True
\end{lstlisting}

Las funciones booleanas se usan a menudo en las sentencias condicionales:

\begin{lstlisting}
if esDivisible(x, y):
  print("x es divisible por y")
else:
  print("x no es divisible por y")
\end{lstlisting}
 Puede parecer tentador escribir algo como:

\begin{lstlisting}
if esDivisible(x, y) == True:
\end{lstlisting}
 Pero la comparación extra es innecesaria.

\section{Más recursión}

\index{recursión} \index{lenguaje completo} \index{lenguaje!completo}
\index{Turing, Alan} \index{Turing, Tésis de}

Hasta aquí, usted sólo ha aprendido un pequeño subconjunto de Python,
pero podría interesarle saber que este subconjunto es un lenguaje
de programación {\em completo}, lo que quiere decir que cualquier
cosa que pueda ser calculada puede ser expresada en este subconjunto.
Cualquier programa escrito alguna vez puede ser reescrito usando solamente
las características que usted ha aprendido hasta ahora (de hecho,
necesitaría algunos comandos mas para manejar dispositivos como el
teclado, el ratón, los discos, etc., pero eso sería todo).

Demostrar esta afirmación no es un ejercicio trivial y fue logrado
por Alan Turing, uno de los primeros científicos de la computación
(algunos dirían que el era un matemático, pero la mayoría de los científicos
pioneros de la computación eran matemáticos). Esto se conoce como
la Tesis de Turing. Si usted toma un curso de Teoría de la Computación
tendrá la oportunidad de ver la demostración.

Para darle una idea de lo que puede hacer con las herramientas que
ha aprendido, vamos a evaluar unas pocas funciones matemáticas definidas
recursivamente.

Una definición recursiva es similar a una circular, ya que éstas contienen
una referencia al concepto que se pretende definir. Una definición
circular verdadera no es muy útil:
\begin{description}
\item [{frabjuoso:}] un adjetivo usado para describir algo que es frabjuoso.
\end{description}
\index{frabjuoso} \index{definición circular} \index{definición!circular}

Si usted viera dicha definición en el diccionario, quedaría confundido.
Por otro lado, si encontrara la definición de la función factorial
hallaría algo como esto:

\vspace{-0.35in}
 
\begin{eqnarray*}
 &  & 0!=1\\
 &  & n!=n(n-1)!
\end{eqnarray*}
\vspace{-0.25in}

Esta definición dice que el factorial de 0 es 1, y que el factorial
de cualquier otro valor, $n$, es $n$ multiplicado por el factorial
de $n-1$.

Así que $3!$ es 3 veces $2!$, que es 2 veces $1!$, que es 1 vez
$0!$. Juntando todo esto, $3!$ es igual a 3 veces 2 veces 1 vez
1, lo que da 6.

\index{función factorial} \index{función!factorial}

Si usted puede escribir una definición recursiva para algo, usualmente
podrá escribir un programa para evaluarlo. El primer paso es decidir
cuales son los parámetros para esta función. Con un poco de esfuerzo
usted concluiría que \texttt{factorial} recibe un único parámetro:

\begin{lstlisting}
def factorial(n):
\end{lstlisting}

Si el argumento es 0, todo lo que hacemos es retornar 1:

\begin{lstlisting}
def factorial(n):
  if n == 0:
    return 1
\end{lstlisting}
 

Sino, y ésta es la parte interesante, tenemos que hacer una llamada
recursiva para encontrar el factorial de $n-1$ y, entonces, multiplicarlo
por $n$:
\begin{lstlisting}
def factorial(n):
  if n == 0:
    return 1
  else:
    recur = factorial(n-1)
    da = n * recur
    return da
\end{lstlisting}

El flujo de ejecución de este programa es similar al flujo de \texttt{conteo}
en la Sección~\ref{recursion}. Si llamamos a \texttt{factorial}
con el valor 3:

\adjustpage{1}

Como 3 no es 0, tomamos la segunda rama y calculamos el factorial
de \texttt{n-1}...
\begin{quote}
Como 2 no es 0, tomamos la segunda rama y calculamos el factorial
de \texttt{n-1}...

\begin{quote}
Como 1 no es 0, tomamos la segunda rama y calculamos el factorial
de \texttt{n-1}...

\begin{quote}
Como 0 {\em es} 0, tomamos la primera rama y retornamos 1 sin hacer
más llamados recursivos. 
\end{quote}
El valor de retorno (1) se multiplica por $n$, que es 1, y el resultado
se retorna. 
\end{quote}
El valor de retorno (1) se multiplica por $n$, que es 2, y el resultado
se retorna. 
\end{quote}
El valor de retorno (2) se multiplica por $n$, que es 3, y el resultado,
6, se convierte en el valor de retorno del llamado de función que
empezó todo el proceso.

Así queda el diagrama de pila para esta secuencia de llamados de función:

\vspace{0.1in}
 \beforefig \centerline{\includegraphics{illustrations/stack3}}
\afterfig \vspace{0.1in}

Los valores de retorno mostrados se pasan hacia arriba a través de
la pila. En cada marco, el valor de retorno es el valor de \texttt{da},
que es el producto de \texttt{n} y \texttt{recur}.

Observe que en el último marco, las variables locales \texttt{recur}
y \texttt{da} no existen porque la rama que las crea no se ejecutó.

\section{El salto de fe}

\index{recursión} \index{salto de fe}

Seguir el flujo de ejecución es una forma de leer programas, pero
rápidamente puede tornarse algo laberíntico. Una alternativa es lo
que denominamos hacer el ``salto de fe.'' Cuando usted llega a un
llamado de función, en lugar de seguir el flujo de ejecución, se {\em
asume} que la función trabaja correctamente y retorna el valor apropiado.

De hecho, usted ya está haciendo el salto de fe cuando usa las funciones
primitivas. Cuando llama a \texttt{math.cos} ó a \texttt{math.exp},
no está examinando las implementaciones de estas funciones. Usted
sólo asume que están correctas porque los que escribieron el módulo
math son buenos programadores.

Lo mismo se cumple para una de sus propias funciones. Por ejemplo,
en la Sección~\ref{boolean}, escribimos una función llamada \texttt{esDivisible}
que determina si un número es divisible por otro. Una vez que nos
hemos convencido de que esta función es correcta —probándola y examinando
el código—podemos usarla sin mirar el código nuevamente.

Lo mismo vale para los programas recursivos. Cuando usted llega a
una llamada recursiva, en lugar de seguir el flujo de ejecución, debería
asumir que el llamado recursivo funciona (retorna el resultado correcto)
y luego preguntarse, ``Asumiendo que puedo encontrar el factorial
de $n-1$, ¿puedo calcular el factorial de $n$?'' En este caso,
es claro que se puede lograr, multiplicándolo por $n$.

Por supuesto que es un poco raro asumir que la función trabaja correctamente
cuando ni siquiera hemos terminado de escribirla, ¡por eso es que
denominamos a esto el salto de fe!.

\section{Un ejemplo más}

\label{one more example}

En el ejemplo anterior usábamos variables temporales para desplegar
los pasos y depurar el código más fácilmente, pero podríamos ahorrar
unas cuantas líneas:

\begin{lstlisting}
def factorial(n):
  if n == 0:
    return 1
  else:
    return n * factorial(n-1)
\end{lstlisting}
 Desde ahora, vamos a usar esta forma más compacta, pero le recomendamos
que use la forma más explícita mientras desarrolla las funciones.
Cuando estén terminadas y funcionando, con un poco de inspiración
se pueden compactar.

\index{La función de Fibonacci}

Después de \texttt{factorial}, el ejemplo más común de función matemática,
definida recursivamente, es la serie de \texttt{fibonacci}, que tiene
la siguiente definición:

\vspace{-0.25in}
 
\begin{eqnarray*}
 &  & fibonacci(0)=1\\
 &  & fibonacci(1)=1\\
 &  & fibonacci(n)=fibonacci(n-1)+fibonacci(n-2);
\end{eqnarray*}
Traducida a Python, luce así:

\begin{lstlisting}
def fibonacci (n):
  if n == 0 or n == 1:
    return 1
  else:
    return fibonacci(n-1) + fibonacci(n-2)
\end{lstlisting}
 Si usted intenta seguir el flujo de ejecución de fibonacci, incluso
para valores pequeños de $n$, le va a doler la cabeza. Pero, si seguimos
el salto de fe, si asumimos que los dos llamados recursivos funcionan
correctamente, es claro que el resultado correcto es la suma de éstos
dos.

\section{Chequeo de tipos}

\index{chequeo de tipos} \index{chequeo de errores} \index{función factorial}

¿Qué pasa si llamamos a \texttt{factorial} y le pasamos a 1.5 como
argumento?

\begin{lstlisting}
>>> factorial (1.5)
RuntimeError: Maximum recursion depth exceeded
\end{lstlisting}
 Parece recursión infinita. ¿Cómo puede darse? Hay un caso base —cuando
\texttt{n == 0}. El problema reside en que los valores de \texttt{n}
se {\em saltan} al caso base .

\index{recursión infinita} \index{recursión!infinita}

En la primera llamada recursiva el valor de \texttt{n} es 0.5. En
la siguiente es -0.5. Desde allí se hace cada vez más pequeño, pero
nunca será 0.

Tenemos dos opciones, podemos intentar generalizar la función \texttt{factorial}
para que trabaje con números de punto flotante, o podemos chequear
el tipo del parámetro que llega. La primera opción se denomina en
matemática la función gama y está fuera del alcance de este libro.
Optaremos por la segunda.

\index{función gama}

Podemos usar la función \texttt{type} para comparar el tipo del parámetro
al tipo de un valor entero conocido (como 1). Mientras estamos en
eso también aseguraremos que el parámetro sea positivo:

\begin{lstlisting}
def factorial (n):
  if type(n) != type(1):
    print("Factorial solo esta definido para enteros.")
    return -1
  elif n < 0:
    print("Factorial solo esta definido para positivos")
    return -1
  elif n == 0:
    return 1
  else:
    return n * factorial(n-1)
\end{lstlisting}

Ahora tenemos tres casos base. El primero atrapa a los valores que
no son enteros, el segundo a los enteros negativos. En ambos casos
el programa imprime un mensaje de error y retorna un valor especial,
-1, para indicar que algo falló:
\begin{lstlisting}
>>> factorial ("pedro")
Factorial solo esta definido para enteros.
-1
>>> factorial (-2)
Factorial solo esta definido para positivos.
-1
\end{lstlisting}

Si pasamos los dos chequeos, tenemos la garantía de que $n$ es un
número entero positivo, y podemos probar que la recursión termina.

Este programa demuestra el uso de un patrón denominado \textbf{guarda}.
Los primeros dos condicionales actúan como guardas, protegiendo al
código interno de los valores que pueden causar un error. Las guardas
hacen posible demostrar que el código es correcto.

\section{Pruebas unitarias con doctest}

Con funciones fructíferas podemos realizar pruebas unitarias. Por
ejemplo, la función área de un cuadrado puede adornarse con un bloque
de comentarios con triple comillas, que explica su propósito:

\begin{lstlisting}
def area(lado):
    """ Calcula el area de un cuadrado
        Par�metros:
            radio: n�mero
    """
    return lado**2
\end{lstlisting}
 

Si al bloque le agregamos una línea probando el llamado de la función,
seguida del valor de retorno que debe entregar:

\begin{lstlisting}
def area(lado):
    """ Calcula el area de un cuadrado
        Par�metros:
            radio: n�mero
        Pruebas:
        >>> area(1)
        1        
    """
    return lado**2
\end{lstlisting}

Logramos obtener una función que se puede probar en un caso particular.
El módulo doctest de Python permite ejecutar automáticamente los casos
de prueba que tengamos en las funciones agregando al final del guión
su importación y el llamado de la función testmod(), como se ilustra
a continuación con la función area, ahora con cuatro casos de prueba:

\begin{lstlisting}
def area(lado):
    """ Calcula el area de un cuadrado
        Par�metros:
            radio: n�mero
        Pruebas:
        >>> area(1)
        1
        >>> area(2)
        4
        >>> area(4)
        16
        >>> area(10)
        100
        
    """
    return lado**2

if __name__ == '__main__':
    import doctest
    doctest.testmod()
\end{lstlisting}

Si se ejecuta el guión se ejecutarán todas las pruebas unitarias de
todas las funciones, esto nos permite atrapar errores rápidamente
y corregirlos. En Unix/Linux, al ejecutar \verb+python -m doctest -v guión.py+
se logran ejecutar los casos de prueba y visualizar detalladamente
en la pantalla.

\section{Glosario}
\begin{description}
\item [{Función fructífera:}] función que retorna un resultado.
\item [{Valor de retorno:}] el valor que entrega como resultado un llamado
de función.
\item [{Variable temporal:}] variable usada para almacenar un valor intermedio
en un cálculo complejo.
\item [{Código muerto:}] parte de un programa que nunca puede ser ejecutada,
a menudo porque aparece después de una sentencia \texttt{return}.
\item [{None:}] valor especial en Python retornado por las funciones que
no tienen una sentencia return, o que tienen una sentencia return
sin un argumento.
\item [{Desarrollo incremental:}] un plan de desarrollo de programas que
evita la depuración, agregando y probando solo pequeñas porciones
de código en cada momento.
\item [{Andamiaje:}] código que se usa durante el desarrollo de programas,
pero no hace parte de la solución final.
\item [{Guarda:}] una condición que chequea y controla circunstancias que
pueden causar errores.

\index{variable temporal} \index{variable!temporal} \index{valor de retorno}
\index{código muerto} \index{None} \index{desarrollo incremental}
\index{andamiaje} \index{guarda}
\end{description}

\section{Ejercicios}
\begin{enumerate}
\item Escriba la función \verb+comparar(a,b)+ que devuelva 1 si $a<b$,
0 si $a=b$, y -1 si $a>b$
\item Tome la solución del último ejercicio del capítulo anterior y conviértala
en una función que retorne la nota definitiva de su curso de programación.
\item Calcule en una función el área de un disco, teniendo como entrada
el radio menor y el radio mayor.
\item Escriba la función \verb+pendiente(x1, y1, x2, y2)+ que calcule la
pendiente de una línea que pasa por los puntos $(x_{1},y_{1})$ y
$(x_{2},y_{2})$.
\item Convierta las funciones de los capítulos pasados, y que se puedan
transformar, a fructíferas.
\item Convierta las funciones que obtuvo en el punto anterior agregando
guardas para protegerlas de las situaciones en que reciben argumentos
de un tipo de dato que no pueden manipular.
\item Agregue pruebas unitarias a las funciones que obtuvo en el punto anterior.
\end{enumerate}


\clearemptydoublepage % funciones fructíferas

\chapter{Iteración }

\index{iteración}

\section{Asignación múltiple}

\index{asignación} \index{sentencia!asignación} \index{asignación múltiple}

Puede que usted ya haya descubierto que es posible realizar más de
una asignación a la misma variable. Una nueva asignación hace que
la variable existente se refiera a un nuevo valor (y deje de referirse
al viejo valor).

\begin{pythoncode}
pedro = 5
print(pedro)
pedro = 7
print(pedro)
\end{pythoncode}
 La salida de este programa es:

\texttt{5 }

\texttt{7} 

porque la primera vez que \texttt{pedro} se imprime, tiene el valor
5, y la segunda vez tiene el valor 7.

Aquí se puede ver cómo luce la \textbf{asignación múltiple } en un
diagrama de estado:

\beforefig \centerline{\includegraphics{illustrations/assign2}}
\afterfig

Con asignación múltiple es muy importante distinguir entre una asignación
y una igualdad. Como Python usa el signo igual (\texttt{=}) para la
asignación podemos caer en la tentación de interpretar a una sentencia
como \texttt{a = b} como si fuera una igualdad. ¡Y no lo es!

Primero, la igualdad es conmutativa y la asignación no lo es. Por
ejemplo, en la matemática si $a=7$ entonces $7=a$. Pero en Python,
la sentencia \texttt{a = 7} es legal aunque \texttt{7 = a} no lo es.

Además, en matemática, una sentencia de igualdad \textit{siempre}
es cierta. Si $a=b$ ahora, entonces $a$ siempre será igual a $b$.
En Python, una sentencia de asignación puede lograr que dos variables
sean iguales pero sólo por un tiempo determinado:

\begin{pythoncode}
a = 5
b = a    # a y b ahora son iguales
a = 3    # a y b no son iguales ahora
\end{pythoncode}
 La tercera línea cambia el valor de \texttt{a} pero no cambia el
valor de \texttt{b}, así que ya no serán iguales. En algunos lenguajes
de programación se usa un signo diferente para la asignación como
\texttt{<-} o \texttt{:=} para evitar la confusión.

Aunque la asignación múltiple puede ser útil se debe usar con precaución.
Si los valores de las variables cambian frecuentemente se puede dificultar
la lectura y la depuración del código.

\section{La sentencia \texttt{while (mientras)} }

\index{sentencia while} \index{sentencia!while} \index{ciclo!while}
\index{iteración}

Los computadores se usan a menudo para automatizar tareas repetitivas.
Esto es algo que los computadores hacen bien y los seres humanos hacemos
mal.

Hemos visto dos programas, \texttt{nLineas} y \texttt{conteo}, que
usan la recursión para lograr repetir, lo que también se denomina
\textbf{iteración}. Como la iteración es tan común, Python proporciona
varios elementos para facilitarla. El primero que veremos es la sentencia
\texttt{while}.

Aquí presentamos a la función \texttt{conteo} usando una sentencia
\texttt{while}:

\begin{pythoncode}
def conteo(n):
  while n > 0:
    print(n)
    n = n-1
  print("Despegue!")
\end{pythoncode}
 Como eliminamos el llamado recursivo, esta función deja de ser recursiva.

La sentencia \texttt{while} se puede leer como en el lenguaje natural.
Quiere decir, ``Mientras \texttt{n} sea mayor que 0, continúe desplegando
el valor de \texttt{n} y reduciendo el valor de \texttt{n} en 1. Cuando
llegue a 0, despliegue la cadena \texttt{Despegue!}''.

Más formalmente, el flujo de ejecución de una sentencia \texttt{while}
luce así:
\begin{enumerate}
\item Evalúa la condición, resultando en \texttt{False} (falso) o \texttt{True}
(cierto).
\item Si la condición es falsa (False), se sale de la sentencia \texttt{while}
y continúa la ejecución con la siguiente sentencia (afuera del while).
\item Si la condición es cierta (True), ejecute cada una de las sentencias
en el cuerpo y regrese al paso 1.
\end{enumerate}
El cuerpo comprende todas las sentencias bajo la cabecera que tienen
la misma indentación.

Este flujo se denomina \textbf{ciclo} porque el tercer paso da la
vuelta hacia el primero. Note que si la condición es falsa la primera
vez que se entra al while, las sentencias internas nunca se ejecutan.

\index{condición} \index{ciclo} \index{ciclo!cuerpo} \index{cuerpo!ciclo}
\index{ciclo infinito} \index{ciclo!infinito}

El cuerpo del ciclo debería cambiar el valor de una o más variables,
de forma que la condición se haga falsa en algún momento y el ciclo
termine. De otra forma, el ciclo se repetirá para siempre, obteniendo
un \textbf{ciclo infinito}. Una broma común entre los científicos
de la computación es interpretar las instrucciones de los champús,
``Aplique champú, aplique rinse, repita,'' como un ciclo infinito.

En el caso de \texttt{conteo}, podemos probar que el ciclo termina
porque sabemos que el valor de \texttt{n} es finito, y podemos ver
que va haciéndose más pequeño cada vez que el while itera (da la vuelta),
así que eventualmente llegaremos a 0. En otros casos esto no es tan
fácil de asegurar:

\begin{pythoncode}
def secuencia(n):
  while n != 1:
    print(n),
    if n%2 == 0:        # n es par
      n = n/2
    else:               # n es impar
      n = n*3+1
\end{pythoncode}
 La condición para este ciclo es \texttt{n != 1}, así que se repetirá
hasta que \texttt{n} sea \texttt{1}, lo que hará que la condición
sea falsa.

En cada iteración del ciclo while el programa despliega el valor de
\texttt{n} y luego chequea si es par o impar. Si es par, el valor
de \texttt{n} se divide por 2. Si es impar el valor se reemplaza por
\texttt{n{*}3+1}. Si el valor inicial (del argumento) es 3, la secuencia
que resulta es 3, 10, 5, 16, 8, 4, 2, 1.

Como \texttt{n} aumenta algunas veces y otras disminuye, no hay una
demostración obvia de que \texttt{n} llegará a ser 1, o de que el
programa termina. Para algunos valores particulares de \texttt{n}
podemos demostrar la terminación. Por ejemplo, si el valor inicial
es una potencia de dos, entonces el valor de \texttt{n} será par en
cada iteración del ciclo hasta llegar a 1. El ejemplo anterior termina
con una secuencia así que empieza con 16.

Dejando los valores particulares de lado, la interesante pregunta
que nos planteamos es si podemos demostrar que este programa termina
para {\em todos} los valores de \texttt{n}. Hasta ahora, ¡nadie
ha sido capaz de probarlo {\em o} refutarlo!.

\section{Tablas}

\label{tables} \index{tabla} \index{logaritmo}

Una gama de aplicaciones donde los ciclos se destacan es la de generación
de información tabular. Antes de que los computadores existieran la
gente tenía que calcular logaritmos, senos, cosenos y otras funciones
matemáticas a mano. Para facilitar la tarea, los libros matemáticos
incluían largas tablas con los valores de dichas funciones. La creación
de las tablas era un proceso lento y aburridor, y tendían a quedar
con muchos errores.

Cuando los computadores entraron en escena, una de las reacciones
iniciales fue ``Esto es maravilloso! Podemos usar los computadores
para generar las tablas, de forma que no habrían errores''. Eso resultó
(casi) cierto, pero poco prospectivo. Poco después los computadores
y las calculadoras se hicieron tan ubicuos que las tablas se hicieron
obsoletas.

Bueno, casi. Para algunas operaciones los computadores usan tablas
de valores para obtener una respuesta aproximada y luego hacer mas
cálculos para mejorar la aproximación. En algunos casos, se han encontrado
errores en las tablas subyacentes, el más famoso ha sido el de la
tabla para realizar la división en punto flotante en los procesadores
Pentium de la compañía Intel.

\index{Intel} \index{Pentium}

Aunque una tabla logarítmica no es tan útil como en el pasado, todavía
sirve como un buen ejemplo de iteración. El siguiente programa despliega
una secuencia de valores en la columna izquierda y sus logaritmos
en la columna derecha: 
\begin{pythoncode}
import math
x = 1.0
while x < 10.0:
  print(x, '\t', math.log(x))
  x = x + 1.0
\end{pythoncode}

La cadena \verb+'\t'+ representa un carácter \textbf{tab} (tabulador).

A medida que los caracteres y las cadenas se despliegan en la pantalla
un marcador invisible denominado \textbf{cursor} lleva pista de dónde
va a ir el siguiente carácter. Después de una sentencia \texttt{print},
el cursor va al comienzo de la siguiente línea.

El carácter tabulador mueve el cursor hacia la derecha hasta que alcanza
un punto de parada (cada cierto número de espacios, que pueden variar
de sistema a sistema). Los tabuladores son útiles para alinear columnas
de texto, como la salida del anterior programa:
\begin{pythoncode}
1.0     0.0
2.0     0.69314718056
3.0     1.09861228867
4.0     1.38629436112
5.0     1.60943791243
6.0     1.79175946923
7.0     1.94591014906
8.0     2.07944154168
9.0     2.19722457734
\end{pythoncode}
Si estos valores parecen extraños, recuerde que la función \texttt{log}
usa la base \texttt{e}. Ya que las potencias de dos son importantes
en la ciencias de la computación, a menudo deseamos calcular logaritmos
en base 2. Para este fin podemos usar la siguiente fórmula:

\begin{equation}
\log_{2}x=\frac{log_{e}x}{log_{e}2}
\end{equation}

Cambiando la salida del ciclo a:
\begin{pythoncode}
print(x, '\t',  math.log(x)/math.log(2.0))
\end{pythoncode}
\begin{pythoncode}
   resulta en:
1.0     0.0
2.0     1.0
3.0     1.58496250072
4.0     2.0
5.0     2.32192809489
6.0     2.58496250072
7.0     2.80735492206
8.0     3.0
9.0     3.16992500144
\end{pythoncode}
Podemos ver que 1, 2, 4, y 8 son potencias de dos porque sus logaritmos
en base 2 son números enteros. Si deseamos calcular el logaritmo de
más potencias de dos podemos modificar el programa así:
\begin{pythoncode}
x = 1.0
while x < 100.0:
  print(x, '\t', math.log(x)/math.log(2.0))
  x = x * 2.0
\end{pythoncode}

Ahora, en lugar de agregar algo a \texttt{x} en cada iteración del
ciclo, produciendo una serie aritmética, multiplicamos a \texttt{x}
por algo constante, produciendo una serie geométrica. El resultado
es:

\index{serie aritmética} \index{serie geométrica}
\begin{pythoncode}
1.0     0.0
2.0     1.0
4.0     2.0
8.0     3.0
16.0    4.0
32.0    5.0
64.0    6.0
\end{pythoncode}
Gracias a los caracteres tabuladores entre las columnas, la posición
de la segunda columna no depende del número de dígitos en la primera.

Puede que las tablas de logaritmos no sirvan en nuestros días, ¡pero
para los científicos de la computación saber las potencias de dos
sí es muy importante!.

\index{secuencia de escape}

El carácter diagonal invertido (backslash) \verb+'\'+ indica el comienzo
de una \textbf{secuencia de escape}. Estas secuencias se utilizan
para representar caracteres invisibles como tabuladores y nuevas líneas.
La secuencia \verb+'\n'+ representa una nueva línea.

Una secuencia de escape puede empezar en cualquier posición de una
cadena; en el ejemplo anterior, la secuencia de escape tabuladora
es toda la cadena.

¿Cómo cree que se representa un diagonal invertido en una cadena?

\section{Tablas de dos dimensiones}

\index{tabla!bidimensional}

Una tabla de dos dimensiones es una en la que los valores se leen
en la intersección de una fila y una columna. Una tabla de multiplicación
es un ejemplo familiar. Digamos que usted desea imprimir una tabla
de multiplicación para los valores del 1 al 6.

Una buena forma de empezar es escribir un ciclo que imprima los múltiplos
de 2, en una sola línea:
\begin{pythoncode}
i = 1
while i <= 6:
  print(2*i, '   ',end=' ')
  i = i + 1

print()
\end{pythoncode}

La primera línea inicializa una variable llamada \texttt{i}, que actúa
como un contador o \textbf{variable de ciclo}. A medida que se ejecuta
el ciclo, el valor de \texttt{i} se incrementa de 1 a 6. Cuando \texttt{i}
es 7, el ciclo termina. En cada iteración del ciclo despliega el valor
\texttt{2{*}i}, seguido de tres espacios.

El uso de \texttt{end=' '} en la función \texttt{print} cambia el
último caracter que se imprime, que por defecto es el de fin de línea.
Después de que el ciclo termina la segunda sentencia \texttt{print}
comienza una nueva línea.

La salida del programa es:
\begin{pythoncode}
2      4      6      8      10     12
\end{pythoncode}
Hasta aquí vamos bien. El paso siguiente es \textbf{encapsular} y
\textbf{generalizar}.

\section{Encapsulamiento y generalización}

\label{encapsulation}

\index{encapsulamiento} \index{generalización} \index{desarrollo de programas!encapsulamiento}
\index{desarrollo de programas!generalización}

Encapsular es el proceso de envolver un trozo de código en una función,
permitiendo tomar ventaja de todas los beneficios que esto conlleva.
Usted ha visto dos ejemplos de encapsulamiento: \texttt{imprimirParidad}
en la Sección~\ref{alternative execution}; y \texttt{esDivisible}
en la Sección~\ref{boolean}.

La generalización es tomar algo específico, tal como imprimir los
múltiplos de 2, y convertirlo en algo más general, tal como imprimir
los múltiplos de cualquier entero.

Esta función encapsula el ciclo anterior y lo generaliza para imprimir
múltiplos de un parámetro \texttt{n}:
\begin{pythoncode}
def imprimaMultiplos(n):
  i = 1
  while i <= 6:
    print(n*i, end='\t')
    i = i + 1
  print()
\end{pythoncode}

Para encapsular el ciclo, todo lo que agregamos fue la primera línea
que declara el nombre de la función y la lista de parámetros. Para
generalizar, lo que hicimos fue reemplazar el valor 2 con el parámetro
\texttt{n}.

Si llamamos a esta función con el argumento 2, obtenemos la misma
salida de la sección anterior. Con el argumento 3, la salida es:
\begin{pythoncode}
3      6      9      12     15     18
\end{pythoncode}
Con el argumento 4, la salida es:
\begin{pythoncode}
4      8      12     16     20     24
\end{pythoncode}
Ahora, usted probablemente imagine cómo imprimir una tabla de multiplicación
—llamando a \texttt{imprimaMultiplos} repetidamente con diferentes
argumentos. De hecho, podemos usar otro ciclo:

\begin{pythoncode}
i = 1
while i <= 6:
  imprimaMultiplos(i)
  i = i + 1
\end{pythoncode}

Observe lo similar que es este ciclo al ciclo interno de la función
\texttt{imprimaMultiplos}. Todo lo que hicimos fue reemplazar la sentencia
\texttt{print} con un llamado de función.

La salida de este programa es una tabla de multiplicación:
\begin{pythoncode}
1      2      3      4      5      6
2      4      6      8      10     12
3      6      9      12     15     18
4      8      12     16     20     24
5      10     15     20     25     30
6      12     18     24     30     36
\end{pythoncode}

\section{Más encapsulamiento}

Para demostrar el encapsulamiento otra vez, tomemos el código de la
sección \ref{encapsulation} y envolvámoslo en una función:
\begin{pythoncode}
def imprimirTablaMultiplicacion():
  i = 1
  while i <= 6:
    imprimaMultiplos(i)
    i = i + 1
\end{pythoncode}

Este proceso es un \textbf{plan de desarrollo} común. Desarrollamos
código escribiendo líneas afuera de cualquier función o digitándolas
en el intérprete. Cuando funcionan, las ponemos dentro de una función.

Este plan de desarrollo es particularmente útil si usted no sabe,
cuando está empezando a programar, cómo dividir el programa en funciones.
Este enfoque permite diseñar a medida que se escribe código.

\section{Variables locales}

\index{variable!local} \index{variable local}

Puede estar preguntándose cómo usamos la misma variable, \texttt{i}
en las dos funciones \texttt{imprimaMultiplos} y \texttt{imprimaTabla}.
¿No ocurren problemas cuando una de ellas cambia el valor de la variable?

La respuesta es no, porque la \texttt{i} en \texttt{imprimaMultiplos}
y la \texttt{i} en \texttt{imprimaTabla} {\em no} son la misma
variable.

Las variables creadas dentro de una definición de función son locales;
no se puede acceder a ellas fuera de su función ``casa''. Esto quiere
decir que usted tiene la libertad de tener múltiples variables con
el mismo nombre en tanto no pertenezcan a la misma función.

El diagrama de pila para este programa muestra que las dos variables
llamadas \texttt{i} no son la misma. Se pueden referir a valores diferentes,
y cambiar una de ellas no altera la otra.

\beforefig \centerline{\includegraphics{illustrations/stack4}}
\afterfig

El valor de \texttt{i} en \texttt{imprimaTabla} va de 1 a 6. En el
diagrama va por el valor 3. En la próxima iteración del ciclo será
4. En cada iteración, \texttt{imprimaTabla} llama a \texttt{imprimaMultiplos}
con el valor actual de \texttt{i} como argumento. Ese valor se le
asigna al parámetro \texttt{n}.

Adentro de \texttt{imprimaMultiplos} el valor de \texttt{i} va de
1 a 6. En el diagrama, es 2. Cambiar esta variable no tiene efecto
en el valor de \texttt{i} en \texttt{imprimaTabla}.

Es legal y muy común tener diferentes variables locales con el mismo
nombre. Particularmente, nombres como \texttt{i} y \texttt{j} se usan
frecuentemente como variables de ciclo. Si usted evita usarlas en
una función porque ya las usó en otra, probablemente dificultará la
lectura del programa.

%\adjustpage{-2}%\pagebreak

\section{Mas generalización}

Como otro ejemplo de generalización, imagine que le piden un programa
que imprima una tabla de multiplicación de cualquier tamaño; no sólo
la tabla de seis por seis. Podría agregarse un parámetro a \texttt{imprimaTabla}:
\begin{pythoncode}
def imprimaTabla(tope):
  i = 1
  while i <= tope:
    imprimaMultiplos(i)
    i = i + 1
\end{pythoncode}

Reemplazamos el valor 6 con el parámetro \texttt{tope}. Si llamamos
a \texttt{imprimaTabla} con el argumento 7, despliega:
\begin{pythoncode}
1      2      3      4      5      6
2      4      6      8      10     12
3      6      9      12     15     18
4      8      12     16     20     24
5      10     15     20     25     30
6      12     18     24     30     36
7      14     21     28     35     42
\end{pythoncode}
Esto está bien, pero quizás deseamos que la tabla sea cuadrada—con
el mismo número de filas y columnas. Para lograrlo, añadimos un parámetro
a imprimaMultiplos que especifique cuántas columnas debe tener la
tabla.

Sólo por confundir, vamos a nombrarlo \texttt{tope}, demostrando que
diferentes funciones pueden tener parámetros con el mismo nombre (igual
que las variables locales). Aquí está el programa completo:
\begin{pythoncode}
def imprimaMultiplos(n, tope):
  i = 1
  while i <= tope:
    print(n*i, end='\t')
    i = i + 1
  print()

def imprimaTabla(tope):
  i = 1
  while i <= tope:
    imprimaMultiplos(i, tope)
    i = i + 1
\end{pythoncode}

Note que cuando agregamos el nuevo parámetro tuvimos que cambiar la
primera línea de la función (la cabecera), y también el lugar donde
la función se llama en \texttt{imprimaTabla}.

Como se esperaba, este programa genera una tabla cuadrada de siete-por-siete:
\begin{pythoncode}
1      2      3      4      5      6      7
2      4      6      8      10     12     14
3      6      9      12     15     18     21
4      8      12     16     20     24     28
5      10     15     20     25     30     35
6      12     18     24     30     36     42
7      14     21     28     35     42     49
\end{pythoncode}
Cuando se generaliza una función adecuadamente, a menudo se obtiene
un programa con capacidades que no se habían planeado. Por ejemplo,
podríamos aprovechar el hecho de que $ab=ba$, que causa que todas
las entradas de la tabla aparezcan dos veces para ahorrar tinta imprimiendo
solamente la mitad de la tabla. Para lograrlo sólo se necesita cambiar
una línea de \texttt{imprimaTabla}. Cambiamos:
\begin{pythoncode}
    imprimaMultiplos(i, tope)
\end{pythoncode}
por
\begin{pythoncode}
    imprimaMultiplos(i, i)
\end{pythoncode}
y se obtiene:
\begin{pythoncode}
1
2      4
3      6      9
4      8      12     16
5      10     15     20     25
6      12     18     24     30     36
7      14     21     28     35     42     49
\end{pythoncode}

\section{Funciones}

\index{función}

Ya hemos mencionado los ``beneficios de las funciones.'' Usted se
estará preguntado cuales son estos beneficios. Aquí hay algunos:
\begin{itemize}
\item Nombrar una secuencia de sentencias aumenta la legibilidad de los
programas y los hace más fáciles de depurar.
\item Dividir un programa grande en funciones permite separar partes de
éste, depurarlas aisladamente, y luego componerlas en un todo coherente.
\item Las funciones facilitan la recursión y la iteración.
\item Las funciones bien diseñadas resultan ser útiles para muchos programas.
Una vez que usted escribe y depura una, se puede reutilizar.
\end{itemize}

\section{Glosario}
\begin{description}
\item [{Asignación múltiple:}] realizar más de una asignación a una
misma variable durante la ejecución de un programa.
\item [{Iteración:}] la ejecución repetida de un grupo de sentencias, ya
sea en una función recursiva o en un ciclo.
\item [{Ciclo:}] una sentencia o grupo de sentencias que se ejecuta repetidamente
hasta que una condición de terminación se cumple.
\item [{Ciclo infinito:}] ciclo en el que la condición de terminación
nunca se cumple.
\item [{Cuerpo:}] las sentencias adentro de un ciclo.
\item [{Variable de ciclo:}] variable que se usa como parte de la condición
de terminación de un ciclo.
\item [{Tab:}] (tabulador) carácter especial que causa el movimiento del
cursor al siguiente punto de parada en la línea actual.
\item [{Nueva línea:}] carácter que causa que el cursor se mueva al principio
de la siguiente línea.
\item [{Cursor:}] marcador invisible que lleva pista de dónde se va a imprimir
el siguiente carácter.
\item [{Secuencia de escape:}] carácter de escape ($\backslash$) seguido
por uno o más caracteres imprimibles que se usan para denotar un carácter
no imprimible.
\item [{Encapsular:}] dividir un programa grande y complejo en componentes
(como funciones) y aislarlos entre sí (usando variables locales, por
ejemplo).
\item [{Generalizar:}] reemplazar algo innecesariamente específico (como
un valor constante) con algo general más apropiado (como una variable
o parámetro). La generalización aumenta la versatilidad del código,
lo hace más reutilizable y en algunos casos facilita su escritura.
\item [{Plan de desarrollo:}] proceso para desarrollar un programa. En
este capítulo demostramos un estilo de desarrollo basado en escribir
código que realiza cómputos simples y específicos, para luego encapsularlo
y generalizarlo.

\index{asignación múltiple} \index{asignación!múltiple} \index{iteración}
\index{ciclo!cuerpo} \index{ciclo} \index{ciclo infinito} \index{secuencia de escape}
\index{cursor} \index{tab} \index{nueva línea} \index{variable de ciclo}
\index{encapsular} \index{generalizar} \index{plan de desarrollo}
\index{programas!desarrollo de}
\end{description}

\section{Ejercicios}
\begin{enumerate}
\item Siga la ejecución de la última versión de \texttt{imprimaTabla} y
descifre cómo trabaja. 
\item Reescriba la función \texttt{nLineas} de la Sección \ref{recursion},
usando iteración en vez de recursión. 
\item Escriba una sola cadena que

\beforeverb 
\begin{pythoncode}
produzca
        esta
                salida.
\end{pythoncode}
\afterverb
\item Escriba un programa que despliegue las potencias de 2 hasta 65,536
(esto es $2^{16}$).
\item Escriba una función \verb+muestra_numeros_triangulares(n)+ que muestre
los primeros n números triangulares. Una llamada a \verb+muestra_numeros_triangulares(5)+
produciría la siguiente salida: \beforeverb 

\begin{pythoncode}
  1       1
  2       3
  3       6
  4       10
  5       15
  
\end{pythoncode}
\afterverb
\item Muchos cálculos de funciones matemáticas se realizan con series infinitas,
por ejemplo:

$\ln{x}=\sum_{n=1}^{\infty}\frac{1}{{n}}\left(\frac{x-1}{x}\right)^{n}=\left(\frac{x-1}{x}\right)+\frac{1}{2}\left(\frac{x-1}{x}\right)^{2}+\frac{1}{3}\left(\frac{x-1}{x}\right)^{3}+\cdots$

que son aproximadas fijando un valor $n$ tal que la precisión, dada
por el número de cifras significativas correctas del valor calculado,
sea buena. Escriba un programa que calcule el logaritmo natural de
un número dado basado en la formula anterior. Para esto debe probar
con varios valores de $n$ hasta que obtenga un buen número de cifras
significativas correctas comparando el resultado de su programa con
el de \verb+math.log+.
\item Busque una formula correcta para calcular el seno de un ángulo y escriba
un programa para calcular este valor basado en la formula, como en
el punto anterior. Compare con \verb+math.sin+.
\item Compare los valores de $n$ que obtuvo para los puntos 6 y 7. Explique
si encuentra diferencias.
\item Escriba una función, \verb+es_primo+, que tome un solo argumento
entero y devuelva \verb+True+ cuando el argumento es un número primo
y \verb+False+ en caso contrario.
\item Agregue el código para chequeo de tipo de datos y para las prueba
unitarias a todas las funciones desarrolladas previamente.
\end{enumerate}
\clearemptydoublepage % iteracion

\chapter{Cadenas }

\label{strings}

\section{Un tipo de dato compuesto}

\index{tipo de dato compuesto} \index{tipo de dato!compuesto}

Hasta aquí hemos visto tres tipos de datos: \texttt{int}, \texttt{float}
y \texttt{string}. Las cadenas son cualitativamente diferentes de
los otros dos tipos porque están compuestas de piezas más pequeñas—los
caracteres.

\index{carácter}

Los tipos que comprenden piezas más pequeñas se denominan \textbf{tipos
de datos compuestos}. Dependiendo de lo que hagamos podemos tratar
un tipo compuesto como unidad o podemos acceder a sus partes. Esta
ambigüedad es provechosa.

\index{operador corchete} \index{operador!corchete}

El operador corchete selecciona un carácter de una cadena.

\begin{pyconcode}
>>> fruta = "banano"
>>> letra = fruta[1]
>>> print(letra)	
\end{pyconcode}

La expresión \texttt{fruta{[}1{]}} selecciona el carácter número 1
de \texttt{fruta}. La variable \texttt{letra} se refiere al resultado.
Cuando desplegamos \texttt{letra}, obtenemos una pequeña sorpresa:
\begin{pythoncode}
a
\end{pythoncode}
La primera letra de ``banano'' no es \texttt{a}. ¡A menos que usted
sea un científico de la computación! Por razones perversas, los científicos
de la computación empiezan a contar desde cero. La letra número 0
de \texttt{``banano''} es \texttt{b}. La letra 1 es a, y la letra
2 es n.

Si usted desea la primera letra de una cadena se pone 0, o cualquier
expresión con el valor 0, dentro de los corchetes:
\begin{pyconcode}
>>> letra = fruta[0]
>>> print(letra)
b
\end{pyconcode}

La expresión en corchetes se denomina \textbf{índice}. Un índice especifica
un miembro de un conjunto ordenado, en este caso el conjunto de caracteres
de la cadena. El índice {\em indica} cual elemento desea usted,
por eso se llama así. Puede ser cualquier expresión entera.

\index{índice}

\section{Longitud}

\index{cadena!longitud} \index{error de tiempo de ejecución}

La función \texttt{len} retorna el número de caracteres en una cadena:

\begin{pyconcode}
>>> fruta = "banano"
>>> len(fruta)
6
\end{pyconcode}
 Para acceder a la última letra de una cadena usted podría caer en
algo como esto:
\begin{pythoncode}
longitud = len(fruta)
ultima = fruta[longitud]       # ERROR!
\end{pythoncode}

Y no funcionará. Causa un error en tiempo de ejecución, \texttt{IndexError: string
index out of range}. La razón yace en que no hay una letra 6 en ``banano''.
Como empezamos a contar desde cero, las seis letras se numeran de
0 a 5. En general, para obtener la última letra, tenemos que restar
1 a la \texttt{longitud}:

\index{error en tiempo de ejecución}

\begin{pythoncode}
longitud = len(fruta)
ultima = fruta[longitud-1]
\end{pythoncode}

Alternativamente, podemos usar índices negativos, que cuentan hacia
atrás desde el fin de la cadena. La expresión \texttt{fruta{[}-1{]}}
retorna la última letra \texttt{fruta{[}-2{]}} retorna la penúltima,
y así sucesivamente.

\index{índice!negativo}

\section{Recorridos en cadenas y el ciclo \texttt{for}}

\label{for} 

\index{recorridos} \index{ciclo!recorrido} \index{ciclo for} \index{ciclo!ciclo for}

Muchos cálculos implican procesar una cadena carácter por carácter.
A menudo empiezan al inicio, seleccionan cada carácter en cada paso,
le hacen algo y continúan hasta el final. Este patrón de procesamiento
se denomina \textbf{recorrido}. Hay una forma de realizarlo con la
sentencia \texttt{while}:
\begin{pythoncode}
indice = 0
while indice < len(fruta):
  letra = fruta[indice]
  print(letra)
  indice = indice + 1
\end{pythoncode}

Este ciclo recorre la cadena y despliega cada letra en una línea independiente.
La condición del ciclo es \texttt{indice < len(fruta)}, así que cuando
\texttt{indice} se hace igual a la longitud de la cadena, la condición
es falsa, y el cuerpo del ciclo no se ejecuta. El último carácter
accedido es el que tiene el índice \texttt{len(fruta)-1}, es decir,
el último.

Usar un índice para recorrer un conjunto de valores es tan común que
Python tiene una sintaxis alternativa más simple—el ciclo \texttt{for}
:

\begin{pythoncode}
for caracter in fruta:
   print(caracter)
\end{pythoncode}

Cada vez que el ciclo itera, el próximo carácter en la cadena se asigna
a la variable \texttt{caracter}. El ciclo continúa hasta que no quedan
más caracteres.

\index{concatenación} \index{lexicográfico} \index{McCloskey, Robert}
\index{Make Way for Ducklings@{\em Make Way for Ducklings}}

El siguiente ejemplo muestra cómo usar la concatenación y un ciclo
\texttt{for} para generar una serie en orden lexicográfico. Lexicográfico
se refiere a una lista en la que los elementos aparecen en orden alfabético.
Por ejemplo, en el libro {\em Make Way for Ducklings} de Robert
McCloskey, los nombres de los patos son Jack, Kack, Lack, Mack, Nack,
Ouack, Pack, and Quack. Este ciclo los despliega en orden:
\begin{pythoncode}
prefijos = "JKLMNOPQ"
sufijo = "ack"

for letra in prefijos:
  print(letra + sufijo)
\end{pythoncode}

La salida de este programa es:
\begin{pythoncode}
Jack
Kack
Lack
Mack
Nack
Oack
Pack
Qack
\end{pythoncode}

Por supuesto que hay un error, ya que ``Ouack'' y ``Quack'' no
están bien deletreados.

\section{Segmentos de cadenas }

\label{slice} \index{segmento} \index{cadena!segmento}

Una porción de una cadena de caracteres se denomina \textbf{segmento}.
Seleccionar un segmento es similar a seleccionar un carácter:
\begin{pyconcode}
>>> s = "Pedro, Pablo, y Maria"
>>> print(s[0:5])
Pedro
>>> print(s[7:12])
Pablo
>>> print(s[16:21])
Maria
\end{pyconcode}

El operador \texttt{{[}n:m{]}} retorna la parte de la cadena que va
desde el carácter n hasta el m, incluyendo el primero y excluyendo
el último. Este comportamiento es contraintuitivo, tiene más sentido
si se imagina que los índices van {\em antes} de los caracteres,
como en el siguiente diagrama:

\beforefig \centerline{\includegraphics{illustrations/banana}}
\afterfig

Si usted omite el primer índice (antes de los puntos seguidos), el
segmento comienza en el inicio de la cadena. Si se omite el segundo
índice, el segmento va hasta el final. Entonces:
\begin{pyconcode}
>>> fruta  = "banano"
>>> fruta[:3]
'ban'
>>> f[3:]
'ano'
\end{pyconcode}
 ¿Que cree que significa \texttt{s{[}:{]}}?

\section{Comparación de cadenas}

\index{comparación de cadenas} \index{comparación!de cadenas}

El operador de comparación funciona con cadenas. Para ver si dos cadenas
son iguales:
\begin{pythoncode}
if palabra == "banana":
  print("No hay bananas!")
\end{pythoncode}
Las otras operaciones de comparación son útiles para poner las palabras
en orden alfabético:
\begin{pythoncode}
if palabra < "banana":
  print("Su palabra," + palabra + ", va antes que banana.")
elif palabra > "banana":
  print("Su palabra," + palabra + ", va después de banana.")
else:
  print("No hay banana!")
\end{pythoncode}

Sin embargo, usted debe ser consciente de que Python no maneja las
letras minúsculas y mayúsculas de la misma forma en que lo hace la
gente. Todas las letras mayúsculas vienen antes de las minúsculas.
Si palabra vale ``Zebra'' la salida sería:
\begin{pythoncode}
Su palabra, Zebra, va antes que banana.
\end{pythoncode}
Este problema se resuelve usualmente convirtiendo las cadenas a un
formato común, todas en minúsculas por ejemplo, antes de hacer la
comparación. Un problema más difícil es lograr que el programa reconozca
que una zebra no es una fruta.

\section{Las cadenas son inmutables}

\index{mutable} \index{cadena inmutable} \index{cadena!inmutable}

Uno puede caer en la trampa de usar el operador \texttt{{[}{]}} al
lado izquierdo de una asignación con la intención de modificar un
carácter en una cadena. Por ejemplo:
\begin{pythoncode}
saludo = "Hola mundo"
saludo[0] = 'J'            # ERROR!
print(saludo)
\end{pythoncode}
 En lugar de desplegar \texttt{Jola mundo!}, se produce un error en
tiempo de ejecución \texttt{TypeError: object doesn't support item
assignment}.

\index{error en tiempo de ejecución}

Las cadenas son \textbf{inmutables}, lo que quiere decir que no se
puede cambiar una cadena existente. Lo máximo que se puede hacer es
crear otra cadena que cambia un poco a la original:
\begin{pythoncode}
saludo = "Hola mundo!"
nuevoSaludo = 'J' + saludo[1:]
print(nuevoSaludo)
\end{pythoncode}

La solución consiste en concatenar la primera nueva letra con un segmento
de \texttt{saludo}. Esto no tiene efecto sobre la primera cadena,
usted puede chequearlo.

\index{concatenación}

\section{Una función \texttt{buscar} }

\label{find} \index{recorrido} \index{recorrido eureka} \index{patrón}
\index{patrón computacional}

¿Qué hace la siguiente función?

\begin{pythoncode}
def buscar(cad, c):
  indice = 0
  while indice < len(cad):
    if cad[indice] == c:
      return indice
    indice = indice + 1
  return -1
\end{pythoncode}
 De cierta manera \texttt{buscar} es el opuesto del operador \texttt{{[}{]}}.
En vez de tomar un índice y extraer el carácter correspondiente, toma
un carácter y encuentra el índice donde éste se encuentra. Si no se
encuentra el carácter en la cadena, la función retorna \texttt{-1}.

Este es el primer ejemplo de una sentencia \texttt{return} dentro
de un ciclo. Si se cumple que \texttt{cadena{[}indice{]} == c}, la
función retorna inmediatamente, rompiendo el ciclo prematuramente.

Si el carácter no está en la cadena, el programa completa todo el
ciclo y retorna \texttt{-1}.

Este patrón computacional se denomina recorrido ``eureka'', ya que
tan pronto encontremos lo que buscamos, gritamos ``Eureka!'' y dejamos
de buscar.

\section{Iterando y contando}

\label{counter} \index{contador} \index{patrón}

El siguiente programa cuenta el número de veces que la letra \texttt{a}
aparece en una cadena:
\begin{pythoncode}
fruta = "banano"
cont = 0
for car in fruta:
  if car == 'a':
    cont = cont + 1
print(cont)
\end{pythoncode}
 Este programa demuestra otro patrón computacional denominado \textbf{contador}.
La variable \texttt{cont} se inicializa a 0 y se incrementa cada vez
que se encuentre una \texttt{a}. ( \textbf{incrementar} es añadir
uno; es el opuesto de \textbf{decrementar}, y no tienen nada que ver
con ``excremento,'' que es un sustantivo.) Cuando el ciclo finaliza,
\texttt{cont} contiene el resultado—el número total de \texttt{a}'s.

\section{Algunos métodos de las cadenas}

\index{cadenas!métodos} \index{método} \index{método!invocación}
\index{métodos!cadenas} \index{métodos sobre cadenas} \index{invocar métodos}

Un \textbf{método} es similar a una función, acepta parámetros y devuelve
un valor, pero la sintaxis es diferente. Por ejemplo, las cadenas
tienen un método denominado \texttt{find} que hace lo mismo que buscar.
Para llamarlo tenemos que especificar la cadena, o la variable que
contiene la cadena, usando la notación punto, en la que el método
se escribe después del punto. La llamada a un método también se denomina
\textbf{invocación}; en este caso, diríamos que estamos invocando
\texttt{find} sobre la cadena \texttt{fruta}.

\begin{pyconcode}
>>> fruta = "banano"
>>> ind = fruta.find("a")
>>> print(ind)
1
\end{pyconcode}

El método \texttt{find} es más general que buscar, también puede buscar
subcadenas, no solo caracteres:
\begin{pyconcode}
>>> "banano".find("na")
2
\end{pyconcode}

También tiene un argumento adicional que especifica el índice desde
el que debe empezar la búsqueda:
\begin{pyconcode}
>>> "banana".find("na",3) 
4
\end{pyconcode}

Igualmente, puede tomar dos argumentos adicionales que especifican
un rango de índices:
\begin{pyconcode}
>>> "bob".find("b",1,2) 
-1
\end{pyconcode}

Aquí la búsqueda falló porque la letra {\em b} no está en en el
rango de índices de \texttt{1} a \texttt{2} (recuerde que no se incluye
el último índice, el \texttt{2}).

\section{El módulo \texttt{string y} clasificación de caracteres}

\index{módulo} \index{módulo string}

\label{in} \index{clasificación de caracteres} \index{clasificación!de caracteres}
\index{mayúsculas} \index{minúsculas} \index{espacios en blanco}

Con frecuencia es útil examinar un carácter y decidir si está en mayúsculas
o en minúsculas, o si es un dígito. El módulo \texttt{string} proporciona
varias constantes que sirven para lograr estos objetivos.

La cadena \texttt{string.lowercase} contiene todas las letras que
el sistema considera como minúsculas. Igualmente, \texttt{string.uppercase}
contiene todas las letras mayúsculas. Intente lo siguiente y vea por
sí mismo:
\begin{pyconcode}
>>> import string
>>> print(string.ascii_lowercase)
>>> print(string.ascii_uppercase)
>>> print(string.digits)
\end{pyconcode}
Podemos usar estas constantes y el método \texttt{find} para clasificar
los caracteres. Por ejemplo, si \texttt{c.find(lowercase)} retorna
un valor distinto de \texttt{-1}, entonces \texttt{c} debe ser una
letra minúscula:
\begin{pythoncode}
import string

def esMinuscula(c):
  return string.ascii_lowercase.find(c) != -1
\end{pythoncode}

Otra alternativa la da el operador \texttt{in} que determina si un
carácter aparece en una cadena:
\begin{pythoncode}
def esMinuscula(c):
  return c in string.ascii_lowercase
\end{pythoncode}
 Y otra alternativa más, con el operador de comparación:
\begin{pythoncode}
def esMinuscula(c):
  return 'a' <= c <= 'z'
\end{pythoncode}

Si \texttt{c} está entre {\em a} y {\em z}, debe ser una letra
minúscula.

Otra constante definida en el módulo \texttt{string} puede sorprenderlo
cuando la imprima:
\begin{pyconcode}
>>> print(string.whitespace)
\end{pyconcode}

Un carácter de los que pertenecen a \textbf{whitespace} mueve el cursor
sin imprimir nada. Crean un espacio en blanco que se puede evidenciar
entre caracteres. La constante \texttt{string.whitespace} contiene
todos los caracteres que representan espacios en blanco: espacio,
tab (\verb+\t+), y nueva línea (\verb+\n+).

\index{módulo string} \index{módulo!string}

Hay otras funciones útiles en el módulo string, pero este libro no
es un manual de referencia. Para esto usted puede consultar la referencia
de las bibliotecas de Python ({\em Python Library Reference}).
Además, hay un gran cúmulo de documentación en el sitio web de Python
\texttt{www.python.org}.

\index{Python Library Reference@{\em Python Library Reference}}

\section{Glosario}
\begin{description}
\item [{Tipo de dato compuesto:}] un tipo de dato en el que los valores
están compuestos por componentes o elementos, que, a su vez, son valores.
\item [{Recorrido:}] iteración sobre todos los elementos de un conjunto
ejecutando una operación similar en cada uno.
\item [{Índice:}] variable o valor que se usa para seleccionar un miembro
de un conjunto ordenado, tal como los caracteres de una cadena. También
se puede usar el término \texttt{posición} como sinónimo de índice.
\item [{Segmento:}] parte de una cadena, especificada por un rango de índices.
\item [{Mutable:}] un tipo de dato compuesto a cuyos elementos pueden asignarseles
nuevos valores.
\item [{Contador:}] una variable que se usa para contar algo, usualmente
se inicializa en cero y se incrementa posteriormente dentro de un
ciclo.
\item [{Incrementar:}] agregar uno al valor de una variable
\item [{Decrementar:}] restar uno al valor de una variable
\item [{Espacio en blanco:}] cualquiera de los caracteres que mueven el
cursor sin imprimir nada visible. La constante \texttt{string.whitespace}
contiene todos los caracteres que representan espacios en blanco.

\index{tipo de dato compuesto} \index{recorrido} \index{índice}
\index{segmento} \index{mutable} \index{contador} \index{incrementar}
\index{decrementar} \index{espacio en blanco}
\end{description}

\section{Ejercicios}

Para cada función, agregue chequeo de tipos y pruebas unitarias.
\begin{enumerate}
\item Escriba una función que tome una cadena como argumento y despliegue
las letras al revés, una por cada línea.
\item Modifique el programa de la sección \ref{for} para corregir el error
con los patos Ouack y Quack.
\item Modifique la función \texttt{buscar} de forma que reciba un tercer
parámetro, el índice en la cadena donde debe empezar a buscar.
\item Encapsule el código de la sección \ref{counter} en una función llamada
\texttt{contarLetras}, y generalícela de forma que reciba la cadena
y la letra como parámetros.
\item Reescriba la función que obtuvo en el punto anterior de forma que
en lugar de recorrer la cadena, llame a la función \texttt{buscar}
que recibe tres parámetros.
\item Discuta qué versión de \texttt{esMinuscula} cree que es la más rápida.
¿Puede pensar en otra razón distinta de la velocidad para preferir
alguna de ellas sobre las otras?
\item Cree un archivo llamado \verb+cadenas.py+ y escriba lo siguiente
en él:

\begin{pythoncode}
  def invertir(s):
    """
      >>> invertir('feliz')
      'zilef'
      >>> invertir('Python')
      'nohtyP'
      >>> invertir("")
      ''
      >>> invertir("P")
      'P'
    """

  if __name__ == '__main__':
    import doctest
    doctest.testmod()
\end{pythoncode}
 Agregue un cuerpo a la función invertir que haga que pase las pruebas
unitarias.

Agregue al archivo \verb+cadenas.py+ cuerpos a cada una de las siguientes
funciones, una a la vez.
\item Reflejar:
\begin{pythoncode}
  def reflejar(s):
    """
      >>> reflejar("bien")
      'bienneib'
      >>> reflejar("si")
      'siis'
      >>> reflejar('Python')
      'PythonnohtyP'
      >>> reflejar("")
      ''
      >>> reflejar("a")
      'aa'
    """
\end{pythoncode}
\item Eliminar letra:
\begin{pythoncode}
  def elimina_letra(letra, cadena):
    """
      >>> elimina_letra('a', 'manzana')
      'mnzn'
      >>> elimina_letra('a', 'banana')
      'bnn'
      >>> elimina_letra('z', 'banana')
      'banana'
      >>> elimina_letra('i', 'Mississippi')
      'Msssspp'
    """
\end{pythoncode}
\item Es palíndromo:
\begin{pythoncode}
  def es_palindromo(s):
    """
      >>> es_palindromo('abba')
      True
      >>> es_palindromo('abab')
      False
      >>> es_palindromo('tenet')
      True
      >>> es_palindromo('banana')
      False
      >>> es_palindromo('zorra arroz')
      True
    """
\end{pythoncode}
\item Cuenta el número de ocurrencias:
\begin{pythoncode}
  def cuenta(sub, s):
    """
      >>> cuenta('is', 'Mississippi')
      2
      >>> cuenta('an', 'banana')
      2
      >>> cuenta('ana', 'banana')
      2
      >>> cuenta('nana', 'banana')
      1
      >>> cuenta('nanan', 'banana')
      0
    """
\end{pythoncode}
\item Eliminar la primera ocurrencia: 
\begin{pythoncode}
  def elimina(sub, s):
    """
      >>> elimina('an', 'banana')
      'bana'
      >>> elimina('cic', 'bicicleta')
      'bileta'
      >>> elimina('iss', 'Mississippi')
      'Missippi'
      >>> elimina('huevo', 'bicicleta')
      'bicicleta'
    """
\end{pythoncode}
\item Eliminar todas las ocurrencias:
\begin{pythoncode}
  def elimina_todo(sub, s):
    """
      >>> elimina_todo('an', 'banana')
      'ba'
      >>> elimina_todo('cic', 'bicicleta')
      'bileta'
      >>> elimina_todo('iss', 'Mississippi')
      'Mippi'
      >>> elimina_todo('huevos', 'bicicleta')
      'bicicleta'
    """
\end{pythoncode}
\end{enumerate}

 
\clearemptydoublepage % cadenas

\chapter{Listas}

\label{cap:listas} \index{lista} \index{elemento} \index{secuencia}

Una \textbf{lista} es un conjunto ordenado de valores que se identifican
por medio de un índice. Los valores que componen una lista se denominan
\textbf{elementos}. Las listas son similares a las cadenas, que son
conjuntos ordenados de caracteres, pero son mas generales, ya que
pueden tener elementos de cualquier tipo de dato. Las listas y las
cadenas—y otras conjuntos ordenados que veremos— se denominan \textbf{secuencias}.

\section{Creación de listas}

Hay varias formas de crear una nueva lista; la más simple es encerrar
los elementos entre corchetes (\verb+[+ y \verb+]+):
\begin{lstlisting}
[10, 20, 30, 40]
["correo", "lapiz", "carro"]
\end{lstlisting}

El primer ejemplo es una lista de cuatro enteros, la segunda, una
lista de tres cadenas. Los elementos de una lista no tienen que tener
el mismo tipo. La siguiente lista contiene una cadena, un flotante,
un entero y (mirabile dictu) otra lista:
\begin{lstlisting}
["hola", 2.0, 5, [10, 20]]
\end{lstlisting}

Cuando una lista está contenida por otra se dice que está \textbf{anidada}.

\index{lista!anidada}

Las listas que contienen enteros consecutivos son muy comunes, así
que Python proporciona una forma de crearlas:
\begin{lstlisting}
>>> list(range(1,5))
[1, 2, 3, 4]
\end{lstlisting}

La función \texttt{range} toma dos argumentos y retorna una colección
que contiene todos los enteros desde el primero hasta el segundo,
¡incluyendo el primero y no el último! Esta colección se puede convertir
a lista con la función  \texttt{ list }

Hay otras formas de usar a \texttt{range}. Con un solo argumento crea
una lista que empieza en 0:
\begin{lstlisting}
>>> list(range(10))
[0, 1, 2, 3, 4, 5, 6, 7, 8, 9]
\end{lstlisting}

Si hay un tercer argumento, este especifica el espacio entre los valores
sucesivos, que se denomina el \textbf{tamaño del paso}. Este ejemplo
cuenta de 1 a 10 con un paso de tamaño 2:
\begin{lstlisting}
>>> list(range(1, 10, 2))
[1, 3, 5, 7, 9]
\end{lstlisting}

Finalmente, existe una lista especial que no contiene elementos. Se
denomina lista vacía, y se denota con \texttt{{[}{]}}.

Con todas estas formas de crear listas sería decepcionante si no pudiéramos
asignar listas a variables o pasarlas como parámetros a funciones.
De hecho, podemos hacerlo:
\begin{lstlisting}
>>> vocabulario = ["mejorar", "castigar", "derrocar"]
>>> numeros = [17, 123]
>>> vacia = []
>>> print(vocabulario, numeros, vacia)
["mejorar", "castigar", "derrocar"] [17, 123] []
\end{lstlisting}

\section{Accediendo a los elementos}

\index{lista!elemento} \index{acceso}

La sintaxis para acceder a los elementos de una lista es la misma
que usamos en las cadenas—el operador corchete (\texttt{{[}{]}}).
La expresión dentro de los corchetes especifica el índice. Recuerde
que los índices o posiciones empiezan desde 0:
\begin{lstlisting}
print(numeros[0])
numeros[1] = 5
\end{lstlisting}

El operador corchete para listas puede aparecer en cualquier lugar
de una expresión. Cuanto aparece al lado izquierdo de una asignación
cambia uno de los elementos de la lista de forma que el elemento 1
de \texttt{numeros}, que tenía el valor 123, ahora es 5.

Cualquier expresión entera puede usarse como índice:
\begin{lstlisting}
>>> numeros[3-2]
5
>>> numeros[1.0]
TypeError: sequence index must be integer
\end{lstlisting}

Si usted intenta leer o escribir un elemento que no existe, obtiene
un error en tiempo de ejecución:

\index{error en tiempo de ejecución}
\begin{lstlisting}
>>> numeros[2] = 5
IndexError: list assignment index out of range
\end{lstlisting}

Si el índice tiene un valor negativo, cuenta hacia atrás desde el
final de la lista:
\begin{lstlisting}
>>> numeros[-1]
5
>>> numeros[-2]
17
>>> numeros[-3]
IndexError: list index out of range
\end{lstlisting}

\texttt{numeros{[}-1{]}} es el último elemento de la lista, \texttt{numeros{[}-2{]}}
es el penúltimo, y \texttt{numeros{[}-3{]}} no existe.

Usualmente se usan variables de ciclo como índices de listas:
\begin{lstlisting}
combatientes = ["guerra", "hambruna", "peste", "muerte"]

i = 0
while i < 4:
  print(combatientes[i])
  i = i + 1
\end{lstlisting}

Este ciclo \texttt{while} cuenta de 0 a 4. Cuando la variable de ciclo
\texttt{i} es 4, la condición falla y el ciclo termina. El cuerpo
del ciclo se ejecuta solamente cuando \texttt{i} es 0, 1, 2, y 3.

En cada iteración del ciclo, la variable \texttt{i} se usa como un
índice a la lista, imprimiendo el \texttt{i}-ésimo elemento. Este
patrón se denomina \textbf{recorrido de una lista}.

\index{lista!recorrido de una} \index{recorrido!lista}

\section{Longitud de una lista}

\index{longitud} \index{lista!longitud}

La función \texttt{len} retorna la longitud de una lista. Es una buena
idea usar este valor como límite superior de un ciclo en vez de una
constante. De ésta forma, si la lista cambia, usted no tendrá que
cambiar todos los ciclos del programa, ellos funcionarán correctamente
para listas de cualquier tamaño:
\begin{lstlisting}
combatientes = ["guerra", "hambruna", "peste", "muerte"]

i = 0
while i < len(combatientes):
  print(combatientes[i])
  i = i + 1
\end{lstlisting}

La última vez que el ciclo se ejecuta \texttt{i} es \texttt{len(combatientes)
- 1}, que es la posición del último elemento. Cuando \texttt{i} es
igual a \texttt{len(combatientes)}, la condición falla y el cuerpo
no se ejecuta, lo que está muy bien , ya que \texttt{len(combatientes)}
no es un índice válido.

Aunque una lista puede contener a otra, la lista anidada se sigue
viendo como un elemento único. La longitud de esta lista es cuatro:
\begin{lstlisting}
['basura!', 1, ['Brie', 'Roquefort', 'Pol le Veq'], 
 [1, 2, 3]]
\end{lstlisting}

\section{Pertenencia}

\index{lista!pertenencia} \index{operador in} \index{operador!in}

\texttt{in} es un operador booleano que chequea la pertenencia de
un valor a una secuencia. Lo usamos en la Sección~\ref{in} con cadenas,
pero también funciona con listas y otras secuencias:
\begin{lstlisting}
>>> combatientes = ["guerra", "hambruna", "peste", "muerte"]
>>> 'peste' in combatientes
True
>>> 'corrupcion' in combatientes
False
\end{lstlisting}

Ya que ``peste'' es un miembro de la lista \texttt{combatientes},
el operador \texttt{in} retorna cierto. Como ``corrupcion'' no está
en la lista, \texttt{in} retorna falso.

Podemos usar el operador lógico \texttt{not} en combinación con el
\texttt{in} para chequear si un elemento no es miembro de una lista:
\begin{lstlisting}
>>> 'corrupcion' not in combatientes
True
\end{lstlisting}

\section{Listas y ciclos \texttt{for}}

\index{ciclo for} \index{lista!ciclo for} \index{recorrido}

El ciclo \texttt{for} que vimos en la Sección~\ref{for} también
funciona con listas. La sintaxis generalizada de un ciclo \texttt{for}
es:
\begin{lstlisting}
for VARIABLE in LISTA:
  CUERPO
\end{lstlisting}

Esto es equivalente a:
\begin{lstlisting}
i = 0
while i < len(LISTA):
  VARIABLE = LISTA[i]
  CUERPO
  i = i + 1
\end{lstlisting}

El ciclo \texttt{for} es más conciso porque podemos eliminar la variable
de ciclo \texttt{i}. Aquí está el ciclo de la sección anterior escrito
con un \texttt{for} en vez de un while: 
\begin{lstlisting}
for combatiente in combatientes:
  print(combatiente)
\end{lstlisting}

Casi se lee como en español: ``Para (cada) combatiente en (la lista
de) combatientes, imprima (el nombre del) combatiente''.

Cualquier expresión que cree una lista puede usarse en un ciclo \texttt{for}:
\begin{lstlisting}
for numero in range(20):
  if numero % 2 == 0:
    print(numero)

for fruta in ["banano", "manzana", "pera"]:
  print("Me gustaria comer " + fruta + "s!")
\end{lstlisting}

El primer ejemplo imprime todos los números pares entre uno y diecinueve.
El segundo expresa entusiasmo sobre varias frutas.

\section{Operaciones sobre listas}

\index{operaciones sobre listas} \index{operación!sobre listas}

El operador \texttt{+} concatena listas:

\index{concatenación!de listas}
\begin{lstlisting}
>>> a = [1, 2, 3]
>>> b = [4, 5, 6]
>>> c = a + b
>>> print(c)
[1, 2, 3, 4, 5, 6]
\end{lstlisting}

Similarmente, el operador \texttt{{*}} repite una lista un número
de veces determinado:

\index{repetición!de listas}
\begin{lstlisting}
>>> [0] * 4
[0, 0, 0, 0]
>>> [1, 2, 3] * 3
[1, 2, 3, 1, 2, 3, 1, 2, 3]
\end{lstlisting}

El primer ejemplo repite \texttt{{[}0{]}} cuatro veces. El segundo
repite \texttt{{[}1, 2, 3{]}} tres veces.

\section{Segmentos de listas}

\index{segmento} \index{lista!segmento}

Las operaciones para sacar segmentos de cadenas que vimos en la Sección~\ref{slice}
también funcionan con listas:
\begin{lstlisting}
>>> lista = ['a', 'b', 'c', 'd', 'e', 'f']
>>> lista[1:3]
['b', 'c']
>>> lista[:4]
['a', 'b', 'c', 'd']
>>> lista[3:]
['d', 'e', 'f']
>>> lista[:]
['a', 'b', 'c', 'd', 'e', 'f']
\end{lstlisting}

\section{Las listas son mutables}

\index{mutable!lista} \index{lista!mutable}

Las listas son mutables y no tienen la restricción de las cadenas,
esto quiere decir que podemos cambiar los elementos internos usando
el operador corchete al lado izquierdo de una asignación.
\begin{lstlisting}
>>> fruta = ["banano", "manzana", "pera"]
>>> fruta[0] = "mandarina"
>>> fruta[-1] = "naranja"
>>> print(fruta)
['mandarina', 'manzana', 'naranja']
\end{lstlisting}
 Con el operador segmento podemos actualizar varios elementos a la
vez:
\begin{lstlisting}
>>> lista = ['a', 'b', 'c', 'd', 'e', 'f']
>>> lista[1:3] = ['x', 'y']
>>> print(lista)
['a', 'x', 'y', 'd', 'e', 'f']
\end{lstlisting}

También podemos eliminar varios elementos asignándoles la lista vacía:
\begin{lstlisting}
>>> lista = ['a', 'b', 'c', 'd', 'e', 'f']
>>> lista[1:3] = []
>>> print(lista)
['a', 'd', 'e', 'f']
\end{lstlisting}

Igualmente, podemos agregar elementos a una lista apretándolos dentro
de un segmento vacío en la posición que deseamos:
\begin{lstlisting}
>>> lista = ['a', 'd', 'f']
>>> lista[1:1] = ['b', 'c']
>>> print(lista)
['a', 'b', 'c', 'd', 'f']
>>> lista[4:4] = ['e']
>>> print(lista)
['a', 'b', 'c', 'd', 'e', 'f']
\end{lstlisting}

\section{Otras operaciones sobre listas}

\index{operación sobre listas}

Usar segmentos para insertar y borrar elementos de una lista es extraño
y propenso a errores. Hay mecanismos alternativos más legibles como
\texttt{del} que elimina un elemento de una lista.

\index{borrado!en listas} \index{borrado en listas} \index{del}
\begin{lstlisting}
>>> a = ['one', 'two', 'three']
>>> del a[1]
>>> a
['one', 'three']
\end{lstlisting}

Como es de esperar, \texttt{del} recibe índices negativos, y causa
errores en tiempo de ejecución si el índice está fuera de rango.

También se puede usar un segmento como argumento a \texttt{del}:
\begin{lstlisting}
>>> lista = ['a', 'b', 'c', 'd', 'e', 'f']
>>> del lista[1:5]
>>> print(lista)
['a', 'f']
\end{lstlisting}

Como de costumbre, los segmentos seleccionan todos los elementos hasta
el segundo índice, sin incluirlo.

La función \texttt{append} agrega un elemento (o una lista) al final
de una lista existente:
\begin{lstlisting}
>>> a = ['uno', 'dos']
>>> a.append('tres')
>>> print(a)
\end{lstlisting}

Observe que se llama con la notación punto, a diferencia de \texttt{len}
y \texttt{del}.

\section{Objetos y valores}

\index{objeto} \index{valor}

Si ejecutamos estas asignaciones
\begin{lstlisting}
a = "banana"
b = "banana"
\end{lstlisting}

sabemos que \texttt{a} y \texttt{b} se referirán a una cadena con
las letras \texttt{``banana''}. Pero no podemos afirmar que sea
la {\em misma} cadena.

Hay dos situaciones posibles:

\beforefig \centerline{\includegraphics{illustrations/list1}}
\afterfig

En un caso, \texttt{a} y \texttt{b} se refieren a cosas distintas
que tienen el mismo valor. En el segundo caso, se refieren a la misma
cosa. Estas ``cosas'' tienen nombres—se denominan \textbf{objetos}.
Un objeto es algo a lo que se puede referir una variable.

Cada objeto tiene un \textbf{identificador} único, que podemos obtener
con la función \texttt{id}. Imprimiendo el identificador de \texttt{a}
y \texttt{b}, podemos saber si se refieren al mismo objeto.
\begin{lstlisting}
>>> id(a)
135044008
>>> id(b)
135044008
\end{lstlisting}

De hecho, obtenemos el mismo identificador dos veces, lo que nos dice
que Python sólo creó una cadena, y que \texttt{a} y \texttt{b} se
refieren a ella.

Las listas, por otro lado, se comportan de manera diferente. Cuando
creamos dos listas obtenemos dos objetos:
\begin{lstlisting}
>>> a = [1, 2, 3]
>>> b = [1, 2, 3]
>>> id(a)
135045528
>>> id(b)
135041704
\end{lstlisting}

Así que el diagrama de estados luce así:

\beforefig \centerline{\includegraphics{illustrations/list2}}
\afterfig

\texttt{a} y \texttt{b} tienen el mismo valor pero no se refieren
al mismo objeto.

\section{Alias}

\index{alias} \index{referencia!alias}

Como las variables se pueden referir a objetos, si asignamos una variable
a otra, las dos se referirán al mismo objeto:
\begin{lstlisting}
>>> a = [1, 2, 3]
>>> b = a
\end{lstlisting}

En este caso el diagrama de estados luce así:

\beforefig \centerline{\includegraphics{illustrations/list3}}
\afterfig

Como la misma lista tiene dos nombres distintos, \texttt{a} y \texttt{b},
podemos decir que b es un \textbf{alias} de a. Los cambios que se
hagan a través de un alias afectan al otro:
\begin{lstlisting}
>>> b[0] = 5
>>> print(a)
[5, 2, 3]
\end{lstlisting}

Aunque este comportamiento puede ser útil, algunas veces puede ser
indeseable. En general, es más seguro evitar los alias cuando se está
trabajando con objetos mutables. Para objetos inmutables no hay problema.
Esta es la razón por la que Python tiene la libertad de crear alias
a cadenas cuando ve la oportunidad de economizar memoria. Pero tenga
en cuenta que esto puede variar en las diferentes versiones de Python;
por lo tanto no es recomendable realizar programas que dependan de
este comportamiento.

\section{Clonando listas}

\index{lista!clonando} \index{clonando}

Si queremos modificar una lista y conservar una copia de la original,
necesitamos realizar una copia de la lista, no sólo de la referencia.
Este proceso se denomina \textbf{clonación}, para evitar la ambigüedad
de la palabra ``copiar''.

La forma más sencilla de clonar una lista es usar el operador segmento:
\begin{lstlisting}
>>> a = [1, 2, 3]
>>> b = a[:]
>>> print(b)
[1, 2, 3]
\end{lstlisting}

Al tomar cualquier segmento de \texttt{a} creamos una nueva lista.
En este caso el segmento comprende toda la lista.

Ahora podemos realizar cambios a \texttt{b} sin preocuparnos por \texttt{a}:
\begin{lstlisting}
>>> b[0] = 5
>>> print(a)
[1, 2, 3]
\end{lstlisting}

¿Como sería un diagrama de estados para \texttt{a} y \texttt{b} antes
y después de este cambio? 

\section{Listas como parámetros}

\index{listas!como parámetros} \index{parámetro} \index{parámetro!lista}

Pasar una lista como argumento es pasar un alias (o una referencia),
no una copia de ella. Por ejemplo, la función \texttt{cabeza} toma
una lista como parámetro y retorna el primer elemento:
\begin{lstlisting}
def cabeza(lista):
  return lista[0]
\end{lstlisting}

Se puede usar así:
\begin{lstlisting}
>>> numeros = [1, 2, 3]
>>> cabeza(numeros)
1
\end{lstlisting}

El parámetro \texttt{lista} y la variable \texttt{numeros} son alias
para el mismo objeto. El diagrama de estados luce así:

\beforefig \centerline{\includegraphics{illustrations/stack5}}
\afterfig

Como el objeto lista está compartido por dos marcos, lo dibujamos
en el medio.

Si una función modifica un parámetro de tipo lista, el que hizo el
llamado ve los cambios. Por ejemplo, \texttt{borrarCabeza} borra el
primer elemento de una lista:
\begin{lstlisting}
def borrarCabeza(lista):
  del lista[0]
\end{lstlisting}

Y se puede usar así::
\begin{lstlisting}
>>> numeros = [1, 2, 3]
>>> borrarCabeza(numeros)
>>> print(numeros)
[2, 3]
\end{lstlisting}

Si una función retorna una lista, retorna una referencia a ella. Por
ejemplo, la función \texttt{cola} retorna una lista que contiene todos
los elementos, excepto el primero:
\begin{lstlisting}
def cola(lista):
  return lista[1:]
\end{lstlisting}

\texttt{cola} se puede usar así:
\begin{lstlisting}
>>> numeros = [1, 2, 3]
>>> resto = cola(numeros)
>>> print(resto)
[2, 3]
\end{lstlisting}

Como el valor de retorno se creó con el operador segmento, es una
nueva lista. La creación de \texttt{resto}, y los cambios subsecuentes
sobre esta variable no tienen efecto sobre \texttt{numeros}.

\section{Listas anidadas}

\label{nested lists} \index{listas anidadas} \index{lista!anidada}

Una lista anidada aparece como elemento dentro de otra lista. En la
siguiente lista, el tercer elemento es una lista anidada:
\begin{lstlisting}
>>> lista = ["hola", 2.0, 5, [10, 20]]
\end{lstlisting}

Si imprimimos \texttt{lista{[}3{]}}, vemos \texttt{{[}10, 20{]}}.
Para tomar un elemento de la lista anidada podemos realizar dos pasos:
\begin{lstlisting}
>>> elt = lista[3]
>>> elt[0]
10
\end{lstlisting}
 O, los podemos combinar:
\begin{lstlisting}
>>> lista[3][1]
20
\end{lstlisting}

Las aplicaciones del operador corchete se evalúan de izquierda a derecha,
así que ésta expresión obtiene el elemento 3 de \texttt{lista} y extrae
de allí el elemento 1.

\section{Matrices}

\index{matriz} \index{lista!anidada}

Las listas anidadas se usan a menudo para representar matrices. Por
ejemplo, la matriz:

\beforefig \centerline{\includegraphics{illustrations/matrix}}
\afterfig

se puede representar así:
\begin{lstlisting}
>>> matriz = [[1, 2, 3], [4, 5, 6], [7, 8, 9]]
\end{lstlisting}

\texttt{matriz} es una lista con tres elementos, cada uno es una fila.
Podemos seleccionar una fila de la manera usual:
\begin{lstlisting}
>>> matriz[1]
[4, 5, 6]
\end{lstlisting}

O podemos extraer un elemento individual de la matriz usando dos índices:
\begin{lstlisting}
>>> matriz[1][1]
5
\end{lstlisting}

El primero escoge la fila, y el segundo selecciona la columna. Aunque
esta forma de representar matrices es común, no es la única posibilidad.
Una pequeña variación consiste en usar una lista de columnas en lugar
de una lista de filas. Más adelante veremos una alternativa más radical,
usando un diccionario.

\index{diccionario} \index{fila} \index{columna}

\section{Cadenas y listas}

\index{función split} \index{función join}

Dos de las funciones más usadas de las cadenas implican listas de
cadenas. \texttt{split} separa una cadena en una lista de palabras.
Por defecto, cualquier número de espacios en blanco sirven como criterio
de separación:
\begin{verbatim}
>>> cancion = "La vida es un ratico..."
>>> str.split(cancion)
['La', 'vida', 'es', 'un', 'ratico...']
\end{verbatim}
Un argumento opcional denominado \textbf{delimitador} se puede usar
para especificar que caracteres usar como criterio de separación.
El siguiente ejemplo usa la cadena \texttt{an} como delimitador:
\begin{lstlisting}
>>> str.split( "La rana que canta", "an")
['La r', 'a que c', 'ta']
\end{lstlisting}

Note que el delimitador no aparece en la lista resultante.

La función \texttt{join} es la inversa de \texttt{split}. Toma una
lista de cadenas y las concatena usando como separador a la cadena
que la llama con notación punto:
\begin{lstlisting}
>>> m = ['La', 'vida', 'es', 'un', 'ratico']
>>> " ".join(m)
'La vida es un ratico'
\end{lstlisting}

Si usamos una cadena diferente de espacio, obtenemos otro resultado:
\begin{lstlisting}
>>> "_".join(m)
'La_vida_es_un_ratico'
\end{lstlisting}

\section{Glosario}
\begin{description}
\item [{Lista:}] colección de objetos que recibe un nombre. Cada objeto
se identifica con un índice o número entero positivo.
\item [{Índice:}] valor o variable entero que indica la posición de un
elemento en una lista.
\item [{Elemento:}] uno de los valores dentro de una lista (u otra secuencia).
El operador corchete selecciona elementos de una lista.
\item [{Secuencia:}] los tipos de datos que contienen un conjunto ordenado
de elementos, identificados por índices.
\item [{Lista anidada:}] lista que es elemento de otra lista.
\item [{Recorrido de una lista:}] es el acceso secuencial de cada elemento
de una lista.
\item [{Objeto:}] una cosa a la que una variable se puede referir.
\item [{Alias:}] cuando varias variables tienen referencias hacia el mismo
objeto.
\item [{Clonar:}] crear un objeto con el mismo valor que un objeto preexistente.
Copiar una referencia a un objeto crea un alias, pero no clona el
objeto.
\item [{Delimitador:}] carácter o cadena que se usa para indicar el lugar
donde una cadena debe ser separada.

\index{lista} \index{índice} \index{secuencia} \index{elemento}
\index{lista anidada} \index{recorrido de una lista} \index{objeto}
\index{alias} \index{clonar} \index{delimitador}
\end{description}

\section{Ejercicios}

Para cada función, agregue chequeo de tipos y pruebas unitarias.
\begin{enumerate}
\item Escriba una función llamada medio que reciba una lista y retorne una
nueva lista que contenga todos los elementos de la lista de entrada
excepto el primero y el último. Por ejemplo, medio({[}1,2,3,4{]})
debe retornar {[}2,3{]}.
\item Escriba una función llamada cortar que reciba una lista y la modifique
eliminando el primer y el último elemento, retornando None.
\item Escriba una función que recorra una lista de cadenas imprimiendo la
longitud de cada una. ¿Qué pasa si usted le pasa un entero a \texttt{len}?
\item Describa la relación entre las expresiones:

\texttt{cadena} \hfill{}\texttt{' '.join(str.split(cadena))}

¿Son iguales para todas las cadenas?

¿Cuando serían diferentes?
\item Escriba una función llamada \verb+esta_ordenada+ que tome una lista
como parámetro y retorne True si la lista está ordenada de forma ascendente
o False si no lo está. Usted puede asumir como precondición que los
elementos son comparables con los operadores relacionales. Por ejemplo:

\verb+esta_ordenada([1,2,2])+ debe retornar True

\verb+esta_ordenada(['b','a'])+ debe retornar False.
\item Dos palabras son anagramas si se pueden reordenar las letras de una
palabra para formar la otra. Escriba una función llamada \verb+es_anagrama+
que tome dos cadenas y retorne True si son anagramas y False en caso
contrario.
\item Escriba una función llamada \verb+eliminar_duplicados+ que reciba
una lista y retorne una nueva lista con los elementos únicos de la
original. No necesitan estar en el mismo orden.
\item Escriba dos versiones de una función que lea el archivo palabras.txt
y construya una lista con un elemento por palabra. Una versión usará
el método append y la otra la construcción t=t+{[}x{]}. ¿Cual es mas
lenta? ¿Por qué? Pista: use el módulo time para medir lo que tarda
la ejecución de las versiones.

palabras.txt: \url{https://github.com/abecerra/thinkcs-py_es/releases/download/thinkcs-py_es_e2-rc1/palabras.txt}

Solución: \url{http://thinkpython.com/code/wordlist.py}
\item Si hay un grupo de 23 personas, ¿cual es la probabilidad de que dos
tengan la misma fecha de nacimiento?. Este valor puede estimarse generando
muestras aleatorias de 23 cumpleaños y contando las coincidencias.
Pista: consulte la función randint del módulo random.

Solución: \url{http://thinkpython.com/code/birthday.py}
\item Dos palabras son un ``par inverso'' si cada una es la inversa de
la otra. Escriba un programa que encuentre todos los pares inversos
del español (palabras.txt).

Solución: \url{http://thinkpython.com/code/reverse_pair.py}
\item Dos palabras se entretejen si tomando las letras de las dos, alternándose,
se puede formar una nueva palabra. Por ejemplo: 'pie' y 'en' se entretejen
en 'peine'.

Solución: \url{http://thinkpython.com/code/interlock.py} 
\end{enumerate}


\clearemptydoublepage % listas

\chapter{Tuplas}

\label{tuplechap} \index{tupla}

\section{Mutabilidad y tuplas}

\index{tupla} \index{tipo de dato!tupla} \index{tipo de dato!inmutable}

Hasta aquí, usted ha visto dos tipos de datos compuestos: cadenas,
que están compuestas de caracteres; y listas, compuestas de elementos
de cualquier tipo. Una de las diferencias que notamos es que los elementos
de una lista pueden modificarse, pero los caracteres en una cadena
no. En otras palabras, las cadenas son \textbf{inmutables} y las listas
son \textbf{mutables}.

\index{mutable} \index{inmutable}

Hay otro tipo de dato en Python denominado \textbf{tupla}, que se
parece a una lista, con la excepción de que es inmutable. Sintácticamente,
una tupla es una lista de valores separados por comas:
\begin{pyconcode}
>>> tupla = 'a', 'b', 'c', 'd', 'e'
\end{pyconcode}

Aunque no es necesario, se pueden encerrar entre paréntesis:
\begin{pyconcode}
>>> tupla = ('a', 'b', 'c', 'd', 'e')
\end{pyconcode}

Para crear una tupla con un único elemento, tenemos que incluir la
coma final:
\begin{pyconcode}
>>> t1 = ('a',)
>>> type(t1)
<type 'tuple'>
\end{pyconcode}

Sin la coma, Python creería que \texttt{('a')} es una cadena en paréntesis:
\begin{pyconcode}
>>> t2 = ('a')
>>> type(t2)
<type 'string'>
\end{pyconcode}

Las operaciones sobre tuplas son las mismas que vimos con las listas.
El operador corchete selecciona un elemento de la tupla.
\begin{pyconcode}
>>> tupla = ('a', 'b', 'c', 'd', 'e')
>>> tupla[0]
'a'
\end{pyconcode}

Y el operador segmento selecciona un rango de elementos:
\begin{pyconcode}
>>> tupla[1:3]
('b', 'c')
\end{pyconcode}

Pero si intentamos modificar un elemento de la tupla obtenemos un
error:

\index{error en tiempo de ejecución}
\begin{pyconcode}
>>> tupla[0] = 'A'
TypeError: object doesn't support item assignment
\end{pyconcode}

Aunque no podemos modificar los elementos, sí podemos modificar toda
la tupla:
\begin{pyconcode}
>>> tupla = ('A',) + tupla[1:]
>>> tupla
('A', 'b', 'c', 'd', 'e')
>>> tupla = (1,2,3)
>>> tupla
\end{pyconcode}

\section{Asignación de tuplas}

\label{tuple assignment} \index{asignación de tuplas} \index{Asignación!de tuplas}

De vez en cuando necesitamos intercambiar los valores de dos variables.
Con el operador de asignación normal tenemos que usar una variable
temporal. Por ejemplo, para intercambiar \texttt{a} y \texttt{b}:
\begin{pyconcode}
>>> temp = a
>>> a = b
>>> b = temp
\end{pyconcode}

Si tenemos que intercambiar variables muchas veces, el código tiende
a ser engorroso. Python proporciona una forma de \textbf{asignación
de tuplas} que resuelve este problema:
\begin{pyconcode}
>>> a, b = b, a
\end{pyconcode}

El lado izquierdo es una tupla de variables; el derecho es una tupla
de valores. Cada valor se asigna a su respectiva variable en el orden
en que se presenta. Las expresiones en el lado derecho se evalúan
antes de cualquier asignación. Esto hace a la asignación de tuplas
una herramienta bastante versátil.

Naturalmente, el número de variables a la izquierda y el número de
valores a la derecha deben coincidir.
\begin{pyconcode}
>>> a, b, c, d = 1, 2, 3
ValueError: unpack tuple of wrong size
\end{pyconcode}

\section{Tuplas como valores de retorno}

\index{tupla} \index{valor!tupla} \index{valor de retorno!tupla}
\index{función!tupla como valor de retorno}

Las funciones pueden tener tuplas como valores de retorno. Por ejemplo,
podríamos escribir una función que intercambie sus dos parámetros:
\begin{pythoncode}
def intercambiar(x, y):
  return y, x
\end{pythoncode}

Así podemos asignar el valor de retorno a una tupla con dos variables:
\begin{pythoncode}
a, b = intercambiar(a, b)
\end{pythoncode}

En este caso, escribir una función \texttt{intercambio} no es muy
provechoso. De hecho, hay un peligro al tratar de encapsular \texttt{intercambio},
que consiste en el siguiente error:
\begin{pythoncode}
def intercambio(x, y):      # version incorrecta
  x, y = y, x
\end{pythoncode}

Si llamamos a esta función así:
\begin{pythoncode}
intercambio(a, b)
\end{pythoncode}

entonces \texttt{a} y \texttt{x} son dos alias para el mismo valor.
Cambiar \texttt{x} dentro de \texttt{intercambio} hace que \texttt{x}
se refiera a un valor diferente, pero no tiene efecto en la \texttt{a}
dentro de \texttt{\_\_main\_\_}. Igualmente, cambiar \texttt{y} no
tiene efecto en \texttt{b}.

Esta función se ejecuta sin errores, pero no hace lo que se pretende.
Es un ejemplo de error semántico.

\index{error semántico}

\section{Números aleatorios}

\index{número aleatorio} \index{número!aleatorio}

La gran mayoría de los programas hacen lo mismo cada vez que se ejecutan,
esto es, son \textbf{determinísticos}. El determinismo generalmente
es una buena propiedad, ya que usualmente esperamos que los cálculos
produzcan el mismo resultado. Sin embargo, para algunas aplicaciones
necesitamos que el computador sea impredecible. Los juegos son un
ejemplo inmediato, pero hay más.

Lograr que un programa sea verdaderamente no determinístico no es
una tarea fácil, pero hay formas de que parezca no determinístico.
Una de ellas es generar números aleatorios y usarlos para determinar
la salida de un programa. Python tiene una función primitiva que genera
números \textbf{pseudoaleatorios}, que, aunque no sean aleatorios
desde el punto de vista matemático, sirven para nuestros propósitos.

El módulo \texttt{random} contiene una función llamada \texttt{random}
que retorna un número flotante entre 0.0 y 1.0. Cada vez que se llama
a \texttt{random}, se obtiene el siguiente número de una serie muy
larga. Para ver un ejemplo ejecute el siguiente ciclo:
\begin{pythoncode}
import random

for i in range(10):
  x = random.random()
  print(x)
\end{pythoncode}

Para generar un número aleatorio entre 0.0 y un límite superior como
\texttt{sup}, multiplique \texttt{x} por \texttt{sup}.

\section{Lista de números aleatorios}

Vamos a escribir una función llamada \texttt{listaAleatoria} que cree
una lista de números aleatorios. Recibirá un parámetro entero que
especifique el número de elementos a generar. Primero, genera una
lista de \texttt{n} ceros. Luego cada vez que itera en un ciclo for,
reemplaza uno de los ceros por un número aleatorio. El valor de retorno
es una referencia a la lista construida:
\begin{pythoncode}
def listaAleatoria(n):
  s = [0] * n
  for i in range(n):
    s[i] = random.random()
  return s
\end{pythoncode}

La probaremos con ocho elementos. Para depurar es una buena idea empezar
con pocos datos:
\begin{pythoncode}
>>> listaAleatoria(8)
0.15156642489
0.498048560109
0.810894847068
0.360371157682
0.275119183077
0.328578797631
0.759199803101
0.800367163582
\end{pythoncode}

Los números generados por \texttt{random} deben distribuirse uniformemente,
lo que significa que cada valor es igualmente probable.

Si dividimos el rango de valores posibles en ``regiones'' del mismo
tamaño y contamos el número de veces que un valor aleatorio cae en
cada región, deberíamos obtener un resultado aproximado en todas las
regiones.

Podemos probar esta hipótesis escribiendo un programa que divida el
rango en regiones y cuente el número de valores que caen en cada una.

\section{Conteo}

\index{conteo}

Un enfoque que funciona en problemas como éste es dividir el problema
en subproblemas que se puedan resolver con un patrón computacional
que ya sepamos.

En este caso, necesitamos recorrer una lista de números y contar el
número de veces que un valor cae en un rango dado. Esto parece familiar.
En la Sección~\ref{counter}, escribimos un programa que recorría
una cadena y contaba el números de veces que aparecía una letra determinada.

Entonces podemos copiar el programa viejo para adaptarlo posteriormente
a nuestro problema actual. El original es:
\begin{pythoncode}
cont = 0
for c in fruta:
  if c == 'a':
    cont = cont + 1
print(cont)
\end{pythoncode}

El primer paso es reemplazar \texttt{fruta} con \texttt{lista} y \texttt{c}
por \texttt{num}. Esto no cambia el programa, sólo lo hace más legible.

El segundo paso es cambiar la prueba. No queremos buscar letras. Queremos
ver si \texttt{num} está entre dos valores dados \texttt{inf} y \texttt{sup}.

\begin{pythoncode}
cont = 0
for num in lista:
  if inf < num < sup:
    cont = cont + 1
print(cont)
\end{pythoncode}

El último paso consiste en encapsular este código en una función denominada
\texttt{enRegion}. Los parámetros son la lista y los valores \texttt{inf}
y \texttt{sup}.
\begin{pythoncode}
def enRegion(lista, inf, sup):
  cont = 0
  for num in lista:
    if inf < num < sup:
      cont = cont + 1
  return cont
\end{pythoncode}

Copiando y modificando un programa existente fuimos capaces de escribir
esta función rápidamente y ahorrarnos un buen tiempo de depuración.
Este plan de desarrollo se denomina \textbf{concordancia de patrones}.
Si se encuentra trabajando en un problema que ya ha resuelto antes,
reutilice la solución.

\section{Muchas regiones}

\label{muchasregiones}

Como el número de regiones aumenta, \texttt{enRegion} es un poco engorroso.
Con dos no esta tan mal:
\begin{pythoncode}
inf = enRegion(a, 0.0, 0.5)
sup = enRegion(a, 0.5, 1)
\end{pythoncode}
 Pero con cuatro:
\begin{pythoncode}
Region1 = enRegion(a, 0.0, 0.25)
Region2 = enRegion(a, 0.25, 0.5)
Region3 = enRegion(a, 0.5, 0.75)
Region4 = enRegion(a, 0.75, 1.0)
\end{pythoncode}

Hay dos problemas. Uno es que siempre tenemos que crear nuevos nombres
de variables para cada resultado. El otro es que tenemos que calcular
el rango de cada región.

Primero resolveremos el segundo problema. Si el número de regiones
está dado por la variable \texttt{numRegiones}, entonces, el ancho
de cada región está dado por la expresión \texttt{1.0/numRegiones}.

Usaremos un ciclo para calcular el rango de cada región. La variable
de ciclo \texttt{i} cuenta de 1 a \texttt{numRegiones-1}:
\begin{pythoncode}
ancho = 1.0 / numRegiones
for i in range(numRegiones):
  inf = i * ancho
  sup = inf + ancho
  print(inf, " hasta ", sup)
\end{pythoncode}

Para calcular el extremo inferior de cada región, multiplicamos la
variable de ciclo por el ancho. El extremo superior está a un \texttt{ancho}
de región de distancia.

Con \texttt{numRegiones = 8}, la salida es:
\begin{verbatim}
0.0 hasta 0.125
0.125 hasta 0.25
0.25 hasta 0.375
0.375 hasta 0.5
0.5 hasta 0.625
0.625 hasta 0.75
0.75 hasta 0.875
0.875 hasta 1.0
\end{verbatim}
Usted puede confirmar que cada región tiene el mismo ancho, que no
se solapan y que cubren el rango completo de 0.0 a 1.0.

Ahora regresemos al primer problema. Necesitamos una manera de almacenar
ocho enteros, usando una variable para indicarlos uno a uno. Usted
debe estar pensando ``¡una lista!''

Tenemos que crear la lista de regiones fuera del ciclo, porque esto
sólo debe ocurrir una vez. Dentro del ciclo, llamaremos a \texttt{enRegion}
repetidamente y actualizaremos el \texttt{i}ésimo elemento de la lista:

\pagebreak

\begin{pythoncode}
numRegiones = 8
Regiones = [0] * numRegiones
ancho = 1.0 / numRegiones
for i in range(numRegiones):
  inf = i * ancho
  sup = inf + ancho
  Regiones[i] = enRegion(lista, inf, sup)
print(Regiones)
\end{pythoncode}

Con una lista de 1000 valores, este código produce la siguiente lista
de conteo:
\begin{verbatim}
[138, 124, 128, 118, 130, 117, 114, 131]
\end{verbatim}
Todos estos valores están muy cerca a 125, que es lo que esperamos.
Al menos, están lo suficientemente cerca como para creer que el generador
de números pseudoaleatorios está funcionando bien.

\section{Una solución en una sola pasada}

\label{histograma} \index{histograma}

Aunque funciona, este programa no es tan eficiente como debería. Cada
vez que llama a \texttt{enRegion}, recorre la lista entera. A medida
que el número de regiones incrementa, va a hacer muchos recorridos.

Sería mejor hacer una sola pasada a través de la lista y calcular
para cada región el índice de la región en la que cae. Así podemos
incrementar el contador apropiado.

En la sección anterior tomamos un índice \texttt{i} y lo multiplicamos
por el \texttt{ancho} para encontrar el extremo inferior de una región.
Ahora vamos a encontrar el índice de la región en la que cae.

Como este problema es el inverso del anterior, podemos intentar dividir
por \texttt{ancho} en vez de multiplicar. ¡Esto funciona!

Como \texttt{ancho = 1.0 / numRegiones}, dividir por \texttt{ancho}
es lo mismo que multiplicar por \texttt{numRegiones}. Si multiplicamos
un número en el rango 0.0 a 1.0 por \texttt{numRegiones}, obtenemos
un número en el rango de 0.0 a \texttt{numRegiones}. Si redondeamos
ese número al entero más cercano por debajo, obtenemos lo que queremos,
un índice de región:
\begin{pythoncode}
numRegiones = 8
Regiones = [0] * numRegiones
for i in lista:
  ind = int(i * numRegiones)
  Regiones[ind] = Regiones[ind] + 1
\end{pythoncode}

Usamos la función \texttt{int} para pasar de número de punto flotante
a entero.

¿Es posible que este programa produzca un índice que esté fuera de
rango (por ser negativo o mayor que \texttt{len(Regiones)-1})?

Una lista como \texttt{Regiones} que almacena los conteos del número
de valores que hay en cada rango se denomina \textbf{histograma}.

\section{Glosario}
\begin{description}
\item [{Tipo inmutable:}] es un tipo de dato en el que los elementos no
pueden ser modificados. Las asignaciones a elementos o segmentos de
tipos inmutables causan errores. Las cadenas y las tuplas son inmutables.
\item [{Tipo mutable:}] tipo de dato en el que los elementos pueden ser
modificados. Todos los tipos mutables son compuestos. Las listas y
los diccionarios son mutables.
\item [{Tupla:}] tipo de dato secuencial similar a la lista, pero inmutable.
Las tuplas se pueden usar donde se requiera un tipo inmutable, por
ejemplo como llaves de un diccionario.
\item [{Asignación de tuplas:}] una asignación a todos los elementos
de una tupla en una sola sentencia. La asignación ocurre en paralelo
y no secuencialmente. Es útil para intercambiar valores de variables.
\item [{Determinístico:}] programa que hace lo mismo cada vez que se llama.
\item [{Pseudoaleatoria:}] secuencia de números que parece aleatoria, pero
en realidad es el resultado de un cómputo determinístico, bien diseñado.
\item [{Histograma:}] lista de enteros en la que cada elemento cuenta el
número de veces que algo sucede.
\item [{Correspondencia de patrones:}] plan de desarrollo de programas
que implica identificar un patrón computacional familiar y copiar
la solución de un problema similar.

\index{tipo mutable} \index{tipo inmutable} \index{tupla} \index{asignación de tupla}
\index{asignación!de tupla} \index{determinístico} \index{pseudoaleatorio}
\index{histograma} \index{correspondencia de patrones}
\end{description}

\section{Ejercicios}

Para cada función, agregue chequeo de tipos y pruebas unitarias.
\begin{enumerate}
\item Escriba una función mas\_frecuentes que tome una cadena e imprima
las letras en orden descendente por frecuencia. Ejecútela con textos
de diferentes lenguajes y observe como varían las frecuencias de letras.
Compare sus resultados con las tablas en:

\url{http://en.wikipedia.org/wiki/Letter_frequencies}

Solución: \url{http://thinkpython.com/code/most_frequent.py}
\item Escriba un programa que lea una lista de palabras de un archivo e
imprima todos los conjuntos de palabras que son anagramas.

Este es un ejemplo de la salida del programa con palabras en inglés:
\begin{verbatim}
['deltas', 'desalt', 'lasted', 'salted', 'slated', 'staled'] 
['retainers', 'ternaries'] 
['generating', 'greatening']
['resmelts', 'smelters', 'termless']
\end{verbatim}
Pista: cree un diccionario que asocie cada conjunto de letras a una
lista de palabras que puedan ser formadas con esas letras. ¿Como se
puede representar el conjunto de letras de forma que pueda ser usado
como llave? Modifique el programa que obtuvo para que imprima en orden
descendente por tamaño los conjuntos de anagramas. En el juego Scrabble,
un 'bingo' se da cuando se juegan las 7 fichas, junto con otra letra
en el tablero, para formar una palabra de 8 letras. ¿Que conjunto
de 8 letras forma los bingos mas posibles?

Solución: \url{http://thinkpython.com/code/anagram_sets.py}
\item Dos palabras forma un 'par de metatesis' si se puede transformar una
en otra intercambiando dos letras. Por ejemplo, 'conversación' y 'conservación'.
Escriba un programa que encuentre todos los pares de metatesis en
el diccionario. Pista: no pruebe todos los pares.

Solución: \url{http://thinkpython.com/code/metathesis.py}

Crédito: el ejercicio está inspirado por un ejemplo de \url{http://puzzlers.org}
\item ¿Cual es la palabra mas larga que sigue siendo válida a medida que
se remueven una a una sus letras? Por ejemplo, en inglés, 'sprite'
sin la 'r' es 'spite', que sin la 'e', es 'spit', que sin la 's' es
'pit', que sin la 'p' es 'it' que sin la 't' es 'i'. Las letras se
pueden remover de cualquier posición, pero no se pueden reordenar.

Escriba un programa que encuentre todas las palabras que pueden reducirse
de esta forma y que encuentre la mas larga.

Pistas:

Escriba una función que tome una palabra y calcule todas las palabras
que pueden formarse al removerle una letra. Estas son las palabras
'hijas'. Recursivamente, una palabra es reducible si alguno de sus
hijas es reducible. El caso base lo da la cadena vacía.

Solución: \url{http://thinkpython.com/code/reducible.py} 
\end{enumerate}

 
\clearemptydoublepage % tuplas

\chapter{Diccionarios}

\index{diccionarios} \index{diccionario} \index{tipo de datos!diccionario}
\index{clave} \index{par clave-valor} \index{índice}

Los tipos compuestos que ha visto hasta ahora (cadenas, listas y tuplas)
usan enteros como índices. Si usted intenta usar cualquier otro tipo
como índice provocará un error.

Los \textbf{diccionarios} son similares a otros tipos compuestos,
excepto en que pueden usar como índice cualquier tipo inmutable. A
modo de ejemplo, crearemos un diccionario que traduzca palabras inglesas
al español. En este diccionario, los índices son cadenas \texttt{(strings)}.

Una forma de crear un diccionario es empezar con el diccionario vacío
y añadir elementos. El diccionario vacío se expresa como \texttt{\{\}}:
\begin{lstlisting}
>>> ing_a_esp = {}
>>> ing_a_esp['one'] = 'uno'
>>> ing_a_esp['two'] = 'dos'
\end{lstlisting}

La primera asignación crea un diccionario llamado \texttt{ing\_a\_esp};
las otras asignaciones añaden nuevos elementos al diccionario. Podemos
desplegar el valor actual del diccionario del modo habitual:
\begin{lstlisting}
>>> print(ing_a_esp)
{'one': 'uno', 'two': 'dos'}
\end{lstlisting}

Los elementos de un diccionario aparecen en una lista separada por
comas. Cada entrada contiene un índice y un valor separado por dos
puntos (:). En un diccionario, los índices se llaman \textbf{claves},
por eso los elementos se llaman \textbf{pares clave-valor}.

Otra forma de crear un diccionario es dando una lista de pares clave-valor
con la misma sintaxis que la salida del ejemplo anterior:
\begin{lstlisting}
>>> ing_a_esp={'one': 'uno', 'two': 'dos', 'three': 'tres'}
\end{lstlisting}

Si volvemos a imprimir el valor de \texttt{ing\_a\_esp}, nos llevamos
una sorpresa:
\begin{lstlisting}
>>> print(ing_a_esp)
{'one': 'uno', 'three': 'tres', 'two': 'dos'}
\end{lstlisting}

¡Los pares clave-valor no están en orden! Afortunadamente, no necesitamos
preocuparnos por el orden, ya que los elementos de un diccionario
nunca se indexan con índices enteros. En lugar de eso, usamos las
claves para buscar los valores correspondientes:
\begin{lstlisting}
>>> print(ing_a_esp['two'])
'dos'
\end{lstlisting}

La clave \texttt{'two'} nos da el valor \texttt{'dos'} aunque aparezca
en el tercer par clave-valor.

\section{Operaciones sobre diccionarios}

\index{diccionario!operaciones} \index{diccionarios!operaciones sobre}

La sentencia \texttt{del} elimina un par clave-valor de un diccionario.
Por ejemplo, el diccionario siguiente contiene los nombres de varias
frutas y el número de esas frutas en un almacén:
\begin{lstlisting}
>>> inventario = {'manzanas': 430, 'bananas': 312, 
       'naranjas': 525,   'peras': 217}
>>> print(inventario)
{'naranjas': 525, 'manzanas': 430, 'peras': 217, 
 'bananas': 312}
\end{lstlisting}
 Si alguien compra todas las peras, podemos eliminar la entrada del
diccionario:
\begin{lstlisting}
>>> del inventario['peras']
>>> print(inventario)
{'naranjas': 525, 'manzanas': 430, 'bananas': 312}
\end{lstlisting}

O si esperamos recibir más peras pronto, podemos simplemente cambiar
el inventario asociado con las peras:
\begin{lstlisting}
>>> inventario['peras'] = 0
>>> print(inventario)
{'naranjas': 525, 'manzanas': 430, 'peras': 0, 
 'bananas': 312}
\end{lstlisting}
 La función \texttt{len} también funciona con diccionarios; devuelve
el número de pares clave-valor:
\begin{lstlisting}
>>> len(inventario)
4
\end{lstlisting}

\section{Métodos del diccionario}

\index{diccionario!métodos} \index{método} \index{método!invocación}
\index{diccionarios!métodos} \index{métodos sobre diccionarios}
\index{invocar métodos}

El método \texttt{keys} acepta un diccionario y devuelve una lista
con las claves que aparecen, pero en lugar de la sintaxis de llamado
de función \texttt{keys(ing\_a\_esp)}, usamos la sintaxis para un
método \texttt{ing\_a\_esp.keys()}.

\index{notación de punto}

\begin{lstlisting}
>>> ing_a_esp.keys()
['one', 'three', 'two']
\end{lstlisting}
 Esta forma de notación punto especifica el nombre de la función,
\texttt{keys}, y el nombre del objeto al que se va a aplicar la función,
\texttt{ing\_a\_esp}. Los paréntesis vacíos indican que este método
no admite parámetros.

El método \texttt{values} es similar; devuelve una lista de los valores
del diccionario:
\begin{lstlisting}
>>> ing_a_esp.values()
['uno', 'tres', 'dos']
\end{lstlisting}

El método \texttt{items} devuelve ambos, una lista de tuplas con los
pares clave-valor del diccionario:
\begin{lstlisting}
>>> ing_a_esp.items()
[('one','uno'), ('three', 'tres'), ('two', 'dos')]
\end{lstlisting}

La sintaxis nos proporciona información muy útil acerca del tipo de
datos. Los corchetes indican que es una lista. Los paréntesis indican
que los elementos de la lista son tuplas.

Para averiguar si una clave aparece en el diccionario, se puede usar
\texttt{in}:
\begin{lstlisting}
>>> 'one' in ing_a_esp
True
>>> 'deux' in ing_a_esp
False
\end{lstlisting}

\index{error en tiempo de ejecución}

\section{Copiado y alias}

\index{asignación de alias} \index{copiado} \index{clonado}

Usted debe estar atento a los alias debido a la mutabilidad de los
diccionarios. Si dos variables se refieren al mismo objeto los cambios
en una afectan a la otra.

Si quiere modificar un diccionario y mantener una copia del original,
se puede usar el método \texttt{copy}. Por ejemplo, \texttt{opuestos}
es un diccionario que contiene pares de opuestos:
\begin{lstlisting}
>>> opuestos = {'arriba': 'abajo', 'derecho': 'torcido', 
  'verdadero': 'falso'}
>>> alias = opuestos
>>> copia = opuestos.copy()
\end{lstlisting}

\texttt{alias} y \texttt{opuestos} se refieren al mismo objeto; \texttt{copia}
se refiere a una copia nueva del mismo diccionario. Si modificamos
\texttt{alias}, \texttt{opuestos} también resulta cambiado:
\begin{lstlisting}
>>> alias['derecho'] = 'sentado'
>>> opuestos['derecho']
'sentado'
\end{lstlisting}

Si modificamos \texttt{copia}, \texttt{opuestos} no varía:
\begin{lstlisting}
>>> copia['derecho'] = 'privilegio'
>>> opuestos['derecho']
'sentado'
\end{lstlisting}

\section{Matrices dispersas}

\index{matriz!dispersa} \index{lista anidada} \index{lista!anidada}

En la Sección~\ref{nested lists} usamos una lista de listas para
representar una matriz. Es una buena opción para una matriz en la
que la mayoría de los valores es diferente de cero, pero piense en
una matriz como ésta:

\beforefig\centerline{\includegraphics{illustrations/sparse}}
%\afterfig

La representación de la lista contiene un montón de ceros:
\begin{lstlisting}
>>> matriz = [ [0,0,0,1,0],
           [0,0,0,0,0],
           [0,2,0,0,0],
           [0,0,0,0,0],
           [0,0,0,3,0] ]
\end{lstlisting}

Una posible alternativa consiste en usar un diccionario. Como claves,
podemos usar tuplas que contengan los números de fila y columna. Ésta
es la representación de la misma matriz por medio de un diccionario:
\begin{lstlisting}
>>> matriz = {(0,3): 1, (2, 1): 2, (4, 3): 3}
\end{lstlisting}

Sólo hay tres pares clave-valor, uno para cada elemento de la matriz
diferente de cero. Cada clave es una tupla, y cada valor es un entero.

Para acceder a un elemento de la matriz, podemos usar el operador
\texttt{{[}{]}}:
\begin{lstlisting}
>>> matriz[0,3]
1
\end{lstlisting}

Observe que la sintaxis para la representación por medio del diccionario
no es la misma de la representación por medio de la lista anidada.
En lugar de dos índices enteros, usamos un índice compuesto que es
una tupla de enteros.

Hay un problema. Si apuntamos a un elemento que es cero, se produce
un error porque en el diccionario no hay una entrada con esa clave:

\index{error en tiempo de ejecución}
\begin{lstlisting}
>>> matriz[1,3]
KeyError: (1, 3)
\end{lstlisting}

El método \texttt{get} soluciona este problema:
\begin{lstlisting}
>>> matriz.get((0,3), 0)
1
\end{lstlisting}

El primer argumento es la clave; el segundo argumento es el valor
que debe devolver \texttt{get} en caso de que la clave no esté en
el diccionario:
\begin{lstlisting}
>>> matriz.get((1,3), 0)
0
\end{lstlisting}

\texttt{get} mejora sensiblemente la semántica del acceso a una matriz
dispersa. ¡Lástima que la sintaxis no sea tan clara!

\section{Pistas}

\index{pista} \index{función de Fibonacci}

Si ha jugado con la función \texttt{fibonacci} de la Sección~\ref{one more example},
es posible que haya notado que cuanto más grande es el argumento que
recibe, más tiempo le cuesta ejecutarse. De hecho, el tiempo de ejecución
aumenta muy rápidamente. En nuestra máquina, \texttt{fibonacci(20)}
acaba instantáneamente, \texttt{fibonacci(30)} tarda más o menos un
segundo, y \texttt{fibonacci(40)} tarda una eternidad.

Para entender por qué, observe este \textbf{gráfico de llamadas} de
\texttt{fibonacci} con \texttt{n=4}:

\beforefig \centerline{\includegraphics{illustrations/fibonacci}}
\afterfig

Un gráfico de llamadas muestra un conjunto de cajas de función con
líneas que conectan cada caja con las cajas de las funciones a las
que llama. En lo alto del gráfico, \texttt{fibonacci} con \texttt{n=4}
llama a \texttt{fibonacci} con \texttt{n=3} y \texttt{n=2}. A su vez,
\texttt{fibonacci} con \texttt{n=3} llama a \texttt{fibonacci} con
\texttt{n=2} y \texttt{n=1}. Y así sucesivamente.

\index{caja de función} \index{caja} \index{gráfico de llamadas}

Cuente cuántas veces se llama a \texttt{fibonacci(0)} y \texttt{fibonacci(1)}.
Esta función es una solución ineficiente para el problema, y empeora
mucho a medida que crece el argumento.

Una buena solución es llevar un registro de los valores que ya se
han calculado almacenándolos en un diccionario. A un valor que ya
ha sido calculado y almacenado para un uso posterior se le llama \textbf{pista}.
Aquí hay una implementación de \texttt{fibonacci} con pistas:
\begin{lstlisting}
anteriores = {0:1, 1:1}

def fibonacci(n):
  if anteriores.has_key(n):
    return anteriores[n]
  else:
    nuevoValor = fibonacci(n-1) + fibonacci(n-2)
    anteriores[n] = nuevoValor
    return nuevoValor
\end{lstlisting}

El diccionario llamado \texttt{anteriores} mantiene un registro de
los valores de Fibonacci que ya conocemos. El programa comienza con
sólo dos pares: 0 corresponde a 1 y 1 corresponde a 1.

Siempre que se llama a \texttt{fibonacci} comprueba si el diccionario
contiene el resultado ya calculado. Si está ahí, la función puede
devolver el valor inmediatamente sin hacer más llamadas recursivas.
Si no, tiene que calcular el nuevo valor. El nuevo valor se añade
al diccionario antes de que la función retorne.

Con esta versión de \texttt{fibonacci}, nuestra máquina puede calcular
\texttt{fibonacci(40)} en un abrir y cerrar de ojos. Pero cuando intentamos
calcular \texttt{fibonacci(50)}, nos encontramos con otro problema:

\index{error en tiempo de ejecución} \index{desbordamiento}

\begin{lstlisting}
>>> fibonacci(50)
OverflowError: integer addition
\end{lstlisting}

La respuesta, como se verá en un momento, es 20.365.011.074. El problema
es que este número es demasiado grande para caber en un entero de
Python. Se \textbf{desborda}. Afortunadamente, hay una solución fácil
para este problema.

\section{Enteros largos}

\index{enteros largos} \index{tipos de datos!enteros largos} \index{enteros!largos}

Python proporciona un tipo llamado \texttt{long int} que puede manejar
enteros de cualquier tamaño. Hay dos formas de crear un valor \texttt{long
int}. Una es escribir un entero con una {\em L} mayúscula al final:

\begin{lstlisting}
>>> type(1L)
<type 'long int'>
\end{lstlisting}
 La otra es usar la función \texttt{long} para convertir un valor
en \texttt{long int}. \texttt{long} acepta cualquier tipo numérico
e incluso cadenas de dígitos:
\begin{lstlisting}
>>> long(1)
1L
>>> long(3.9)
3L
>>> long('57')
57L
\end{lstlisting}

Todas las operaciones matemáticas funcionan sobre los datos de tipo
\texttt{long int}, así que no tenemos que hacer mucho para adaptar
\texttt{fibonacci}:
\begin{lstlisting}
>>> anteriores = {0:1L, 1:1L}
>>> fibonacci(50)
20365011074L
\end{lstlisting}

Simplemente cambiando el contenido inicial de \texttt{anteriores}
cambiamos el comportamiento de \texttt{fibonacci}. Los primeros dos
números de la secuencia son de tipo \texttt{long int}, así que todos
los números subsiguientes lo serán también.

\index{forzado de tipo de datos} \index{coerción!tipo}

\section{Contar letras}

\index{recuento} \index{histograma} \index{compresión}

En el capítulo~\ref{strings} escribimos una función que contaba
el número de apariciones de una letra en una cadena. Una versión más
genérica de este problema es crear un histograma de las letras de
la cadena, o sea, cuántas veces aparece cada letra.

Ese histograma podría ser útil para comprimir un archivo de texto.
Como las diferentes letras aparecen con frecuencias distintas, podemos
comprimir un archivo usando códigos cortos para las letras más habituales
y códigos más largos para las que aparecen con menor frecuencia.

Los diccionarios facilitan una forma elegante de generar un histograma:
\begin{lstlisting}
>>> cuentaLetras = {}
>>> for letra in "Mississippi":
      cuentaLetras[letra] = cuentaLetras.get(letra, 0)+1

>>> cuentaLetras
{'M': 1, 's': 4, 'p': 2, 'i': 4}
\end{lstlisting}

Inicialmente, tenemos un diccionario vacío. Para cada letra de la
cadena, buscamos el recuento actual (posiblemente cero) y la incrementamos.
Al final, el diccionario contiene pares de letras y sus frecuencias.

Puede ser más atractivo mostrar el histograma en orden alfabético.
Podemos hacerlo con los métodos \texttt{items} y \texttt{sort}:
\begin{lstlisting}
>>> itemsLetras = cuentaLetras.items()
>>> list(itemsLetras).sort()
>>> print(itemsLetras)
[('M', 1), ('i', 4), ('p', 2), ('s', 4)]
\end{lstlisting}

Usted ya ha visto el método \texttt{items} aplicable a los diccionarios;
\texttt{sort} es un método aplicable a listas. Hay varios más, como
\texttt{append}, \texttt{extend} y \texttt{reverse}. Consulte la documentación
de Python para ver los detalles.

\index{método!lista} \index{método de lista}

\section{Glosario}
\begin{description}
\item [{Diccionario:}] es una colección de pares clave-valor que establece
una correspondencia entre claves y valores. Las claves pueden ser
de cualquier tipo inmutable, los valores pueden ser de cualquier tipo.
\item [{Clave:}] un valor que se usa para buscar una entrada en un diccionario.
\item [{Par clave-valor:}] uno de los elementos de un diccionario, también
llamado ``asociación''.
\item [{Método:}] tipo de función al que se llama con una sintaxis diferente
y al que se invoca ``sobre'' un objeto.
\item [{Invocar:}] llamar a un método.
\item [{Pista:}] almacenamiento temporal de un valor precalculado, para
evitar cálculos redundantes.
\item [{Desbordamiento:}] resultado numérico que es demasiado grande para
representarse en formato numérico.

\index{diccionario} \index{clave} \index{par clave-valor} \index{pista}
\index{método} \index{invocar}
\end{description}

\section{Ejercicios}

Para cada función, agregue chequeo de tipos y pruebas unitarias.
\begin{enumerate}
\item Como ejercicio, modifique \texttt{factorial} de forma que produzca
un \texttt{long int} como resultado.
\item Escriba una función booleana que averigüe si una lista tiene algún
elemento duplicado usando un diccionario.
\item Una cadena de ARN puede representarse como una lista en la que los
elementos pueden ser los caracteres A,C,G y U. Escriba una función
booleana que averigüe si una lista de caracteres es una cadena de
ARN válida.
\item Generalice la función anterior de forma que reciba una biosecuencia
(ADN, ARN ó Proteína) y un alfabeto de referencia para averiguar si
está bien formada. Para ADN el alfabeto es A,C,T y G, para las proteínas
es:

A,R,N,D,C,E,Q,G,H,I,L,K,M,F,P,S,T,W,Y,V,U,O
\item Escriba una función que reciba una cadena de ADN y cuente cuantos
nucleótidos de cada tipo tiene (cuantas veces tiene A,C,G y T) usando
un diccionario.
\item Generalice la función anterior de forma que reciba una biosecuencia
(ADN, ARN ó Proteína) y un alfabeto. Debe contar cuantos elementos
tiene de cada tipo usando un diccionario.
\item Escriba una función producto que reciba una matriz dispersa M, implementada
con un diccionario, y un número. Debe retornar la matriz que resulta
de multiplicar cada elemento de M por el número.
\item Escriba una función que reciba dos matrices dispersas, implementadas
con diccionarios, y las sume, produciendo otra matriz dispersa.
\item Escriba una función que reciba dos matrices dispersas, implementadas
con diccionarios, y las multiplique, produciendo otra matriz dispersa.
Base su solución en las dos soluciones anteriores.
\item Escriba una función booleana que reciba una matriz dispersa y averigüe
si es la matriz identidad. 
\end{enumerate}


\clearemptydoublepage % diccionarios

\chapter{Archivos y excepciones}

\index{archivos}

Cuando un programa se está ejecutando, sus datos están en la memoria.
Cuando un programa termina, o se apaga el computador, los datos de
la memoria desaparecen. Para almacenar los datos de forma permanente
se deben poner en un \textbf{archivo}. Normalmente los archivos se
guardan en un disco duro, disquete o CD-ROM.

Cuando hay un gran número de archivos, suelen estar organizados en
\textbf{directorios} (también llamados ``carpetas''). Cada archivo
se identifica con un nombre único, o una combinación de nombre de
archivo y nombre de directorio.

Leyendo y escribiendo archivos, los programas pueden intercambiar
información entre ellos y generar formatos imprimibles como PDF.

Trabajar con archivos se parece mucho a hacerlo con libros. Para usar
un libro, hay que abrirlo. Cuando uno ha terminado, hay que cerrarlo.
Mientras el libro está abierto, se puede escribir en él o leer de
él. En cualquier caso, uno sabe en qué lugar del libro se encuentra.
Casi siempre se lee un libro según su orden natural, pero también
se puede ir saltando de página en página.

Todo esto sirve también para los archivos. Para abrir un archivo,
se especifica su nombre y se indica si se desea leer o escribir.

La apertura de un archivo crea un objeto archivo. En este ejemplo,
la variable \texttt{f} apunta al nuevo objeto archivo.\inputencoding{latin9}
\begin{lstlisting}
>>> f = open("test.dat","w")
>>> print(f)
<open file 'test.dat', mode 'w' at fe820>
\end{lstlisting}
\inputencoding{utf8}
La función open toma dos argumentos: el primero, es el nombre del
archivo y el segundo, el modo. El modo {\verb+"w"+} significa que
lo estamos abriendo para escribir.

Si no hay un archivo llamado \texttt{test.dat} se creará. Si ya hay
uno, el archivo que estamos escribiendo lo reemplazará.

Al imprimir el objeto archivo, vemos el nombre del archivo, el modo
y la localización del objeto.

Para escribir datos en el archivo invocamos al método \texttt{write}
sobre el objeto archivo:\inputencoding{latin9}
\begin{lstlisting}
>>> f.write("Ya es hora")
>>> f.write("de cerrar el archivo")
\end{lstlisting}
\inputencoding{utf8}
El cierre del archivo le dice al sistema que hemos terminado de escribir
y deja el archivo listo para leer:\inputencoding{latin9}
\begin{lstlisting}
>>> f.close()
\end{lstlisting}
\inputencoding{utf8}
Ya podemos abrir el archivo de nuevo, esta vez para lectura, y poner
su contenido en una cadena. Esta vez el argumento de modo es {\verb+"r"+},
para lectura:\inputencoding{latin9}
\begin{lstlisting}
>>> f = open("test.dat","r")
\end{lstlisting}
\inputencoding{utf8}
Si intentamos abrir un archivo que no existe, recibimos un mensaje
de error:

\index{error en tiempo de ejecución}\inputencoding{latin9}
\begin{lstlisting}
>>> f = open("test.cat","r")
IOError: [Errno 2] No such file or directory: 'test.cat'
\end{lstlisting}
\inputencoding{utf8}
Como era de esperar, el método \texttt{read} lee datos del archivo.
Sin argumentos, lee el archivo completo:\inputencoding{latin9}
\begin{lstlisting}
>>> texto = f.read()
>>> print(texto)
Ya es horade cerrar el archivo
\end{lstlisting}
\inputencoding{utf8}
No hay un espacio entre ``hora'' y ``de'' porque no escribimos
un espacio entre las cadenas. \texttt{read} también puede aceptar
un argumento que le indica cuántos caracteres leer:\inputencoding{latin9}
\begin{lstlisting}
>>> f = open("test.dat","r")
>>> print(f.read(7))
Ya es h
\end{lstlisting}
\inputencoding{utf8}
Si no quedan suficientes caracteres en el archivo, \texttt{read} devuelve
los que haya. Cuando llegamos al final del archivo, \texttt{read}
devuelve una cadena vacía:\inputencoding{latin9}
\begin{lstlisting}
>>> print(f.read(1000006))
orade cerrar el archivo
>>> print(f.read())

>>>
\end{lstlisting}
\inputencoding{utf8}
La siguiente función copia un archivo, leyendo y escribiendo los caracteres
de cincuenta en cincuenta. El primer argumento es el nombre del archivo
original; el segundo es el nombre del archivo nuevo:\inputencoding{latin9}
\begin{lstlisting}
def copiaArchivo(archViejo, archNuevo):
  f1 = open(archViejo, "r")
  f2 = open(archNuevo, "w")
  while True:
    texto = f1.read(50)
    if texto == "":
      break
    f2.write(texto)
  f1.close()
  f2.close()
  return
\end{lstlisting}
\inputencoding{utf8}
La sentencia \texttt{break} es nueva. Su ejecución interrumpe el ciclo;
el flujo de la ejecución pasa a la primera sentencia después del while.

\index{sentencia break} \index{sentencia!break}

En este ejemplo, el ciclo \texttt{while} es infinito porque la condición
\texttt{True} siempre es verdadera. La {\em única} forma de salir
del ciclo es ejecutar \texttt{break}, lo que sucede cuando \texttt{texto}
es una cadena vacía, y esto pasa cuando llegamos al final del archivo.

\section{Archivos de texto}

\index{archivo de texto} \index{archivo!texto}

Un \textbf{archivo de texto} contiene caracteres imprimibles y espacios
organizados en líneas separadas por caracteres de salto de línea.
Como Python está diseñado específicamente para procesar archivos de
texto, proporciona métodos que facilitan esta tarea.

Para hacer una demostración, crearemos un archivo de texto con tres
líneas de texto separadas por saltos de línea:\inputencoding{latin9}
\begin{lstlisting}
>>> f = open("test.dat","w")
>>> f.write("l�nea uno\nl�nea dos\nl�nea tres\n")
>>> f.close()
\end{lstlisting}
\inputencoding{utf8}
El método \texttt{readline} lee todos los caracteres hasta, e incluyendo,
el siguiente salto de línea:\inputencoding{latin9}
\begin{lstlisting}
>>> f = open("test.dat","r")
>>> print(f.readline())
l�nea uno

>>>
\end{lstlisting}
\inputencoding{utf8}
\texttt{readlines} devuelve todas las líneas que queden como una lista
de cadenas:\inputencoding{latin9}
\begin{lstlisting}
>>> print(f.readlines())
['l�nea dos\012', 'l�nea tres\012']
\end{lstlisting}
\inputencoding{utf8}
En este caso, la salida está en forma de lista, lo que significa que
las cadenas aparecen con comillas y el carácter de salto de línea
aparece como la secuencia de escape \texttt{012}.

Al final del archivo, \texttt{readline} devuelve una cadena vacía
y \texttt{readlines} devuelve una lista vacía:\inputencoding{latin9}
\begin{lstlisting}
>>> print(f.readline())

>>> print(f.readlines())
[]
\end{lstlisting}
\inputencoding{utf8}
Lo que sigue es un ejemplo de un programa de proceso de líneas. \texttt{filtraArchivo}
hace una copia de \texttt{archViejo}, omitiendo las líneas que comienzan
por \texttt{\#}:\inputencoding{latin9}
\begin{lstlisting}
def filtraArchivo(archViejo, archNuevo):
  f1 = open(archViejo, "r")
  f2 = open(archNuevo, "w")
  while 1:
    texto = f1.readline()
    if texto == "":
      break
    if texto[0] == '#':
      continue
    f2.write(texto)
  f1.close()
  f2.close()
  return
\end{lstlisting}
\inputencoding{utf8}
La sentencia \texttt{continue} termina la iteración actual del ciclo,
pero sigue haciendo las que le faltan. El flujo de ejecución pasa
al principio del ciclo, comprueba la condición y continúa normalmente.

\index{sentencia continue} \index{sentencia!continue}

Así, si \texttt{texto} es una cadena vacía, el ciclo termina. Si el
primer carácter de \texttt{texto} es una almohadilla \texttt{(\#)},
el flujo de ejecución va al principio del ciclo. Sólo si ambas condiciones
fallan copiamos \texttt{texto} en el archivo nuevo.

\section{Escribir variables}

\index{operador de formato} \index{cadena de formato} \index{operador!formato}

El argumento de \texttt{write} debe ser una cadena, así que si queremos
poner otros valores en un archivo, tenemos que convertirlos previamente
en cadenas. La forma más fácil de hacerlo es con la función \texttt{str}:\inputencoding{latin9}
\begin{lstlisting}
>>> x = 52
>>> f.write (str(x))
\end{lstlisting}
\inputencoding{utf8}
Una alternativa es usar el \textbf{operador de formato} \texttt{\%}.
Cuando aplica a enteros, \texttt{\%} es el operador de módulo. Pero
cuando el primer operando es una cadena, \texttt{\%} es el operador
de formato.

El primer operando es la \textbf{cadena de formato}, y el segundo,
una tupla de expresiones. El resultado es una cadena que contiene
los valores de las expresiones, formateados de acuerdo con la cadena
de formato.

A modo de ejemplo simple, la \textbf{secuencia de formato} {\verb+"%d"+}
significa que la primera expresión de la tupla debería formatearse
como un entero. Aquí la letra {\em d} quiere decir ``decimal'':\inputencoding{latin9}
\begin{lstlisting}
>>> motos = 52
>>> "%d" % motos
'52'
\end{lstlisting}
\inputencoding{utf8}
El resultado es la cadena \texttt{'52'}, que no debe confundirse con
el valor entero \texttt{52}.

Una secuencia de formato puede aparecer en cualquier lugar de la cadena
de formato, de modo que podemos incrustar un valor en una frase:\inputencoding{latin9}
\begin{lstlisting}
>>> motos = 52
>>> "En julio vendimos %d motos." % motos
'En julio vendimos 52 motos.'
\end{lstlisting}
\inputencoding{utf8}
La secuencia de formato {\verb+"%f"+} formatea el siguiente elemento
de la tupla como un número en punto flotante, y {\verb+"%s"+} formatea
el siguiente elemento como una cadena:\inputencoding{latin9}
\begin{lstlisting}
>>> "En %d dias ganamos %f millones de %s." % (4,1.2,'pesos')
'En 4 dias ganamos 1.200000 millones de pesos.'
\end{lstlisting}
\inputencoding{utf8}
Por defecto, el formato de punto flotante imprime seis decimales.

El número de expresiones en la tupla tiene que coincidir con el número
de secuencias de formato de la cadena. Igualmente, los tipos de las
expresiones deben coincidir con las secuencias de formato:

\index{error en tiempo de ejecución}\inputencoding{latin9}
\begin{lstlisting}
>>> "%d %d %d" % (1,2)
TypeError: not enough arguments for format string
>>> "%d" % 'dolares'
TypeError: illegal argument type for built-in operation
\end{lstlisting}
\inputencoding{utf8}
En el primer ejemplo no hay suficientes expresiones; en el segundo,
la expresión es de un tipo incorrecto.

Para tener más control sobre el formato de los números, podemos detallar
el número de dígitos como parte de la secuencia de formato:\inputencoding{latin9}
\begin{lstlisting}
>>> "%6d" % 62
'    62'
>>> "%12f" % 6.1
'    6.100000'
\end{lstlisting}
\inputencoding{utf8}
El número tras el signo de porcentaje es el número mínimo de espacios
que ocupará el número. Si el valor necesita menos dígitos, se añaden
espacios en blanco delante del número. Si el número de espacios es
negativo, se añaden los espacios tras el número:\inputencoding{latin9}
\begin{lstlisting}
>>> "%-6d" % 62
'62    '
\end{lstlisting}
\inputencoding{utf8}
También podemos especificar el número de decimales para los números
en coma flotante:\inputencoding{latin9}
\begin{lstlisting}
>>> "%12.2f" % 6.1
'        6.10'
\end{lstlisting}
\inputencoding{utf8}
En este ejemplo, el resultado ocupa doce espacios e incluye dos dígitos
tras la coma. Este formato es útil para imprimir cantidades de dinero
con las comas alineadas.

\index{diccionario}

Imagine, por ejemplo, un diccionario que contiene los nombres de los
estudiantes como clave y las tarifas horarias como valores. He aquí
una función que imprime el contenido del diccionario como de un informe
formateado:\inputencoding{latin9}
\begin{lstlisting}
def informe (tarifas) :
  estudiantes = tarifas.keys()
  estudiantes.sort()
  for estudiante in estudiantes :
    print("%-20s %12.02f"%(estudiante, tarifas[estudiante]))
\end{lstlisting}
\inputencoding{utf8}
Para probar la función, crearemos un pequeño diccionario e imprimiremos
el contenido:\inputencoding{latin9}
\begin{lstlisting}
>>> tarifas = {'maria': 6.23, 'jose': 5.45, 'jesus': 4.25}
>>> informe (tarifas)
jose                         5.45
jesus                        4.25
maria                        6.23
\end{lstlisting}
\inputencoding{utf8}
Controlando el ancho de cada valor nos aseguramos de que las columnas
van a quedar alineadas, siempre que los nombre tengan menos de veintiún
caracteres y las tarifas sean menos de mil millones la hora.

\section{Directorios}

\index{directorio}

Cuando se crea un archivo nuevo abriéndolo y escribiendo, este va
a quedar en el directorio en uso (aquél en el que se estuviese al
ejecutar el programa). Del mismo modo, cuando se abre un archivo para
leerlo, Python lo busca en el directorio en uso.

Si usted quiere abrir un archivo de cualquier otro sitio, tiene que
especificar la \textbf{ruta} del archivo, que es el nombre del directorio
(o carpeta) donde se encuentra este:\inputencoding{latin9}
\begin{lstlisting}
>>>   f = open("/usr/share/dict/words","r")
>>>   print(f.readline())
Aarhus
\end{lstlisting}
\inputencoding{utf8}
Este ejemplo abre un archivo denominado \texttt{words}, que se encuentra
en un directorio llamado \texttt{dict}, que está en \texttt{share},
en en \texttt{usr}, que está en el directorio de nivel superior del
sistema, llamado \texttt{/}.

\index{ruta} \index{delimitador}

No se puede usar \texttt{/} como parte del nombre de un archivo; está
reservado como delimitador entre nombres de archivo y directorios.

El archivo \texttt{/usr/share/dict/words} contiene una lista de palabras
en orden alfabético, la primera de las cuales es el nombre de una
universidad danesa.

\section{Encurtido}

\index{encurtido}

Para poner valores en un archivo, se deben convertir a cadenas. Usted
ya ha visto cómo hacerlo con \texttt{str}:\inputencoding{latin9}
\begin{lstlisting}
>>> f.write (str(12.3))
>>> f.write (str([1,2,3]))
\end{lstlisting}
\inputencoding{utf8} El problema es que cuando se vuelve a leer el valor, se obtiene una
cadena. Se ha perdido la información del tipo de dato original. En
realidad, no se puede distinguir dónde termina un valor y dónde comienza
el siguiente:

\inputencoding{latin9}\begin{lstlisting}
>>>   f.readline()
'12.3[1, 2, 3]'
\end{lstlisting}
\inputencoding{utf8} La solución es el \textbf{encurtido}, llamado así porque ``encurte''
estructuras de datos. El módulo \texttt{pickle} contiene las órdenes
necesarias. Para usarlo, se importa \texttt{pickle} y luego se abre
el archivo de la forma habitual:\inputencoding{latin9}
\begin{lstlisting}
>>> import pickle
>>> f = open("test.pck","w")
\end{lstlisting}
\inputencoding{utf8}
Para almacenar una estructura de datos, se usa el método \texttt{dump}
y luego se cierra el archivo de la forma habitual:\inputencoding{latin9}
\begin{lstlisting}
>>> pickle.dump(12.3, f)
>>> pickle.dump([1,2,3], f)
>>> f.close()
\end{lstlisting}
\inputencoding{utf8}
Ahora podemos abrir el archivo para leer y cargar las estructuras
de datos que volcamos ahí:

\inputencoding{latin9}\begin{lstlisting}
>>> f = open("test.pck","r")
>>> x = pickle.load(f)
>>> x
12.3
>>> type(x)
<type 'float'>
>>> y = pickle.load(f)
>>> y
[1, 2, 3]
>>> type(y)
<type 'list'>
\end{lstlisting}
\inputencoding{utf8} Cada vez que invocamos \texttt{load} obtenemos un valor del archivo
completo con su tipo original.

\section{Excepciones}

\index{sentencia try} \index{sentencia!try} \index{lanzar una excepción}
\index{manejar una excepción} \index{sentencia except} \index{sentencia!except}
\index{excepción}

Siempre que ocurre un error en tiempo de ejecución, se crea una \textbf{excepción}.
Normalmente el programa se para y Python presenta un mensaje de error.

Por ejemplo, la división por cero crea una excepción:\inputencoding{latin9}
\begin{lstlisting}
>>> print(55/0)
ZeroDivisionError: integer division or modulo
\end{lstlisting}
\inputencoding{utf8}
Un elemento no existente en una lista hace lo mismo:\inputencoding{latin9}
\begin{lstlisting}
>>> a = []
>>> print(a[5])
IndexError: list index out of range
\end{lstlisting}
\inputencoding{utf8}
O el acceso a una clave que no está en el diccionario:\inputencoding{latin9}
\begin{lstlisting}
>>> b = {}
>>> print(b['qu�'])
KeyError: qu�
\end{lstlisting}
\inputencoding{utf8}
En cada caso, el mensaje de error tiene dos partes: el tipo de error
antes de los dos puntos y detalles sobre el error después de los dos
puntos. Normalmente, Python también imprime una traza de dónde se
encontraba el programa, pero la hemos omitido en los ejemplos.

\index{traza}

A veces queremos realizar una operación que podría provocar una excepción,
pero no queremos que se pare el programa. Podemos \textbf{manejar}
la excepción usando las sentencias \texttt{try} y \texttt{except}.

Por ejemplo, podemos preguntar al usuario por el nombre de un archivo
y luego intentar abrirlo. Si el archivo no existe, no queremos que
el programa se aborte; queremos manejar la excepción.\inputencoding{latin9}
\begin{lstlisting}
nombreArch = input('Introduce un nombre de archivo: ')
try:
  f = open (nombreArch, "r")
except:
  print('No hay ning�n archivo que se llame', nombreArch)
\end{lstlisting}
\inputencoding{utf8}
La sentencia \texttt{try} ejecuta las sentencias del primer bloque.
Si no se produce ninguna excepción, pasa por alto la sentencia \texttt{except}.
Si ocurre cualquier excepción, ejecuta las sentencias de la rama \texttt{except}
y después continúa.

Podemos encapsular esta capacidad en una función: \texttt{existe},
que acepta un nombre de archivo y devuelve verdadero si el archivo
existe y falso si no:\inputencoding{latin9}
\begin{lstlisting}
def existe(nombreArch):
  try:
    f = open(nombreArch)
    f.close()
    return True
  except:
    return False
\end{lstlisting}
\inputencoding{utf8}Se pueden usar múltiples bloques \texttt{except} para manejar diferentes
tipos de excepciones. El {\em Manual de Referencia de Python} contiene
los detalles.

Si su programa detecta una condición de error, se puede lanzar (\textbf{raise}
en inglés) una excepción. Aquí hay un ejemplo que acepta una entrada
del usuario y comprueba si es 17. Suponiendo que 17 no es una entrada
válida por cualquier razón, lanzamos una excepción.\inputencoding{latin9}
\begin{lstlisting}
# -*- coding: utf-8 -*-
def tomaNumero () :                 
  x = input ('Elige un n�mero: ')   
  if x == 17 :
    raise 'ErrorNumeroMalo', '17 es un mal n�mero'
  return x
\end{lstlisting}
\inputencoding{utf8}
La sentencia \texttt{raise} acepta dos argumentos: el tipo de excepción
e información específica acerca del error. \texttt{ErrorNumeroMalo}
es un nuevo tipo de excepción que hemos inventado para esta aplicación.

Si la función llamada \texttt{tomaNumero} maneja el error, el programa
puede continuar; en caso contrario, Python imprime el mensaje de error
y sale:\inputencoding{latin9}
\begin{lstlisting}
>>> tomaNumero ()
Elige un n�mero: 17
ErrorNumeroMalo: 17 es un mal n�mero
\end{lstlisting}
\inputencoding{utf8}
El mensaje de error incluye el tipo de excepción y la información
adicional proporcionada.

\section{Glosario}

\index{archivo} \index{archivo de texto} \index{sentencia break}
\index{sentencia!break} \index{sentencia continue} \index{sentencia!continue}
\index{operador de formato} \index{cadena de formato} \index{operador!formato}
\index{directorio} \index{encurtido} \index{try} \index{lanzar excepción}
\index{manejar excepción} \index{sentencia except} \index{excepción}
\begin{description}
\item [{Archivo:}] entidad con nombre, normalmente almacenada en un disco
duro, disquete o CD-ROM, que contiene una secuencia de caracteres.
\item [{Directorio:}] colección de archivos, con nombre, también llamado
carpeta.
\item [{Ruta:}] secuencia de nombres de directorio que especifica la localización
exacta de un archivo.
\item [{Archivo de texto:}] un archivo que contiene caracteres imprimibles
organizados en líneas separadas por caracteres de salto de línea.
\item [{Sentencia break:}] es una sentencia que provoca que el flujo de
ejecución salga de un ciclo.
\item [{Sentencia continue:}] sentencia que provoca que termine la iteración
actual de un ciclo. El flujo de la ejecución va al principio del ciclo,
evalúa la condición, y procede en consecuencia.
\item [{Operador de formato:}] el operador \texttt{\%} toma una cadena
de formato y una tupla de expresiones y entrega una cadena que incluye
las expresiones, formateadas de acuerdo con la cadena de formato.
\item [{Cadena de formato:}] una cadena que contiene caracteres imprimibles
y secuencias de formato que indican cómo dar formato a valores.
\item [{Secuencia de formato:}] secuencia de caracteres que comienza con
\texttt{\%} e indica cómo dar formato a un valor.
\item [{Encurtir:}] escribir el valor de un dato en un archivo junto con
la información sobre su tipo de forma que pueda ser reconstituido
más tarde.
\item [{Excepción:}] error que ocurre en tiempo de ejecución.
\item [{Manejar:}] impedir que una excepción detenga un programa utilizando
las sentencias \texttt{except} y \texttt{try}.
\item [{Lanzar:}] causar una excepción usando la sentencia \texttt{raise}.
\end{description}

\section{Ejercicios}

Para cada función, agregue chequeo de tipos y pruebas unitarias.
\begin{enumerate}
\item Escriba una función que use \texttt{tomaNumero} para leer un número
del teclado y que maneje la excepción \texttt{ErrorNumeroMalo}
\item Escriba una función que reciba el nombre de un archivo y averigüe
si contiene una cadena de ADN válida. Es decir, cada carácter del
archivo es A,C,T, ó G.
\item Generalice la función anterior para trabajar con cualquier biosecuencia
(ARN, ADN, proteína).
\item Escriba una función que permita escribir una matriz, implementada
como una lista de listas, en un archivo.
\item Escriba una función que permita leer una matriz, implementada como
una lista de listas, de un archivo.
\item Escriba una función que permita escribir una matriz dispersa, implementada
con un diccionario, en un archivo.
\item Escriba una función que permita leer una matriz dispersa, implementada
con un diccionario, de un archivo.
\end{enumerate}


\clearemptydoublepage % archivos y excepciones

\chapter{Clases y objetos}

\index{clase} \index{objeto}

\section{Tipos compuestos definidos por el usuario}

\label{point} \index{tipo de datos compuestos} \index{tipo de datos!compuesto}
\index{tipo de datos definido por el usuario} \index{tipo de datos!definido por el usuario}
\index{constructor}

Una vez utilizados algunos de los tipos internos de Python, estamos
listos para crear un tipo definido por el usuario: el \texttt{Punto}.

Piense en el concepto de un punto matemático. En dos dimensiones,
un punto tiene dos números (coordenadas) que se tratan colectivamente
como un solo objeto. En notación matemática, los puntos suelen escribirse
entre paréntesis con una coma separando las coordenadas. Por ejemplo,
$(0,0)$ representa el origen, y $(x,y)$ representa el punto $x$
unidades a la derecha e $y$ unidades hacia arriba desde el origen.

Una forma natural de representar un punto en Python es con dos valores
en punto flotante. La cuestión es, entonces, cómo agrupar esos dos
valores en un objeto compuesto. La solución rápida y burda es utilizar
una lista o tupla, y para algunas aplicaciones esa podría ser la mejor
opción.

\index{coma flotante}

Una alternativa es que el usuario defina un nuevo tipo de dato compuesto,
también llamado una \textbf{clase}. Esta aproximación exige un poco
más de esfuerzo, pero tiene algunas ventajas que pronto se harán evidentes.

Una definición de clase se parece a esto:
\begin{pythoncode}
class Punto:
  pass
\end{pythoncode}

Las definiciones de clase pueden aparecer en cualquier lugar de un
programa, pero normalmente están al principio (tras las sentencias
\texttt{import}). Las reglas sintácticas de la definición de clases
son las mismas que para las otras sentencias compuestas. (ver la Sección~\ref{alternative execution}).

Esta definición crea una nueva clase llamada \texttt{Punto}. La sentencia
\textbf{pass} no tiene efectos; sólo es necesaria porque una sentencia
compuesta debe tener algo en su cuerpo.

Al crear la clase \texttt{Punto} hemos creado un nuevo tipo, que también
se llama \texttt{Punto}. Los miembros de este tipo se llaman \textbf{instancias}
del tipo u \textbf{objetos}. La creación de una nueva instancia se
llama \textbf{instanciación}. Para instanciar un objeto \texttt{Punto}
ejecutamos una función que se llama \texttt{Punto}:

\index{instancia!objeto} \index{instancia de un objeto} \index{instanciación}
\begin{pythoncode}
limpio = Punto()
\end{pythoncode}

A la variable \texttt{limpio} se le asigna una referencia a un nuevo
objeto \texttt{Punto}. A una función como \texttt{Punto} que crea
un objeto nuevo se le llama \textbf{constructor}.

\section{Atributos}

\index{atributos}

Podemos añadir nuevos datos a una instancia utilizando la notación
de punto:
\begin{pyconcode}
>>> limpio.x = 3.0
>>> limpio.y = 4.0
\end{pyconcode}

Esta sintaxis es similar a la sintaxis para seleccionar una variable
de un módulo, como \texttt{math.pi} o \texttt{string.uppercase}. En
este caso, sin embargo, estamos seleccionando un dato de una instancia.
Estos datos con nombre se denominan \textbf{atributos}.

El diagrama de estados que sigue muestra el resultado de esas asignaciones:

\beforefig \centerline{\includegraphics{illustrations/point}}
\afterfig

La variable \texttt{limpio} apunta a un objeto Punto, que contiene
dos atributos. Cada atributo apunta a un número en punto flotante.

Podemos leer el valor de un atributo utilizando la misma sintaxis:

\begin{pyconcode}
>>> print(limpio.y)
4.0
>>> x = limpio.x
>>> print(x)
3.0
\end{pyconcode}

La expresión \texttt{limpio.x} significa, ``ve al objeto al que apunta
\texttt{limpio} y toma el valor de \texttt{x}''. En este caso, asignamos
ese valor a una variable llamada \texttt{x}. No hay conflicto entre
la variable \texttt{x} y el atributo \texttt{x}. El propósito de la
notación punto es identificar de forma inequívoca a qué variable se
refiere el programador.

Se puede usar la notación punto como parte de cualquier expresión.
Así, las sentencias que siguen son correctas:
\begin{pythoncode}
print('(' + str(limpio.x) + ', ' + str(limpio.y) + ')')
distanciaAlCuadrado = limpio.x * limpio.x + 
                      limpio.y * limpio.y
\end{pythoncode}

La primera línea presenta \texttt{(3.0, 4.0)}; la segunda línea calcula
el valor 25.0.

Usted puede estar tentado a imprimir el propio valor de \texttt{limpio}:
\begin{pyconcode}
>>> print(limpio)
<__main__.Point instance at 80f8e70>
\end{pyconcode}

El resultado indica que \texttt{limpio} es una instancia de la clase
\texttt{Punto} que se definió en \texttt{\_\_main\_\_}. \texttt{80f8e70}
es el identificador único de este objeto, escrito en hexadecimal.
Probablemente ésta no es la manera más clara de mostrar un objeto
\texttt{Punto}. En breve veremos cómo cambiar esto.

\index{imprimir!objeto}

\section{Instancias como parámetro}

\index{instancia} \index{parámetro}

Se puede pasar una instancia como parámetro de la forma habitual.
Por ejemplo:
\begin{pyconcode}
def imprimePunto(p):
  print('(' + str(p.x) + ', ' + str(p.y) + ')')
\end{pyconcode}

\texttt{imprimePunto} acepta un punto como argumento y lo muestra
en el formato estándar de la matemática. Si llamas a \texttt{imprimePunto(limpio)},
el resultado es \texttt{(3.0, 4.0)}.

\section{Mismidad}

\index{mismidad}

El significado de la palabra ``mismo'' parece totalmente claro hasta
que uno se detiene a pensarlo un poco y se da cuenta de que hay algo
más de lo que se supone comúnmente.

\index{ambigüedad} \index{lenguaje natural} \index{lenguaje}

Por ejemplo, si alguien dice ``Pepe y yo tenemos la misma moto'',
lo que quiere decir es que su moto y la de Pepe son de la misma marca
y modelo, pero que son dos motos distintas. Si dice ``Pepe y yo tenemos
la misma madre'', quiere decir que su madre y la de Pepe son la misma
persona\footnote{No todas las lenguas tienen el mismo problema. Por ejemplo, el alemán
tiene palabras diferentes para los diferentes tipos de identidad.
``Misma moto'' en este contexto sería ``gleiche Motorrad'' y ``misma
madre'' sería ``selbe Mutter''.}. Así que la idea de ``identidad'' es diferente según el contexto.

Cuando uno habla de objetos, hay una ambigüedad parecida. Por ejemplo,
si dos \texttt{Puntos} son el mismo, ¿significa que contienen los
mismos datos (coordenadas) o que son de verdad el mismo objeto?

Para averiguar si dos referencias se refieren al mismo objeto, se
utiliza el operador \texttt{==}. Por ejemplo:
\begin{pyconcode}
>>> p1 = Punto()
>>> p1.x = 3
>>> p1.y = 4
>>> p2 = Punto()
>>> p2.x = 3
>>> p2.y = 4
>>> p1 == p2
False
\end{pyconcode}

Aunque \texttt{p1} y \texttt{p2} contienen las mismas coordenadas,
no son el mismo objeto. Si asignamos \texttt{p1} a \texttt{p2}, las
dos variables son alias del mismo objeto:
\begin{pyconcode}
>>> p2 = p1
>>> p1 == p2
True
\end{pyconcode}

Este tipo de igualdad se llama \textbf{igualdad superficial}, porque
sólo compara las referencias, pero no el contenido de los objetos.

\index{igualdad} \index{identidad} \index{igualdad superficial}
\index{igualdad profunda}

Para comparar los contenidos de los objetos (\textbf{igualdad profunda})
podemos escribir una función llamada \texttt{mismoPunto}:

\begin{pythoncode}
def mismoPunto(p1, p2) :
  return (p1.x == p2.x) and (p1.y == p2.y)
\end{pythoncode}
 Si ahora creamos dos objetos diferentes que contienen los mismos
datos podremos usar \texttt{mismoPunto} para averiguar si representan
el mismo punto:

\begin{pyconcode}
>>> p1 = Punto()
>>> p1.x = 3
>>> p1.y = 4
>>> p2 = Punto()
>>> p2.x = 3
>>> p2.y = 4
>>> mismoPunto(p1, p2)
True
\end{pyconcode}
 Por supuesto, si las dos variables apuntan al mismo objeto \texttt{mismoPunto}
devuelve verdadero.

\section{Rectángulos}

\label{embedded} \index{rectángulo}

Digamos que queremos una clase que represente un rectángulo. La pregunta
es, ¿qué información tenemos que proporcionar para definir un rectángulo?
Para simplificar las cosas, supongamos que el rectángulo está orientado
vertical u horizontalmente, nunca en diagonal.

Tenemos varias posibilidades: podemos señalar el centro del rectángulo
(dos coordenadas) y su tamaño (ancho y altura); o podemos señalar
una de las esquinas y el tamaño; o podemos señalar dos esquinas opuestas.
Un modo convencional es señalar la esquina superior izquierda del
rectángulo y el tamaño.

De nuevo, definiremos una nueva clase:
\begin{pythoncode}
class Rectangulo:	
  pass
\end{pythoncode}

Y la instanciaremos:
\begin{pythoncode}
caja = Rectangulo()
caja.ancho = 100.0
caja.altura = 200.0
\end{pythoncode}

Este código crea un nuevo objeto \texttt{Rectangulo} con dos atributos
flotantes. ¡Para señalar la esquina superior izquierda podemos incrustar
un objeto dentro de otro!
\begin{pythoncode}
caja.esquina = Punto()
caja.esquina.x = 0.0;
caja.esquina.y = 0.0;
\end{pythoncode}

El operador punto compone. La expresión \texttt{caja.esquina.x} significa
``ve al objeto al que se refiere \texttt{caja} y selecciona el atributo
llamado \texttt{esquina}; entonces ve a ese objeto y selecciona el
atributo llamado x''.

La figura muestra el estado de este objeto:

\beforefig \centerline{\includegraphics{illustrations/rectangle}}
\afterfig

\section{Instancias como valores de retorno}

\index{instancia} \index{valor de retorno}

Las funciones pueden devolver instancias. Por ejemplo, \texttt{encuentraCentro}
acepta un \texttt{Rectangulo} como argumento y devuelve un \texttt{Punto}
que contiene las coordenadas del centro del \texttt{Rectangulo}:
\begin{pythoncode}
def encuentraCentro(caja):
  p = Punto()
  p.x = caja.esquina.x + caja.ancho/2.0
  p.y = caja.esquina.y + caja.altura/2.0
  return p
\end{pythoncode}

Para llamar a esta función, se pasa \texttt{caja} como argumento y
se asigna el resultado a una variable:
\begin{pyconcode}
>>> centro = encuentraCentro(caja)
>>> imprimePunto(centro)
(50.0, 100.0)
\end{pyconcode}

\section{Los objetos son mutables}

\index{objeto!mutable} \index{objeto mutable}

Podemos cambiar el estado de un objeto efectuando una asignación sobre
uno de sus atributos. Por ejemplo, para cambiar el tamaño de un rectángulo
sin cambiar su posición, podemos cambiar los valores de \texttt{ancho}
y \texttt{altura}:
\begin{pythoncode}
caja.ancho = caja.ancho + 50
caja.altura = caja.altura + 100
\end{pythoncode}

Podemos encapsular este código en un método y generalizarlo para agrandar
el rectángulo en cualquier cantidad:

\index{encapsulamiento} \index{generalización}
\begin{pythoncode}
def agrandarRect(caja, dancho, daltura) :
  caja.ancho = caja.ancho + dancho
  caja.altura = caja.altura + daltura
\end{pythoncode}

Las variables \texttt{dancho} y \texttt{daltura} indican cuánto debe
agrandarse el rectángulo en cada dirección. Invocar este método tiene
el efecto de modificar el \texttt{Rectangulo} que se pasa como argumento.

Por ejemplo, podemos crear un nuevo \texttt{Rectangulo} llamado \texttt{b}
y pasárselo a la función \texttt{agrandarRect}:
\begin{pyconcode}
>>> b = Rectangulo()
>>> b.ancho = 100.0
>>> b.altura = 200.0
>>> b.esquina = Punto()
>>> b.esquina.x = 0.0;
>>> b.esquina.y = 0.0;
>>> agrandarRect(b, 50, 100)
\end{pyconcode}

Mientras \texttt{agrandarRect} se está ejecutando, el parámetro \texttt{caja}
es un alias de \texttt{b}. Cualquier cambio que se haga a \texttt{caja}
afectará también a \texttt{b}.

\section{Copiado}

\index{uso de alias} \index{copiado} \index{módulo copy} \index{módulo!copy}

El uso de un alias puede hacer que un programa sea difícil de leer,
porque los cambios hechos en un lugar pueden tener efectos inesperados
en otro lugar. Es difícil estar al tanto de todas las variables que
pueden apuntar a un objeto dado.

Copiar un objeto es, muchas veces, una alternativa a la creación de
un alias. El módulo \texttt{copy} contiene una función llamada \texttt{copy}
que puede duplicar cualquier objeto:
\begin{pyconcode}
>>> import copy
>>> p1 = Punto()
>>> p1.x = 3
>>> p1.y = 4
>>> p2 = copy.copy(p1)
>>> p1 == p2
False
>>> mismoPunto(p1, p2)
True
\end{pyconcode}

Una vez que hemos importado el módulo \texttt{copy}, podemos usar
el método \texttt{copy} para hacer un nuevo \texttt{Punto}. \texttt{p1}
y \texttt{p2} no son el mismo punto, pero contienen los mismos datos.

Para copiar un objeto simple como un \texttt{Punto}, que no contiene
objetos incrustados, \texttt{copy} es suficiente. Esto se llama \textbf{copiado
superficial}.

Para algo como un \texttt{Rectangulo}, que contiene una referencia
a un \texttt{Punto}, \texttt{copy} no lo hace del todo bien. Copia
la referencia al objeto \texttt{Punto}, de modo que tanto el \texttt{Rectangulo}
viejo como el nuevo apuntan a un único \texttt{Punto}.

Si creamos una caja, \texttt{b1}, de la forma habitual y entonces
hacemos una copia, \texttt{b2}, usando \texttt{copy}, el diagrama
de estados resultante se ve así:

\beforefig \centerline{\includegraphics{illustrations/rectangle2}}
\afterfig

Es casi seguro que esto no es lo que queremos. En este caso, la invocación
de \texttt{agrandaRect} sobre uno de los \texttt{Rectangulo}s no afectaría
al otro, ¡pero la invocación de \texttt{mueveRect} sobre cualquiera
afectaría a ambos! Este comportamiento es confuso y propicia los errores.

Afortunadamente, el módulo \texttt{copy} contiene un método llamado
\texttt{deepcopy} que copia no sólo el objeto, sino también cualesquiera
objetos incrustados en él. No lo sorprenderá saber que esta operación
se llama \textbf{copia profunda} (deep copy).

\begin{pyconcode}
>>> b2 = copy.deepcopy(b1)
\end{pyconcode}
 Ahora \texttt{b1} y \texttt{b2} son objetos totalmente independientes.

Podemos usar \texttt{deepcopy} para reescribir \texttt{agrandaRect}
de modo que en lugar de modificar un \texttt{Rectangulo} existente,
cree un nuevo \texttt{Rectangulo} que tiene la misma localización
que el viejo pero nuevas dimensiones:

\begin{pythoncode}
def agrandaRect(caja, dancho, daltura) :
  import copy
  nuevaCaja = copy.deepcopy(caja)
  nuevaCaja.ancho = nuevaCaja.ancho + dancho
  nuevaCaja.altura = nuevaCaja.altura + daltura
  return nuevaCaja
\end{pythoncode}
 

\section{Glosario}
\begin{description}
\item [{Clase:}] tipo compuesto definido por el usuario. También se puede
pensar en una clase como una plantilla para los objetos que son instancias
de la misma.
\item [{Instanciar:}] Crear una instancia de una clase.
\item [{Instancia:}] objeto que pertenece a una clase.
\item [{Objeto:}] tipo de dato compuesto que suele usarse para representar
una cosa o concepto del mundo real.
\item [{Constructor:}] método usado para crear nuevos objetos.
\item [{Atributo:}] uno de los elementos de datos con nombre que constituyen
una instancia.
\item [{Igualdad superficial:}] igualdad de referencias, o dos referencias
que apuntan al mismo objeto.
\item [{Igualdad profunda:}] igualdad de valores, o dos referencias que
apuntan a objetos que tienen el mismo valor.
\item [{Copia superficial:}] copiar el contenido de un objeto, incluyendo
cualquier referencia a objetos incrustados; implementada por la función
\texttt{copy} del módulo \texttt{copy}.
\item [{Copia profunda:}] copiar el contenido de un objeto así como cualesquiera
objetos incrustados, y los incrustados en estos, y así sucesivamente.
Está implementada en la función \texttt{deepcopy} del módulo \texttt{copy}.

\index{clase} \index{instanciar} \index{instancia} \index{objeto}
\index{constructor} \index{atributo} \index{igualdad superficial}
\index{igualdad profunda} \index{copia superficial} \index{copia profunda}
\end{description}

\section{Ejercicios}
\begin{enumerate}
\item Cree e imprima un objeto \texttt{Punto} y luego use \texttt{id} para
imprimir el identificador único del objeto. Traduzca el número hexadecimal
a decimal y asegúrese de que coincidan.
\item Reescriba la función \texttt{distancia} de la Sección~\ref{program development}
de forma que acepte dos \texttt{Puntos} como parámetros en lugar de
cuatro números.
\item Escriba una función llamada \texttt{mueveRect} que tome un \texttt{Rectangulo}
y dos parámetros llamados \texttt{dx} y \texttt{dy}. Tiene que cambiar
la posición del rectángulo añadiendo en la \texttt{esquina}: \texttt{dx}
a la coordenada \texttt{x} y \texttt{dy} a la coordenada \texttt{y}.
\item Reescriba \texttt{mueveRect} de modo que cree y devuelva un nuevo
\texttt{Rectangulo} en lugar de modificar el viejo.
\item Investigue la documentación del módulo Turtle de Python. Escriba una
función que dibuje un rectángulo con la tortuga.
\item Escriba un programa que dibuje dos tortugas haciendo un movimiento
circular permanente.Pista: Piense en como intercalar los movimientos
de las dos tortugas.
\end{enumerate}


\clearemptydoublepage % clases y objetos

\chapter{Clases y funciones}

\label{time} \index{función} \index{método}

\section{Hora}

Como otro ejemplo de tipo de dato definido por el usuario definiremos
una clase llamada \texttt{Hora}:
\begin{lstlisting}
class Hora:
  pass
\end{lstlisting}

Ahora podemos crear un nuevo objeto \texttt{Hora} y asignarle atributos
para las horas, minutos y segundos:
\begin{lstlisting}
tiempo = Hora()
tiempo.hora = 11
tiempo.minutos = 59
tiempo.segundos = 30
\end{lstlisting}

El diagrama para el objeto \texttt{Hora} luce así:

\beforefig \centerline{\includegraphics{illustrations/time}} \afterfig

\section{Funciones puras}

\index{función pura} \index{tipo función!pura}

En las siguientes secciones escribiremos dos versiones de una función
denominada \texttt{sumarHoras}, que calcule la suma de dos \texttt{Horas}.
Esto demostrará dos clases de funciones: las puras y los modificadores.

La siguiente es una versión de \texttt{sumarHoras}:

\begin{lstlisting}
def sumarHoras(t1, t2):
  sum = Hora()
  sum.hora = t1.hora + t2.hora
  sum.minutos = t1.minutos + t2.minutos
  sum.segundos = t1.segundos + t2.segundos
  return sum
\end{lstlisting}
 La función crea un nuevo objeto \texttt{Hora}, inicializa sus atributos
y retorna una referencia hacia el nuevo objeto. Esto se denomina \textbf{función
pura}, porque no modifica ninguno de los objetos que se le pasaron
como parámetro ni tiene efectos laterales, como desplegar un valor
o leer entrada del usuario.

Aquí hay un ejemplo de uso de esta función. Crearemos dos objetos
\texttt{Hora}: \texttt{horaPan}, que contiene el tiempo que le toma
a un panadero hacer pan y \texttt{horaActual}, que contiene la hora
actual. Luego usaremos \texttt{sumarHoras} para averiguar a qué hora
estará listo el pan. Si no ha terminado la función \texttt{imprimirHora}
aún, adelántese a la Sección \ref{printTime} antes de intentar esto:

\begin{lstlisting}
>>> horaActual = Hora()
>>> horaActual.hora = 9
>>> horaActual.minutos = 14
>>> horaActual.segundos =  30

>>> horaPan = Hora()
>>> horaPan.hora =  3
>>> horaPan.minutos =  35
>>> horaPan.segundos =  0

>>> horaComer = sumarHoras(horaActual, horaPan)
>>> imprimirHora(horaComer)
\end{lstlisting}
 La salida de este programa es \texttt{12:49:30}, que está correcta.
Por otro lado, hay casos en los que no funciona bien. ¿Puede pensar
en uno?

El problema radica en que esta función no considera los casos donde
el número de segundos o minutos suman más de sesenta. Cuando eso ocurre
tenemos que ``acarrear'' los segundos extra a la columna de minutos.
También puede pasar lo mismo con los minutos.

Aquí hay una versión correcta:
\begin{lstlisting}
def sumarHoras(t1, t2):
  sum = Hora()
  sum.hora = t1.hora + t2.hora
  sum.minutos = t1.minutos + t2.minutos
  sum.segundos = t1.segundos + t2.segundos

  if sum.segundos >= 60:
    sum.segundos = sum.segundos - 60
    sum.minutos = sum.minutos + 1

  if sum.minutos >= 60:
    sum.minutos = sum.minutos - 60
    sum.hora = sum.hora + 1

  return sum
\end{lstlisting}

Aunque ahora ha quedado correcta, ha empezado a agrandarse. Más adelante
sugeriremos un enfoque alternativo que produce un código más corto.

\section{Modificadoras}

\label{increment} \index{modificadora} \index{tipo función!modificadora}

A veces es deseable que una función modifique uno o varios de los
objetos que recibe como parámetros. Usualmente, el código que hace
el llamado a la función conserva una referencia a los objetos que
está pasando, así que cualquier cambio que la función les haga será
evidenciado por dicho código. Este tipo de funciones se denominan
\textbf{modificadoras}.

\texttt{incrementar}, que agrega un número de segundos a un objeto
\texttt{Hora}, se escribiría más naturalmente como función modificadora.
Un primer acercamiento a la función luciría así:
\begin{lstlisting}
def incrementar(h, segundos):
  h.segundos = h.segundos + segundos

  if h.segundos >= 60:
    h.segundos = h.segundos - 60
    h.minutos = h.minutos + 1

  if h.minuto >= 60:
    h.minutos = h.minutos - 60
    h.hora = h.hora + 1

  return h
\end{lstlisting}

La primera línea ejecuta la operación básica, las siguientes consideran
los casos especiales que ya habíamos visto.

¿Es correcta esta función? ¿Que pasa si el parámetro \texttt{segundos}
es mucho más grande que sesenta? En ese caso, no sólo es suficiente
añadir uno, tenemos que sumar de uno en uno hasta que \texttt{segundos}
sea menor que sesenta. Una solución consiste en reemplazar las sentencias
\texttt{if} por sentencias \texttt{while}:
\begin{lstlisting}
def incrementar(hora, segundos):
  hora.segundos = hora.segundos + segundos

  while hora.segundos >= 60:
    hora.segundos = hora.segundos - 60
    hora.minutos = hora.minutos + 1

  while hora.minutos >= 60:
    hora.minutos = hora.minutos - 60
    hora.hora = hora.hora + 1

  return hora

  time.segundos = time.segundos + segundos
\end{lstlisting}

Ahora, la función sí es correcta, aunque no sigue el proceso más eficiente.

\section{¿Cual es el mejor estilo?}

\index{estilo de programación funcional}

Todo lo que puede hacerse con modificadoras también se puede hacer
con funciones puras. De hecho, algunos lenguajes de programación sólo
permiten funciones puras. La evidencia apoya la tesis de que los programas
que usan solamente funciones puras se desarrollan más rápido y son
menos propensos a errores que los programas que usan modificadoras.
Sin embargo, las funciones modificadoras, a menudo, son convenientes
y, a menudo, los programas funcionales puros son menos eficientes.

En general, le recomendamos que escriba funciones puras cada vez que
sea posible y recurrir a las modificadoras solamente si hay una ventaja
en usar este enfoque. Esto se denomina un \textbf{estilo de programación
funcional}.

\section{Desarrollo con prototipos vs. planificación}

\label{convert} \index{desarrollo con prototipos}

En este capítulo mostramos un enfoque de desarrollo de programas que
denominamos \textbf{desarrollo con prototipos}. Para cada problema
escribimos un bosquejo (o prototipo) que ejecutará el cálculo básico
y lo probará en unos cuantos casos de prueba, corrigiendo errores
a medida que surgen.

Aunque este enfoque puede ser efectivo, puede conducirnos a código
innecesariamente complicado —ya que considera muchos casos especiales—y
poco confiable—ya que es difícil asegurar que hemos descubierto todos
los errores.

Una alternativa es el \textbf{desarrollo planificado}, en el que la
profundización en el dominio del problema puede darnos una comprensión
profunda que facilita bastante la programación. En el caso anterior,
comprendimos que un objeto \texttt{Hora} realmente es un número de
tres dígitos en base 60! El componente \texttt{segundos} contiene
las ``unidades,'' el componente \texttt{minutos} la ``columna de
sesentas,'' y el componente \texttt{hora} contiene la ``columna
de tres mil seiscientos.''

Cuando escribimos \texttt{sumarHoras} e \texttt{incrementar}, realmente
estábamos haciendo una suma en base 60, razón por la cual teníamos
que efectuar un acarreo de una columna a la siguiente.

Esta observación sugiere otro enfoque al problema—podemos convertir
un objeto \texttt{Hora} en un número único y aprovecharnos del hecho
de que el computador sabe realizar aritmética. La siguiente función
convierte un objeto \texttt{Hora} en un entero:
\begin{lstlisting}
def convertirASegundos(t):
  minutos = t.hora * 60 + t.minutos
  segundos = minutos * 60 + t.segundos
  return segundos
\end{lstlisting}

Ahora necesitamos una forma de convertir desde entero a un objeto
\texttt{Hora}:
\begin{lstlisting}
def crearHora(segundos):
  h = Hora()
  h.hora = segundos/3600
  segundos = segundos - h.hora * 3600
  h.minutos = segundos/60
  segundos = segundos - h.minutos * 60
  h.segundos = segundos
  return h
\end{lstlisting}

Usted debe pensar unos minutos para convencerse de que esta técnica
sí convierte, de una base a otra, correctamente. Asumiendo que ya
está convencido, se pueden usar las funciones anteriores para reescribir
\texttt{sumarHoras}:

\begin{lstlisting}
def sumarHoras(t1, t2):
  segundos = convertirASegundos(t1) + convertirASegundos(t2)
  return crearHora(segundos)
\end{lstlisting}
 Esta versión es mucho más corta que la original, y es mucho más fácil
de demostrar que es correcta (asumiendo, como de costumbre, que las
funciones que llama son correctas).

\section{Generalización}

\index{generalización}

Desde cierto punto de vista, convertir de base 60 a base 10 y viceversa
es más difícil que calcular solamente con horas. La conversión de
bases es más abstracta, mientras que nuestra intuición para manejar
horas está más desarrollada.

Pero si tenemos la intuición de tratar las horas como números en base
60 y hacemos la inversión de escribir las funciones de conversión
(\texttt{convertirASegundos} y \texttt{crearHora}), obtenemos un programa
más corto, legible, depurable y confiable.

También facilita la adición de más características. Por ejemplo, piense
en el problema de restar dos \texttt{Hora}s para averiguar el tiempo
que transcurre entre ellas. La solución ingenua haría resta llevando
préstamos. En cambio, usar las funciones de conversión sería mas fácil.

Irónicamente, algunas veces el hacer de un problema algo más difícil
(o más general) lo hace más fácil (porque hay menos casos especiales
y menos oportunidades para caer en errores).

\section{Algoritmos}

\index{algoritmo}

Cuando usted escribe una solución general para una clase de problemas,
en vez de encontrar una solución específica a un solo problema, ha
escrito un \textbf{algoritmo}. Mencionamos esta palabra antes, pero
no la definimos cuidadosamente. No es fácil de definir, así que intentaremos
dos enfoques.

Primero, considere algo que no es un algoritmo. Cuando usted aprendió
a multiplicar dígitos, probablemente memorizó la tabla de multiplicación.
De hecho, usted memorizó 100 soluciones específicas. Este tipo de
conocimiento no es algorítmico.

Pero si usted fuera ``perezoso,'' probablemente aprendió a hacer
trampa por medio de algunos trucos. Por ejemplo, para encontrar el
producto entre $n$ y 9, usted puede escribir $n-1$ como el primer
dígito y $10-n$ como el segundo. Este truco es una solución general
para multiplicar cualquier dígito por el 9. ¡Este es un algoritmo!

Similarmente, las técnicas que aprendió para hacer suma con acarreo
( llevando para la columna hacia la derecha), resta con préstamos,
y división larga, todas son algoritmos. Una de las características
de los algoritmos es que no requieren inteligencia para ejecutarse.
Son procesos mecánicos en el que cada paso sigue al anterior de acuerdo
con un conjunto de reglas sencillas.

En nuestra opinión, es vergonzoso que los seres humanos pasemos tanto
tiempo en la escuela aprendiendo a ejecutar algoritmos que, literalmente,
no requieren inteligencia.

Por otro lado, el proceso de diseñar algoritmos es interesante, intelectualmente
desafiante y una parte central de lo que denominamos programación.

Algunas cosas que la gente hace naturalmente sin dificultad o pensamiento
consciente, son las mas difíciles de expresar algorítmicamente. Entender
el lenguaje natural es una de ellas. Todos lo hacemos, pero hasta
ahora nadie ha sido capaz de explicar {\em como} lo hacemos, al
menos no con un algoritmo.

\section{Glosario}
\begin{description}
\item [{Función pura:}] función que no modifica ninguno de los objetos
que recibe como parámetros. La mayoría de las funciones puras son
fructíferas.
\item [{Modificadora:}] función que cambia uno o varios de los objetos
que recibe como parámetros. La mayoría de los modificadoras no retornan
nada.
\item [{Estilo de programación funcional}] estilo de diseño de programas
en el que la mayoría de funciones son puras.
\item [{Desarrollo con prototipos:}] es la forma de desarrollar programas
empezando con un prototipo que empieza a mejorarse y probarse gradualmente.
\item [{Desarrollo planeado:}] es la forma de desarrollar programas que
implica un conocimiento de alto nivel sobre el problema y mas planeación
que el desarrollo incremental o el desarrollo con prototipos.
\item [{Algoritmo:}] conjunto de instrucciones para resolver una clase
de problemas por medio de un proceso mecánico, no inteligente.

\index{función pura} \index{modificadora} \index{estilo de programación funcional}
\index{desarrollo incremental} \index{desarrollo!incremental} \index{desarrollo planeado}
\index{desarrollo!planeado} \index{algoritmo}
\end{description}

\section{Ejercicios}
\begin{enumerate}
\item Reescriba la función \texttt{incrementar} de forma que no contenga
ciclos y siga siendo correcta.
\item Reescriba \texttt{incrementar} usando convertirASegundos y crearHora.
\item Reescriba \texttt{incrementar} como una función pura, y escriba llamados
a funciones de las dos versiones.
\item Escriba una función \texttt{imprimirHora} que reciba un objeto \texttt{Hora}
como argumento y lo imprima de la forma \texttt{horas:minutos:segundos}.
\item Escriba una función booleana \texttt{despues} que reciba dos objetos
\texttt{Hora}, \texttt{t1} y \texttt{t2} como argumentos, y retorne
cierto si \texttt{t1} va después de \texttt{t2} cronológicamente y
falso en caso contrario.
\end{enumerate}


\clearemptydoublepage % clases y funciones

\chapter{Clases y métodos}

\section{Características de orientación a objetos}

\index{lenguaje de programación orientado a objetos} \index{programación orientada a objetos}

Python es un \textbf{lenguaje de programación orientado a objetos},
lo que quiere decir que proporciona características que soportan la
\textbf{programación orientada a objetos}.

No es fácil definir la programación orientada a objetos, pero ya hemos
notado algunos de sus elementos clave:
\begin{itemize}
\item Los programas se construyen a partir de definiciones de objetos y
definiciones de funciones; la mayoría de los cómputos se hacen con
base en objetos.
\item Cada definición de objetos corresponde a algún concepto o cosa del
mundo real, y las funciones que operan sobre esos objetos corresponden
a las maneras en que los conceptos o cosas reales interactúan.
\end{itemize}
Por ejemplo, la clase \texttt{Hora}, definida en el Capítulo~\ref{time},
corresponde a la forma en que la gente registra las horas del día
y las funciones que definimos corresponden a la clase de cosas que
la gente hace con horas. Similarmente, las clases \texttt{Punto} y
\texttt{Rectangulo} corresponden a los conocidos conceptos geométricos

Hasta aquí, no hemos aprovechado las características que Python proporciona
para soportar la programación orientada a objetos. De hecho, estas
características no son necesarias. La mayoría sólo proporciona una
sintaxis alternativa para cosas que ya hemos logrado; pero, en muchos
casos, esta forma alternativa es más concisa y comunica de una manera
mas precisa la estructura de los programas.

Por ejemplo, en el programa \texttt{Hora} no hay una conexión obvia
entre la definición de clase y las definiciones de funciones. Después
de examinarlo un poco, es evidente que todas las funciones toman como
parámetro al menos un objeto \texttt{Hora}.

Esta observación es la motivación para los \textbf{métodos}. Ya hemos
visto algunos métodos como \texttt{keys} y \texttt{values}, que llamamos
sobre diccionarios. Cada método se asocia con una clase y está pensado
para invocarse sobre instancias de dicha clase.

\index{método} \index{función} \index{instancia!objeto} \index{objeto instancia}

Los métodos son como las funciones, pero con dos diferencias:
\begin{itemize}
\item Los métodos se definen adentro de una definición de clase, a fin de
marcar explícitamente la relación entre la clase y éstos.
\item La sintaxis para llamar o invocar un método es distinta que para las
funciones.
\end{itemize}
En las siguientes secciones tomaremos las funciones de los capítulos
anteriores y las transformaremos en métodos. Esta transformación es
totalmente mecánica; se puede llevar a cabo siguiendo una secuencia
de pasos. Si usted se siente cómodo al transformar de una forma a
la otra, será capaz de escoger lo mejor de cada lado para resolver
los problemas que tenga a la mano.

\section{\texttt{imprimirHora}}

\label{printTime} \index{imprimir!objetos}

En el capítulo~\ref{time}, definimos una clase denominada \texttt{Hora}
y usted escribió una función denominada \texttt{imprimirHora}, que
lucía así:
\begin{pythoncode}
class Hora:
  pass

def imprimirHora(h):
  print(str(h.hora) + ":" + 
        str(h.minutos) + ":" + 
        str(h.segundos))
\end{pythoncode}

Para llamar esta función, le pasamos un objeto \texttt{Hora} como
parámetro:
\begin{pyconcode}
>>> horaActual = Hora()
>>> horaActual.hora = 9
>>> horaActual.minutos = 14
>>> horaActual.segundos = 30
>>> imprimirHora(horaActual)
\end{pyconcode}

Para convertir \texttt{imprimirHora} en un método todo lo que tenemos
que hacer es ponerla adentro de la definición de clase. Note como
ha cambiado la indentación.

\begin{pythoncode}
class Hora:
  def imprimirHora(h):
    print( str(h.hora) + ":" + 
          str(h.minutos) + ":" + 
          str(h.segundos))
\end{pythoncode}
 Ahora podemos llamar a \texttt{imprimirHora} usando la notación punto.

\index{notación punto}
\begin{pyconcode}
>>> horaActual.imprimirHora()
\end{pyconcode}

Como de costumbre, el objeto en el que el método se llama aparece
antes del punto y el nombre del método va a la derecha. El objeto
al cual se invoca el método se asigna al primer parámetro, así que
\texttt{horaActual} se asigna al parámetro \texttt{h}.

Por convención, el primer parámetro de un método se denomina \texttt{self}
(en inglés, eso es algo como ``sí mismo''). La razón para hacerlo
es un poco tortuosa, pero se basa en una metáfora muy útil.

La sintaxis para una llamada de función, \texttt{imprimirHora(horaActual)},
sugiere que la función es el agente activo. Dice algo como ``Hey,
\texttt{imprimirHora}! Aquí hay un objeto para que imprimas''.

En la programación orientada a objetos, los objetos son los agentes
activos. Una invocación como \texttt{horaActual.imprimirHora()} dice
algo como ``Hey, objeto \texttt{horaActual}! Por favor, imprímase
a sí mismo!''.

Este cambio de perspectiva parece ser sólo ``cortesía'', pero puede
ser útil. En los ejemplos que hemos visto no lo es. Pero, el transferir
la responsabilidad desde las funciones hacia los objetos hace posible
escribir funciones más versátiles y facilita la reutilización y el
mantenimiento de código.

\section{Otro ejemplo}

Convirtamos \texttt{incrementar} (de la Sección~\ref{increment})
en un método. Para ahorrar espacio, omitiremos los métodos que ya
definimos, pero usted debe conservarlos en su programa:

\pagebreak

\begin{pythoncode}
class Hora:
  # Las definiciones anteriores van aquí...
  
  def incrementar(self, segundos):
    self.segundos = self.segundos + segundos

    if self.segundos >= 60:
      self.segundos = self.segundos - 60
      self.minutos = self.minutos + 1

    if self.minutos >= 60:
      self.minutos = self.minutos - 60
      self.hora = self.hora + 1

    return self
\end{pythoncode}
 La transformación es totalmente mecánica —ponemos la definición del
método adentro de la clase y cambiamos el nombre del primer parámetro.

Ahora podemos llamar a \texttt{incrementar} como método:
\begin{pythoncode}
horaActual.incrementar(500)
\end{pythoncode}

Nuevamente, el objeto con el cual se invoca el método se asigna al
primer parámetro, \texttt{self}. El segundo parámetro, \texttt{segundos}
recibe el valor \texttt{500}.

\section{Un ejemplo más complejo}

El método \texttt{despues} es un poco más complejo ya que opera sobre
dos objetos \texttt{Hora}, no sólo uno. Solamente podemos convertir
uno de los parámetros a \texttt{self}; el otro continúa igual:

\pagebreak

\begin{pythoncode}
class Hora:
  # Las definiciones anteriores van aqui...

  def despues(self, hora2):
    if self.hora > hora2.hora:
      return True
    if self.hora < hora2.hora:
      return False

    if self.minutos > hora2.minutos:
      return True
    if self.minutos < hora2.minutos:
      return False

    if self.segundos > hora2.segundos:
      return True
    return False
\end{pythoncode}
 Llamamos a este método sobre un objeto y le pasamos el otro como
argumento:
\begin{pythoncode}
if horaComer.despues(horaActual):
  print("El pan estará listo para comer en un momento.")
\end{pythoncode}

Casi se puede leer el llamado en lenguaje natural:``Si la hora para
Comer viene después de la hora Actual, entonces ...''.

\section{Argumentos opcionales}

Hemos visto varias funciones primitivas que toman un número variable
de argumentos. Por ejemplo, \texttt{string.find} puede tomar dos,
tres o cuatro.

Es posible escribir funciones con listas de argumentos opcionales.
Por ejemplo, podemos mejorar nuestra versión de \texttt{buscar} para
que sea tan sofisticada como \texttt{string.find}.

Esta es la versión original que introdujimos en la Sección~\ref{find}:

\pagebreak{}

\begin{pythoncode}
def buscar(cad, c):
  indice = 0
  while indice < len(cad):
    if cad[indice] == c:
      return indice
    indice = indice + 1
  return -1
\end{pythoncode}
 Esta es la nueva versión, mejorada:
\begin{pythoncode}
def buscar(cad, c,ini=0):
  indice = ini
  while indice < len(cad):
    if cad[indice] == c:
      return indice
    indice = indice + 1
  return -1
\end{pythoncode}

El tercer parámetro, \texttt{ini}, es opcional, ya que tiene un valor
por defecto, \texttt{0}. Si llamamos a \texttt{buscar} con dos argumentos,
se usa el valor por defecto y la búsqueda se hace desde el principio
de la cadena:
\begin{pyconcode}
>>> buscar("apple", "p")
1
\end{pyconcode}

Si se pasa el tercer parámetro, este \textbf{sobreescribe} el valor
por defecto:
\begin{pyconcode}
>>> buscar("apple", "p", 2)
2
>>> buscar("apple", "p", 3)
-1
\end{pyconcode}

\section{El método de inicialización}

\index{método de inicialización} \index{método!de inicialización}

El \textbf{de inicialización} es un método especial que se llama cuando
se crea un objeto. El nombre de este método es \texttt{\_\_init\_\_}
(dos caracteres de subrayado, seguidos por \texttt{init}, y luego
dos caracteres de subrayado más). Un método de inicialización para
la clase \texttt{Hora} se presenta a continuación:
\begin{pythoncode}
class Hora:
  def __init__(self, hora=0, minutos=0, segundos=0):
    self.hora = hora
    self.minutos = minutos
    self.segundos = segundos
\end{pythoncode}

No hay conflicto entre el atributo \texttt{self.hora} y el parámetro
\texttt{hora}. La notación punto especifica a qué variable nos estamos
refiriendo.

\index{notación punto}

Cuando llamamos al método constructor de \texttt{Hora}, los argumentos
se pasan a \texttt{init}:
\begin{pyconcode}
>>> horaActual = Hora(9, 14, 30)
>>> horaActual.imprimirHora()
>>> 9:14:30
\end{pyconcode}

Como los parámetros son opcionales, se pueden omitir:
\begin{pyconcode}
>>> horaActual = Hora()
>>> horaActual.imprimirHora()
>>> 0:0:0
\end{pyconcode}

O podemos pasar solo un parámetro:
\begin{pyconcode}
>>> horaActual = Hora(9)
>>> horaActual.imprimirHora()
>>> 9:0:0
\end{pyconcode}

O, sólo los dos primeros:
\begin{pyconcode}
>>> horaActual = Hora(9, 14)
>>> horaActual.imprimirHora()
>>> 9:14:0
\end{pyconcode}

Finalmente, podemos proporcionar algunos parámetros, nombrándolos
explícitamente:
\begin{pyconcode}
>>> horaActual = Hora(segundos = 30, hora = 9)
>>> horaActual.imprimirHora()
>>> 9:0:30
\end{pyconcode}

\section{Reconsiderando la clase Punto}

\index{clase Punto} \index{clase!Punto}

Reescribamos la clase \texttt{Punto} de la Sección~\ref{point} en
un estilo más orientado a objetos:
\begin{pythoncode}
class Punto:
  def __init__(self, x=0, y=0):
    self.x = x
    self.y = y

  def __str__(self):
    return '(' + str(self.x) + ', ' + str(self.y) + ')'
\end{pythoncode}

El método de inicialización toma los valores $x$ y $y$ como parámetros
opcionales, el valor por defecto que tienen es 0.

El método \texttt{\_\_str\_\_}, retorna una representación de un objeto
\texttt{Punto} en forma de cadena de texto. Si una clase proporciona
un método denominado \texttt{\_\_str\_\_}, éste sobreescribe el comportamiento
por defecto de la función primitiva \texttt{str}.
\begin{pyconcode}
>>> p = Punto(3, 4)
>>> str(p)
'(3, 4)'
\end{pyconcode}

Imprimir un objeto \texttt{Punto} implícitamente invoca a \texttt{\_\_str\_\_}
o sobre éste, así que definir a \texttt{\_\_str\_\_} también cambia
el comportamiento de la sentencia \texttt{print}:
\begin{pyconcode}
>>> p = Punto(3, 4)
>>> print(p)
(3, 4)
\end{pyconcode}

Cuando escribimos una nueva clase, casi siempre empezamos escribiendo
\texttt{\_\_init\_\_}, ya que facilita la instanciación de objetos,
y \texttt{\_\_str\_\_}, que casi siempre es esencial para la depuración.

\section{Sobrecarga de operadores}

\label{operator overloading} \index{operadores sobrecarga de} \index{operadores!sobrecarga}
\index{producto punto} \index{multiplicación escalar}

Algunos lenguajes hacen posible cambiar la definición de los operadores
primitivos cuando se aplican sobre tipos definidos por el programador.
Esta característica se denomina \textbf{sobrecarga de operadores}.
Es especialmente útil para definir tipos de datos matemáticos.

Por ejemplo, para sobrecargar el operador suma, \texttt{+}, proporcionamos
un método denominado \texttt{\_\_add\_\_}:

\pagebreak

\begin{pythoncode}
class Punto:
  # los métodos definidos previamente van aquí...

  def __add__(self, otro):
    return Punto(self.x + otro.x, self.y + otro.y)
\end{pythoncode}

Como de costumbre, el primer parámetro es el objeto con el cual se
invoca el método. El segundo parámetro se denomina con la palabra
\texttt{otro} para marcar la distinción entre éste y \texttt{self}.
Para sumar dos \texttt{Punto}s, creamos y retornamos un nuevo \texttt{Punto}
que contiene la suma de las coordenadas en el eje $x$ y la suma de
las coordenadas en el eje $y$.

Ahora, cuando aplicamos el operador \texttt{+} a dos objetos \texttt{Punto},
Python hace el llamado del método \texttt{\_\_add\_\_}:
\begin{pyconcode}
>>>   p1 = Punto(3, 4)
>>>   p2 = Punto(5, 7)
>>>   p3 = p1 + p2
>>>   print(p3)
(8, 11)
\end{pyconcode}

La expresión \texttt{p1 + p2} es equivalente a \texttt{p1.\_\_add\_\_(p2)},
pero luce mucho mas elegante.

Hay varias formas de sobrecargar el comportamiento del operador multiplicación:
definiendo un método \texttt{\_\_mul\_\_}, o \texttt{\_\_rmul\_\_},
o ambos.

Si el operando izquierdo de \texttt{{*}} es un \texttt{Punto}, Python
invoca a \texttt{\_\_mul\_\_}, asumiendo que el otro operando también
es un \texttt{Punto}. Calcula el \textbf{producto escalar} de los
dos puntos de acuerdo a las reglas del álgebra lineal:
\begin{pythoncode}
def __mul__(self, otro):
  return self.x * otro.x + self.y * otro.y
\end{pythoncode}

Si el operando izquierdo de \texttt{{*}} es un tipo primitivo y el
operando derecho es un \texttt{Punto}, Python llama a \texttt{\_\_rmul\_\_},
que ejecuta la \textbf{multiplicación escalar }:
\begin{pythoncode}
def __rmul__(self, otro):
  return Punto(otro * self.x,  otro * self.y)
\end{pythoncode}

El resultado ahora es un nuevo \texttt{Punto} cuyas coordenadas son
múltiplos de las originales. Si \texttt{otro} pertenece a un tipo
que no se puede multiplicar por un número de punto flotante, la función
\texttt{\_\_rmul\_\_} producirá un error.

Este ejemplo ilustra las dos clases de multiplicación:
\begin{pyconcode}
>>> p1 = Punto(3, 4)
>>> p2 = Punto(5, 7)
>>> print(p1 * p2)
43
>>> print(2 * p2)
(10, 14)
\end{pyconcode}
 ¿Que pasa si tratamos de evaluar \texttt{p2 {*} 2}? Ya que el primer
parámetro es un \texttt{Punto}, Python llama a \texttt{\_\_mul\_\_}
con \texttt{2} como segundo argumento. Dentro de \texttt{\_\_mul\_\_},
el programa intenta acceder al valor \texttt{x} de \texttt{otro},
lo que falla porque un número entero no tiene atributos:

\begin{pyconcode}
>>> print(p2 * 2)
AttributeError: 'int' object has no attribute 'x'
\end{pyconcode}
 Desafortunadamente, el mensaje de error es un poco opaco. Este ejemplo
demuestra una de las dificultades de la programación orientada a objetos.
Algunas veces es difícil saber qué código está ejecutándose.

Para un ejemplo completo de sobrecarga de operadores vea el capítulo
\ref{overloading}.

\section{Polimorfismo}

\index{polimorfismo}

La mayoría de los métodos que hemos escrito sólo funcionan para un
tipo de dato específico. Cuando se crea un nuevo tipo de objeto, se
escriben métodos que operan sobre ese tipo.

Pero hay ciertas operaciones que se podrían aplicar a muchos tipos,
un ejemplo de éstas son las operaciones aritméticas de las secciones
anteriores. Si muchos tipos soportan el mismo conjunto de operaciones,
usted puede escribir funciones que trabajen con cualquiera de estos
tipos.

Por ejemplo la operación \texttt{multsuma} (que se usa en el álgebra
lineal) toma tres parámetros, multiplica los primeros dos y luego
suma a esto el tercero. En Python se puede escribir así:

\begin{pythoncode}
def multsuma (x, y, z):
  return x * y + z
\end{pythoncode}
 Este método funcionará para cualesquier valores de \texttt{x} e \texttt{y}
que puedan multiplicarse, y para cualquier valor de \texttt{z} que
pueda sumarse al producto.

Podemos llamarla sobre números:

\begin{pyconcode}
>>> multsuma (3, 2, 1)
7
\end{pyconcode}
 O sobre \texttt{Punto}s:
\begin{pyconcode}
>>> p1 = Punto(3, 4)
>>> p2 = Punto(5, 7)
>>> print(multsuma (2, p1, p2))
(11, 15)
>>> print(multsuma (p1, p2, 1))
44
\end{pyconcode}

En el primer caso, el \texttt{Punto} se multiplica por un escalar
y luego se suma a otro \texttt{Punto}. En el segundo caso, el producto
punto produce un valor numérico, así que el tercer parámetro también
tiene que ser un número.

Una función como ésta, que puede tomar parámetros con tipos distintos
se denomina \textbf{polimórfica}.

Otro ejemplo es la función \texttt{derechoyAlReves}, que imprime una
lista dos veces, al derecho y al revés:

\begin{pythoncode}
def derechoyAlReves(l):
  import copy
  r = copy.copy(l)
  r.reverse()
  print(str(l) + str(r))
\end{pythoncode}
 Como el método \texttt{reverse} es una función modificadora, tenemos
que tomar la precaución de hacer una copia de la lista antes de llamarlo.
De esta forma la lista que llega como parámetro no se modifica.

Aquí hay un ejemplo que aplica \texttt{derechoyAlReves} a una lista:

\begin{pyconcode}
>>> miLista = [1, 2, 3, 4]
>>> derechoyAlReves(miLista)
[1, 2, 3, 4][4, 3, 2, 1]
\end{pyconcode}
 Por supuesto que funciona para listas, esto no es sorprendente. Lo
que sería sorprendente es que pudiéramos aplicarla a un \texttt{Punto}.

Para determinar si una función puede aplicarse a un nuevo tipo de
dato usamos la regla fundamental del polimorfismo:
\begin{quote}
\textbf{Si todas las operaciones adentro de la función pueden aplicarse
al otro tipo, la función puede aplicarse al tipo.} 
\end{quote}
Las operaciones que usa el método son \texttt{copy}, \texttt{reverse},
y \texttt{print}.

\texttt{copy} funciona para cualquier objeto, y como ya hemos escrito
un método \texttt{\_\_str\_\_} para los \texttt{Punto}s, lo único
que nos falta es el método \texttt{reverse} dentro de la clase \texttt{Punto}:

\begin{pythoncode}
def reverse(self):
  self.x , self.y = self.y, self.x
\end{pythoncode}
 Entonces podemos aplicar \texttt{derechoyAlReves} a objetos \texttt{Punto}:

\begin{pyconcode}
>>>   p = Punto(3, 4)
>>>   derechoyAlReves(p)
(3, 4)(4, 3)
\end{pyconcode}
 El mejor tipo de polimorfismo es el que no se pretendía lograr, aquel
en el que se descubre que una función escrita puede aplicarse a un
tipo para el cual no se había planeado hacerlo.

\section{Glosario}
\begin{description}
\item [{Lenguaje orientado a objetos:}] lenguaje que tiene características,
como las clases definidas por el usuario y la herencia, que facilitan
la programación orientada a objetos.
\item [{Programación orientada a objetos:}] estilo de programación en
el que los datos y las operaciones que los manipulan se organizan
en clases y métodos.
\item [{Método:}] función que se define adentro de una clase y se llama
sobre instancias de ésta.
\item [{Sobreescribir:}] reemplazar un valor preexistente. Por ejemplo,
se puede reemplazar un parámetro por defecto con un argumento particular
y un método ya definido, proporcionando un nuevo método con el mismo
nombre.
\item [{Método de inicialización:}] método especial que se llama automáticamente
cuando se crea un nuevo objeto. Inicializa los atributos del objeto.
\item [{Sobrecarga de operadores:}] extender el significado de los operadores
primitivos (\texttt{+}, \texttt{-}, \texttt{{*}}, \texttt{>}, \texttt{<},
etc.) de forma que acepten tipos definidos por el usuario.
\item [{Producto punto:}] operación del álgebra lineal que multiplica
dos \texttt{Punto}s y produce un valor numérico.
\item [{Multiplicación escalar:}] operación del álgebra lineal que multiplica
cada una de las coordenadas de un \texttt{Punto} por un valor numérico.
\item [{Polimórfica:}] función que puede operar sobre varios tipos de datos.
Si todas las operaciones que se llaman dentro de la función se le
pueden aplicar al tipo de dato, entonces la función puede aplicársela
al tipo.

\index{lenguaje de programación orientado a objetos} \index{método}
\index{método de inicialización} \index{sobreescribir} \index{sobrecarga}
\index{sobrecarga de operadores} \index{producto punto} \index{multiplicación escalar}
\index{polimórfica}
\end{description}

\section{Ejercicios}
\begin{enumerate}
\item Convierta \texttt{convertirASegundos} (de la Sección~\ref{convert})
a un método de la clase \texttt{Hora}.
\item Añada un cuarto parámetro \texttt{fin} a la función \texttt{buscar}
que especifique hasta donde continuar la búsqueda.

Advertencia: Este ejercicio tiene una cascarita. El valor por defecto
de \texttt{fin} debería ser \texttt{len(cad)}, pero esto no funciona.
Los valores por defecto se evalúan en el momento de definición de
las funciones, no cuando se llaman. Cuando se define \texttt{buscar},
\texttt{cad} no existe todavía, así que no se puede obtener su longitud.
\item Agregue un método \texttt{\_\_sub\_\_(self, otro)} que sobrecargue
el operador resta de la clase \texttt{Punto}, y pruébelo.
\end{enumerate}


\clearemptydoublepage % clases y metodos
%%FIX %
%\setcounter{page}{168}

\chapter{Conjuntos de objetos}

\section{Composición}

\index{composición} \index{estructura anidada}

En este momento usted ya ha visto varios ejemplos de composición.
Uno de los primeros fue una invocación de un método como parte de
una expresión. Otro ejemplo es la estructura anidada de sentencias;
por ejemplo, se puede colocar una sentencia \texttt{if} dentro de
un ciclo \texttt{while}, dentro de otra sentencia \texttt{if}.

Después de observar esto y haber aprendido sobre listas y objetos
no debería sorprenderse al saber que se pueden crear listas de objetos.
También pueden crearse objetos que contengan listas (como atributos),
listas que contengan listas, objetos que contengan objetos, y así
sucesivamente.

En este capítulo y el siguiente, mostraremos algunos ejemplos de estas
combinaciones, usando objetos \texttt{Carta}.

\section{Objeto \texttt{Carta} }

\index{Carta} \index{clase!Carta}

Si usted no tiene familiaridad con juegos de cartas este es un buen
momento para conseguir una baraja, de lo contrario este capítulo no
tendrá mucho sentido. Hay cincuenta y dos cartas en una baraja, cada
una pertenece a una de las cuatro figuras y uno de los trece valores.
Las figuras son Picas, Corazones, Diamantes y Tréboles. Los valores
son As, 2, 3, 4, 5, 6, 7, 8, 9, 10, J, Q, K. Dependiendo del juego,
el valor del As puede ser más alto que el de un rey o más bajo que
2.

\index{valor} \index{figura}

Si deseamos definir un nuevo objeto para representar una carta del
naipe, parece obvio que los atributos deberían ser \texttt{valor}
y \texttt{figura}. No es tan obvio que tipo de dato asignar a estos
atributos. Una posibilidad consiste en usar cadenas de texto con palabras
como \texttt{Picas} para las figuras y \texttt{Reina} para los valores.
Un problema de esta implementación es que no sería tan fácil comparar
cartas para ver cuál tiene un valor mayor o una figura mayor.

\index{codificar} \index{encriptar} \index{correspondencia}

Una alternativa consiste en usar enteros para \textbf{codificar} los
valores y las figuras. Por ``codificar'', no estamos haciendo alusión
a encriptar o traducir a un código secreto. Lo que un científico de
la computación considera ``codificar'' es ``definir una correspondencia
entre una secuencia de números y los objetos que deseamos representar''.
Por ejemplo:

\beforefig %
\begin{tabular}{lcl}
Picas  & $\mapsto$  & 3 \tabularnewline
Corazones  & $\mapsto$  & 2 \tabularnewline
Diamantes  & $\mapsto$  & 1 \tabularnewline
Tréboles  & $\mapsto$  & 0 \tabularnewline
\end{tabular}\afterfig

Una característica notable de esta correspondencia es que las figuras
aparecen en orden decreciente de valor así como los enteros van disminuyendo.
De esta forma, podemos comparar figuras mediante los enteros que las
representan. Una correspondencia para los valores es bastante sencilla;
cada número se relaciona con el entero correspondiente, y para las
cartas que se representan con letras tenemos lo siguiente:

\beforefig %
\begin{tabular}{lcl}
A  & $\mapsto$  & 1 \tabularnewline
J  & $\mapsto$  & 11 \tabularnewline
Q  & $\mapsto$  & 12 \tabularnewline
K  & $\mapsto$  & 13 \tabularnewline
\end{tabular}\afterfig

La razón para usar notación matemática en estas correspondencias es
que ellas no hacen parte del programa en Python. Son parte del diseño,
pero nunca aparecen explícitamente en el código fuente. La definición
de la clase \texttt{Carta} luce así:\inputencoding{latin9}
\begin{lstlisting}
class Carta:
  def __init__(self, figura=0, valor=0):
    self.figura = figura
    self.valor = valor
\end{lstlisting}
\inputencoding{utf8} Como de costumbre, proporcionamos un método de inicialización que
toma un parámetro opcional para cada atributo.

\index{constructor}

Para crear un objeto que represente el 3 de tréboles, usamos este
comando:\inputencoding{latin9}
\begin{lstlisting}
tresTreboles = Carta(0, 3)
\end{lstlisting}
\inputencoding{utf8}
El primer argumento, \texttt{0}, representa la figura (tréboles).

\section{Atributos de clase y el método \texttt{\_\_str\_\_}}

\index{atributo de clase} \index{atributo!de clase}

Para imprimir objetos \texttt{Carta} en una forma que la gente pueda
leer fácilmente, queremos establecer una correspondencia entre códigos
enteros y palabras. Una forma natural de hacerlo es con listas de
cadenas de texto. Asignamos estas listas a \textbf{atributos de clase}
al principio de la clase:\inputencoding{latin9}
\begin{lstlisting}
class Carta:
  listaFiguras = ["Treboles", "Diamantes", "Corazones", 
  "Picas"]
  listaValores = ["narf", "As", "2", "3", "4", "5", "6", 
  "7","8", "9", "10", "Jota", "Reina", "Rey"]

  # se omite el metodo init

  def __str__(self):
    return self.listaFiguras[self.valor] + " de " + 
            self.listaValores[self.figura]
\end{lstlisting}
\inputencoding{utf8}
Un atributo de clase se define afuera de los métodos y puede ser accedido
desde cualquiera de ellos.

Dentro de \texttt{\_\_str\_\_}, podemos usar a \texttt{listaFiguras}
y \texttt{listaValores} para establecer una correspondencia entre
los valores numéricos de \texttt{figura}, \texttt{valor} y los nombres
de las cartas. La expresión \verb+self.listaFiguras[self.figura]+
significa ``use el atributo \texttt{figura} del objeto \texttt{self}
como índice dentro del atributo de clase \texttt{listaFiguras}, esto
seleccionará la cadena de texto apropiada''.

La razón para el \texttt{``narf''} en el primer elemento de \texttt{listaValores}
consiste en ocupar el elemento cero de la lista que no va a ser usado
en el programa. Los valores válidos son de 1 a 13. Este elemento desperdiciado
no es necesario, podríamos haber empezado a contar desde 0, pero es
menos confuso codificar 2 como 2, 3 como 3 ... y 13 como 13.

Con los métodos que tenemos hasta aquí, podemos crear e imprimir cartas:\inputencoding{latin9}
\begin{lstlisting}
>>> c1 = Carta(1, 11)
>>> print(c1)
Jota de Diamantes
\end{lstlisting}
\inputencoding{utf8} Los atributos de clase como \texttt{listaFiguras} se comparten por
todos los objetos \texttt{Carta}. La ventaja de esto es que podemos
usar cualquier objeto \texttt{Carta} para acceder a ellos:

\inputencoding{latin9}\begin{lstlisting}
>>> c2 = Carta(1, 3)
>>> print(c2)
3 de Diamantes
>>> print(c2.listaFiguras[1])
Diamantes
\end{lstlisting}
\inputencoding{utf8} La desventaja es que si modificamos un atributo de clase, afecta
a todas las otras instancias de la clase. Por ejemplo, si decidimos
que ``Jota de Diamantes'' debería llamarse ``Caballero de Rombos
rojos,'' podríamos ejecutar:

\index{instancia!objeto} \index{objeto instancia}

\inputencoding{latin9}\begin{lstlisting}
>>> c1.listaFiguras[1] = "Caballero de Rombos rojos"
>>> print(c1)
Caballero de Rombos rojos
\end{lstlisting}
\inputencoding{utf8} El problema es que {\em todos} los Diamantes ahora son Rombos
rojos:

\inputencoding{latin9}\begin{lstlisting}
>>> print(c2)
3 de Rombos rojos
\end{lstlisting}
\inputencoding{utf8} Usualmente no es una buena idea modificar los atributos de clase.

\section{Comparando cartas}

\label{comparecard} \index{operador!condicional} \index{operador condicional}

Para los tipos primitivos contamos con los operadores (\texttt{<},
\texttt{>}, \texttt{==}, etc.) que determinan cuándo un valor es mayor,
menor, mayor o igual, menor o igual, o igual al otro. Para los tipos
definidos por el programador podemos sobrecargar el comportamiento
de los operadores predefinidos proporcionando un método llamado \texttt{\_\_cmp\_\_}.
Por convención, \texttt{\_\_cmp\_\_} toma dos parámetros, \texttt{self}
y \texttt{otro}, y retorna 1 si el primer objeto es más grande, -1
si el segundo es más grande y 0 si son iguales entre si.

\index{sobrecargar} \index{sobrecarga de operadores} \index{orden}
\index{orden total} \index{orden parcial}

Algunos tipos tienen un orden total, lo que quiere decir que cualquier
pareja de elementos se puede comparar para decidir cuál de ellos es
mayor. Por ejemplo, los números enteros y los de punto flotante tienen
un orden total. Algunos conjuntos no tienen relación de orden, lo
que quiere decir que no hay una manera sensata de determinar que un
elemento es mayor que otro. Por ejemplo, las frutas no tienen una
relación de orden, y esta es la razón por la que no se pueden comparar
manzanas con naranjas.

El conjunto de cartas tiene un orden parcial, lo que quiere decir
que algunas veces se pueden comparar elementos, y otras veces no.
Por ejemplo, el 3 de Picas es mayor que el 2 de picas, y el 3 de Diamantes
es mayor que el 3 de Picas. Pero, ¿que es más alto, el 3 de Picas
o el 2 de Diamantes? Uno tiene un valor más alto, pero el otro tiene
una figura más alta.

\index{comparable}

Para lograr comparar las cartas, hay que tomar una decisión sobre
la importancia del valor y de la figura. Para ser honestos, esta decisión
es arbitraria. Así que tomaremos la opción de determinar qué figura
es más importante, basándonos en que un mazo de cartas nuevo viene
con las Picas (en orden), luego los Diamantes, y así sucesivamente.

Con esta decisión \texttt{\_\_cmp\_\_} queda así:\inputencoding{latin9}
\begin{lstlisting}
def __cmp__(self, otro):
  # chequea las figuras
  if self.figura > otro.figura: return 1
  if self.figura < otro.figura: return -1
  # Si tienen la misma figura... 
  if self.valor > otro.valor: return 1
  if self.valor < otro.valor: return -1
  # si los valores son iguales... hay un empate
  return 0
\end{lstlisting}
\inputencoding{utf8}
Con este orden los Ases valen menos que los Dos.

\section{Mazos}

\index{lista!de objetos} \index{objeto!lista de} \index{mazo}

Ahora que tenemos objetos para representar \texttt{Carta}s, el siguiente
paso lógico consiste en definir una clase para representar un \texttt{Mazo}.
Por supuesto, un mazo (o baraja) está compuesto por cartas, así que
cada instancia de \texttt{Mazo} contendrá como atributo una lista
de cartas.

\index{método de inicialización} \index{método!de inicialización}

La siguiente es la definición de la clase \texttt{Mazo}. El método
de inicialización crea el atributo \texttt{cartas} y genera el conjunto
usual de cincuenta y dos cartas:

\index{composición} \index{ciclo!anidado}

\inputencoding{latin9}\begin{lstlisting}
class Mazo:
  def __init__(self):
    self.cartas = []
    for figura in range(4):
      for valor in range(1, 14):
        self.cartas.append(Carta(figura, valor))
\end{lstlisting}
\inputencoding{utf8} La forma más sencilla de llenar el mazo consiste en usar un ciclo
anidado. El ciclo exterior enumera las figuras de 0 a 3. El ciclo
interno enumera los valores de 1 a 13. Como el ciclo exterior itera
cuatro veces y el interno itera trece veces, el número total de iteraciones
es cincuenta y dos ($4\times13$). Cada iteración crea una nueva instancia
de \texttt{Carta} y la pega a la lista \texttt{cartas}.

El método \texttt{append} acepta secuencias mutables como las listas
y no acepta tuplas.

\index{método append} \index{método de lista} \index{método!de lista}

\section{Imprimiendo el mazo}

\label{printdeck} \index{imprimir!objeto mazo}

Como de costumbre, cuando definimos un nuevo tipo de objeto, deseamos
tener un método que imprima su contenido. Para imprimir un \texttt{Mazo},
recorremos la lista e imprimimos cada objeto \texttt{Carta}:\inputencoding{latin9}
\begin{lstlisting}
class Mazo:
  ...
  def imprimirMazo(self):
    for carta in self.cartas:
      print(carta)
\end{lstlisting}
\inputencoding{utf8}
En este ejemplo y en los que siguen, los puntos suspensivos indican
que hemos omitido los otros métodos de la clase.

Otra alternativa a \texttt{imprimirMazo} puede ser escribir un método
\texttt{\_\_str\_\_} para la clase \texttt{Mazo}. La ventaja de \texttt{\_\_str\_\_}
radica en su mayor flexibilidad. Además de imprimir el contenido del
objeto, genera una representación de él en una cadena de texto que
puede manipularse en otros lugares del programa, incluso antes de
imprimirse.

A continuación hay una versión de \texttt{\_\_str\_\_} que retorna
una representación de un \texttt{Mazo}. Para añadir un estilo de cascada,
cada carta se imprime un espacio mas hacia la derecha que la anterior:\inputencoding{latin9}
\begin{lstlisting}
class Mazo:
  ...
  def __str__(self):
    s = ""
    for i in range(len(self.cartas)):
      s = s + " "*i + str(self.cartas[i]) + "\n"
    return s
\end{lstlisting}
\inputencoding{utf8}
Este ejemplo demuestra varios puntos. Primero, en vez de recorrer
los elementos de la lista \texttt{self.cartas}, estamos usando a \texttt{i}
como variable de ciclo que lleva la posición de cada elemento en la
lista de cartas.

Segundo, estamos usando el operador multiplicación aplicado a un número
y una cadena, de forma que la expresión \verb+" "*i+ produce un número
de espacios igual al valor actual de \texttt{i}.

Tercero, en vez de usar el comando \texttt{print} para realizar la
impresión, utilizamos la función \texttt{str}. Pasar un objeto como
argumento a \texttt{str} es equivalente a invocar el método \texttt{\_\_str\_\_}
sobre el objeto.

\index{acumulador}

Finalmente, estamos usando a la variable \texttt{s} como \textbf{acumulador}.
Inicialmente \texttt{s} es la cadena vacía. En cada iteración del
ciclo se genera una nueva cadena y se concatena con el valor viejo
de \texttt{s} para obtener el nuevo valor. Cuando el ciclo finaliza,
\texttt{s} contiene la representación completa del \texttt{Mazo},
que se despliega (parcialmente) así:

\inputencoding{latin9}\begin{lstlisting}
>>> mazo = Mazo()
>>> print(mazo)
As de Picas
 2 de Picas
  3 de Picas
   4 de Picas
    5 de Picas
     6 de Picas
      7 de Picas
       8 de Picas
        9 de Picas
         10 de Picas
          J de Picas
           Reina de Picas
            Rey de Picas
             As de Diamantes
\end{lstlisting}
\inputencoding{utf8} Aunque el resultado se despliega en 52 líneas, es una sola cadena
que contiene caracteres nueva linea \verb+(\n)+.

\section{Barajando el mazo}

\index{barajar}

Si un mazo se baraja completamente, cualquier carta tiene la misma
probabilidad de aparecer en cualquier posición, y cualquier posición
tiene la misma probabilidad de contener a cualquier carta.

\index{random} \index{randrange}

Para barajar el mazo, usamos la función \texttt{randrange} que pertenece
al módulo del sistema \texttt{random}. \texttt{randrange} recibe dos
parámetros enteros \texttt{a} y \texttt{b}, y se encarga de escoger
al azar un valor perteneciente al rango \texttt{a <= x < b}. Como
el límite superior es estrictamente menor que \texttt{b}, podemos
usar el número de elementos de una lista como el segundo parámetro
y siempre obtendremos un índice válido como resultado. Por ejemplo,
esta expresión escoge al azar el índice de una carta en un mazo:

\inputencoding{latin9}\begin{lstlisting}
random.randrange(0, len(self.cartas))
\end{lstlisting}
\inputencoding{utf8} Una manera sencilla de barajar el mazo consiste en recorrer todas
las cartas intercambiando cada una con otra carta escogida al azar.
Es posible que la carta se intercambie consigo misma, pero esto no
causa ningún problema. De hecho, si prohibiéramos esto, el orden de
las cartas no sería tan aleatorio:

\inputencoding{latin9}\begin{lstlisting}
class Mazo:
  ...
  def barajar(self):
    import random
    nCartas = len(self.cartas)
    for i in range(nCartas):
      j = random.randrange(i, nCartas)
      self.cartas[i], self.cartas[j] = self.cartas[j], \
                                       self.cartas[i]
\end{lstlisting}
\inputencoding{utf8} En vez de asumir que hay 52 cartas en el mazo, obtenemos el número
de ellas a través de la función len y lo almacenamos en la variable
\texttt{nCartas}.

\index{intercambiar} \index{asignación de tuplas} \index{asignación!de tuplas}

Para cada carta en el mazo, escogemos, aleatoriamente, una carta de
las que no han sido barajadas todavía. Intercambiamos la carta actual
(con índice \texttt{i}) con la seleccionada (con índice \texttt{j}).
Para intercambiar las cartas usamos asignación de tuplas, como en
la sección ~\ref{tuple assignment}:\inputencoding{latin9}
\begin{lstlisting}
self.cartas[i], self.cartas[j] = self.cartas[j], \
                                 self.cartas[i]
\end{lstlisting}
\inputencoding{utf8}
\section{Eliminando y entregando cartas}

\index{eliminando cartas}

Otro método que sería útil para la clase \texttt{Mazo} es \texttt{eliminarCarta},
que toma una carta como parámetro, la remueve y retorna True si la
encontró en el mazo o False si no estaba:

\inputencoding{latin9}\begin{lstlisting}
class Mazo:
  ...
  def eliminarCarta(self, carta):
    if carta in self.cartas:
      self.cartas.remove(carta)
      return True
    else: 
      return True
\end{lstlisting}
\inputencoding{utf8} El operador \texttt{in} retorna True si el primer operando se encuentra
dentro del segundo, que debe ser una secuencia. Si el primer operando
es un objeto, Python usa el método \texttt{\_\_cmp\_\_} para determinar
la igualdad de elementos en la lista. Como la función \texttt{\_\_cmp\_\_}
en la clase \texttt{Carta} detecta igualdad profunda, el método \texttt{eliminarCarta}
detecta igualdad profunda.

\index{operador in} \index{operador!in}

Para entregar cartas necesitamos eliminar y retornar la primera carta
del mazo. El método \texttt{pop} de las listas proporciona esta funcionalidad:\inputencoding{latin9}
\begin{lstlisting}
class Mazo:
  ...
  def entregarCarta(self):
    return self.cards.pop()
\end{lstlisting}
\inputencoding{utf8} En realidad, \texttt{pop} elimina la {\em última} carta de la
lista, así que realmente estamos entregando cartas por la parte inferior,
y esto no causa ningún inconveniente.

\index{función booleana} \index{función!booleana}

Una operación más que podemos requerir es la función booleana \texttt{estaVacio},
que retorna True si el mazo está vacío:

\inputencoding{latin9}\begin{lstlisting}
class Mazo:
  ...
  def estaVacio(self):
    return (len(self.cartas) == 0)
\end{lstlisting}
\inputencoding{utf8}
\section{Glosario}
\begin{description}
\item [{Codificar:}] representar un conjunto de valores usando otro conjunto
de valores estableciendo una correspondencia entre ellos.
\item [{Atributo de clase:}] variable de una clase que está fuera de todos
los métodos. Puede ser accedida desde todos los métodos y se comparte
por todas las instancias de la clase.
\item [{Acumulador:}] variable que se usa para acumular una serie de valores
en un ciclo. Por ejemplo, concatenar varios valores en una cadena
o sumarlos.

\index{codificar} \index{atributo de clase} \index{atributo!de clase}
\index{acumulador}
\end{description}

\section{Ejercicios}
\begin{enumerate}
\item Modifique \texttt{\_\_cmp\_\_} para que los Ases tengan mayor puntaje
que los reyes.
\item Reescriba el intercambio que se hace en \texttt{barajar} sin usar
asignación de tuplas.
\item Escriba una clase \texttt{secuenciaADN} que permita representar una
secuencia de ADN con un método \texttt{\_\_init\_\_} adecuado.
\item Agregue cuatro métodos a la clase para averiguar la cantidad de cada
nucleótido en la secuencia, cuantas A, G, C, T.
\end{enumerate}


\clearemptydoublepage % conjuntos de objetos

\chapter{Herencia}

\section{Definición}

\index{herencia} \index{programación orientada a objetos} \index{clase madre}
\index{clase hija} \index{subclase}

La característica más asociada con la programación orientada a objetos
quizás sea la \textbf{herencia}. Ésta es la capacidad de definir una
nueva clase que constituye la versión modificada de una clase preexistente.

La principal ventaja de la herencia consiste en que se pueden agregar
nuevos métodos a una clase sin modificarla. El nombre ``herencia''
se usa porque la nueva clase hereda todos los métodos de la clase
existente. Extendiendo esta metáfora, la clase preexistente se denomina
la clase \textbf{madre}. La nueva clase puede llamarse clase \textbf{hija}
o, ``subclase.''

\index{diseño orientado a objetos}

La herencia es muy poderosa. Algunos programas complicados se pueden
escribir de una manera más sencilla y compacta a través del uso de
la herencia. Además, facilita la reutilización de código, ya que se
puede especializar el comportamiento de una clase madre sin modificarla.
En algunos casos, las relaciones entre las clases reflejan la estructura
de las entidades del mundo real que se presentan en un problema, y
esto hace que los programas sean más comprensibles.

Por otra parte, la herencia puede dificultar la lectura de los programas.
Cuando un método se llama, puede que no sea muy claro donde está definido.
El código relevante puede estar disperso entre varios módulos. Además,
muchas soluciones que se pueden escribir con el uso de herencia, también
se pueden escribir sin ella, y de una manera elegante (algunas veces
más elegante). Si la estructura natural de las entidades que intervienen
en un problema no se presta a pensar en términos de herencia, este
estilo de programación puede causar más perjuicios que beneficios.

En este capítulo demostraremos el uso de la herencia como parte de
un programa que implementa el juego de cartas la solterona. Una de
las metas es escribir una base de código que pueda reutilizarse para
implementar otros juegos de cartas.

\section{Una mano de cartas}

Para casi todos los juegos de cartas vamos a requerir representar
una mano de cartas. Hay una clara semejanza entre un mazo y una mano
de cartas, ambos son conjuntos de cartas y requieren operaciones como
agregar o eliminar cartas. También, podríamos barajar mazos y manos
de cartas.

Por otro lado, una mano se diferencia de un mazo; dependiendo del
juego que estemos considerando podríamos ejecutar algunas operaciones
sobre las manos que no tienen sentido en un mazo. Por ejemplo, en
el póker, podríamos clasificar manos como un par, una terna, 2 pares,
escalera, o podríamos comparar un par de manos. En el bridge, podríamos
calcular el puntaje para una mano con el objetivo de hacer una apuesta.

Esta situación sugiere el uso de la herencia. Si una \texttt{Mano}
es una subclase de \texttt{Mazo}, tendrá todos los métodos definidos
en ésta y se podrán agregar nuevos métodos.

\index{clase madre} \index{clase!madre}

En la definición de una clase hija, el nombre de la clase madre se
encierra entre paréntesis:

\begin{pythoncode}
class Mano(Mazo):
  pass
\end{pythoncode}
 Esta sentencia indica que la nueva clase \texttt{Mano} hereda de
la clase \texttt{Mazo}.

El constructor de \texttt{Mano} inicializa los atributos \texttt{nombre}
y \texttt{cartas}. La cadena \texttt{nombre} identifica cada mano,
probablemente con el nombre del jugador que la sostiene. Por defecto,
el parámetro nombre es una cadena vacía. \texttt{cartas} es la lista
de cartas en la mano, inicializada a una lista vacía:

\begin{pythoncode}
class Mano(Mazo):
  def __init__(self, nombre=""):
    self.cartas = []
    self.nombre = nombre
\end{pythoncode}
 Para la gran mayoría de juegos de cartas es necesario agregar y remover
cartas del mazo. Como \texttt{Mano} hereda \texttt{eliminarCarta}
de \texttt{Mazo}, ya tenemos la mitad del trabajo hecho, solo tenemos
que escribir \texttt{agregarCarta}:
\begin{pythoncode}
class Mano(Mazo):
  ...
  def agregarCarta(self,carta) :
    self.cartas.append(carta)
\end{pythoncode}

Recuerde que la elipsis (...) indica que estamos omitiendo los otros
métodos. El método \texttt{append} de las listas permite agregar la
nueva carta.

\section{Repartiendo cartas}

\index{repartiendo cartas}

Ahora que tenemos una clase \texttt{Mano} deseamos transferir cartas
del \texttt{Mazo} a las manos. No es fácil decidir que el método que
implemente esta transferencia se incluya en la clase \texttt{Mano}
o en la clase \texttt{Mazo}, pero como opera sobre un solo mazo y
(probablemente) sobre varias manos, es más natural incluirlo en \texttt{Mazo}.

El método \texttt{repartir} debería ser muy general, ya que diferentes
juegos tienen diversas reglas. Podríamos transferir todo el mazo de
una vez, o repartir carta a carta alternando por cada mano, como se
acostumbra en los casinos.

El método \texttt{repartir} toma dos parámetros, una lista (o tupla)
de manos y el número total de cartas que se deben entregar. Si no
hay suficientes cartas en el mazo para repartir, entrega todas las
cartas y se detiene:

\begin{pythoncode}
class Mazo:
  ...
  def repartir(self, manos, nCartas=999):
    nManos = len(manos)
    for i in range(nCartas):
      if self.estaVacia(): 
            break    # se detiene si no hay cartas
      # toma la carta en el tope
      carta = self.eliminarCarta() 
      # siguiente turno!
      mano = manos[i % nManos] 
      # agrega la carta a la mano
      mano.agregarCarta(carta) 
\end{pythoncode}
 El segundo parámetro, \texttt{nCartas}, es opcional; y por defecto
es un número entero grande que garantice que todas las cartas del
mazo se repartan.

\index{variable de ciclo} \index{variable!de ciclo}

La variable de ciclo \texttt{i} va de 0 a \texttt{nCartas-1}. En cada
iteración remueve una carta al tope del mazo usando el método \texttt{pop},
que elimina y retorna el último elemento de la lista.

\index{operador residuo} \index{operador!residuo}

El operador residuo (\texttt{\%}) nos permite repartir cartas de manera
circular (una carta a una mano distinta en cada iteración). Cuando
\texttt{i} es igual al número de manos en la lista, la expresión \texttt{i
\% nManos} \textit{da la vuelta} retornando la primera posición de
la lista (la posición 0).

\section{Imprimiendo una mano}

\index{imprimiendo!manos de cartas}

Para imprimir el contenido de una mano podemos aprovechar los métodos
\texttt{\_\_str\_\_} e \texttt{imprimirMazo} heredados de \texttt{Mazo}.
Por ejemplo:

\begin{pyconcode}
>>> mazo = Mazo()
>>> mazo.barajar()
>>> mano = Mano("Rafael")
>>> mazo.repartir([mano], 5)
>>> print(mano)
La Mano Rafael contiene
2 de Picas
 3 de Picas
  4 de Picas
   As de Corazones
    9 de tréboles
\end{pyconcode}
 No es una gran mano, pero se puede mejorar.

Aunque es conveniente heredar los métodos existentes, hay un dato
adicional que un objeto \texttt{Mano} puede incluir cuando se imprime,
para lograr esto implementamos \texttt{\_\_str\_\_} sobrecargando
el que está definido en la clase \texttt{Mazo}:

\begin{pythoncode}
class Mano(Mazo)
  ...
  def __str__(self):
    s = "Mano " + self.nombre
    if self.estaVacia():
      s = s + " esta vacia\n"
    else:
      s = s + " contiene\n"
    return s + Mazo.__str__(self)
\end{pythoncode}
 Inicialmente, \texttt{s} es una cadena que identifica la mano. Si
está vacía, se añade la cadena \texttt{esta vacia}. Si esto no es
así se añade la cadena \texttt{contiene} y la representación textual
de la clase \texttt{Mazo}, que se obtiene aplicando el método \texttt{\_\_str\_\_}
a \texttt{self}.

Parece extraño aplicar el método \texttt{\_\_str\_\_} de la clase
\texttt{Mazo} a \texttt{self} que se refiere a la \texttt{Mano} actual.
Para disipar cualquier duda, recuerde que \texttt{Mano} es una clase
de \texttt{Mazo}. Los objetos \texttt{Mano} pueden hacer todo lo que
los objetos \texttt{Mazo} hacen, así que esto es legal.

\index{subclase} \index{clase padre} \index{clase!padre}

En general, siempre se puede usar una instancia de una subclase en
vez de una instancia de la clase padre.

\section{La clase \texttt{JuegoDeCartas}}

La clase \texttt{JuegoDeCartas} se encarga de algunas operaciones
básicas comunes a todos los juegos, tales como crear el mazo y barajarlo:

\begin{pythoncode}
class JuegoDeCartas:
  def __init__(self):
    self.mazo = Mazo()
    self.mazo.barajar()
\end{pythoncode}
 En este ejemplo vemos que la inicialización realiza algo más importante
que asignar valores iniciales a los atributos.

Para implementar un juego específico podemos heredar de \texttt{JuegoDeCartas}
y agregar las características del juego particular que estemos desarrollando.

A manera de ejemplo, escribiremos una simulación del juego La Solterona.

El objetivo de La Solterona es deshacerse de todas las cartas. Cada
jugador hace esto emparejando cartas por figura y valor. Por ejemplo
el 4 de Treboles se empareja con el 4 de Picas, por que son cartas
negras. La J de Corazones se empareja con la J de Diamantes, porque
son cartas rojas.

Para empezar, la reina de Treboles se elimina del mazo, de forma que
la reina de Picas no tenga pareja. Las otras 51 cartas se reparten
equitativamente entre los jugadores. Después de repartir cada jugador
busca parejas en su mano y las descarta.

Cuando ningún jugador pueda descartar más se empieza a jugar por turnos.
Cada jugador escoge una carta de su vecino a la izquierda (sin mirarla).
Si la carta escogida se empareja con alguna carta en la mano del jugador,
el par se elimina. Si esto no es así, la carta debe agregarse a la
mano del jugador que escoge. Poco a poco, se realizarán todos los
emparejamientos posibles, dejando únicamente a la reina de Picas en
la mano del perdedor.

En nuestra simulación del juego, el computador juega todas las manos.
Desafortunadamente, algunos matices del juego real se pierden en la
simulación por computador. En un juego real, el jugador con la Solterona
(la reina de Picas) intenta deshacerse de ella de diferentes formas,
ya sea desplegándola de una manera prominente, o ocultándola de alguna
manera. El programa simplemente escogerá una carta de algún vecino
aleatoriamente.

\section{La clase \texttt{ManoSolterona}}

\index{clase!ManoSolterona}

Una mano para jugar a la Solterona requiere algunas capacidades que
no están presentes en la clase \texttt{Mano}. Vamos a definir una
nueva clase \texttt{ManoSolterona}, que hereda de \texttt{Mano} y
provee un método adicional llamado \texttt{quitarPareja}:

\begin{pythoncode}
class ManoSolterona(Mano):
  def quitarPareja(self):
    conteo = 0
    cartasOriginales = self.cartas[:]
    for carta in cartasOriginales:
      m = Carta(3 - carta.figura, carta.valor)
      if pareja in self.cartas:
        self.cartas.remove(carta)
        self.cartas.remove(m)
        print("Mano %s: %s se empareja con %s" % \
              (self.name,carta,m))
        cont = cont + 1
    return cont
\end{pythoncode}
 Empezamos haciendo una copia de la lista de cartas, de forma que
podamos recorrerla y simultáneamente eliminar cartas. Como \texttt{self.cartas}
se modifica en el ciclo, no vamos a utilizarlo para controlar el recorrido.
¡Python puede confundirse totalmente si empieza a recorrer una lista
que está cambiando!

\index{recorrido}

Para cada carta en la mano, averiguamos si se empareja con una carta
escogida de la mano de otra persona. Para esto, tienen que tener el
mismo valor y la otra figura del mismo color. La expresión \texttt{3
- carta.figura} convierte un trébol (figura 0) en una Pica (figura
3) y a un Diamante (figura 1) en un Corazón (figura 2). Usted puede
comprobar que las operaciones inversas también funcionan. Si hay una
carta que se empareje, las dos se eliminan.

El siguiente ejemplo demuestra cómo usar \texttt{quitarPareja}:

\begin{pyconcode}
>>> juego = JuegoDeCartas()
>>> mano = ManoSolterona("frank")
>>> juego.mazo.repartir([mano], 13)
>>> print(mano)
Mano frank contiene
As de Picas
 2 de Diamantes
  7 de Picas
   8 de Treboles
    6 de Corazones
     8 de Picas
      7 de Treboles
       Reina de Treboles
        7 de Diamantes
         5 de Treboles
          Jota de Diamantes
           10 de Diamantes
            10 de Corazones

>>> mano.quitarPareja()
Mano frank: 7 de Picas se empareja con 7 de Treboles
Mano frank: 8 de Picas se empareja con 8 de Treboles
Mano frank: 10 de Diamantes se empareja con 10 de Corazones
>>> print(mano)
Mano frank contiene
Ace de Picas
 2 de Diamantes
  6 de Corazones
   Reina de Treboles
    7 de Diamantes
     5 de Treboles
      Jota de Diamantes
\end{pyconcode}
 Tenga en cuenta que no hay método \texttt{\_\_init\_\_} en la clase
\texttt{ManoSolterna}. Lo heredamos de \texttt{Mano}.

\section{La clase \texttt{JuegoSolterona}}

\index{clase!JuegoSolterona}

Ahora podemos dedicarnos a desarrollar el juego. \texttt{JuegoSolterona}
es una subclase de \texttt{JuegoDeCartas} con un nuevo método llamado
\texttt{jugar} que recibe una lista de jugadores como parámetro.

Como \texttt{\_\_init\_\_} se hereda de \texttt{JuegoDeCartas}, tendremos
la garantía de que un objeto de tipo \texttt{JuegoSolterona} contendrá
un mazo barajado:

\pagebreak

\begin{pythoncode}
class JuegoSolterona(JuegoDeCartas):
  def jugar(self, nombres):
    # elimina la Reina de Treboles
    self.mazo.eliminarCarta(Carta(0,12))

    # prepara una mano para cada jugador
    self.manos = []
    for nombre in nombres :
      self.manos.append(ManoJuegoSolterona(nombre))

    # reparte las  cartas
    self.mazo.repartir(self.cartas)
    print("---------- Cartas repartidas!")
    self.imprimirManos()

    # quitar parejas iniciales
    parejas = self.eliminarParejas()
    print("---------- Parejas descartadas, comienza el juego")
    self.imprimirManos()

    # jugar hasta que las 50 cartas sean descartadas
    turno = 0
    numManos = len(self.manos)
    while parejas < 25:
      parejas = parejas+self.jugarUnTurno(turno)
      turno = (turno + 1) % numManos

    print("-------- Juego terminado")
    self.printManos()
\end{pythoncode}

\pagebreak

 Algunos de las etapas del juego se han separado en métodos. \texttt{eliminarParejas}
recorre la lista de manos invocando \texttt{eliminarPareja} en cada
una de ellas:
\begin{pythoncode}
class JuegoSolterona(JuegoDeCartas):
  ...
  def eliminarParejas(self):
    cont = 0
    for mano in self.manos:
      cont = cont + mano.eliminarParejas()
    return cont
\end{pythoncode}

\texttt{cont} es un acumulador que lleva cuenta del número de parejas
que se encontraron en cada mano.

\index{acumulador}

Cuando el número total de parejas encontradas llega a ser veinticinco,
se han quitado cincuenta cartas de las manos, y esto implica que la
única carta que resta es la reina de Picas y que el juego ha terminado.

La variable \texttt{turno} lleva la pista de cual es el jugador que
tiene el turno para jugar. Empieza en 0 y se incrementa de uno en
uno; cuando alcanza el valor \texttt{numManos}, el operador residuo
lo reinicia en 0.

El método \texttt{jugarUnTurno} toma un parámetro que indica el jugador
que tiene el turno. El valor de retorno es el número de parejas encontradas
durante este turno:

\begin{pythoncode}
class JuegoSolterona(JuegoDeCartas):
  ...
  def jugarUnTurno(self, i):
    if self.manos[i].estaVacia():
      return 0
    vecino = self.encontrarVecino(i)
    cartaEscogida = self.manos[vecino].eliminarCarta()
    self.manos[i].agregarCarta(cartaEscogida)
    print("Mano", self.manos[i].nombre, 
          "escogio", cartaEscogida)
    cont = self.manos[i].eliminarParejas()
    self.manos[i].barajar()
    return cont
\end{pythoncode}
 Si la mano de un jugador está vacía, este jugador está fuera del
juego, así que no hace ninguna acción y retorna 0.

Si esto no es así, un turno consiste en encontrar el primer jugador
en la izquierda que tenga cartas, tomar una de él, y buscar por parejas.
Antes de retornar, las cartas en la mano se barajan para que la elección
del siguiente jugador sea al azar.

El método \texttt{encontrarVecino} comienza con el jugador a la izquierda
y continua buscando de manera circular hasta que encuentra un jugador
que tenga cartas:
\begin{pythoncode}
class JuegoSolterona(JuegoDeCartas):
  ...
  def encontrarVecino(self, i):
    numManos = len(self.manos)
    for siguiente in range(1,numManos):
      vecino = (i + siguiente) % numManos
      if not self.manos[vecino].estaVacia():
        return vecino
\end{pythoncode}

Si \texttt{encontrarVecino} diera toda la vuelta sin encontrar cartas,
retornaría \texttt{None} y causaría un error en algún otro lugar del
programa. Afortunadamente, usted puede comprobar que esto nunca va
a pasar (siempre y cuando el fin del juego se detecte correctamente).

Hemos omitido el método \texttt{imprimaManos}. Usted puede escribirlo.

La siguiente salida es de un juego en el que solo las primeras 15
cartas mas altas (con valor 10 y superior) se repartieron a tres jugadores.
Con este pequeño mazo, el juego termina después de siete parejas encontradas,
en vez de veinticinco.
\begin{pyconcode}
>>> import cartas
>>> juego = cartas.JuegoSolterona()
>>> juego.jugar(["Allen","Jeff","Chris"])
---------- Las cartas se han repartido
Mano Allen contiene
Rey de Corazones
 Jota de Treboles
  Reina de Picas
   Rey de Picas
    10 de Diamantes

Mano Jeff contiene
Reina de Corazones
 Jota de Picas
  Jota de Corazones
   Rey de Diamantes
    Reina de Diamantes

Mano Chris contiene
Jota de Diamantes
 Rey de Treboles
  10 de Picas
   10 de Corazones
    10 de Treboles

Mano Jeff: Reina de Corazones se empareja con Reina de 
Diamantes
Mano Chris: 10 de Picas se empareja con 10 de Treboles
----------  Parejas eliminadas, comienza el juego
Mano Allen contiene
Rey de Corazones
 Jota de Treboles
  Reina de Picas
   Rey de Picas
    10 de Diamantes

Mano Jeff contiene
Jota de Picas
 Jota de Corazones
  Rey de Diamantes

Mano Chris contiene
Jota de Diamantes
 Rey de Treboles
  10 de Corazones

Mano Allen escogio Rey de Diamantes
Mano Allen: Rey de Corazones se empareja con Rey de 
Diamantes
Mano Jeff escogio 10 de Corazones
Mano Chris escogio Jota de Treboles
Mano Allen escogio Jota de Corazones
Mano Jeff escogio Jota de Diamantes
Mano Chris escogio Reina de Picas
Mano Allen escogio Jota de Diamantes
Mano Allen: Jota de Corazones se empareja con Jota de 
Diamantes
Mano Jeff escogio Rey de Treboles
Mano Chris escogio Rey de Picas
Mano Allen escogio 10 de Corazones
Mano Allen: 10 de Diamantes se empareja con 10 de Corazones
Mano Jeff escogio Reina de Picas
Mano Chris escogio Jota de Picas
Mano Chris: Jota de Treboles se empareja con Jota de Picas
Mano Jeff escogio Rey de Picas
Mano Jeff: Rey de Treboles se empareja con Rey de Picas
---------- Game is Over
Mano Allen esta vacia

Mano Jeff contiene
Reina de Picas

Mano Chris esta vacia
\end{pyconcode}
\begin{verbatim}

\end{verbatim}
Así que Jeff pierde.

\section{Glosario}
\begin{description}
\item [{Herencia:}] es la capacidad de definir una clase, modificando una
clase definida previamente.
\item [{Clase madre:}] esta es la clase de la que una clase hereda.
\item [{Clase hija:}] nueva clase creada por medio de herencia, también
recibe el nombre de ``subclase.''

\index{herencia} \index{clase madre} \index{clase madre} \index{subclase}
\end{description}

\section{Ejercicios}
\begin{enumerate}
\item Escriba \texttt{imprimaManos} que recorre la lista \texttt{self.manos}
e imprime cada mano.
\item Con base en la clase \texttt{secuenciaADN}, escriba una clase más
general que se denomine \texttt{biosecuencia} que maneje el alfabeto
de la biosecuencia como un atributo.
\item Agregue un método para chequear que la secuencia está bien formada,
esto es, cada elemento pertenece al alfabeto.
\end{enumerate}


\clearemptydoublepage % herencia

\chapter{Listas enlazadas}

\label{lista} \index{lista}

\section{Referencias incrustadas}

\index{referencia} \index{referencia incrustada} \index{referencia!incrustada}
\index{lista enlazada} \index{lista!enlazada} \index{nodo} \index{carga}

Hemos visto ejemplos de atributos (denominados \textbf{referencias
incrustadas}) que se refieren a otros objetos en la sección \ref{embedded}.
Una estructura de datos muy común (la \textbf{lista enlazada}), toma
ventaja de esta posibilidad.

Las listas enlazadas están hechas de \textbf{nodos}, que contienen
una referencia al siguiente nodo en la lista. Además, cada nodo contiene
una información denominada la \textbf{carga}.

Una lista enlazada se considera como una \textbf{estructura de datos
recursiva} si damos la siguiente definición.
\begin{quote}
Una lista enlazada es: 

\begin{itemize}
\item la lista vacía, representada por el valor \texttt{None}, o
\item un nodo que contiene una carga y una referencia a una lista enlazada.
\end{itemize}
\end{quote}
\index{estructura de datos recursiva} \index{estructura de datos!recursiva}

Las estructuras de datos recursivas se implementan naturalmente con
métodos recursivos.

\section{La clase \texttt{Nodo} }

\index{clase Nodo} \index{clase!Nodo}

Empezaremos con los métodos básicos de inicialización y el \texttt{\_\_str\_\_}
para que podamos crear y desplegar objetos:

\beforeverb 
\begin{pythoncode}
class Nodo:
  def __init__(self, carga=None, siguiente=None):
    self.carga = carga
    self.siguiente  = siguiente

  def __str__(self):
    return str(self.carga)
\end{pythoncode}
\afterverb Los parámetros para el método de inicialización son opcionales.
Por defecto la carga y el enlace \texttt{siguiente}, reciben el valor
\texttt{None}.

La representación textual de un nodo es la representación de la carga.
Como cualquier valor puede ser pasado a la función \texttt{str} ,
podemos almacenar cualquier tipo de valor en la lista.

Para probar la implementación, podemos crear un \texttt{Nodo} e imprimirlo:

\beforeverb 
\begin{pyconcode}
>>> nodo = Nodo("test")
>>> print(nodo)
test
\end{pyconcode}
\afterverb Para hacerlo más interesante, vamos a pensar en una lista
con varios nodos:

\beforeverb 
\begin{pyconcode}
>>> nodo1 = Nodo(1)
>>> nodo2 = Nodo(2)
>>> nodo3 = Nodo(3)
\end{pyconcode}
\afterverb Este código crea tres nodos, pero todavía no tenemos una
lista porque estos no estan \textbf{enlazados}. El diagrama de estados
luce así:

\beforefig \centerline{\includegraphics[scale=0.7]{illustrations/link1}}
\afterfig

Para enlazar los nodos, tenemos que lograr que el primer nodo se refiera
al segundo, y que el segundo se refiera al tercero:

\beforeverb 
\begin{pyconcode}
>>> nodo1.siguiente = nodo2
>>> nodo2.siguiente = nodo3
\end{pyconcode}
\afterverb La referencia del tercer nodo es \texttt{None}, lo que
indica que es el último nodo de la lista. Ahora el diagrama de estados
luce así:

\beforefig \centerline{\includegraphics[scale=0.9]{illustrations/link2}}
\afterfig

Ahora usted sabe cómo crear nodos y enlazarlos para crear listas.
Lo que todavía no está claro, es el por qué hacerlo.

\section{Listas como colecciones}

\index{colección}

Las listas son útiles porque proporcionan una forma de ensamblar múltiples
objetos en una entidad única, a veces llamada \textbf{colección}.
En el ejemplo, el primer nodo de la lista sirve como referencia a
toda la lista.

\index{lista!imprimir} \index{lista!como parámetro}

Para pasar la lista como parámetro, sólo tenemos que pasar una referencia
al primer nodo. Por ejemplo, la función \texttt{imprimirLista} toma
un solo nodo como argumento. Empieza con la cabeza de la lista, imprime
cada nodo hasta llegar al final:

\beforeverb 
\begin{pythoncode}
def imprimirLista(nodo):
  while nodo:
    print(nodo),
    nodo = nodo.siguiente
  print
\end{pythoncode}
\afterverb Para llamar este método, pasamos una referencia al primer
nodo:

\beforeverb 
\begin{pyconcode}
>>> imprimirLista(nodo1)
1 2 3
\end{pyconcode}
\afterverb Dentro de \texttt{imprimirLista} tenemos una referencia
al primer nodo de la lista, pero no hay variable que se refiera a
los otros nodos. Tenemos que usar el valor \texttt{siguiente} de cada
nodo para obtener el siguiente nodo.

Para recorrer una lista enlazada, es muy común usar una variable de
ciclo como \texttt{nodo} para que se refiera a cada uno de los nodos
en cada momento.

\index{variable de ciclo} \index{lista!recorrido} \index{recorrido}

Este diagrama muestra el valor de \texttt{lista} y los valores que
\texttt{nodo} toma:

\beforefig \centerline{\includegraphics{illustrations/link3}}
\afterfig
\begin{quote}
{\em Por convención, las listas se imprimen entre corchetes y los
elementos se separan por medio de comas, como en el ejemplo \texttt{{[}1,
2, 3{]}}. Como ejercicio modifique \texttt{imprimirLista} de forma
que muestre la salida en este formato.} 
\end{quote}

\section{Listas y recursión}

\label{listrecursion} \index{lista!recorrido recursivo} \index{recorrido}

Es natural implementar muchas operaciones sobre listas por medio de
métodos recursivos. Por ejemplo, el siguiente algoritmo recursivo
imprime una lista al revés:
\begin{enumerate}
\item Separe la lista en dos partes: el primer nodo (la cabeza de la lista);
y el resto.
\item Imprima el resto al revés.
\item Imprima la cabeza.
\end{enumerate}
Por supuesto, el paso 2, el llamado recursivo asume que ya tenemos
una forma de imprimir una lista al revés. Si asumimos que esto es
así —el salto de fe—entonces podemos convencernos de que el algoritmo
trabaja correctamente.

\index{salto de fe} \index{listas!imprimiendo al revés}

Todo lo que necesitamos es un caso base y una forma de demostrar que
para cualquier lista, eventualmente llegaremos al caso base. Dada
la definición recursiva de una lista, un caso base natural es la lista
vacía, representada por \texttt{None}:

\beforeverb 
\begin{pythoncode}
def imprimirAlReves(lista):
  if lista == None: 
    return
  cabeza = lista
  resto = lista.siguiente
  imprimirAlReves(resto)
  print(cabeza),
\end{pythoncode}
\afterverb La primera línea resuelve el caso base. Las siguientes
separan la \texttt{cabeza} y el \texttt{resto}. Las últimas dos líneas
imprimen la lista. La coma al final de la última línea evita que Python
introduzca una nueva línea después de cada nodo.

Ahora llamamos a este método:

\beforeverb 
\begin{pyconcode}
>>> imprimirAlReves(nodo1)
3 2 1
\end{pyconcode}
\afterverb El efecto es una impresión la lista, al revés.

Una pregunta natural que usted se puede estar formulando es, ¿por
qué razón \texttt{imprimirAlReves} e \texttt{imprimirLista} son funciones
y no métodos en la clase \texttt{Nodo}? La razón es que queremos usar
a \texttt{None} para representar la lista vacía y no se puede llamar
un método sobre \texttt{None} en Python. Esta limitación hace un poco
engorroso escribir el código para manipulación de listas siguiendo
la programación orientada a objetos.

¿Podemos demostrar que \texttt{imprimirAlReves} va a terminar siempre?
En otras palabras, ¿llegará siempre al caso base? De hecho, la respuesta
es negativa, algunas listas causarán un error de ejecución.

\section{Listas infinitas }

\index{lista infinita} \index{lista!infinita} \index{ciclos!en listas}
\index{lista!ciclo}

No hay manera de evitar que un nodo se refiera a un nodo anterior
en la lista hacia ``atrás''. Incluso, puede referirse a sí mismo.
Por ejemplo, la siguiente figura muestra una lista con dos nodos,
uno de los cuales se refiere a sí mismo:

\beforefig \centerline{\includegraphics{illustrations/link4}}
\afterfig

Si llamamos a \texttt{imprimirLista} sobre esta lista, iteraría para
siempre. Si llamamos a \texttt{imprimirAlReves}, se haría recursión
hasta causar un error en tiempo de ejecución. Este comportamiento
hace a las listas circulares muy difíciles de manipular.

Sin embargo, a veces son muy útiles. Por ejemplo, podemos representar
un número como una lista de dígitos y usar una lista infinita para
representar una fracción periódica.

Así que no es posible demostrar que \texttt{imprimirLista} e \texttt{imprimirAlReves}
terminen. Lo mejor que podemos hacer es probar la sentencia, ``Si
la lista no tiene referencias hacia atrás, los métodos terminarán.''.
Esto es una \textbf{precondición}. Impone una restricción sobre los
parámetros y describe el comportamiento del método si ésta se cumple.
Más adelante veremos otros ejemplos.

\index{precondición}

\section{El teorema de la ambigüedad fundamental}

\index{ambigüedad!teorema fundamental} \index{teorema!fundamental de la ambigüedad}

Una parte de \texttt{imprimirAlReves} puede haber suscitado su curiosidad:

\beforeverb 
\begin{pythoncode}
    cabeza = lista
    resto = lista.siguiente
\end{pythoncode}
\afterverb Después de la primera asignación \texttt{cabeza} y \texttt{lista}
tienen el mismo tipo y el mismo valor. ¿Por qué creamos una nueva
variable?

La respuesta yace en que las dos variables tienen roles distintos.
\texttt{cabeza} es una referencia a un nodo y lista es una referencia
a toda la lista. Estos ``roles'' están en la mente del programador
y le ayudan a mantener la coherencia de los programas.

\index{variable!roles} \index{rol!variable}

En general, no podemos decir inmediatamente qué rol juega una variable
en un programa. Esta ambigüedad puede ser útil, pero también dificulta
la lectura. Los nombres de las variables pueden usarse para documentar
la forma en que esperamos que se use una variable, y, a menudo, podemos
crear variables adicionates como \texttt{nodo} y \texttt{lista} para
eliminar ambigüedades.

Podríamos haber escrito \texttt{imprimirAlReves} de una manera más
concisa sin las variables \texttt{cabeza} y \texttt{resto}, pero esto
también dificulta su lectura:

\beforeverb 
\begin{pythoncode}
def imprimirAlReves(lista) :
  if lista == None : 
     return
  imprimirAlReves(lista.siguiente)
  print(lista),
\end{pythoncode}
\afterverb Cuando leamos el código, tenemos que recordar que \texttt{imprimirAlReves}
trata a su argumento como una colección y \texttt{print} como a un
solo nodo.

El \textbf{teorema de la ambigüedad fundamental} describe la ambigüedad
inherente en la referencia a un nodo:
\begin{quote}
\textbf{Una variable que se refiera a un nodo puede tratar el nodo
como un objeto único o como el acceso a la lista de nodos} 
\end{quote}

\section{Modificando listas}

\index{lista!modificando} \index{modificando listas}

Hay varias formas de modificar una lista enlazada. La obvia consiste
en cambiar la carga de uno de sus nodos. Las mas interesantes son
las que agregan, eliminan o reordenan los nodos.

Como ejemplo, escribamos un método que elimine el segundo nodo en
la lista y retorne una referencia al nodo eliminado

\beforeverb 
\begin{pythoncode}
def eliminarSegundo(lista):
  if lista == None: 
     return
  primero = lista
  segundo = lista.siguiente
  # hacemos que el primer nodo se refiera al tercero
  primero.siguiente = segundo.siguiente
  # desconectamos el segundo nodo de la lista
  segundo.siguiente = None
  return segundo
\end{pythoncode}
\afterverb Aquí también estamos usando variables temporales para
aumentar la legibilidad. Aquí hay un ejemplo de uso del método:

\beforeverb 
\begin{pyconcode}
>>> imprimirLista(nodo1)
1 2 3
>>> borrado = eliminarSegundo(nodo1)
>>> imprimirLista(borrado)
2
>>> imprimirLista(nodo1)
1 3
\end{pyconcode}
\afterverb Este diagrama de estado muestra el efecto de la operación:

\beforefig \centerline{\includegraphics{illustrations/link5}}
\afterfig

¿Qué pasa si usted llama este método con una lista que contiene un
solo elemento (un \textbf{singleton})? ¿Qué pasa si se llama con la
lista vacía como argumento? ¿Hay precondiciones para este método?
Si las hay, corríjalo de forma que maneje de manera razonable las
violaciones a la precondición.

\index{singleton}

\section{Funciones facilitadoras (wrappers) y auxiliares (helpers)}

\index{método facilitador} \index{método!facilitador} \index{función facilitadora}
\index{función!facilitadora} \index{método auxiliar} \index{método!auxiliar}

Es bastante útil dividir las operaciones de listas en dos métodos.
Con la impresión al revés podemos ilustrarlo, para desplegar \texttt{{[}3,
2, 1{]}} en pantalla podemos llamar el método \texttt{imprimirAlReves}
que desplegará \texttt{3, 2}, y llamar otro método para imprimir los
corchetes y el primer nodo. Nombrémosla así:

\beforeverb 
\begin{pythoncode}
def imprimirAlRevesBien(lista):
  print("["),
  if lista != None:
    cabeza = lista
    resto = lista.siguiente
    imprimirAlReves(resto)
    print(cabeza),
  print("]"),
\end{pythoncode}
\afterverb Es conveniente chequear que estos métodos funcionen bien
para casos especiales como la lista vacía o una lista con un solo
elemento (singleton).

\index{singleton}

Cuando usamos este método en algún programa, llamamos directamente
a la función \texttt{imprimirAlRevesBien} para que llame a \texttt{imprimirAlReves}.
En este sentido, \texttt{imprimirAlRevesBien} es una función \textbf{facilitadora},
que utiliza a la otra, \texttt{imprimirAlReves} como función \textbf{auxiliar}.

\section{La clase \texttt{ListaEnlazada}}

\index{ListaEnlazada} \index{clase!ListaEnlazada}

Hay problemas más sutiles en nuestra implementación de listas que
vamos a ilustrar desde los efectos a las causas, a partir de una implementación
alternativa exploraremos los problemas que resuelve.

Primero, crearemos una clase nueva llamada \texttt{ListaEnlazada}.
Tiene como atributos un entero con el número de elementos de la lista
y una referencia al primer nodo. Las instancias de \texttt{ListaEnlazada}
sirven como mecanismo de control de listas compuestas por instancias
de la clase \texttt{Nodo}:

\beforeverb 
\begin{pythoncode}
class ListaEnlazada :
  def __init__(self) :
    self.numElementos = 0
    self.cabeza   = None
\end{pythoncode}
\afterverb Lo bueno de la clase \texttt{ListaEnlazada} es que proporciona
un lugar natural para definir las funciones facilitadores como \texttt{imprimirAlRevesBien}
como métodos:

\beforeverb 
\begin{pythoncode}
class ListaEnlazada:
  ...
  def imprimirAlReves(self):
    print("["),
    if self.cabeza != None:
      self.cabeza.imprimirAlReves()
    print("]"),

class Nodo:
  ...
  def imprimirAlReves(self):
    if self.siguiente != None:
      resto = self.siguiente
      resto.imprimirAlReves()
    print(self.carga),
\end{pythoncode}
\afterverb Aunque inicialmente pueda parecer un poco confuso, vamos
a renombrar a la función \texttt{imprimirAlRevesBien}. Ahora vamos
a implementar dos métodos con el mismo nombre \texttt{imprimirAlReves}:
uno en la clase \texttt{Nodo} (el auxiliar); y uno en la clase \texttt{ListaEnlazada}
(el facilitador). Cuando el facilitador llama al otro método, \texttt{self.cabeza.imprimirAlReves},
está invocando al auxiliar, porque \texttt{self.cabeza} es una instancia
de la clase \texttt{Nodo}.

Otro beneficio de la clase \texttt{ListaEnlazada} es que facilita
agregar o eliminar el primer elemento de una lista. Por ejemplo, \texttt{agregarAlPrincipio}
es un método de la clase \texttt{ListaEnlazada} que toma una carga
como argumento y la pone en un nuevo nodo al principio de la lista:

\beforeverb 
\begin{pythoncode}
class ListaEnlazada:
  ...
  def agregarAlPrincipio(self, carga):
    nodo = Nodo(carga)
    nodo.siguiente = self.cabeza
    self.cabeza = nodo
    self.numElementos = self.numElementos + 1
\end{pythoncode}
\afterverb Como de costumbre, usted debe revisar este código para
verificar qué sucede con los casos especiales. Por ejemplo, ¿qué pasa
si se llama cuando la lista está vacía?

\section{Invariantes}

\index{Invariante} \index{Invariante de objetos} \index{lista!bien formada}

Algunas listas están ``bien formadas``. Por ejemplo,
si una lista contiene un ciclo, causará problemas graves a nuestros
métodos, así que deseamos evitar a toda costa que las listas tengan
ciclos. Otro requerimiento de las listas es que el número almacenado
en el atributo \texttt{numElementos} de la clase \texttt{ListaEnlazada}
sea igual al número de elementos en la lista.

Estos requerimientos se denominan \textbf{Invariantes} porque, idealmente,
deberían ser ciertos para todo objeto de la clase en todo momento.
Es una muy buena práctica especificar los Invariantes para los objetos
porque permite comprobar de manera mas sencilla la corrección del
código, revisar la integridad de las estructuras de datos y detectar
errores.

Algo que puede confundir acerca de los invariantes es que hay ciertos
momentos en que son violados. Por ejemplo, en el medio de \texttt{agregarAlPrincipio},
después de que hemos agregado el nodo, pero antes de incrementar el
atributo \texttt{numElementos}, el Invariante se viola. Esta clase
de violación es aceptable, de hecho, casi siempre es imposible modificar
un objeto sin violar un Invariante, al menos momentáneamente. Normalmente,
requerimos que cada método que viole un invariante, lo establezca
nuevamente.

Si hay una parte significativa de código en la que el Invariante se
viola, es importante documentarlo claramente, de forma que no se ejecuten
operaciones que dependan del Invariante.

\index{documentar}

\section{Glosario}

\index{referencia incrustada} \index{referencia!incrustada} \index{estructura de datos recursiva}
\index{estructura de datos!recursiva} \index{lista enlazada} \index{lista!enlazada}
\index{nodo} \index{dato} \index{enlace} \index{precondición}
\index{invariante} \index{facilitador} \index{método auxiliar}
\index{teorema fundamental de la ambigüedad} \index{singleton}
\begin{description}
\item [{Referencia incrustada:}] referencia almacenada en un atributo
de un objeto.
\item [{Lista enlazada:}] es la estructura de datos que implementa una
colección por medio de una secuencia de nodos enlazados.
\item [{Nodo:}] elemento de la lista, usualmente implementado como un objeto
que contiene una referencia hacia otro objeto del mismo tipo.
\item [{Carga:}] dato contenido en un nodo.
\item [{Enlace:}] referencia incrustada, usada para enlazar un objeto con
otro.
\item [{Precondición:}] condición lógica (o aserción) que debe ser cierta
para que un método funcione correctamente.
\item [{Teorema fundamental de la ambigüedad:}] la referencia a un nodo
de una lista puede interpretarse hacia un nodo determinado o como
la referencia a toda la lista de nodos.
\item [{Singleton:}] lista enlazada con un solo nodo.
\item [{Facilitador:}] método que actúa como intermediario entre alguien
que llama un método y un método auxiliar. Se crean normalmente para
facilitar los llamados y hacerlos menos propensos a errores.
\item [{Método auxiliar:}] es un método que el programador no llama directamente,
sino que es usado por otro método para realizar parte de una operación.
\item [{Invariante:}] aserción que debe ser cierta para un objeto en todo
momento (excepto cuando el objeto está siendo modificado).
\end{description}


\clearemptydoublepage % listas enlazadas

\chapter{Pilas}

\section{Tipos abstractos de datos}

\index{tipos abstractos de datos} \index{TAD} \index{encapsulamiento}

Los tipos de datos que ha visto hasta el momento son concretos, en
el sentido que hemos especificado completamente como se implementan.
Por ejemplo, la clase \texttt{Carta} representa una carta por medio
de dos enteros. Pero esa no es la única forma de representar una carta;
hay muchas representaciones alternativas.

Un \textbf{tipo abstracto de datos}, o TAD, especifica un conjunto
de operaciones (o métodos) y la semántica de las operaciones (lo que
hacen), pero no especifica la implementación de las operaciones. Eso
es lo que los hace abstractos.

¿Qué es lo que los hace tan útiles?
\begin{itemize}
\item La tarea de especificar un algoritmo se simplifica si se pueden denotar
las operaciones sin tener que pensar al mismo tiempo como se implementan.
\item Como usualmente hay muchas formas de implementar un TAD, puede ser
provechoso escribir un algoritmo que pueda usarse con cualquier implementación
alternativa.
\item Los TADs bien conocidos, como el TAD Pila de este capítulo, a menudo
se encuentran implementados en las bibliotecas estándar de los lenguajes
de programación, así que pueden escribirse una sola vez y usarse muchas
veces.
\item Las operaciones de los TADs nos proporcionan un lenguaje de alto nivel
para especificar algoritmos.
\end{itemize}
Cuando hablamos de TADs hacemos la distinción entre el código que
utiliza el TAD, denominado código \textbf{cliente}, del código que
implementa el TAD, llamado código \textbf{proveedor}.

\index{cliente} \index{proveedor}

\section{El TAD Pila}

\index{pila} \index{colección} \index{TAD!Pila}

Como ya hemos aprendido a usar otras colecciones como los diccionarios
y las listas, en este capítulo exploraremos un TAD muy general, la
\textbf{pila}.

Una pila es una colección, esto es, una estructura de datos que contiene
múltiples elementos. \index{interfaz}

Un TAD se define por las operaciones que se pueden ejecutar sobre
él, lo que recibe el nombre de \textbf{interfaz}. La interfaz de una
pila comprende las siguientes operaciones:
\begin{description}
\item [{texttt{\_\_init\_\_}:}] inicializa una pila vacía.
\item [{texttt{meter}:}] agrega un objeto a la pila.
\item [{texttt{sacar}:}] elimina y retorna un elemento de la pila. El
objeto que se retorna siempre es el último que se agregó.
\item [{texttt{estaVacia}:}] revisa si la pila está vacía.
\end{description}
Una pila también se conoce como una estructura``último que Entra,
Primero que Sale '' o UEPS, porque el último dato que entró es el
primero que va a salir.

\section{Implementando pilas por medio de listas de Python}

\index{Pila} \index{clase!Pila} \index{Estructura de datos genérica}
\index{Estructura de datos!genérica}

Las operaciones de listas que Python proporciona son similares a las
operaciones que definen una pila. La interfaz no es lo que uno se
espera, pero podemos escribir código que traduzca desde el TAD pila
a las operaciones primitivas de Python.

Este código se denomina la \textbf{implementación} del TAD Pila. En
general, una implementación es un conjunto de métodos que satisfacen
los requerimientos sintácticos y semánticos de una interfaz.

Aquí hay una implementación del TAD Pila que usa una lista de Python:

\beforeverb 
\begin{pythoncode}
class Pila :
  def __init__(self) :
    self.items = []

  def meter(self, item) :
    self.items.append(item)

  def sacar(self) :
    return self.items.pop()

  def estaVacia(self) :
    return (self.items == [])
\end{pythoncode}
\afterverb Una objeto \texttt{Pila} contiene un atributo llamado
\texttt{items} que es una lista de los objetos que están en la Pila.
El método de inicialización le asigna a \texttt{items} la lista vacía.

Para meter un nuevo objeto en la Pila, \texttt{meter} lo pega a \texttt{items}.
Para sacar un objeto de la Pila, \texttt{sacar} usa al método \texttt{pop}
que proporciona Python para eliminar el último elemento de una lista.

Finalmente, para verificar si la Pila está vacía, \texttt{estaVacia}
compara a \texttt{items} con la lista vacía.

\index{barniz}

Una implementación como ésta, en la que los métodos son simples llamados
de métodos existentes, se denomina \textbf{barniz}. En la vida real,
el barniz es una delgada capa de protección que se usa algunas veces
en la fabricación de muebles para ocultar la calidad de la madera
que recubre. Los científicos de la computación usan esta metáfora
para describir una pequeña porción de código que oculta los detalles
de una implementación para proporcionar una interfaz más simple o
más estandarizada.

\section{Meter y sacar}

\index{meter} \index{sacar} \index{estructura de datos genérica}
\index{estructura de datos!genérica}

Una Pila es una \textbf{estructura de datos genérica}, o sea que podemos
agregar datos de cualquier tipo a ella. El siguiente ejemplo mete
dos enteros y una cadena en la Pila:

\beforeverb 
\begin{pyconcode}
>>> s = Pila()
>>> s.meter(54)
>>> s.meter(45)
>>> s.meter("+")
\end{pyconcode}
\afterverb Podemos usar los métodos \texttt{estaVacia} y \texttt{sacar}
para eliminar e imprimir todos los objetos en la Pila:

\beforeverb 
\begin{pythoncode}
while not s.estaVacia() :
  print(s.sacar()),
\end{pythoncode}
\afterverb La salida es \texttt{+ 45 54}. En otras palabras, acabamos
de usar la Pila para imprimir los objetos al revés, ¡y de una manera
muy sencilla!

Compare esta porción de código con la implementación de \texttt{imprimirAlReves}
en la Sección~\ref{listrecursion}. Hay una relación muy profunda
e interesante entre la versión recursiva de \texttt{imprimirAlReves}
y el ciclo anterior. La diferencia reside en que \texttt{imprimirAlReves}
usa la Pila que provee el ambiente de ejecución de Python para llevar
pista de los nodos mientras recorre la lista, y luego los imprime
cuando la recursión se empieza a devolver. El ciclo anterior hace
lo mismo, pero explícitamente por medio de un objeto Pila.

\section{Evaluando expresiones postfijas con una Pila}

\index{postfija} \index{infija} \index{expresión}

En la mayoría de los lenguajes de programación las expresiones matemáticas
se escriben con el operador entre los operandos, como en \texttt{1+2}.
Esta notación se denomina \textbf{infija}. Una alternativa que algunas
calculadoras usan se denomina notación \textbf{postfija}. En la notación
postfija, el operador va después de los operandos, como en \texttt{1
2 +}.

La razón por la que esta notación es útil reside en que hay una forma
muy natural de evaluar una expresión postfija usando una Pila:
\begin{itemize}
\item Comenzando por el inicio de la expresión, ir leyendo cada término
(operador u operando).

\begin{itemize}
\item Si el término es un operando, meterlo en la Pila.
\item Si el término es un operador, sacar dos operandos de la Pila, ejecutar
la operación sobre ellos, y meter el resultado en la Pila.
\end{itemize}
\item Cuando llegue al final de la expresión, tiene que haber un solo aperando
en la Pila, ese es el resultado de la expresión.
\end{itemize}
\begin{quote}
{\em Como ejercicio, aplique este algoritmo a la expresión \texttt{1
2 + 3 {*}}.} 
\end{quote}
Este ejemplo demuestra una de las ventajas de la notación postfija—no
hay necesidad de usar paréntesis para controlar el orden de las operaciones.
Para obtener el mismo resultado en notación infija tendríamos que
haber escrito \texttt{(1 + 2) {*} 3}.
\begin{quote}
{\em Como ejercicio, escriba una expresión postfija equivalente
a \texttt{1 + 2 {*} 3}.} 
\end{quote}

\section{Análisis sintáctico}

\index{análisis sintáctico} \index{lexema} \index{delimitador}
\index{expresión regular}

Para implementar el algoritmo anterior, necesitamos recorrer una cadena
y separarla en operandos y operadores. Este proceso es un ejemplo
de \textbf{análisis sintáctico}, y los resultados –los trozos individuales
que obtenemos—se denominan \textbf{lexemas}. Tal vez recuerde estos
conceptos introducidos en el capítulo 2.

Python proporciona el método \texttt{split} en los módulos \texttt{string}
y \texttt{re} (expresiones regulares). La función \texttt{string.split}
parte una cadena en una lista de cadenas usando un caracter como \textbf{delimitador}.
Por ejemplo:

\beforeverb 
\begin{pyconcode}
>>> import string
>>> string.split("Ha llegado la hora"," ")
['Ha', 'llegado', 'la', 'hora']
\end{pyconcode}
\afterverb En este caso, el delimitador es el caracter espacio, así
que la cadena se separa cada vez que se encuentra un espacio.

La función \texttt{re.split} es mas poderosa, nos permite especificar
una expresión regular en lugar de un delimitador. Una expresión regular
es una forma de especificar un conjunto de cadenas. Por ejemplo, \verb+[A-Z]+
es el conjunto de todas las letras y \verb+[0-9]+ es el conjunto
de todos los números. El operador \verb+^+ niega un conjunto, así
que \verb+[^0-9]+ es el conjunto complemento al de números (todo
lo que no es un número), y esto es exactamente lo que deseamos usar
para separar una expresión postfija:

\beforeverb 
\begin{pyconcode}
>>> import re
>>> re.split("([^0-9])", "123+456*/")
['123', '+', '456', '*', '', '/', '']
\end{pyconcode}
\afterverb Observe que el orden de los argumentos es diferente al
que se usa en la función \texttt{string.split}; el delimitador va
antes de la cadena.

La lista resultante contiene a los operandos \texttt{123} y \texttt{456},
y a los operadores \texttt{{*}} y \texttt{/}. También incluye dos
cadenas vacías que se insertan después de los operandos.

\section{Evaluando expresiones postfijas}

Para evaluar una expresión postfija, utilizaremos el analizador sintáctico
de la sección anterior y el algoritmo de la anterior a esa. Por simplicidad,
empezamos con un evaluador que solo implementa los operadores \texttt{+}
y \texttt{{*}}:

\adjustpage{-3} %\pagebreak

%\beforeverb
\begin{pythoncode}
def evalPostfija(expr):
  import re
  listaLexemas = re.split("([^0-9])", expr)
  Pila = Pila()
  Por lexema in listaLexemas:
    if  lexema == '' or lexema == ' ':
      continue
    if  lexema == '+':
      suma = Pila.sacar() + Pila.sacar()
      Pila.meter(suma)
    elif lexema == '*':
      producto = Pila.sacar() * Pila.sacar()
      Pila.meter(producto)
    else:
      Pila.meter(int(lexema))
  return Pila.sacar()
\end{pythoncode}
%\afterverbLa primera condición ignora los espacios y las cadenas
vacías. Las siguientes dos condiciones detectan los operadores. Asumimos
por ahora —intrépidamente—, que cualquier caracter no numérico es
un operando.

Verifiquemos la función evaluando la expresión \texttt{(56+47){*}2}
en notación postfija:

\beforeverb 
\begin{pyconcode}
>>> print(evalPostfija ("56 47 + 2 *"))
206
\end{pyconcode}
\afterverb Bien, por ahora.

\section{Clientes y proveedores}

\index{encapsulamiento} \index{TAD}

Una de los objetivos fundamentales de un TAD es separar los intereses
del proveedor, el que escribe el código que implementa el Tipo Abstracto
de Datos, y los del cliente, el que lo usa. El proveedor sólo tiene
que preocuparse por que la implementación sea correcta —de acuerdo
con la especificación del TAD—y no por el cómo va a ser usado.

Por otro lado, el cliente {\em asume} que la implementación del
TAD es correcta y no se preocupa por los detalles. Cuando usted utiliza
los tipos primitivos de Python, se está dando el lujo de pensar como
cliente exclusivamente.

Por supuesto, cuando se implementa un TAD, también hay que escribir
algún código cliente que permita chequear su funcionamiento. En ese
caso, usted asume los dos roles y la labor puede ser un tanto confusa.
Hay que concentrarse para llevar la pista del rol que se está jugando
en un momento determinado.

\section{Glosario}

\index{TAD} \index{cliente} \index{proveedor} \index{infija} \index{postfija}
\index{análisis sintáctico} \index{lexema} \index{delimitador}
\begin{description}
\item [{Tipo Abstracto de Datos (TAD):}] Un tipo de datos (casi siempre
es una colección de objetos) que se define por medio de un conjunto
de operaciones y que puede ser implementado de diferentes formas.
\item [{Interfaz:}] conjunto de operaciones que define al TAD.
\item [{Implementación:}] código que satisface los requerimientos sintácticos
y semánticos de una interfaz de un TAD.
\item [{Cliente:}] un programa (o la persona que lo escribió) que usa un
TAD.
\item [{Proveedor:}] el código (o la persona que lo escribió) que implementa
un TAD.
\item [{Barniz:}] una definición de clase que implementa un TAD con métodos
que son llamados a otros métodos, a menudo realizando unas transformaciones
previas. El barniz no realiza un trabajo significativo, pero sí mejora
o estandariza las interfaces a las que accede el cliente.
\item [{Estructura de datos genérica:}] estructura de datos que puede
contener objetos de todo tipo.
\item [{Infija:}] una forma de escribir expresiones matemáticas con los
operadores entre los operandos.
\item [{Postfija:}] una forma de escribir expresiones matemáticas con los
operadores después de los operandos.
\item [{Análisis sintáctico:}] leer una cadena de caracteres o lexemas
y analizar su estructura gramatical.
\item [{Lexema:}] conjunto de caracteres que se considera como una unidad
para los propósitos del análisis sintáctico, tal como las palabras
del lenguaje natural.
\item [{Delimitador:}] caracter que se usa para separar lexemas, tal como
los signos de puntuación en el lenguaje natural.
\end{description}


\clearemptydoublepage % pilas

\chapter{Colas}

\label{Cola} \index{cola} \index{TAD!Cola} \index{cola de prioridad}
\index{TAD!cola de prioridad} \index{PEPS} \index{política para meter}
\index{cola de prioridad}

Este capítulo presenta dos TADs: la cola y la cola de prioridad. En
la vida real una \textbf{Cola} es una línea de clientes esperando
por algún servicio. En la mayoría de los casos, el primer cliente
en la línea es el próximo en ser atendido. Sin embargo, hay excepciones.
En los aeropuertos, los clientes cuyos vuelos están próximos a partir
se atienden, sin importar su posición en la cola. En los supermercados,
un cliente cortés puede dejar pasar a alguien que va a pagar unos
pocos víveres.

La regla que dictamina quién se atiende a continuación se denomina
\textbf{política de atención}. La más sencilla se denomina \textbf{PEPS},
por la frase ``Primero que Entra - Primero que Sale''. La política
más general es la que implementa una \textbf{cola de prioridad}, en
la que cada cliente tiene asignada una prioridad y siempre se atiende
el cliente con la prioridad más alta, sin importar el orden de llegada.
Es la política más general en el sentido de que la prioridad puede
asignarse bajo cualquier criterio: la hora de partida de un vuelo,
cuántos víveres se van a pagar, o qué tan importante es el cliente.
No todas las políticas de atención son ``justas,'' pero la justicia
está definida por el que presta el servicio.

El TAD Cola y la cola de prioridad TAD tienen el mismo conjunto de
operaciones. La diferencia está en la semántica de ellas: una cola
utiliza la política PEPS y una cola de prioridad usa la política de
prioridad.

\adjustpage{1}

\section{El TAD Cola}

\index{TAD!Cola} \index{Cola TAD} \index{implementación!Cola}
\index{Cola!implementación con lista}

El TAD Cola se define por la siguiente interfaz:
\begin{description}
\item [{texttt{\_\_init\_\_}:}] inicializa una Cola vacía.
\item [{texttt{meter}:}] agrega un nuevo objeto a la cola.
\item [{texttt{sacar}:}] elimina y retorna un objeto de la Cola. Entre
todos los que están dentro de la cola, el objeto retornado fue el
primero en agregarse
\item [{texttt{estaVacia}:}] revisa si la cola está vacía.
\end{description}

\section{Cola enlazada}

\index{Cola enlazada} \index{Cola!implementación enlazada}

Esta primera implementación del TAD Cola se denomina \textbf{Cola
enlazada} porque está compuesta de objetos \texttt{Nodo} enlazados.
Aquí está la definición:

\beforeverb 
\begin{pythoncode}
class Cola:
  def __init__(self):
    self.numElementos = 0
    self.primero = None

  def estaVacia(self):
    return (self.numElementos == 0)

  def meter(self, carga):
    nodo = Nodo(carga)
    nodo.siguiente = None
    if self.primero == None:
      # si esta vacia este nodo sera el primero
      self.primero = nodo
    else:
      # encontrar el ultimo nodo
      ultimo = self.primero
      while ultimo.siguiente: ultimo = ultimo.siguiente
      # pegar el nuevo
      ultimo.siguiente = nodo
    self.numElementos = self.numElementos + 1

  def sacar(self):
    carga = self.primero.carga
    self.primero = self.primero.siguiente
    self.numElementos = self.numElementos - 1
    return carga
\end{pythoncode}
\afterverb El método \texttt{estaVacia} es idéntico al de la \texttt{ListaEnlazada},
\texttt{sacar} es quitar el enlace del primer nodo. El método \texttt{meter}
es un poco más largo.

Si la cola está vacía, le asignamos a \texttt{primero} el nuevo nodo.

Si tiene elementos, recorremos la lista hasta el último nodo y pegamos
el nuevo nodo al final. Podemos detectar si hemos llegado al final
de la lista porque el atributo \texttt{siguiente} tiene el valor \texttt{None}.

Hay dos invariantes que un objeto \texttt{Cola} bien formado debe
cumplir. El valor de \texttt{numElementos} debe ser el número de nodos
en la Cola, y el último nodo debe tener en \texttt{siguiente} el valor
\texttt{None}. Verifique que este método satisfaga los dos invariantes.

\section{Desempeño}

\index{desempeño}

Usualmente, cuando llamamos un método, no nos interesan los detalles
de la implementación. Sin embargo, hay un ``detalle'' que quisiéramos
saber —el desempeño del nuevo método. ¿Cuánto tarda en ejecutarse
y cómo cambia el tiempo de ejecución a medida que el número de objetos
en la Cola crece?

Primero observemos al método \texttt{sacar}.

No hay ciclos ni llamados a funciones, así que el tiempo de ejecución
de este método es el mismo cada vez que se ejecuta. Los métodos de
este tipo se denominan operaciones de \textbf{tiempo constante}. De
hecho, el método puede ser mas rápido cuando la lista está vacía ya
que no entra al cuerpo del condicional, pero esta diferencia no es
significativa.

\index{tiempo constante}

El desempeño de \texttt{meter} es muy diferente. En el caso general,
tenemos que recorrer la lista para encontrar el último elemento.

Este recorrido toma un tiempo proporcional al atributo numElementos
de la lista. Ya que el tiempo de ejecución es una función lineal de
numElementos, se dice que este método tiene un tiempo de ejecución
de \textbf{tiempo lineal}. Comparado con el tiempo constante, es bastante
malo.

\index{tiempo lineal}

\section{Cola Enlazada mejorada}

\index{Cola!implementación mejorada} \index{Cola mejorada}

Nos gustaría contar con una implementación del TAD Cola, cuyas operaciones
tomen tiempo constante. Una forma de hacerlo es manteniendo una referencia
al último nodo, como se ilustra en la figura siguiente:

\beforefig \centerline{\includegraphics{illustrations/queue1}}
\afterfig

La implementación de \texttt{ColaMejorada} es la siguiente:

\beforeverb 
\begin{pythoncode}
class ColaMejorada:
  def __init__(self):
    self.numElementos = 0
    self.primero   = None
    self.ultimo   = None

  def estaVacia(self):
    return (self.numElementos == 0)
\end{pythoncode}
\afterverb Hasta aquí el único cambio es al nuevo atributo \texttt{ultimo}.
Este debe ser usado en los métodos \texttt{meter} y \texttt{sacar}:

\beforeverb 
\begin{pythoncode}
class ColaMejorada:
  ...
  def meter(self, carga):
    nodo = nodo(carga)
    nodo.siguiente = None
    if self.numElementos == 0:
      # si está vacía, el nuevo nodo es primero y ultimo
      self.primero = self.ultimo = nodo
    else:
      # encontrar el ultimo nodo
      ultimo = self.ultimo
      # pegar el nuevo nodo
      ultimo.siguiente = nodo
      self.ultimo = nodo
    self.numElementos = self.numElementos + 1
\end{pythoncode}
\afterverb Ya que \texttt{ultimo} lleva la pista del último nodo,
no tenemos que buscarlo. Como resultado, este método tiene un tiempo
constante de ejecución.

Hay un precio que pagar por esta mejora. Tenemos que agregar un caso
especial a \texttt{sacar} que asigne a \texttt{ultimo} el valor \texttt{None}
cuando se saca el único elemento:

\beforeverb 
\begin{pythoncode}
class ColaMejorada:
  ...
  def sacar(self):
    carga     = self.primero.carga
    self.primero = self.primero.siguiente
    self.numElementos = self.numElementos - 1
    if self.numElementos == 0:
      self.ultimo = None
    return carga
\end{pythoncode}
\afterverb Esta implementación es más compleja que la inicial y es
más difícil demostrar su corrección. La ventaja es que hemos logrado
el objetivo —\texttt{meter} y \texttt{sacar} son operaciones que se
ejecutan en un tiempo constante.
\begin{quote}
{\em Como ejercicio, escriba una implementación del TAD Cola usando
una lista de Python. Compare el desempeño de esta implementación con
el de la \texttt{ColaMejorada} para un distintos valores de numElementos.} 
\end{quote}

\section{Cola de prioridad}

\index{cola de prioridad!TAD} \index{TAD!cola de prioridad}

El TAD cola de prioridad tiene la misma interfaz que el TAD Cola,
pero su semántica es distinta.
\begin{description}
\item [{texttt{\_\_init\_\_}:}] inicializa una Cola vacía.
\item [{texttt{meter}:}] agrega un objeto a la Cola.
\item [{texttt{sacar}:}] saca y retorna un objeto de la Cola. El objeto
que se retorna es el de la mas alta prioridad.
\item [{texttt{estaVacia}:}] verifica si la Cola está vacía.
\end{description}
La diferencia semántica está en el objeto que se saca, que necesariamente
no es el primero que se agregó. En vez de esto, es el que tiene el
mayor valor de prioridad. Las prioridades y la manera de compararlas
no se especifica en la implementación de la cola de prioridad. Depende
de cuáles objetos estén en la Cola.

Por ejemplo, si los objetos en la Cola tienen nombres, podríamos escogerlos
en orden alfabético. Si son puntajes de bolos iríamos sacando del
más alto al más bajo; pero si son puntajes de golf, iríamos del más
bajo al más alto. En tanto que podamos comparar los objetos en la
Cola, podemos encontrar y sacar el que tenga la prioridad más alta.

Esta implementación de la cola de prioridad tiene como atributo una
lista de Python que contiene los elementos en la Cola.

\beforeverb 
\begin{pythoncode}
class ColaPrioridad:
  def __init__(self):
    self.items = []

  def estaVacia(self):
    return self.items == []

  def meter(self, item):
    self.items.append(item)
\end{pythoncode}
\afterverb El método de inicialización, \texttt{estaVacia}, y \texttt{meter}
solo son barniz para operaciones sobre listas. El único interesante
es \texttt{sacar}:

\beforeverb 
\begin{pythoncode}
class ColaPrioridad:
  ...
  def sacar(self):
    maxi = 0
    for i in range(1,len(self.items)):
      if self.items[i] > self.items[maxi]:
        maxi = i
    item = self.items[maxi]
    self.items[maxi:maxi+1] = []
    return item
\end{pythoncode}
\afterverb Al iniciar cada iteración, \texttt{maxi} almacena el índice
del ítem más grande (con la prioridad más alta) que hayamos encontrado
{\em hasta el momento}. En cada iteración, el programa compara
el \texttt{i}ésimo ítem con el que iba ganando. Si el nuevo es mejor,
el valor de \texttt{maxi} se actualiza con el de \texttt{i}.

\index{recorrido}

Cuando el \texttt{for} se completa, \texttt{maxi} es el índice con
el mayor ítem de todos. Éste se saca de la lista y se retorna.

Probemos la implementación:

\beforeverb 
\begin{pyconcode}
>>> q = ColaPrioridad()
>>> q.meter(11)
>>> q.meter(12)
>>> q.meter(14)
>>> q.meter(13)
>>> while not q.estaVacia(): print(q.sacar())
14
13
12
11
\end{pyconcode}
\afterverb Si la Cola contiene números o cadenas, se sacan en orden
alfabético o numérico, del más alto al más bajo. Python puede encontrar
el mayor entero o cadena a través de los operadores de comparación
primitivos.

Si la Cola contiene otro tipo de objeto, creado por el programador,
tiene que proporcionar el método \texttt{\_\_cmp\_\_}. Cuando \texttt{sacar}
use al operador \texttt{>} para comparar items, estaría llamando el
método \texttt{\_\_cmp\_\_} sobre el primero y pasándole al segundo
como parámetro. En tanto que \texttt{\_\_cmp\_\_} funcione correctamente,
la cola de prioridad será correcta.

\section{La clase \texttt{golfista}}

\index{golfista} \index{clase!golfista} \index{prioridad} \index{sobrecarga de operadores}
\index{sobrecarga!operador}

Un ejemplo poco usado de definición de prioridad es la clase \texttt{golfista}
que lleva el registro de los nombres y los puntajes de jugadores de
golf. Primero definimos \texttt{\_\_init\_\_} y \texttt{\_\_str\_\_}:

\beforeverb 
\begin{pythoncode}
class golfista:
  def __init__(self, nombre, puntaje):
    self.nombre = nombre
    self.puntaje= puntaje

  def __str__(self):
    return "%-16s: %d" % (self.nombre, self.puntaje)
\end{pythoncode}
\afterverb \texttt{\_\_str\_\_} utiliza el operador de formato para
poner los nombres y los puntajes en dos columnas.

\index{operador de formato} \index{operador!de formato}

A continuación definimos una versión de \texttt{\_\_cmp\_\_} en la
que el puntaje mas bajo tenga la prioridad mas alta. Recuerde que
para Python \texttt{\_\_cmp\_\_} retorna 1 si \texttt{self} es ``mayor
que'' \texttt{otro}, -1 si \texttt{self} es ``menor'' otro, y 0
si son iguales.

\beforeverb 
\begin{pythoncode}
class golfista:
  ...
  def __cmp__(self, otro):
    # el menor tiene mayor prioridad
    if self.puntaje < otro.puntaje: return  1   
    if self.puntaje > otro.puntaje: return -1
    return 0
\end{pythoncode}
\afterverb Ahora estamos listos para probar la cola de prioridad
almacenando instancias de la clase \texttt{golfista}:

\beforeverb 
\begin{pyconcode}
>>> tiger = golfista("Tiger Woods",    61)
>>> phil  = golfista("Phil Mickelson", 72)
>>> hal   = golfista("Hal Sutton",     69)
>>>
>>> pq = ColaPrioridad()
>>> pq.meter(tiger)
>>> pq.meter(phil)
>>> pq.meter(hal)
>>> while not pq.estaVacia(): print(pq.sacar())
Tiger Woods    : 61
Hal Sutton     : 69
Phil Mickelson : 72
\end{pyconcode}
\afterverb
\begin{quote}
{\em Como ejercicio, escriba una implementación del TAD cola de
prioridad TAD usando una lista enlazada. Ésta debe mantenerse ordenada,
de forma que sacar sea una operación de tiempo constante. Compare
el desempeño de esta implementación con la implementación basada en
listas de Python.} 
\end{quote}

\section{Glosario}

\index{Cola} \index{política de atención} \index{PEPS} \index{cola de prioridad}
\index{barniz} \index{tiempo constante} \index{tiempo lineal} \index{desempeño}
\index{Cola enlazada} \index{buffer circular} \index{clase abstracta}
\index{interfaz}
\begin{description}
\item [{Cola:}] conjunto ordenado de objetos (o personas) esperando a que
se les preste algún servicio
\item [{Cola:}] TAD con las operaciones que se realizan en una Cola.
\item [{Política de atención:}] reglas que determinan cuál es el siguiente
objeto que se saca (atiende) en una Cola.
\item [{PEPS:}] ``Primero que Entra, Primero que Sale'' , política de
atención en la que se saca el primer elemento de la Cola.
\item [{Atención por prioridad:}] una política de atención en la que
se saca el elemento de la Cola que tenga la mayor prioridad.
\item [{Cola de prioridad:}] un TAD que define las operaciones que se
pueden realizar en una cola de prioridad.
\item [{Cola enlazada:}] implementación de una Cola que utiliza una lista
enlazada.
\item [{Desempeño:}] toda función de un TAD realiza un número de operaciones
básicas que dependen del número de elementos que éste contiene en
un momento dado. Por medio de este número de operaciones básicas se
pueden comparar distintas alternativas de implementación de una operación.
\item [{Tiempo constante:}] desempeño de una operación, cuyo tiempo de
ejecución no depende del tamaño de la estructura de datos.
\item [{Tiempo lineal:}] desempeño de una operación, cuyo tiempo de ejecución
es una función lineal del tamaño de la estructura de datos.
\end{description}


\clearemptydoublepage % colas (normales y de prioridad)

\chapter{Árboles}

\index{árbol} \index{nodo} \index{nodo de un árbol} \index{carga}
\index{referencia incrustada} \index{árbol binario}

Como las listas enlazadas, los árboles están compuestos de nodos.
Una clase muy común es el \textbf{árbol binario}, en el que cada nodo
contiene una referencia a otros dos nodos (posiblemente nulas). Estas
referencias se denominan los subárboles izquierdo y derecho. Como
los nodos de las listas, los nodos de los árboles también contienen
una carga. Un diagrama de estados para los árboles luce así:

\label{tree} \beforefig \centerline{\includegraphics{illustrations/tree1}}
\afterfig

Para evitar el caos en las figuras, a menudo omitimos los valores
\texttt{None}.

El inicio del árbol (al nodo al que \texttt{árbol} se refiere) se
denomina \textbf{raíz}. Para conservar la metáfora con los árboles,
los otros nodos se denominan ramas, y los nodos que tienen referencias
nulas se llaman \textbf{hojas}. Parece extraño el dibujo con la raíz
en la parte superior y las hojas en la inferior, pero esto es sólo
el principio.

\index{nodo raíz} \index{nodo hoja} \index{nodo padre} \index{nodo hijo}
\index{nivel}

Los científicos de la computación también usan otra metáfora—el árbol
genealógico. El nodo raíz se denomina \textbf{padre} y los nodos a
los que se refiere \textbf{hijos}, los nodos que tienen el mismo padre
se denominan \textbf{hermanos}.

Finalmente, hay un vocabulario geométrico para referirse a los árboles.
Ya mencionamos la distinción entre izquierda y derecha, también se
acostumbra diferenciar entre ``arriba'' (hacia el padre/raíz) y
``abajo'' (hacia los hijos/hojas). Además, todos los nodos que están
a una misma distancia de la raíz comprenden un \textbf{nivel}.

Probablemente no necesitemos estas metáforas para describir los árboles,
pero se usan extensivamente.

Como las listas enlazadas, los árboles son estructuras de datos recursivas
ya que su definición es recursiva.

\index{estructuras de datos recursivas} \index{estructuras de datos!recursivas}
\begin{quote}
Un árbol es:

\begin{itemize}
\item el árbol vacío, representado por \texttt{None}, o
\item Un nodo que contiene una referencia a un objeto y referencias a otros
aŕboles.
\end{itemize}
\end{quote}
\index{árbol!vacío}

\section{Construyendo árboles}

El proceso de construir un árbol es similar al de una lista enlazada.
La llamada al constructor arma un árbol con un solo nodo.

\beforeverb 
\begin{pythoncode}
class arbol:
  def __init__(self, carga, izquierdo=None, derecho=None):
    self.carga = carga
    self.izquierdo  = izquierdo
    self.derecho = derecho

  def __str__(self):
    return str(self.carga)
\end{pythoncode}
\afterverb La \texttt{carga} puede tener cualquier tipo, pero los
parámetros \texttt{izquierdo} y \texttt{derecho} deben ser nodos.
En \texttt{\_\_init\_\_}, \texttt{izquierdo} y \texttt{derecho} son
opcionales; su valor por defecto es \texttt{None}. Imprimir un nodo
equivale a imprimir su carga.

Una forma de construir un árbol es de abajo hacia arriba. Primero
se construyen los nodos hijos:

\beforeverb 
\begin{pythoncode}
izquierdo = arbol(2)
derecho = arbol(3)
\end{pythoncode}
\afterverb Ahora se crea el padre y se enlazan los hijos:

\beforeverb 
\begin{pythoncode}
arbol = arbol(1, izquierdo, derecho);
\end{pythoncode}
\afterverb Podemos escribir esto de una manera más compacta anidando
los llamados:

\beforeverb 
\begin{pyconcode}
>>> arbol = arbol(1, arbol(2), arbol(3))
\end{pyconcode}
\afterverb Con las dos formas se obtiene como resultado el árbol
que ilustramos al principio del capítulo.

\section{Recorridos sobre árboles}

\index{árbol!recorrido} \index{recorrer} \index{recursión}

Cada vez que se encuentre con una nueva estructura de datos su primera
pregunta debería ser, ``¿Cóomo la recorro?''. La forma más natural
de recorrer un árbol es recursiva. Si el árbol contiene números enteros
en la carga, esta función calcula su suma :

\beforeverb 
\begin{pythoncode}
def total(arbol):
  if arbol == None: 
     return 0
  else:
     return total(arbol.izquierdo) + total(arbol.derecho) 
            + arbol.carga
\end{pythoncode}
\afterverb El caso base se da cuando el argumento es el árbol vacío,
que no tiene carga, así que la suma se define como 0. El paso recursivo
realiza dos llamados recursivos para encontrar la suma de los árboles
hijos, cuando finalizan, se suma a estos valores la carga del padre.

\section{Árboles de expresiones}

\index{árbol!expresión} \index{árbol para una expresión} \index{postfija}
\index{infija} \index{operador binario} \index{operador!binario}

Un árbol representa naturalmente la estructura de una expresión. Además,
lo puede realizar sin ninguna ambigüedad. Por ejemplo, la expresión
infija \texttt{1 + 2 {*} 3} es ambigua a menos que se establezca que
la multiplicación se debe realizar antes que la suma.

Este árbol representa la misma expresión:

\beforefig \centerline{\includegraphics{illustrations/tree2}}
\afterfig

Los nodos de un árbol para una expresión pueden ser operandos como
\texttt{1} y \texttt{2}, también operadores como \texttt{+} y \texttt{{*}}.
Los operandos deben ser nodos hoja; y los nodos que contienen operadores
tienen referencias a sus operandos. Todos estos operadores son \textbf{binarios},
así que solamente tienen dos operandos.

Un árbol como el siguiente representa la figura anterior:

\beforeverb 
\begin{pyconcode}
>>> arbol = arbol('+', arbol(1), 
                       arbol('*', arbol(2), arbol(3)))
\end{pyconcode}
\afterverb Observando la figura no hay ninguna duda sobre el orden
de las operaciones; la multiplicación ocurre primero para que se calcule
el segundo operando de la suma.

Los árboles de expresiones tienen muchos usos. El ejemplo de este
capítulo utiliza árboles para traducir expresiones entre las notaciones
postfija, prefija, e infija. Árboles similares se usan en los compiladores
para analizar sintácticamente, optimizar y traducir programas.

\section{Recorrido en árboles}

\index{árbol!recorrido} \index{recorrido} \index{recursión} \index{Preorden}
\index{Postorden} \index{EnOrden}

Podemos recorrer un árbol de expresiones e imprimir el contenido de
la siguiente forma:

\beforeverb 
\begin{pythoncode}
def imprimirarbol(arbol):
  if arbol == None: 
     return
  print(arbol.carga),
  imprimirarbol(arbol.izquierdo)
  imprimirarbol(arbol.derecho)
\end{pythoncode}
\afterverb \index{Preorden} \index{prefija}

En otras palabras, para imprimir un árbol, primero se imprime el contenido
(carga) de la raíz, luego todo el subárbol izquierdo, y a continuación
todo el subárbol derecho. Este recorrido se denomina \textbf{preorden},
porque el contenido de la raíz se despliega {\em antes} que el
contenido de los hijos. Para el ejemplo anterior, la salida es:

\beforeverb 
\begin{pyconcode}
>>> arbol = arbol('+', arbol(1), 
                       arbol('*', arbol(2), arbol(3)))
>>> imprimirarbol(arbol)
+ 1 * 2 3
\end{pyconcode}
\afterverb Esta notación diferente a la infija y a la postfija, se
denomina \textbf{prefija}, porque los operadores aparecen antes que
sus operandos.

Usted puede sospechar que si se recorre el árbol de una forma distinta
se obtiene otra notación. Por ejemplo si se despliegan los dos subárboles
primero y a continuación el nodo raíz, se obtiene

\beforeverb 
\begin{pythoncode}
def imprimirarbolPostorden(arbol):
  if arbol == None: 
     return
  else 
     imprimirarbolPostorden(arbol.izquierdo)
     imprimirarbolPostorden(arbol.derecho)
     print(arbol.carga),
\end{pythoncode}
\afterverb \index{postorden} \index{en orden} El resultado, \texttt{1
2 3 {*} +}, está en notación postfija!. Por esta razón este recorrido
se denomina \textbf{postorden}.

Finalmente, para recorrer el árbol \textbf{en orden}, se imprime el
árbol izquierdo, luego la raíz y, por último, el árbol derecho:

\beforeverb 
\begin{pythoncode}
def imprimirabolEnOrden(árbol):
  if arbol == None: 
     return
  imprimirabolEnOrden(arbol.izquierdo)
  print(arbol.carga),
  imprimirabolEnOrden(arbol.derecho)
\end{pythoncode}
\afterverb El resultado es \texttt{1 + 2 {*} 3}, la expresión en
notación infija.

Por precisión debemos anotar que hemos omitido una complicación importante.
Algunas veces cuando escribimos una expresión infija, tenemos que
usar paréntesis para preservar el orden de las operaciones. Así que
un recorrido en orden no es suficiente en todos los casos para generar
una expresión infija.

Sin embargo, con unas mejoras adicionales, los árboles de expresiones
y los tres recorridos recursivos proporcionan una forma general de
traducir expresiones de un formato al otro.
\begin{quote}
{\em Como ejercicio, modifique \texttt{imprimirarbolEnOrden} para
que despliegue paréntesis alrededor de cada operador y pareja de operandos.
¿La salida es correcta e inequívoca? ¿Siempre son necesarios los paréntesis?
} 
\end{quote}
Si realizamos un recorrido en orden y llevamos pista del nivel en
el que vamos podemos generar una representación gráfica del árbol:

\beforeverb 
\begin{pythoncode}
def imprimirarbolSangrado(arbol, nivel=0):
  if arbol == None: 
     return
  imprimirarbolSangrado(arbol.derecho, nivel+1)
  print('  '*nivel + str(arbol.carga))
  imprimirarbolSangrado(arbol.izquierdo, nivel+1)
\end{pythoncode}
\afterverb El parámetro \texttt{nivel} lleva el nivel actual. Por
defecto es 0. Cada vez que hacemos un llamado recursivo pasamos \texttt{nivel+1},
porque el nivel de los hijos siempre es uno más del nivel del padre.
Cada objeto se sangra o indenta con dos espacios por nivel. El resultado
para el árbol de ejemplo es:

\beforeverb 
\begin{pyconcode}
>>> imprimirarbolSangrado(arbol)
    3
  *
    2
+
  1
\end{pyconcode}
\afterverb Si rota el libro 90 grados en el sentido de las manecillas
del reloj verá una forma simplificada del dibujo al principio del
capítulo.

\section{Construyendo un árbol para una expresión}

\index{árbol para una expresión} \index{árbol!expresión} \index{análisis sintáctico}
\index{lexema}

En esta sección, analizaremos sintácticamente expresiones infijas
para construir su respectivo árbol de expresión. Por ejemplo, la expresión
\texttt{(3+7){*}9} se representa con el siguiente árbol:

\beforefig \centerline{\includegraphics{illustrations/tree3}}
\afterfig

Note que hemos simplificado el diagrama ocultando los nombres de los
atributos.

El análisis sintáctico se hará sobre expresiones que incluyan números,
paréntesis, y los operadores \texttt{+} y \texttt{{*}}. Asumimos que
la cadena de entrada ha sido separada en una lista de lexemas; por
ejemplo, para \texttt{(3+7){*}9} la lista de lexemas es:

\beforeverb 
\begin{pythoncode}
['(', 3, '+', 7, ')', '*', 9, 'fin']
\end{pythoncode}
\afterverb La cadena \texttt{fin} sirve para prevenir que el analizador
sintáctico siga leyendo más allá del final de la lista.
\begin{quote}
{\em Como ejercicio escriba una función que reciba una cadena de
texto con una expresión y retorne la lista de lexemas (con la cadena
\texttt{fin} al final).} 
\end{quote}
La primera función que escribiremos es \texttt{obtenerLexema}, que
toma una lista de lexemas y un lexema esperado como parámetros. Compara
el lexema esperado con el primero de la lista: si son iguales, elimina
el lexema de la lista y retorna True, si no son iguales, retorna False:

\beforeverb 
\begin{pythoncode}
def obtenerLexema(listaLexemas, esperado):
  if listaLexemas[0] == esperado:
    del listaLexemas[0]
    return 1
  else:
    return 0
\end{pythoncode}
\afterverb Como \texttt{listaLexemas} se refiere a un objeto mutable,
los cambios que hacemos son visibles en cualquier otra parte del programa
que tenga una referencia a la lista.

La siguiente función, \texttt{obtenerNumero}, acepta operandos. Si
el siguiente lexema en \texttt{listaLexemas} es un número, \texttt{obtenerNumero}
lo elimina y retorna un nodo hoja cuya carga será el número; si no
es un número retorna \texttt{None}.

\beforeverb 
\begin{pythoncode}
def obtenerNumero(listaLexemas):
  x = listaLexemas[0]
  if type(x) != type(0): 
     return None
  del listaLexemas[0]
  return arbol (x, None, None)
\end{pythoncode}
\afterverb Probemos a \texttt{obtenerNumero} con una lista pequeña
de números. Después del llamado, imprimimos el árbol resultante y
lo que queda de la lista:

\beforeverb 
\begin{pyconcode}
>>> listaLexemas = [9, 11, 'fin']
>>> x = obtenerNumero(listaLexemas)
>>> imprimirarbolPostorden(x)
9
>>> print(listaLexemas)
[11, 'fin']
\end{pyconcode}
\afterverb El siguiente método que necesitamos es \texttt{obtenerProducto},
que construye un árbol de expresión para productos. Un producto sencillo
tiene dos números como operandos, como en \texttt{3 {*} 7}.

\beforeverb 
\begin{pythoncode}
def obtenerProducto(listaLexemas):
  a = obtenerNumero(listaLexemas)
  if obtenerLexema(listaLexemas, '*'):
    b = obtenerNumero(listaLexemas)
    return árbol ('*', a, b)
  else:
    return a
\end{pythoncode}
\afterverb Asumiendo que \texttt{obtenerNumero} retorna un árbol,
le asignamos el primer operando a \texttt{a}. Si el siguiente carácter
es \texttt{{*}}, obtenemos el segundo número y construimos un árbol
de expresión con \texttt{a}, \texttt{b}, y el operador.

Si el siguiente carácter es cualquier otro, retornamos el nodo hoja
con \texttt{a}. Aquí hay dos ejemplos:

\beforeverb 
\begin{pyconcode}
>>> listaLexemas = [9, '*', 11, 'fin']
>>> arbol = obtenerProducto(listaLexemas)
>>> imprimirarbolPostorden(árbol)
9 11 *
\end{pyconcode}
\afterverb

\beforeverb 
\begin{pyconcode}
>>> listaLexemas = [9, '+', 11, 'fin']
>>> arbol = obtenerProducto(listaLexemas)
>>> imprimirarbolPostorden(arbol)
9
\end{pyconcode}
\afterverb El segundo ejemplo implica que consideramos que un solo
operando sea tratado como una clase de producto. Esta definición de
``producto'' es contraintuitiva, pero resulta ser muy provechosa.

Ahora, tenemos que manejar los productos compuestos, como \texttt{3
{*} 5 {*} 13}. Esta expresión es un producto de productos, vista así:
\texttt{3 {*} (5 {*} 13)}. El árbol resultante es:

\beforefig \centerline{\includegraphics{illustrations/tree4}}
\afterfig

Con un pequeño cambio en \texttt{obtenerProducto}, podemos analizar
un producto arbitrariamente largo:

\beforeverb 
\begin{pythoncode}
def obtenerProducto(listaLexemas):
  a = obtenerNumero(listaLexemas)
  if obtenerLexema(listaLexemas, '*'):
    b = obtenerProducto(listaLexemas)   # esta línea cambió
    return arbol ('*', a, b)
  else:
    return a
\end{pythoncode}
\afterverb En otras palabras, un producto puede ser un árbol singular
o un árbol con \texttt{{*}} en la raíz, un número en el subárbol izquierdo,
y un producto en el subárbol derecho.

Esta clase de definición recursiva debería empezar a ser familiar.

\index{producto} \index{definición!recursiva} \index{definición recursiva}

Probemos la nueva versión con un producto compuesto:

\beforeverb 
\begin{pyconcode}
>>> listaLexemas = [2, '*', 3, '*', 5 , '*', 7, 'fin']
>>> arbol = obtenerProducto(listaLexemas)
>>> imprimirarbolPostorden(arbol)
2 3 5 7 * * *
\end{pyconcode}
\afterverb Ahora, agregaremos la posibilidad de analizar sumas. Otra
vez daremos una definición contraintuitiva a la ``suma.'' Una suma
puede ser un árbol con \texttt{+} en la raíz, un producto en el subárbol
izquierdo y una suma en el subárbol derecho. O, una suma puede ser
sólo un producto.

\index{suma}

Si usted analiza esta definición encontrará que tiene una propiedad
muy bonita: podemos representar cualquier expresión (sin paréntesis)
como una suma de productos. Esta propiedad es el fundamento de nuestro
algoritmo de análisis sintáctico.

\texttt{obtenerSuma} intenta construir un árbol con un producto en
izquierdo y una suma en derecho. Pero si no encuentra un \texttt{+},
solamente construye un producto.

\beforeverb 
\begin{pythoncode}
def obtenerSuma(listaLexemas):
  a = obtenerProducto(listaLexemas)
  if obtenerLexema(listaLexemas, '+'):
    b = obtenerSuma(listaLexemas)
    return arbol ('+', a, b)
  else:
    return a
\end{pythoncode}
\afterverb Probemos con \texttt{9 {*} 11 + 5 {*} 7}:

\beforeverb 
\begin{pyconcode}
>>> listaLexemas = [9, '*', 11, '+', 5, '*', 7, 'fin']
>>> arbol = obtenerSuma(listaLexemas)
>>> imprimirarbolPostorden(árbol)
9 11 * 5 7 * +
\end{pyconcode}
\afterverb Casi terminamos, pero todavía faltan los paréntesis. En
cualquier posición de una expresión donde podamos encontrar un número
puede también haber una suma completa cerrada entre paréntesis. Necesitamos
modificar \texttt{obtenerNumero} para que sea capaz de manejar \textbf{subexpresiones}:

\index{subexpresión}

\beforeverb 
\begin{pythoncode}
def obtenerNumero(listaLexemas):
  if obtenerLexema(listaLexemas, '('):
    # obtiene la subexpresión
    x = obtenerSuma(listaLexemas)  
    # elimina los paréntesis
    obtenerLexema(listaLexemas, ')') 
    return x
  else:
    x = listaLexemas[0]
    if type(x) != type(0): 
       return None
    listaLexemas[0:1] = []
    return árbol (x, None, None)    
\end{pythoncode}
\afterverb Probemos esto con \texttt{9 {*} (11 + 5) {*} 7}:

\beforeverb 
\begin{pyconcode}
>>> listaLexemas = [9, '*', '(', 11, '+', 5, ')','*', 7, 
'fin']
>>> arbol = obtenerSuma(listaLexemas)
>>> imprimirarbolPostorden(arbol)
9 11 5 + 7 * *
\end{pyconcode}
\afterverb %\adjustpage{-2}%\pagebreak

El analizador manejó los paréntesis correctamente, la suma se hace
antes que la multiplicación.

En la versión final del programa, sería bueno nombrar a \texttt{obtenerNumero}
con un rol más descriptivo.

\section{Manejo de errores}

\index{manejo de errores} \index{errores!manejo de}

En todo momento hemos asumido que las expresiones están bien formadas.
Por ejemplo, cuando llegamos al final de una subexpresión, asumimos
que el siguiente carácter es un paréntesis derecho. Si hay un error
y el siguiente carácter es algo distinto debemos manejar esta situación.

\beforeverb 
\begin{pythoncode}
def obtenerNumero(listaLexemas):
  if obtenerLexema(listaLexemas, '('):
    x = obtenerSuma(listaLexemas)       
    if not obtenerLexema(listaLexemas, ')'):
      raise 'ErrorExpresionMalFormada', 'falta paréntesis'
    return x
  else:
    # el resto del código se omite
\end{pythoncode}
\afterverb La sentencia \texttt{raise} crea una excepción. En este
caso creamos una nueva clase de excepción llamada \texttt{ErrorExpresionMalFormada}.
Si la función que llamó a \texttt{obtenerNumero}, o una de las funciones
en la traza que causante de su llamado maneja la excepción, el programa
puede continuar. De otra forma, Python imprimirá un mensaje de error
y abortará la ejecución.
\begin{quote}
{\em Como ejercicio, encuentre otros lugares donde pueden ocurrir
errores de este tipo y agregue sentencias \texttt{raise} apropiadas.
Pruebe su código con expresiones mal formadas.} 
\end{quote}

\section{El árbol de animales}

\index{juego de animales} \index{juego!de animales} \index{base de conocimiento}

En esta sección desarrollaremos un pequeño programa que usa un árbol
para representar una base de conocimiento.

El programa interactúa con el usuario para crear un árbol de preguntas
y nombres de animales. Aquí hay una ejecución de ejemplo:

\adjustpage{-3} %\pagebreak

\beforeverb 
\begin{pythoncode}
¿Esta pensando en un animal? s
¿Es un pájaro? n
¿Cual es el nombre del animal? perro
¿Que pregunta permite distinguir entre un perro 
y un pájaro? Puede volar
¿Si el animal fuera perro la respuesta sería? n

¿Esta pensando en un animal? s
¿Puede volar? n
¿Es un perro? n
¿Cual es el nombre del animal? gato
¿Que pregunta permite distinguir un gato 
de un perro? Ladra
¿Si el animal fuera un gato 
la respuesta sería? n

¿Esta pensando en un animal? y
¿Puede volar? n
¿Ladra? s
¿Es un perro? s
¡Soy el mejor!

\end{pythoncode}
\afterverb Este es el árbol que el diálogo genera:

\beforefig \centerline{\includegraphics{illustrations/tree5}}
\afterfig

Al principio de cada ronda, el programa empieza en la raíz del árbol
y hace la primera pregunta. Dependiendo de la respuesta se mueve al
subárbol izquierdo o derecho y continúa hasta que llega a un nodo
hoja. En ese momento conjetura. Si falla, le pregunta al usuario el
nombre del animal y una pregunta que le permitiría distinguir el animal
conjeturado del real. Con esta información agrega un nodo al árbol
con la nueva pregunta y el nuevo animal.

Aquí está el código fuente:

\adjustpage{-2} %\pagebreak

\beforeverb 
\begin{pythoncode}
def animal():
  # Un solo nodo
  raiz = arbol("pajaro")

  # Hasta que el usuario salga
  while True:
    print
    if not si("Esta pensando en un animal? "): 
       break

    # Recorrer el arbol
    arbol = raiz
    while arbol.obtenerizquierdo() != None:
      pregunta = arbol.obtenercarga() + "? "
      if si(pregunta):
        arbol = arbol.obtenerderecho()
      else:
        arbol = arbol.obtenerizquierdo()

    # conjetura
    conjetura = arbol.obtenercarga()
    pregunta = "¿Es un" + conjetura + "? "
    if si(pregunta):
      print("¡Soy el mejor!")
      continue

    # obtener mas informacion
    pregunta  = "¿Cual es el nombre el animal? "
    animal  = input(pregunta)
    pregunta  = "¿Que pregunta permitiria 
                  distinguir un %s de un %s? "
    q = input(pregunta % (animal,conjetura))

    # agrega un nuevo nodo arbol
    arbol.asignarcarga(q)
    pregunta = "¿Si el animal fuera %s 
                la respuesta sería? "
    if si(pregunta % animal):
      arbol.asignarizquierdo(arbol(conjetura))
      árbol.asignarderecho(arbol(animal))
    else:
      arbol.asignarizquierdo(arbol(animal))
      arbol.asignarderecho(arbol(conjetura))
\end{pythoncode}
\afterverb La función \texttt{si} es auxiliar, imprime una pregunta
y recibe la respuesta del usuario. Si la respuesta empieza con {\em
s} o {\em S}, retorna cierto:

\beforeverb 
\begin{pythoncode}
def si(preg):
  from string import lower
  r = lower(input(preg))
  return (r[0] == 's')
\end{pythoncode}
\afterverb La condición del ciclo es \texttt{True}, lo que implica
que el ciclo iterará hasta que la sentencia \texttt{break} se ejecute
cuando el usuario deje de pensar en animales.

El ciclo \texttt{while} interno recorre el árbol desde la raíz hasta
el fondo, guiándose por las respuestas del usuario.

Cuando se agrega un nuevo nodo al árbol, la nueva pregunta reemplaza
la carga, y los dos hijos son el animal nuevo y la carga original.

Una limitación seria de este programa es que cuando finaliza, ¡olvida
todo lo que se le ha enseñado!
\begin{quote}
{\em Como ejercicio, piense en diferentes maneras de guardar este
árbol de conocimiento en un archivo. Implemente la que parezca más
sencilla.} 
\end{quote}

\section{Glosario}

\index{árbol binario} \index{nodo} \index{nodo raíz} \index{nodo hoja}
\index{nodo padre} \index{nodo hijo} \index{nodo hermano} \index{nivel}
\index{prefija} \index{preorden} \index{postorden} \index{en orden}
\index{operador binario} \index{operador!binario}
\begin{description}
\item [{Arbol binario:}] un árbol en el que cada nodo se refiere a cero,
uno o dos nodos, llamados hijos.
\item [{Raíz:}] nodo inicial en un árbol, es el único que no tiene padre.
\item [{Hoja:}] nodo que no tiene hijos y se encuentra lo mas abajo posible.
\item [{Padre:}] nodo que tiene la referencia hacia otro nodo.
\item [{Hijo:}] uno de los nodos referenciados por un nodo.
\item [{Hermanos:}] nodos que comparten un padre común.
\item [{Nivel:}] conjunto de nodos equidistantes a la raíz.
\item [{Operador binario:}] operador que acepta dos operandos.
\item [{Subexpresión:}] expresión en paréntesis que se puede ver como un
solo operando en una expresión más grande.
\item [{Preorden:}] es el recorrido sobre un árbol en el que se visita
cada nodo antes que a sus hijos.
\item [{Notación prefija:}] notación para escribir una expresión matemática
en la que cada operador aparece antes de sus operandos.
\item [{Postorden:}] forma de recorrer un árbol, visitando los hijos de
un nodo antes que a este.
\item [{En orden:}] forma de recorrer un árbol, visitando primero el subárbol
izquierdo, luego la raíz y por último, el subárbol derecho.
\end{description}


\clearemptydoublepage % arboles


\appendix
%dummy comment inserted by tex2lyx to ensure that this paragraph is not empty

\chapter{Depuración }

\index{depuración}

Hay diferentes tipos de error que pueden suceder en un programa y
es muy útil distinguirlos a fin de rastrearlos más rápidamente:
\begin{itemize}
\item Los errores sintácticos se producen cuando Python traduce el código
fuente en código objeto. Usualmente indican que hay algún problema
en la sintaxis del programa. Por ejemplo, omitir los puntos seguidos
al final de una sentencia \texttt{def} produce un mensaje de error
un poco redundante \texttt{SyntaxError: invalid syntax}.
\item Los errores en tiempo de ejecución se producen por el sistema de ejecución,
si algo va mal mientras el programa corre o se ejecuta. La mayoría
de errores en tiempo de ejecución incluyen información sobre la localización
del error y las funciones que se estaban ejecutando. Ejemplo: una
recursión infinita eventualmente causa un error en tiempo de ejecución
de ``maximum recursion depth exceeded.''
\item Los errores semánticos se dan en programas que compilan y se ejecutan
normalmente, pero no hacen lo que se pretendía. Ejemplo: una expresión
podría evaluarse en un orden inesperado, produciendo un resultado
incorrecto.
\end{itemize}
\index{error en tiempo de compilación} \index{error sintáctico}
\index{error en tiempo de ejecución} \index{error semántico} \index{error!en tiempo de compilación}
\index{error!sintaxis} \index{error!tiempo de ejecución} \index{error!semántica}
\index{excepción}

El primer paso en la depuración consiste en determinar la clase de
error con la que se está tratando. Aunque las siguientes secciones
se organizan por tipo de error, algunas técnicas se aplican en más
de una situación.

\section{Errores sintácticos}

\index{mensajes de error} \index{compilador}

Los errores sintácticos se corrigen fácilmente una vez que usted ha
determinado a qué apuntan. Desafortunadamente, en algunas ocasiones
los mensajes de error no son de mucha ayuda. Los mensajes de error
más comunes son \texttt{SyntaxError: invalid syntax} y \texttt{SyntaxError: invalid
token}, que no son muy informativos.

Por otro lado, el mensaje sí dice dónde ocurre el problema en el programa.
Más precisamente, dice dónde fue que Python encontró un problema,
que no necesariamente es el lugar dónde está el error. Algunas veces
el error está antes de la localización que da el mensaje de error,
a menudo en la línea anterior.

\index{desarrollo incremental de programas}

Si usted está construyendo los programas incrementalmente, debería
tener una buena idea de dónde se localiza el error. Estará en la última
línea que se agregó.

Si usted está copiando código desde un libro, comience por comparar
su código y el del libro muy cuidadosamente. Chequee cada carácter.
Al mismo tiempo, recuerde que el libro puede tener errores, así que
si encuentra algo que parece un error sintáctico, entonces debe serlo.

Aquí hay algunas formas de evitar los errores sintácticos más comunes:

\index{syntax}
\begin{enumerate}
\item Asegúrese de no usar una palabra reservada de Python como nombre de
variable
\item Chequee que haya colocado dos puntos seguidos al final de la cabecera
de cada sentencia compuesta, incluyendo los ciclos \texttt{for}, \texttt{while},
los condicionales \texttt{if}, las definiciones de función \texttt{def}
y las \texttt{clases}.
\item Chequee que la indentación o sangrado sea consistente. Se puede indentar
con espacios o tabuladores, pero es mejor no mezclarlos. Cada nivel
debe sangrarse la misma cantidad de espacios o tabuladores.
\item Asegúrese de que las cadenas en los programas estén encerradas entre
comillas.
\item Si usted tiene cadenas multilínea creadas con tres comillas consecutivas,
verifique su terminación. Una cadena no terminada puede causar un
error denominado \texttt{invalid token} al final de su programa, o
puede tratar la siguiente parte del programa como si fuera una cadena,
hasta toparse con la siguiente cadena. En el segundo caso, ¡puede
que Python no produzca ningún mensaje de error!
\item Un paréntesis sin cerrar—\verb+(+, \verb+{+, o \verb+[+—hace que
Python continúe con la siguiente línea como si fuera parte de la sentencia
actual. Generalmente, esto causa un error inmediato en la siguiente
línea.
\item Busque por la confusión clásica entre \texttt{=} y \texttt{==}, adentro
y afuera de los condicionales.
\end{enumerate}
Si nada de esto funciona, avance a la siguiente sección...

\subsection{No puedo ejecutar mi programa sin importar lo que haga}

Si el compilador dice que hay un error y usted no lo ha visto, eso
puede darse porque usted y el compilador no están observando el mismo
código. Chequee su ambiente de programación para asegurarse de que
el programa que está editando es el que Python está tratando de ejecutar.
Si no está seguro, intente introducir deliberadamente un error sintáctico
obvio al principio del programa. Ahora ejecútelo o impórtelo de nuevo.
Si el compilador no encuentra el nuevo error, probablemente hay algún
problema de configuración de su ambiente de programación.

Si esto pasa, una posible salida es empezar de nuevo con un programa
como ``Hola todo el mundo!,'' y asegurarse de que pueda ejecutarlo
correctamente. Después, añadir gradualmente a éste los trozos de su
programa.

\section{Errores en tiempo de ejecución}

Cuando su programa está bien sintácticamente Python puede importarlo
y empezar a ejecutarlo. ¿Qué podría ir mal ahora?

\subsection{Mi programa no hace absolutamente nada}

Este problema es el más común cuando su archivo comprende funciones
y clases pero no hace ningún llamado para empezar la ejecución. Esto
puede ser intencional si usted sólo planea importar este módulo para
proporcionar funciones y clases.

Si no es intencional, asegúrese de que está llamando a una función
para empezar la ejecución, o ejecute alguna desde el indicador de
entrada (prompt). Revise la sección posterior sobre el ``Flujo de
Ejecución''.

\subsection{Mi programa se detiene}

\index{ciclo infinito} \index{recursión infinita} \index{detención}

Si un programa se detiene y parece que no está haciendo nada, decimos
que está ``detenido.'' Esto a veces sucede porque está atrapado
en un ciclo infinito o en una recursión infinita.
\begin{itemize}
\item Si hay un ciclo sospechoso de ser la causa del problema, añada un
\texttt{print} inmediatamente antes del ciclo que diga ``entrando
al ciclo'' y otra inmediatamente después, que diga ``saliendo del
ciclo.''

Ejecute el programa. Si obtiene el primer mensaje y no obtiene el
segundo, ha encontrado su ciclo infinito. revise la sección posterior
``Ciclo Infinito''.
\item La mayoría de las veces, una recursión infinita causará que el programa
se ejecute por un momento y luego produzca un error ``RuntimeError:
Maximum recursion depth exceeded''. Si esto ocurre revise la sección
posterior ``Recursión Infinita''.

Si no está obteniendo este error, pero sospecha que hay un problema
con una función recursiva o método, también puede usar las técnicas
de la sección ``Recursión Infinita''.
\item Si ninguno de estos pasos funciona, revise otros ciclos y otras funciones
recursivas, o métodos.
\item Si eso no funciona entonces es posible que usted no comprenda el flujo
de ejecución que hay en su programa. Vaya a la sección posterior ``Flujo
de ejecución''.
\end{itemize}

\subsubsection{Ciclo infinito}

\index{ciclo infinito} \index{ciclo!infinito} \index{condición}
\index{ciclo!condición}

Si usted cree que tiene un ciclo infinito añada un \texttt{print}
al final de éste que imprima los valores de las variables de ciclo
(las que aparecen en la condición) y el valor de la condición.

Por ejemplo:\inputencoding{latin9}
\begin{lstlisting}
while x > 0 and y < 0 :
  # hace algo con x
  # hace algo con y

  print( "x: ", x )
  print( "y: ", y )
  print( "condicion: ", (x > 0 and y < 0))
\end{lstlisting}
\inputencoding{utf8} Ahora, cuando ejecute el programa, usted verá tres líneas de salida
para cada iteración del ciclo. En la última iteración la condición
debe ser \texttt{falsa}. Si el ciclo sigue, usted podrá ver los valores
de \texttt{x} y \texttt{y}, y puede deducir por qué no se están actualizando
correctamente.

\subsubsection{Recursión infinita}

\index{recursión Infinita} \index{recursión Infinita}

La mayoría de las veces una recursión infinita causará que el programa
se ejecute durante un momento y luego producirá un error: \texttt{Maximum
recursion depth exceeded}.

Si sospecha que una función o método está causando una recursión infinita,
empiece por chequear la existencia de un caso base. En otras palabras,
debe haber una condición que haga que el programa o método retorne
sin hacer un llamado recursivo. Si no lo hay, es necesario reconsiderar
el algoritmo e identificar un caso base.

Si hay un caso base, pero el programa no parece alcanzarlo, añada
un \texttt{print} al principio de la función o método que imprima
los parámetros. Ahora, cuando ejecute el programa usted verá unas
pocas líneas de salida cada vez que la función o método es llamada,
y podrá ver los parámetros. Si no están cambiando de valor acercándose
al caso base, usted podrá deducir por qué ocurre esto.

\subsubsection{Flujo de ejecución}

\index{Flujo de Ejecución} \index{ejecución!flujo de}

Si no está seguro de cómo se mueve el flujo de ejecución a través
de su programa, añada \texttt{print} al comienzo de cada función con
un mensaje como ``entrando a la función \texttt{foo},'' donde \texttt{foo}
es el nombre de la función.

Ahora, cuando ejecute el programa, se imprimirá una traza de cada
función a medida que van siendo llamadas.

\subsection{Cuando ejecuto el programa obtengo una excepción}

\index{excepción} \index{error en tiempo de ejecución}

Si algo va mal durante la ejecución, Python imprime un mensaje que
incluye el nombre de la excepción, la línea del programa donde ocurrió,
y un trazado inverso.

\index{trazado inverso}

El trazado inverso identifica la función que estaba ejecutándose,
la función que la llamó, la función que llamó a {\em esta} última,
y así sucesivamente. En otras palabras, traza el camino de llamados
que lo llevaron al punto actual de ejecución. También incluye el número
de línea en su archivo, donde cada uno de estos llamados tuvo lugar.

El primer paso consiste en examinar en el programa el lugar donde
ocurrió el error y ver si se puede deducir qué pasó. Aquí están algunos
de los errores en tiempo de ejecución más comunes:
\begin{description}
\item [{NameError:}] usted está tratando de usar una variable que no existe
en el ambiente actual. Recuerde que las variables locales son locales.
No es posible referirse a ellas afuera de la función donde se definieron.

\index{NameError} \index{TypeError}
\item [{TypeError:}] hay varias causas:

\begin{itemize}
\item Usted está tratando de usar un valor impropiamente. Por ejemplo: indexar
una cadena, lista o tupla con un valor que no es entero.

\index{índice}
\item No hay correspondencia entre los elementos en una cadena de formato
y los elementos pasados para hacer la conversión. Esto puede pasar
porque el número de elementos no coincide o porque se está pidiendo
una conversión invalida.

\index{operador de formato} \index{operador!formato}
\item Usted está pasando el número incorrecto de argumentos a una función
o método. Para los métodos, mire la definición de métodos y chequee
que el primer parámetro sea \texttt{self}. Luego mire el llamado,
asegúrese de que se hace el llamado sobre un objeto del tipo correcto
y de pasar los otros parámetros correctamente.
\end{itemize}
\item [{KeyError:}] usted está tratando de acceder a un elemento de un
diccionario usando una llave que éste no contiene.

\index{KeyError} \index{diccionario}
\item [{AttributeError:}] está tratando de acceder a un atributo o método
que no existe.

\index{AttributeError}
\item [{IndexError:}] el índice que está usando para acceder a una lista,
cadena o tupla es más grande que su longitud menos uno. Inmediatamente,
antes de la línea del error, agregue un \texttt{print} para desplegar
el valor del índice y la longitud de la secuencia. ¿Tiene ésta el
tamaño correcto? ¿Tiene el índice el valor correcto?

\index{IndexError}
\end{description}

\subsection{Agregué tantos \texttt{prints} que estoy inundado de texto de salida}

\index{función print} \index{función!print}

Uno de los problemas de \texttt{print} para depurar es que uno puede
terminar inundado de salida. Hay dos formas de proceder: simplificar
la salida o simplificar el programa.

Para simplificar la salida se pueden eliminar o comentar los \texttt{print}
que no son de ayuda, o se pueden combinar, o se puede dar formato
a la salida, de forma que quede más fácil de entender.

Para simplificar el programa hay varias cosas que se pueden hacer.
Primero, disminuya la escala del problema que el programa intenta
resolver. Por ejemplo, si usted está ordenando un arreglo, utilice
uno {\em pequeño} como entrada. Si el programa toma entrada del
usuario, pásele la entrada más simple que causa el problema.

Segundo, limpie el programa. Borre el código muerto y reorganízelo
para hacerlo lo más legible que sea posible. Por ejemplo, si usted
sospecha que el problema está en una sección de código profundamente
anidada, intente reescribir esa parte con una estructura más sencilla.
Si sospecha de una función grande, trate de partirla en funciones
mas pequeñas y pruébelas separadamente.

Este proceso de encontrar el caso mínimo de prueba que activa el problema
a menudo permite encontrar el error. Si usted encuentra que el programa
funciona en una situación, pero no en otras, esto le da una pista
de lo que está sucediendo.

Similarmente, reescribir un trozo de código puede ayudar a encontrar
errores muy sutiles. Si usted hace un cambio que no debería alterar
el comportamiento del programa, y sí lo hace, esto es una señal de
alerta.

\section{Errores semánticos}

\index{error semántico} \index{error!semántico}

Estos son los mas difíciles de depurar porque ni el compilador ni
el sistema de ejecución dan información sobre lo que está fallando.
Sólo usted sabe lo que el programa debe hacer y sólo usted sabe por
qué no lo está haciendo bien.

El primer paso consiste en hacer una conexión entre el código fuente
del programa y el comportamiento que está dándose. Ahora, usted necesita
una hipótesis sobre lo que el programa está haciendo realmente. Una
de las cosas que complica el asunto es que la ejecución de programas
en un computador moderno es muy rápida.

A veces deseará desacelerar el programa hasta una velocidad humana,
y con algunos programas depuradores esto es posible. Pero el tiempo
que toma insertar unos \texttt{print} bien situados a menudo es mucho
más corto comparado con la configuración del depurador, la inserción
y eliminación de puntos de quiebre (breakpoints en inglés) y ``saltar''
por el programa al punto donde el error se da.

\subsection{Mi programa no funciona}

Usted debe hacerse estas preguntas:
\begin{itemize}
\item ¿Hay algo que el programa debería hacer, pero no hace? Encuentre la
sección de código que tiene dicha funcionalidad y asegúrese de que
se ejecuta en los momentos adecuados.
\item ¿Está pasando algo que no debería? Encuentre código en su programa
que tenga una funcionalidad y vea si ésta se ejecuta cuando no debería.
\item ¿Hay una sección de código que produce un efecto que no esperaba usted?
Asegúrese de que entiende dicha sección de código, especialmente si
tiene llamados a funciones o métodos en otros módulos. Lea la documentación
para las funciones que usted llama. Intente escribir casos de prueba
más sencillos y chequee los resultados.
\end{itemize}
Para programar, usted necesita un modelo mental de cómo trabajan los
programas. Si usted escribe un programa que no hace lo que se espera,
muy frecuentemente el problema no está en el programa, sino en su
modelo mental.

\index{modelo!mental} \index{modelo mental}

La mejor forma de corregir su modelo mental es descomponer el programa
en sus componentes (usualmente funciones y métodos) para luego probarlos
independientemente. Una vez encuentre la discrepancia entre su modelo
y la realidad, el problema puede resolverse.

Por supuesto, usted debería construir y probar componentes a medida
que desarrolla el programa. Si encuentra un problema, debería haber
una pequeña cantidad de código nuevo que puede estar incorrecto.

\subsection{He obtenido una expresión grande y peluda que no hace lo que espero}

\index{expresión!grande y peluda}

Escribir expresiones complejas está bien en tanto queden legibles,
sin embargo, puede ser difícil depurarlas. Es una buena idea separar
una expresión compleja en una serie de asignaciones a variables temporales.

Por ejemplo:\inputencoding{latin9}
\begin{lstlisting}
self.manos[i].agregarCarta( \
  self.manos[self.encontrarVecino(i)].eliminarCarta())
 Esto puede reescribirse como:
 
vecino = self.encontrarVecino (i)
cartaEscogida = self.manos[vecino].eliminarCarta()
self.manos[i].agregarCarta(cartaEscogida)
\end{lstlisting}
\inputencoding{utf8} La versión explícita es más fácil de leer porque los nombres de variables
proporcionan una documentación adicional y porque se pueden chequear
los tipos de los valores intermedios desplegándolos.

\index{variable temporal} \index{variable!temporal} \index{orden de evaluación}
\index{precedencia}

Otro problema que ocurre con las expresiones grandes es que el orden
de evaluación puede no ser el que usted espera. Por ejemplo, si usted
está traduciendo la expresión $\frac{x}{2\pi}$ a Python, podría escribir:

\inputencoding{latin9}\begin{lstlisting}
y = x / 2 * math.pi;
\end{lstlisting}
\inputencoding{utf8} Esto es incorrecto, porque la multiplicación y la división tienen
la misma precedencia y se evalúan de izquierda a derecha. Así que
ese código calcula $x\pi/2$.

Una buena forma de depurar expresiones es agregar paréntesis para
hacer explícito el orden de evaluación:\inputencoding{latin9}
\begin{lstlisting}
 y = x / (2 * math.pi);
\end{lstlisting}
\inputencoding{utf8} Cuando no esté seguro del orden de evaluación, use paréntesis. No
sólo corregirá el programa si había un error, sino que también lo
hará mas legible para otras personas que no se sepan las reglas de
precedencia.

\subsection{Tengo una función o método que no retorna lo que debería}

\index{sentencia return} \index{sentencia!return}

Si usted tiene una sentencia \texttt{return} con una expresión compleja
no hay posibilidad de imprimir el valor del \texttt{return} antes
de retornar. Aquí también se puede usar una variable temporal. Por
ejemplo, en vez de:

\inputencoding{latin9}\begin{lstlisting}
return self.manos[i].eliminarParejas()
\end{lstlisting}
\inputencoding{utf8} se podría escribir:

\inputencoding{latin9}\begin{lstlisting}
cont = self.manos[i].eliminarParejas()
return cont
\end{lstlisting}
\inputencoding{utf8} Ahora usted tiene la oportunidad de desplegar el valor de \texttt{count}
antes de retornar.

\subsection{Estoy REALMENTE atascado y necesito ayuda}

Primero, intente alejarse del computador por unos minutos. Los computadores
emiten ondas que afectan al cerebro causando estos efectos:
\begin{itemize}
\item Frustración e ira.
\item Creencias supersticiosas (``el computador me odia'') y pensamiento
mágico (``el programa sólo funciona cuando me pongo la gorra al revés'').
\item Programación aleatoria (el intento de programar escribiendo cualquier
programa posible y escogiendo posteriormente el que funcione correctamente).
\end{itemize}
Si usted está sufriendo de alguno de estos síntomas, tómese un paseo.
Cuando ya esté calmado, piense en el programa. ¿Qué está haciendo?
¿Cuáles son las causas posibles de éste comportamiento? ¿Cuándo fue
la última vez que funcionaba bien, y qué hizo usted después de eso?

Algunas veces, sólo toma un poco de tiempo encontrar un error. A menudo
encontramos errores cuando estamos lejos del computador y dejamos
que la mente divague. Algunos de los mejores lugares para encontrar
errores son los trenes, duchas y la cama, justo antes de dormir.

\subsection{No, realmente necesito ayuda}

Esto sucede. Hasta los mejores programadores se atascan alguna vez.
A veces se ha trabajado tanto tiempo en un programa que ya no se puede
ver el error. Un par de ojos frescos es lo que se necesita.

Antes de acudir a alguien más, asegúrese de agotar todas las técnicas
descritas aquí. Su programa debe ser tan sencillo como sea posible
y usted debería encontrar la entrada más pequeña que causa el error.
Debería tener varios \texttt{print} en lugares apropiados (y la salida
que despliegan debe ser comprensible). Usted debe entender el problema
lo suficientemente bien como para describirlo concisamente.

Cuando acuda a alguien, asegúrese de darle la información necesaria:
\begin{itemize}
\item Si hay un mensaje de error, ¿cuál es, y a qué parte del programa se
refiere?
\item ¿Cuál fue el último cambio antes de que se presentara el error? ¿Cuáles
fueron las últimas líneas de código que escribió usted o cuál es el
nuevo caso de prueba que falla?
\item ¿Qué ha intentado hasta ahora, y qué ha aprendido sobre el programa?
\end{itemize}
Cuando encuentre el error, tómese un segundo para pensar sobre lo
que podría haber realizado para encontrarlo más rápido. La próxima
vez que le ocurra algo similar, será capaz de encontrar el error rápidamente.

Recuerde, el objetivo no sólo es hacer que el programa funcione. El
objetivo es aprender cómo hacer que los programas funcionen.

\clearemptydoublepage % depuracion

\chapter{Interludio 1: Triqui }

\label{cap:inter1:triqui} \index{Triqui}

\section{Motivación}

Con el fin de poner en práctica los conceptos de los capítulos anteriores
vamos a desarrollar un sencillo juego de triqui para dos jugadores.
La idea es seguir un desarrollo iterativo que toma un pequeño programa
y lo convierte, poco a poco, en un juego de triqui con toda la funcionalidad
esperada.

El código fuente tiene diez versiones, comenzando desde \texttt{triqui0.py}
hasta \texttt{triqui9.py}. Los diez programas puede descargarse de:

\url{}

Para comprender el capítulo hay que ejecutar cada versión del programa
a medida que se avanza en la lectura. En las secciones a continuación
se discuten los fragmentos del programa a medida que se van agregando,
cada fragmento tiene el nombre del archivo en que se introdujo como
comentario inicial.

\section{Estructuras de datos}

El código fuente de un programa extenso puede ser difícil de comprender.
Esta es la razón por la que lo primero que se debe explicar de un
software son sus estructuras de datos: las variables, listas, matrices
que usa para representar la información.

En nuestro triqui, el tablero de juego es una matriz de 3 filas y
3 columnas que se almacena en una lista de listas usando la representación
usual del capítulo \ref{cap:listas} en la que cada fila de la matriz
es una sublista.

Todos los elementos de la matriz serán caracteres de texto con la
convención que sigue. Si el elemento de la matriz es:
\begin{description}
\item [{' ' (esto es el caracter espacio):}] nadie ha jugado en esa casilla. 
\item [{'O': }] el primer jugador jugó en esa casilla. 
\item [{'X':}] el segundo jugador jugó en esa casilla. 
\end{description}

\section{Inicio}

Con la convención anterior, nuestro triqui empieza humildemente:

\inputencoding{latin9}\begin{lstlisting}
# triqui0.py
def crear():
    m =  [ [' ',' ',' '],
           [' ',' ',' '],
           [' ',' ',' '] ]
    return m


def imprimir(tablero):
    for i in range(3):
        print("|"),
        for j in range(3):
            print(tablero[i][j]),
        print("|")
        
triqui = crear()
imprimir(triqui)
\end{lstlisting}
\inputencoding{utf8}
Ahora podemos agregar un ciclo para jugar, sencillo, con un solo jugador:\pagebreak{}

\inputencoding{latin9}\begin{lstlisting}
# triqui1.py
while True:
    print("Juegue jugador O")
    f = input("fila? ")
    c = input("columna? ")
    triqui[f][c] = "O"
    imprimir(triqui)
\end{lstlisting}
\inputencoding{utf8}
Agregar al segundo jugador es casi idéntico al fragmento anterior
y está en el archivo triqui2.py.

\section{¿Quien gana?}

Para verificar si un jugador gana la partida, vamos a empezar por
las diagonales, implementando un ciclo para la verificar si alguien
gana en la diagonal principal:

\inputencoding{latin9}\begin{lstlisting}
# triqui3.py
def ganaDiagonal1(jugador,tablero):
    for i in range(3):
        if tablero[i][i]!=jugador:
            return False
    return True
\end{lstlisting}
\inputencoding{utf8}
La idea es que si encuentra algo diferente del símbolo del jugador
('X' ó 'O'), retorna False. Sino, retorna True. La otra diagonal requiere
mas trabajo, usamos el hecho de que \texttt{tablero{[}i{]}{[}2-i{]}}
va dando los elementos de la segunda diagonal para i de 0 a 2. ¡Verifiquelo!

\inputencoding{latin9}\begin{lstlisting}
# triqui3.py
def ganaDiagonal2(jugador,tablero):
    for i in range(3):
        if tablero[i][2-i]!=jugador:
            return False
    return True
\end{lstlisting}
\inputencoding{utf8}
Falta llamar las funciones en el ciclo del juego, y si alguien gana,
terminamos el juego con la sentencia \texttt{break}. Por ejemplo,
para el primer jugador:

\inputencoding{latin9}\begin{lstlisting}
# triqui3.py
print("Juegue jugador O")
    f = input("fila? ")
    c = input("columna? ")
    triqui[f][c] = "O"
    imprimir(triqui)
    if ganaDiagonal1("O",triqui) or ganaDiagonal2("O",triqui):
        print("Gana el jugador O!!!!")
        break
\end{lstlisting}
\inputencoding{utf8}
Agregar las funciones para verificar si alguien gana por alguna fila
es sencillo. Seguimos la misma estructura de las diagonales, creando
una función \texttt{ganaFila}, que verifica si el jugador gana en
una de las filas del tablero.

\inputencoding{latin9}\begin{lstlisting}
# triqui4.py
def ganaFila(fila,jugador,tablero):
    """Chequea si el jugador gana en la fila dada"""
    for i in range(3):
        if tablero[fila][i]!=jugador:
            return False
    return True
\end{lstlisting}
\inputencoding{utf8}
La función anterior debe ser llamada para todas las filas:

\inputencoding{latin9}\begin{lstlisting}
# triqui4.py

def ganaHorizontal(jugador,tablero):
    for i in range(3):
        if ganaFila(i,jugador,tablero):
            return True
    return False
\end{lstlisting}
\inputencoding{utf8}
Las funciones para chequear las columnas son muy parecidas. Para llamarlas
modificamos el ciclo del juego. Por ejemplo, para el jugador 'X':

\inputencoding{latin9}\begin{lstlisting}
# triqui4.py
while True:
    print("Juegue jugador X")
    f = input("fila? ")
    c = input("columna? ")
    triqui[f][c] = "X"
    imprimir(triqui)
    if ganaDiagonal1("X",triqui) or ganaDiagonal2("X",triqui) or \
       ganaHorizontal("X",triqui) or ganaVertical("X",triqui):
        print("Gana el jugador X!!!!")
        break
\end{lstlisting}
\inputencoding{utf8}
\section{Reestructurando el código}

\index{reestructuración}

Casi siempre que se está desarrollando un programa y uno encuentra
que está copiando y pegando código para hacer pequeños cambios vale
la pena analizar si se pueden crear funciones para evitar problemas
futuros. Una función correcta, que se llama desde varias partes del
programa es más fácil de mantener que una serie de porciones de código
parecidas, pero con cambios, que se han copiado, pegado y modificado.

En el triqui podemos observar que el código dentro del ciclo para
el jugador 'O' y el 'X' es muy parecido. Así que lo podemos poner
en una función que tenga como parámetro el símbolo del jugador:

\inputencoding{latin9}\begin{lstlisting}
# triqui5.py
def jugar(jugador,tablero):
    print("Juegue jugador ", jugador)
    f = input("fila? ")
    c = input("columna? ")
    tablero[f][c] = jugador
    imprimir(triqui)
    diag = ganaDiagonal1(jugador,tablero) or \
           ganaDiagonal2(jugador,tablero)
    linea = ganaHorizontal(jugador,tablero) or \ 
            ganaVertical(jugador,tablero)
    return  diag or linea
\end{lstlisting}
\inputencoding{utf8}
Con este cambio nuestro ciclo de juego es más pequeño, y el programa
es más fácil de mantener:

\inputencoding{latin9}\begin{lstlisting}
# triqui5.py
while True:
    if jugar("O",triqui):
        print("Gana el jugador O !!!!")
        break
    if jugar("X",triqui):
        print("Gana el jugador X !!!!")
        break
\end{lstlisting}
\inputencoding{utf8}
\section{Validaciones}

\index{validación}

Los usuarios pueden cometer errores, por esta razón los programas
deben revisar todos los datos que generan para ver si cumplen las
condiciones para operar. El código que revisa una condición o restricción
de este tipo se llama validación.

En el triqui podemos agregar validación al juego. Tanto \texttt{f}
como \texttt{c},los valores que el usuario digita para jugar en una
fila y columna deben ser enteros en el intervalo {[}0,2{]} para que
podamos representarlos en la matriz de 3 filas y 3 columnas. Además,
la casilla \texttt{tablero{[}f{]}{[}c{]}} debe estar vacía para que
una jugada nueva pueda hacerse allí.

Estas validaciones pueden ponerse en un ciclo que le pida al jugador
digitar los valores para f y c varias veces, hasta que sean correctos:

\inputencoding{latin9}\begin{lstlisting}
# triqui6.py

def valido(n):
    return 0<=n<=2
    
def jugar(jugador,tablero):
    while True:     
        print("Juegue jugador ", jugador)
        f = int(input("fila? "))
        c = int(input("columna? "))
        if type(f)==type(c)==type(1) and valido(f) 
           and valido(c) and tablero[f][c]==' ':
            tablero[f][c] = jugador
            break      

    imprimir(tablero)
    diag = ganaDiagonal1(jugador,tablero) or \
           ganaDiagonal2(jugador,tablero)
    linea = ganaHorizontal(jugador,tablero) or \
            ganaVertical(jugador,tablero)
    return  diag or linea
\end{lstlisting}
\inputencoding{utf8}
\section{Empates}

Ahora agregamos una función para chequear si hay empate entre los
jugadores. Esto sucede si el tablero está lleno, o sea que no hay
ninguna casilla vacía (con el carácter ' '):

\inputencoding{latin9}\begin{lstlisting}
# triqui7.py

def empate(tablero):
    for i in range(3):
        for j in range(3):
            if tablero[i][j]==' ':
                return False
    return True
\end{lstlisting}
\inputencoding{utf8}
Llamamos a empate después de cada jugador: 

\inputencoding{latin9}\begin{lstlisting}
# triqui8.py
while True:
    if jugar("O",triqui):
        print("Gana el jugador O !!!!")
        break
    if empate(triqui):
        print("Empate !!!")
        break
    if jugar("X",triqui):
        print("Gana el jugador X !!!!")
        break
    if empate(triqui):
        print("Empate !!!")
        break
\end{lstlisting}
\inputencoding{utf8}
Y también agregamos un mensaje de retroalimentación para el jugador
cuando no ha escogido una casilla válida:

\inputencoding{latin9}\begin{lstlisting}
# triqui8.py
def jugar(jugador,tablero):
    while True:     
        print("Juegue jugador ", jugador)
        f = input("fila? ")
        c = input("columna? ")
        if type(f)==type(c)==type(1) and valido(f) 
           and valido(c) and tablero[f][c]==' ':
            tablero[f][c] = jugador
            break
        else:
            print("Posici�n inv�lida!")

    imprimir(triqui)
    diag = ganaDiagonal1(jugador,tablero) or \
           ganaDiagonal2(jugador,tablero)
    linea = ganaHorizontal(jugador,tablero) or \
            ganaVertical(jugador,tablero)
    return  diag or linea
\end{lstlisting}
\inputencoding{utf8}
\section{Reestructurando más}

Podemos crear una función \texttt{gana}, fructífera, que nos permita
que jugar sea mas pequeña:

\inputencoding{latin9}\begin{lstlisting}
# triqui9.py
def gana(jugador,tablero):
    """ Analiza si el jugador gana la partida """
    diag = ganaDiagonal1(jugador,tablero) or \
           ganaDiagonal2(jugador,tablero)
    linea = ganaHorizontal(jugador,tablero) or \
            ganaVertical(jugador,tablero)
    return  diag or linea
\end{lstlisting}
\inputencoding{utf8}
Con este cambio también ganamos algo: la verificación de quien gana
el juego puede hacerse con una sola función en otro programa, por
ejemplo uno con una interfaz gráfica de usuario, como el del capítulo
\ref{triqui-kivy}.

Ahora, \texttt{gana} se llama en el ciclo principal. Y todo esto se
puede poner en la parte principal del programa:

\inputencoding{latin9}\begin{lstlisting}
# triqui9.py
if __name__ == '__main__':
    triqui = crear()
    imprimir(triqui)

    while True:
        jugar("O",triqui)
        if gana("O",triqui):
            print("Gana el jugador O !!!!")
            break
        if empate(triqui):
            print("Empate !!!")
            break
        jugar("X",triqui)
        if gana("X",triqui):
            print("Gana el jugador X !!!!")
            break
        if empate(triqui):
            print("Empate !!!")
            break
\end{lstlisting}
\inputencoding{utf8}
Así terminamos con triqui9.py, un programa con 12 funciones y un ciclo
de juego que tiene en total 124 líneas de código, ¡pero empezó como
una simple impresión de una matriz vacía!

\section{Resumen}
\begin{description}
\item [{triqui0.py:}] crea e imprime el tablero vacío 
\item [{triqui1.py:}] permite que un solo jugador juegue por siempre 
\item [{triqui2.py:}] permite que dos jugadores jueguen por siempre 
\item [{triqui3.py:}] revisa si algún jugador gana en las diagonales 
\item [{triqui4.py:}] revisa si algún jugador gana en filas o columnas 
\item [{triqui5.py:}] evita la duplicación de código creando una función
'jugar' 
\item [{triqui6.py:}] introduce un ciclo en 'jugar' para validar las jugadas 
\item [{triqui7.py:}] introduce la verificación de empates en el juego 
\item [{triqui8.py:}] llama a empate 2 veces en lugar de 1 y añade un mensaje
al jugador 
\item [{triqui9.py:}] crea una función 'gana' para que 'jugar' sea mas
pequeña y pone el ciclo del juego en la parte principal. 
\end{description}

\section{Glosario}
\begin{description}
\item [{validación:}] análisis de los datos que genera un usuario humano
para que estén dentro de los límites de operación del software.
\item [{reestructuración:}] reescritura de porciones del código para mejorar
la calidad del programa. Por ejemplo, se puede mejorar la legibilidad
del código, también se puede eliminar la redundancia.

\index{validación} \index{reestructuración}
\end{description}

\section{Ejercicios}
\begin{enumerate}
\item Modifique el triqui para que el computador juegue automáticamente,
seleccionando una casilla vacía al azar.
\item Modifique el triqui para que tenga un menú de entrada en el que se
pueda escoger entre dos modalidades: 1 jugador contra el computador,
2 jugadores entre si por turnos y 3, salir del programa. Ahora, cada
vez que termine una partida el flujo de ejecución del programa debe
volver al menú.
\end{enumerate}



\chapter{Interludio 2: Creando un nuevo tipo de datos }

\label{overloading} \index{tipo de dato!definido por el usuario}

Los lenguajes de programación orientados a objetos permiten a los
programadores crear nuevos tipos de datos que se comportan de manera
muy similar a los tipos primitivos. Exploraremos esta característica
construyendo una clase \texttt{Fraccionario} que se comporte como
los tipos de datos numéricos primitivos (enteros y flotantes).

Los números fraccionario o racionales son valores que se pueden expresar
como una división entre dos números enteros, como $\frac{1}{3}$.
El número superior es el numerador y el inferior es es el denominador.

\index{racional} \index{fracción} \index{numerador} \index{denominador}

La clase \texttt{Fraccion} empieza con un método constructor que recibe
como parámetros al numerador y al denominador:

\begin{pythoncode}
class Fraccion:
  def __init__(self, numerador, denominador=1):
    self.numerador = numerador
    self.denominador = denominador
\end{pythoncode}
El denominador es opcional. Una Fracción con un solo parámetro representa
a un número entero. Si el numerador es $n$, construimos la fracción
$n/1$.

El siguiente paso consiste en escribir un método \texttt{\_\_str\_\_}
que despliegue las fracciones de una manera natural. Como estamos
acostumbrados a la notación ``numerador/denominador'', lo más natural
es:

\begin{pythoncode}
class Fraccion:
  ...
  def __str__(self):
    return "%d/%d" % (self.numerador, self.denominador)
\end{pythoncode}
 Para realizar pruebas, ponemos este código en un archivo \texttt{Fraccion.py}
y lo importamos en el intérprete de Python. Ahora creamos un objeto
fracción y lo imprimimos.

\begin{pyconcode}
>>> from Fraccion import Fraccion
>>> s = Fraccion(5,6)
>>> print("La fraccion es", s)
La fraccion es 5/6
\end{pyconcode}
 El método \texttt{print}, automáticamente invoca al método \texttt{\_\_str\_\_}
de manera implícita.

\section{Multiplicación de fracciones}

\index{multiplicación!de fracciones} \index{fracciones!multiplicación}

Nos gustaría aplicar los mismos operadores de suma, resta, multiplicación
y división a las fracciones. Para lograr esto podemos sobrecargar
los operadores matemáticos en la clase \texttt{Fraccion}.

\index{sobrecarga} \index{operadores!sobrecarga de} \index{operador matemático}

La multiplicación es la operación más sencilla entre fraccionarios.
El resultado de multiplicar dos fracciones a y v es una nueva fracción
en la que el numerador es el producto de los dos numeradores (de a
y b) y el denominador es el producto de los dos denominadores (de
a y b).

Python define que el método \texttt{\_\_mul\_\_} se puede definir
en una clase para sobrecargar el operador \texttt{{*}}:
\begin{pythoncode}
class Fraccion:
  ...
  def __mul__(self, otro):
    return Fraccion(self.numerador*otro.numerador, \
                    self.denominador*otro.denominador)
\end{pythoncode}

Podemos probar este método calculando un producto sencillo:
\begin{pyconcode}
>>> print(Fraccion(5,6) * Fraccion(3,4))
15/24
\end{pyconcode}

Funciona, pero se puede mejorar. Podemos manejar el caso en el que
se multiplique una fracción por un número entero. Por medio de la
función \texttt{type} se puede probar si \texttt{otro} es un entero
y convertirlo a una fracción antes de realizar el producto:
\begin{pythoncode}
class Fraccion:
  ...
  def __mul__(self, otro):
    if type(otro) == type(5):
      otro = Fraccion(otro)
    return Fraccion(self.numerador   * otro.numerador, \
                    self.denominador * otro.denominador)
\end{pythoncode}

Ahora, la multiplicación entre enteros y fracciones funciona, pero
sólo si la fracción es el operando a la izquierda :

\begin{pyconcode}
>>> print(Fraccion(5,6) * 4)
20/6
>>> print(4 * Fraccion(5,6))
TypeError: __mul__ nor __rmul__ defined for these operands
\end{pyconcode}
 Para evaluar un operador binario como la multiplicación, Python chequea
el operando izquierdo primero, para ver si su clase define el método
\texttt{\_\_mul\_\_}, y que tenga soporte para el tipo del segundo
operando. En este caso el operador primitivo para multiplicar enteros
no soporta las fracciones.

Después, Python chequea si el operando a la derecha provee un método
\texttt{\_\_rmul\_\_} que soporte el tipo del operando de la izquierda.
En este caso, como no hay definición de \texttt{\_\_rmul\_\_} en la
clase \texttt{Fraccion}, se genera un error de tipo.

Hay una forma sencilla de definir \texttt{\_\_rmul\_\_}:
\begin{pythoncode}
class Fraccion:
  ...
  __rmul__ = __mul__
\end{pythoncode}
 Esta asignación dice que \texttt{\_\_rmul\_\_} contiene el mismo
código que \texttt{\_\_mul\_\_}. Si ahora evaluamos \texttt{4 {*}
Fraccion(5,6)}, Python llama a \texttt{\_\_rmul\_\_} y le pasa al
4 como parámetro:

\begin{pyconcode}
>>> print(4 * Fraccion(5,6))
20/6
\end{pyconcode}
 Como \texttt{\_\_rmul\_\_} tiene el mismo código que \texttt{\_\_mul\_\_},
y el método \texttt{\_\_mul\_\_} puede recibir un parámetro entero,
nuestra multiplicación de fracciones funciona bien.

\section{Suma de fracciones}

\index{suma!de fracciones} \index{fracciones!suma}

La suma es más complicada que la multiplicación. La suma $a/b+c/d$
da como resultado $\frac{(ad+cb)}{bd}$.

Basándonos en la multiplicación, podemos escribir los métodos \texttt{\_\_add\_\_}
y \texttt{\_\_radd\_\_}:

\begin{pythoncode}
class Fraccion:
  ...
  def __add__(self, otro):
    if type(otro) == type(5):
      otro = Fraccion(otro)
    return Fraccion(self.numerador   * otro.denominador + \
                    self.denominador * otro.numerador,    \
                    self.denominador * otro.denominador) 

  __radd__ = __add__
\end{pythoncode}
 Podemos probar estos métodos con objetos \texttt{Fraccion} y con
números enteros.

\begin{pyconcode}
>>> print(Fraccion(5,6) + Fraccion(5,6))
60/36
>>> print(Fraccion(5,6) + 3)
23/6
>>> print(2 + Fraccion(5,6))
17/6
\end{pyconcode}
 Los primeros ejemplos llaman al método \texttt{\_\_add\_\_}; el último
ejemplo llama al método \texttt{\_\_radd\_\_}.

\section{El algoritmo de Euclides}

\index{máximo divisor común} \index{Euclides} \index{pseudocódigo}
\index{simplificar}

En el ejemplo anterior, calculamos $5/6+5/6$ y obtuvimos $60/36$.
Es correcto, pero no es la manera más sencilla de presentar la respuesta.
Para \textbf{simplificar} la fracción tenemos que dividir el numerador
y el denominador por su \textbf{máximo divisor común (MDC)}, que para
este caso es 12. Entonces, un resultado mas sencillo es $5/3$.

En general, cada vez que creamos un nuevo objeto de tipo \texttt{Fraccion}
deberíamos simplificarlo dividiendo el numerador y el denominador
por su MDC. Si la fracción no se puede simplificar, el MDC es 1.

Euclides de Alejandría (aprox. 325–265 A.C) presentó un algoritmo
para encontrar el MDC de dos números enteros $m$ y $n$:
\begin{quote}
Si $n$ divide a $m$ exactamente, entonces $n$ es el MDC. Sino,
el MDC de $m$ y $n$ es el MDC de $n$ y el residuo de la división
$m/n$. 
\end{quote}
Esta definición recursiva se puede implementar en una función:

\begin{pythoncode}
def MDC (m, n):
  if m % n == 0:
    return n
  else:
    return MDC(n, m%n)
\end{pythoncode}
 En la primera línea el operador residuo nos permite chequear si n
divide a n exactamente. En la última línea, lo usamos para calcular
el residuo de la división.

Como todas las operaciones que hemos escrito crean nuevas fracciones
como resultado, podemos simplificar todos los valores de retorno modificando
el método constructor.

\begin{pythoncode}
class Fraccion:
  def __init__(self, numerador, denominador=1):
    g = MDC (numerador, denominador)
    self.numerador   =   numerador / g
    self.denominador = denominador / g
\end{pythoncode}
 Ahora, cada vez que creamos una nueva \texttt{Fraccion}, ¡se simplifica!.

\begin{pyconcode}
>>> Fraccion(100,-36)
-25/9
\end{pyconcode}
 Una característica adicional que nos provee \texttt{MDC} es que si
la fracción es negativa, el signo menos siempre se mueve hacia el
numerador.

\section{Comparando fracciones}

\index{comparación!de fracciones} \index{fracciones!comparación de}

Si vamos a comparar dos objetos \texttt{Fraccion}, digamos \texttt{a}
y \texttt{b}, evaluando la expresión \texttt{a == b}. Como la implementación
de \texttt{==} chequea igualdad superficial de objetos por defecto,
sólo retornará cierto si \texttt{a} y \texttt{b} \textit{son} el mismo
objeto.

Es mucho más probable que deseemos retornar cierto si $a$ y $b$
tienen el mismo valor —esto es, chequear igualdad profunda.

Tenemos que enseñarle a las fracciones cómo compararse entre sí. Como
veremos en la sección \ref{comparecard}, podemos sobrecargar todos
los operadores de comparación por medio de la implementación de un
método \texttt{\_\_cmp\_\_}.

Por convención, el método \texttt{\_\_cmp\_\_} retorna un número negativo
si \texttt{self} es menos que \texttt{otro}, cero si son iguales,
y un número positivo si \texttt{self} es mayor que \texttt{otro}.

La forma más sencilla de comparar fracciones consiste en hacer una
multiplicación cruzada. Si $a/b>c/d$, entonces $ad>bc$. Con esto
en mente, implementamos \texttt{\_\_cmp\_\_}:

\begin{pythoncode}
class Fraccion:
  ...
  def __cmp__(self, otro):
    dif = (self.numerador  * otro.denominador - \
            otro.numerador * self.denominador)
    return dif
\end{pythoncode}
 Si \texttt{self} es mayor que \texttt{otro}, entonces \texttt{dif}
será positivo. Si \texttt{otro} es mayor, \texttt{dif} será negativo.
Si son iguales, \texttt{dif} es cero.

\section{Extendiendo las fracciones}

Todavía no hemos terminado. Tenemos que implementar la resta sobrecargando
el método \texttt{\_\_sub\_\_} y la división con el método \texttt{\_\_div\_\_}.

Podemos restar por medio de la suma si antes negamos (cambiamos de
signo) al segundo operando. También podemos dividir por medio de la
multiplicación si antes invertimos el segundo operando.

Siguiendo este razonamiento, una forma de realizar las operaciones
resta y división consiste en definir primero la negación por medio
de la sobrecarga de \texttt{\_\_neg\_\_} y la inversión sobre sobrecargando
a \texttt{\_\_invert\_\_}.

Un paso adicional sería implementar \texttt{\_\_rsub\_\_} y \texttt{\_\_rdiv\_\_}.
Desafortunadamente no podemos usar el mismo truco que aplicamos para
la suma y la multiplicación, porque la resta y la división no son
conmutativas. En estas operaciones el orden de los operandos altera
el resultado, así que no podemos asignar a \texttt{\_\_rsub\_\_} y
a \texttt{\_\_rdiv\_\_} los método \texttt{\_\_sub\_\_} y \texttt{\_\_div\_\_},
respectivamente.

Para realizar la \textbf{negación unaria}, sobrecargamos a \texttt{\_\_neg\_\_}.

\index{operador unario} \index{negación}

Podemos calcular potencias sobrecargando a \texttt{\_\_pow\_\_}, pero
la implementación tiene un caso difícil: si el exponente no es un
entero, puede que no sea posible representar el resultado como una
\texttt{Fraccion}. Por ejemplo, la siguiente expresión \texttt{Fraccion(2)
{*}{*} Fraccion(1,2)} es la raíz cuadrada de 2, que es un número irracional
(no puede representarse por ninguna fracción). Así que no es fácil
escribir una función general para \texttt{\_\_pow\_\_}.

\index{irracional}

Hay otra extensión a la clase \texttt{Fraccion} que usted puede imaginar.
Hasta aquí, hemos asumido que el numerador y el denominador son enteros.
También podemos permitir que sean de tipo long.

\section{Glosario}
\begin{description}
\item [{Máximo divisor común (MDC):}] el entero positivo más grande
que divide exactamente a dos números (por ejemplo, el numerador y
el denominador en una fracción).
\item [{Simplificar:}] cambiar una fracción en otra equivalente que tenga
un MDC de 1.
\item [{negación unaria:}] la operación que calcula un inverso aditivo,
usualmente representada con un signo menos. Es ``unaria'' en contraposición
con el menos binario que representa a la resta.

\index{máximo divisor común} \index{simplificar} \index{negación unaria}
\end{description}

\section{Ejercicios}
\begin{enumerate}
\item Complemente la implementación de la clase \texttt{Fraccion} para que
soporte denominadores y numeradores de tipo long (enteros grandes). 
\item Agregue la operación resta
\item Agregue la operación división
\item Agregue la operación exponenciación. 
\end{enumerate}



\chapter{Postludio: Triqui con interfaz gráfica }

\label{triqui-kivy}

\index{Triqui}

\section{Motivación}

En el capítulo \ref{cap:inter1:triqui} desarrollamos un juego de
triqui completo para integrar varios conceptos de programación. Ahora,
para integrar varios conceptos de programación con objetos haremos
casi lo mismo, construiremos un triqui con interfaz gráfica de usuario.

El programa utiliza la biblioteca Kivy, que permite realizar interfaces
gráficas que corran en celulares, tablets (con pantallas sensibles
al tacto) y computadores tradicionales (corriendo MacOS X, Windows
y Linux). Hay que conseguir el instalador de:

\url{ http://kivy.org}

Los archivos triqui0.py a triqui8.py están en:

\url{https://github.com/abecerra/thinkcs-py_es/releases/download/thinkcs-py_es_e2-rc1/triqui-kivy-python3.zip}

Para comprender el capítulo hay que ejecutar cada versión del programa
a medida que se avanza en la lectura. En las secciones a continuación
se discuten los fragmentos del programa a medida que se van agregando,
cada fragmento tiene el nombre del archivo en que se introdujo como
comentario inicial.

\section{Ganar y empatar}

Para empezar tomamos el código final del triqui desarrollado en el
capítulo \ref{cap:inter1:triqui} y lo convertimos en un módulo que
nos permite verificar si alguien gana el juego o si hay un empate.
El proceso es sencillo: eliminamos todo el código que no tenga que
ver con verificar quien gana o si hay empate, tomando \texttt{triqui9.py}
para transformarlo en \texttt{validar.py}, un módulo con 9 funciones:
crear, ganaDiagonal1, ganaDiagonal2, ganaFila, ganaHorizontal, ganaColumna,
ganaVertical, gana y empate. Mas adelante lo importaremos, para reutilizar
el código.

\index{reutilización}

\section{Programación orientada a eventos}

\index{Programación orientada a eventos}

En muchas bibliotecas gráficas como kivy existe un gran ciclo que
procesa los eventos generados a través del teclado, apuntador o la
pantalla sensible al tacto. Este ciclo es infinito, pero se le hace
un break cuando se cierra la ventana de la aplicación. En la programación
orientada a eventos debemos acostumbrarnos a varias cosas.

Primero, el flujo del programa está determinado por lo que haga el
usuario con los elementos gráficos de la pantalla. Kivy procesará
cada evento (click, tecla digitada) de manera predeterminada, por
ejemplo cerrar la ventana hará un break en el gran ciclo.

\index{evento}

Segundo, las bibliotecas como Kivy se encargan de redibujar automáticamente
partes de la ventana cuando otra ventana se pase por encima, o cuando
se cambie de tamaño.

Nuestro primer programa importa lo que necesita de kivy:

\begin{pythoncode}
# triqui0.py
import kivy
kivy.require('1.9.1')
from kivy.app import App
from kivy.uix.gridlayout import GridLayout

class Triqui(GridLayout):
    def __init__(self, **kwargs):
        super(Triqui, self).__init__(**kwargs)
   

class Programa(App):
    def build(self):
        self.title = 'Triqui'
        return Triqui()

if __name__ == '__main__':
    Programa().run()
\end{pythoncode}

El gran ciclo reside en la clase \texttt{App} de Kivy, de la que heredamos
la clase \texttt{Programa}. Cuando arranca el programa, al ejecutar
\texttt{run()} se corre este gran ciclo que procesa eventos.

El método build de programa retorna un objeto Triqui, que es nuestra
ventana. La clase Triqui hereda de GridLayout, una ventana que contiene
elementos en disposición de cuadrícula (como una matriz). El método
\texttt{init} de Triqui llama al método \texttt{init} de la clase
madre y le pasa una lista con un número de argumentos variable ({*}{*}kwargs).

Por ahora nuestra ventana, instancia de la clase \texttt{Triqui},
está vacía.

\section{Widgets}

Las ventanas suelen tener muchos elementos gráficos como menús, botones,
paneles entre otros. En bibliotecas como kivy se llaman widgets. Por
ejemplo, un botón es un tipo de widget que se define en la clase \texttt{Button}.

Como el flujo de los programas gráficos no está determinado por el
programador, sino por el usuario al interactuar con los widgets de
la ventana, el mecanismo que se utiliza para reaccionar ante los eventos
es el de registrar métodos que serán invocados automáticamente por
el gran ciclo.

\begin{pythoncode}
# triqui1.py
class Triqui(GridLayout):
    def __init__(self, **kwargs):
        super(Triqui, self).__init__(**kwargs)
        self.add_widget(Button(font_size=100, 
                        on_press=self.boton_presionado))
    
    def boton_presionado(self, w):
        pass
\end{pythoncode}

Los widgets se agregan a una ventana mediante el método \texttt{add\_widget}.
Aquí agregamos un botón y registramos un método que responde al evento
de presionarlo (\texttt{on\_press}). Por ahora el método no hace nada.

\section{El Triqui}

A continuación, definimos la geometría de la ventana como una matriz
de 3 filas y 3 columnas en la que cada elemento es un botón. Ahora,
en el método \texttt{boton\_presionado} vamos a mostrar un cuadro
de diálogo que muestra un texto sencillo.

\begin{pythoncode}
# triqui2.py
class Triqui(GridLayout):
    def __init__(self, **kwargs):
        super(Triqui, self).__init__(**kwargs)
        self.cols = 3
        self.rows = 3
        for i in range(3):
            for j in range(3):
                self.add_widget(Button(font_size=100, \
                                on_press=self.boton_presionado))
        
    def boton_presionado(self, w):
        MostrarMensaje("Titulo","Presionaste una casilla")
\end{pythoncode}

\texttt{MostrarMensaje} es una clase que heredamos de \texttt{PopUp},
la clase que tiene kivy para cuadros de diálogo:

\begin{pythoncode}
# triqui2.py
class MostrarMensaje(Popup): 
    def __init__(self, titulo, mensaje, **kwargs):
        self.size_hint_x = self.size_hint_y = .5
        self.title = titulo
        super(MostrarMensaje, self).__init__(**kwargs)
        self.add_widget(Button(text=mensaje, \
                        on_press=lambda x:self.dismiss()))
        self.open()
\end{pythoncode}

El cuadro de diálogo tiene un título y un botón que, al ser presionado,
cierra todo el cuadro. La acción del botón se registra asignando al
parámetro \texttt{on\_press}
una función anónima \texttt{lambda x: self.dismiss()}.
Esta función recibe un parámetro x que no necesitamos. El método dismiss
de la clase Popup cierra el cuadro de diálogo.

\section{Jugando por turnos}

Como el flujo de ejecución depende de los usuarios, vamos a llevar
pista en el programa de quien tiene el turno de juego con un atributo
en la clase Triqui. Hay que crear el atributo en el método de inicialización
y modificarlo en cada jugada. El \texttt{turno} será 'O' para el primer
jugador y 'X' para el segundo.

\begin{pythoncode}
# triqui3.py
class Triqui(GridLayout):
    def __init__(self, **kwargs):
        super(Triqui, self).__init__(**kwargs)
        self.cols = 3
        self.rows = 3
        for i in range(3):
            for j in range(3):
                self.add_widget(Button(font_size=100, \
                  on_press=self.boton_presionado, text=' '))
        self.turno = 'O'

    def boton_presionado(self, w):
        if w.text != ' ':
            MostrarMensaje('Error!', "Ya se ha jugado \
                            en esa casilla!")
            return
        if self.turno == 'O':
            w.text =  'O'
            self.turno = 'X'
        else:
            w.text = 'X'
            self.turno = 'O'
\end{pythoncode}

Cuando se presiona un botón se verifica si la casilla está vacía para
poder jugar en ella. Si no es así se cambia el texto del botón y se
cambia el turno para el otro jugador. Observe que en el método de
inicialización de \texttt{Triqui} al texto de todos los botones se
le asigna un espacio.

\section{Reutilización}

\index{reutilización}

Agregamos un botón que tiene propiedades para registrar la fila y
la columna, heredando de la clase \texttt{Button}.

\begin{pythoncode}
# triqui4.py
class Boton(Button):
    fila = NumericProperty(0)
    columna = NumericProperty(0)
    
    def __init__(self, **kwargs):      
        super(Boton, self).__init__(**kwargs)
        self.font_size=100
        self.text=' '
\end{pythoncode}

Esto nos permite pasar el estado de los botones a una matriz con el
siguiente método de la clase \texttt{Triqui}:

\begin{pythoncode}
# triqui4.py
    def botones_a_matriz(self,tablero):
        for i in self.children:
            f = i.fila
            c = i.columna
            self.tablero[f][c]=i.text
\end{pythoncode}

En Kivy, children contiene todos los widgets que se han agregado a
una ventana, en nuestro caso los 9 botones.

Así podremos reutilizar el módulo \texttt{validar}, creando la matriz
que lleva el estado del juego :

\begin{pythoncode}
# triqui4.py

from validar import *
    def __init__(self, **kwargs):
        super(Triqui, self).__init__(**kwargs)
        self.cols = 3
        self.rows = 3
        for i in range(3):
            for j in range(3):
                self.add_widget(Boton(on_press=self.boton_presionado,\
                       fila=i,columna=j))
        self.turno = 'O'
        self.tablero = crear()
\end{pythoncode}

Ahora estamos en condiciones de poner valores a la matriz cada vez
que se halla realizado una jugada:

\begin{pythoncode}
# triqui5.py
    def boton_presionado(self, w):
        if w.text != ' ':
            MostrarMensaje('Error!', \
                  "Ya se ha jugado en esa casilla!")
            return
        else:
            if self.turno == 'O':
                w.text =  'O'                
                self.turno = 'X'
                self.botones_a_matriz()
                if gana("O",self.tablero):
                    MostrarMensaje("Fin", "Gana el jugador O")
            else:
                # Muy similar para el otro jugador!
\end{pythoncode}

\section{Reset}

Podemos reiniciar el juego cada vez que un jugador gane, mediante
la creación del siguiente método de \texttt{Triqui}:

\begin{pythoncode}
# triqui6.py
   def reset(self):
        for i in self.children:
            i.text = ' '
        self.turno = 'O'
\end{pythoncode}

Ahora lo llamamos cada vez que un jugador gana, así el tablero de
juego quedará limpio para que se inicie otra partida.

\begin{pythoncode}
# triqui6.py
    def boton_presionado(self, w):
        # Todo lo anterior igual
        else:
                w.text = 'X'
                self.turno = 'O'
                self.botones_a_matriz()
                if gana("X",self.tablero):
                    MostrarMensaje("Fin", "Gana el jugador X")
                    self.reset()
\end{pythoncode}

Aprovechando al método reset, añadimos el chequeo de empates entre
los dos jugadores:

\begin{pythoncode}
# triqui7.py
    def boton_presionado(self, w):
        # Todo lo anterior igual
            if empate(self.tablero):
                MostrarMensaje("Fin", "Empate!")
                self.reset()
\end{pythoncode}

\section{Reestructurando, otra vez}

Otra vez tenemos código parecido para revisar el estado del juego
con los dos jugadores que es mejor consolidar en una función para
mejorar la calidad del programa. Para esto definimos el método revisar,
que:
\begin{itemize}
\item Si el jugador actual gana, muestra el mensaje y resetea.
\item Si hay empate, muestra el mensaje y resetea.
\item Si nadie gana y no hay empate, pasa el turno al otro jugador. 
\end{itemize}

\begin{pythoncode}
# triqui8.py
  def revisar(self):          
        if gana(self.turno,self.tablero):
            mensaje = "Gana el jugador "+self.turno+"."
            MostrarMensaje("Fin", mensaje)
            self.reset()
        elif empate(self.tablero):
            MostrarMensaje("Fin", "Empate!")
            self.reset()
        else:
            self.turno = self.otro()
\end{pythoncode}

Que depende del método otro:

\begin{pythoncode}
# triqui8.py
    def otro(self):
        if self.turno == 'O':
            return 'X'
        else:
            return 'O'
\end{pythoncode}

Así terminamos con un programa que tiene en total 4 clases, con 9
métodos distribuidos en ellas, además de las 9 funciones del módulo
validar. Tiene 96 líneas de código en triqui9.py y 66 en validar,
para un total de 162.

Ilustra algo que siempre pasa con los programas textuales, cuando
se convierten a gráficos se aumenta el código substancialmente. La
ventaja, aparte de la estética, es que el Triqui con la biblioteca
kivy puede ejecutarse en Linux, Windows, Mac OS X, Android y iOS (el
sistema operativo de los teléfonos iphone y de los tablets ipad).

Por ejemplo, el paquete para Android del triqui puede descargarse
de:

\url{https://github.com/abecerra/thinkcs-py_es/releases/download/thinkcs-py_es_e2-rc1/KivyTriquiABe-1.1.0-debug.apk}

Para instalarlo en un smartphone o tablet.

\section{Resumen}
\begin{description}
\item [{triqui0.py:}] crea una ventana vacía 
\item [{triqui1.py:}] Agrega un botón a la ventana (se ve feo!) 
\item [{triqui2.py:}] Agrega 9 botones para formar el tablero del triqui 
\item [{triqui3.py:}] Permite jugar a los dos jugadores sin ningún chequeo 
\item [{triqui4.py:}] Agrega una clase heredada de Button para llevar fila
y columna 
\item [{triqui5.py:}] Cada vez que se juega se copia el estado de los botones
a una matriz 
\item [{triqui6.py:}] Se valida si los jugadores ganan el juego con el
código del triqui viejo y se resetea el juego. 
\item [{triqui7.py:}] Se revisa si hay empate, si lo hay, se resetea el
juego. 
\item [{triqui8.py:}] Se mejora el código evitando la duplicación. 
\end{description}

\section{Glosario}
\begin{description}
\item [{Evento:}] Acción generada por el usuario de una aplicación gráfica.
\item [{Widget:}] Elemento de una aplicación gráfica que se coloca en una
ventana. Hay botones, paneles, barras deslizadoras, áreas de texto,
entre otras clases de widgets.
\end{description}

\section{Ejercicios}
\begin{enumerate}
\item Modifique el triqui para que el computador juegue automáticamente,
seleccionando una casilla vacía al azar.
\item Agregue un sonido cada vez que se haga una jugada válida y otro sonido
de error para cuando el jugador seleccione una casilla en la que ya
se ha jugado.
\item Modifique el triqui para que tenga un menú gráfico de entrada en el
que se pueda escoger entre dos modalidades: 1 jugador contra el computador,
2 jugadores entre si por turnos y 3, salir del programa. Ahora, cada
vez que termine una partida el flujo de ejecución del programa debe
volver al menú.
\end{enumerate}


\clearemptydoublepage % creando un nuevo tipo de dato

\chapter{Programas completos}

\section{Clase punto}

\begin{pythoncode}
class Punto:
  def __init__(self, x=0, y=0):
    self.x = x
    self.y = y

  def __str__(self):
    return '(' + str(self.x) + ',' + str(self.y) + ')'

  def __add__(self, otro):
    return Punto(self.x + otro.x, self.y + otro.y)

  def __sub__(self, otro):
    return Punto(self.x - otro.x, self.y - otro.y)

  def __mul__(self, otro):
    return self.x * otro.x + self.y * otro.y

  def __rmul__(self, otro):
    return Punto(otro * self.x, otro * self.y)

  def invertir(self):
    self.x, self.y = self.y, self.x

  def DerechoYAlReves(derecho):
    from copy import copy
    alreves = copy(derecho)
    alreves.invertir()
    print(str(derecho) + str(alreves))
\end{pythoncode}

\section{Clase hora}

\begin{pythoncode}
class Hora:
  def __init__(self, hora=0, minutos=0, segundos=0):
    self.hora = hora
    self.minutos = minutos
    self.segundos = segundos

  def __str__(self):
    return str(self.hora) + ":" + str(self.minutos) + 
           ":" + str(self.segundos)

  def convertirAsegundoss(self):
    minutos = self.hora * 60 + self.minutos
    segundos = self.minutos * 60 + self.segundos
    return segundos

  def incrementar(self, segs):
    segs = segs + self.segundos

    self.hora = self.hora + segs/3600
    segs = segs % 3600
    self.minutos = self.minutos + segs/60
    segs = segs % 60
    self.segundos = segs

def crearHora(segs):
  H = Hora()
  H.hora = segs/3600
  segs = segs - H.hora * 3600
  H.minutos = segs/60
  segs = segs - H.minutos * 60
  H.segundos = segs
  return H
\end{pythoncode}

\section{Cartas, mazos y juegos}

\begin{pythoncode}
import random

class Carta:
  listaFiguras = ["Treboles", "Diamantes", 
                  "Corazones", "Picas"]
  listaValores = [ "narf", "As", "2", "3", "4", "5", "6", 
               "7","8", "9", "10","Jota", "Reina", "Rey"]

  def __init__(self, figura=0, valor=0):
    self.figura = figura
    self.valor = valor

  def __str__(self):
    return self.listaValores[self.valor] + " de " 
           + self.listaFiguras[self.figura]

  def __cmp__(self, otro):
    # revisa las figuras
    if self.figura > otro.figura: 
       return 1
    if self.figura < otro.figura: 
       return -1
    # las figuras son iguales, se chequean los valores
    if self.valor > otro.valor: 
      return 1
    if self.valor < otro.valor: 
      return -1
    # los valores son iguales,  hay empate
    return 0

class Mazo:
  def __init__(self):
    self.Cartas = []
    for figura in range(4):
      for valor in range(1, 14):
        self.Cartas.append(Carta(figura, valor))

  def imprimirMazo(self):
    for Carta in self.Cartas:
      print(Carta)

  def __str__(self):
    s = ""
    for i in range(len(self.Cartas)):
      s = s + " "*i + str(self.Cartas[i]) + "\n"
    return s

  def barajar(self):
    import random
    nCartas = len(self.Cartas)
    for i in range(nCartas):
      j = random.randrange(i, nCartas)
      [self.Cartas[i], self.Cartas[j]] = [self.Cartas[j], 
                                          self.Cartas[i]]

  def eliminarCarta(self, Carta):
    if Carta in self.Cartas:
      self.Cartas.remove(Carta)
      return 1
    else: return 0

  def entregarCarta(self):
    return self.Cartas.pop()

  def estaVacio(self):
    return (len(self.Cartas) == 0)

  def repartir(self, manos, nCartas=999):
    nmanos = len(manos)
    for i in range(nCartas):
      if self.estaVacio(): 
         break    # rompe el ciclo si no hay cartas
      # quita la carta del tope
      Carta = self.entregarCarta()      
      # quien tiene el proximo turnoo?
      mano = manos[i % nmanos]    
      # agrega la carta a la mano
      mano.agregarCarta(Carta)         

class mano(Mazo):
  def __init__(self, nombre=""):
    self.Cartas = []
    self.nombre = nombre

  def agregarCarta(self,Carta) :
    self.Cartas.append(Carta)

  def __str__(self):
    s = "mano " + self.nombre
    if self.estaVacio():
      s = s + " esta vacia\n"
    else:
      s = s + " contiene\n"
    return s + Mazo.__str__(self)

class JuegoCartas:
  def __init__(self):
    self.Mazo = Mazo()
    self.Mazo.barajar()

class ManoJuegoSolterona(mano):
  def eliminarParejas(self):
    cont = 0
    originalCartas = self.Cartas[:]
    for carta in originalCartas:
      m = Carta(3-carta.figura, carta.valor)
      if m in self.Cartas:
        self.Cartas.remove(carta)
        self.Cartas.remove(m)
        print("mano %s: %s parejas %s" % 
                 (self.nombre,carta,m))
        cont = cont+1
    return cont

class JuegoSolterona(JuegoCartas):
  def jugar(self, nombres):
    # elimina la reina de treboles
    self.Mazo.eliminarCarta(Carta(0,12))

    # crea manos con base en los nombres
    self.manos = []
    for nombre in nombres : 
        self.manos.append(ManoJuegoSolterona(nombre))

    # reparte las Cartas
    self.Mazo.repartir(self.manos)
    print("---------- Cartas se han repartido")
    self.imprimirmanos()

    # eliminar parejas iniciales
    parejas = self.eliminarParejas()
    print("----- parejas descartadas, empieza el juego")
    self.imprimirmanos()

    # jugar hasta que se eliminan 50 cartas
    turno = 0
    nummanos = len(self.manos)
    while parejas < 25:
      parejas = parejas + self.jugarUnturno(turno)
      turno = (turno + 1) % nummanos

    print("---------- Juego Terminado")
    self.imprimirmanos ()

  def eliminarParejas(self):
    cont = 0
    for mano in self.manos:
      cont = cont + mano.eliminarParejas()
    return cont

  def jugarUnturno(self, i):
    if self.manos[i].estaVacio():
      return 0
    vecino = self.encontrarvecino(i)
    cartaEscogida = self.manos[vecino].entregarCarta()
    self.manos[i].agregarCarta(cartaEscogida)
    print("mano", self.manos[i].nombre, 
          "escogió", cartaEscogida)
    count = self.manos[i].eliminarParejas()
    self.manos[i].barajar()
    return count

  def encontrarvecino(self, i):
    nummanos = len(self.manos)
    for siguiente in range(1,nummanos):
      vecino = (i + siguiente) % nummanos
      if not self.manos[vecino].estaVacio():
        return vecino

  def imprimirmanos(self) :
    for mano in self.manos :
      print(mano)
\end{pythoncode}

\section{Listas enlazadas}

\begin{pythoncode}
def imprimirlista(Nodo) :
  while Nodo :
    print(Nodo),
    Nodo = Nodo.siguiente
  print()

def imprimirAlReves(lista) :
  if lista == None : 
      return
  cabeza = lista
  resto = lista.siguiente
  imprimirAlReves(resto)
  print(cabeza),

def imprimirAlRevesBien(lista) :
  print("("),
  if lista != None :
    cabeza = lista
    resto = lista.siguiente
    imprimirAlReves(resto)
    print(cabeza),
  print(")"),

def eliminarSegundo(lista) :
  if lista == None : return
  first  = lista
  second = lista.siguiente
  first.siguiente = second.siguiente
  second.siguiente = None
  return second

class Nodo :
  def __init__(self, carga=None) :
    self.carga = carga
    self.siguiente  = None

  def __str__(self) :
    return str(self.carga)

  def imprimirAlReves(self) :
    if self.siguiente != None :
      resto = self.siguiente
      resto.imprimirAlReves()
    print(self.carga),

class ListaEnlazada :
  def __init__(self) :
    self.numElementos = 0
    self.cabeza   = None

  def imprimirAlReves(self) :
    print("("),
    if self.cabeza != None :
      self.cabeza.imprimirAlReves()
    print(")"),

  def agregarAlPrincipio(self, carga) :
    nodo = Nodo(carga)
    nodo.siguiente = self.cabeza
    self.cabeza = nodo
    self.numElementos = self.numElementos + 1
\end{pythoncode}

\section{Clase pila}

\begin{pythoncode}
class Pila:              
  def __init__(self):
    self.items = []

  def meter(self, item):
    self.items.append(item)

  def sacar(self):
    return self.items.pop()

  def estaVacia(self):
    return(self.items == [])

def evalPostfija(expr) :
  import re
  expr = re.split("([^0-9])", expr)
  pila = Pila()
  for lexema in expr :
    if  lexema == '' or lexema == ' ':
      continue
    if  lexema == '+' :
      suma = pila.sacar() + pila.sacar()
      pila.meter(suma)
    elif lexema == '*' :
      producto = pila.sacar() * pila.sacar()
      pila.meter(producto)
    else :
      pila.meter(int(lexema))
  return pila.sacar()
\end{pythoncode}

\section{Colas PEPS y de colas de prioridad}

\begin{pythoncode}
class Cola:
  def __init__(self):
    self.numElementos = 0
    self.primero = None

  def estaVacia(self):
    return (self.numElementos == 0)

  def meter(self, carga):
    nodo = Nodo(carga) 
    nodo.siguiente = None
    if self.primero == None:
       # si esta vacia este nodo sera el primero
       self.primero = nodo
    else:
       # encontrar el ultimo nodo
       ultimo = self.primero
       while ultimo.siguiente:
     ultimo = ultimo.siguiente
     # pegar el nuevo
     ultimo.siguiente = nodo
       self.numElementos = self.numElementos + 1

  def sacar(self):
    carga = self.primero.carga
    self.primero = self.primero.siguiente
    self.numElementos = self.numElementos - 1
    return carga

class ColaMejorada:
  def __init__(self):
    self.numElementos = 0
    self.primero = None
    self.ultimo = None
        
  def estaVacia(self):
    return (self.numElementos == 0)

  def meter(self, carga):
    nodo = Nodo(carga) 
    nodo.siguiente = None
    if self.numElementos == 0:
      # si esta vacia, el nuevo nodo 
      # es primero y ultimo
      self.primero = self.ultimo = nodo
    else:
      # encontrar el ultimo nodo
      ultimo = self.ultimo
      # pegar el nuevo nodo
      ultimo.siguiente = nodo
      self.ultimo = nodo
      self.numElementos = self.numElementos + 1

    def sacar(self):
      carga = self.primero.carga
      self.primero = self.primero.siguiente
      self.numElementos = self.numElementos - 1
      if self.numElementos == 0:
        self.ultimo = None
      return carga

class ColaPrioridad:
  def __init__(self):
    self.items = []
      
  def estaVacia(self):
    return self.items == []
  
  def meter(self, item):
    self.items.append(item)

  def eliminar(self) :
    maxi = 0
    for i in range(1,len(self.items)) :
      if self.items[i] > self.items[maxi] :
    maxi = i
    item = self.items[maxi]
    self.items[maxi:maxi+1] = []
    return item

  def sacar(self):
    maxi = 0
    for i in range(0,len(self.items)): 
      if self.items[i] > self.items[maxi]:
    maxi = i
      item = self.items[maxi]
      self.items[maxi:maxi+1] = []
      return item


class golfista:
  def __init__(self, nombre, puntaje):
    self.nombre = nombre
    self.puntaje= puntaje
    
  def __str__(self):
    return "%-16s: %d" % (self.nombre, self.puntaje)

  def __cmp__(self, otro):
    # el menor tiene mayor prioridad
    if self.puntaje < otro.puntaje: return 1
      if self.puntaje > otro.puntaje: return -1
        return 0
\end{pythoncode}

\section{Árboles}

\begin{pythoncode}
class Arbol:
  def __init__(self, carga, izquierdo=None, derecho=None):
    self.carga = carga
    self.izquierdo = izquierdo
    self.derecho = derecho
      
  def __str__(self):
    return str(self.carga)
  
  def obtenerizquierdo(self):
    return self.izquierdo
  
  def obtenerderecho(self):
    return self.derecho
  
  def obtenercarga(self):
    return self.carga
    
  def asignarcarga(self, carga):
    self.carga = carga
  
  def asignarizquierdo(self, i):
    self.izquierdo = i
    
  def asignarderecho(self, d):
    self.derecho = d

def total(arbol):
  if arbol.izquierdo == None or arbol.derecho== None:
    return arbol.carga
  else:
    return total(arbol.izquierdo) + total(arbol.derecho) + 
             arbol.carga
   
def imprimirarbol(arbol):
  if arbol == None:
    return 
  else:
    print(arbol.carga)
    imprimirarbol(arbol.izquierdo)
    imprimirarbol(arbol.derecho)

def imprimirarbolPostorden(arbol):
  if arbol == None:
    return
  else:
    imprimirarbolPostorden(arbol.izquierdo)
    imprimirarbolPostorden(arbol.derecho)
    print(arbol.carga)   

def imprimirabolEnOrden(arbol):
  if arbol == None:
    return
  imprimirabolEnOrden(arbol.izquierdo)
  print(arbol.carga,imprimirabolEnOrden(arbol.derecho))

def imprimirarbolSangrado(arbol, nivel=0):
  if arbol == None:
    return
  imprimirarbolSangrado(arbol.derecho, nivel+1)
  print(" "*nivel + str(arbol.carga))
  imprimirarbolSangrado(arbol.izquierdo, nivel+1)
\end{pythoncode}

\section{Árboles de expresiones}

\begin{pythoncode}
def obtenerLexema(listaLexemas, esperado):
  if listaLexemas[0] == esperado:
      del listaLexemas[0]
      return 1
  else:
      return 0

def obtenerNumero(listaLexemas):
  x = listaLexemas[0]
  if type(x) != type(0):
      return None
  del listaLexemas[0]
  return arbol (x, None, None)


def obtenerProducto(listaLexemas):
  a = obtenerNumero(listaLexemas)
  if obtenerLexema(listaLexemas, "*"):
    b = obtenerProducto(listaLexemas) 
    return arbol ("*", a, b)
  else:
    return a

def obtenerNumero(listaLexemas):
  if obtenerLexema(listaLexemas, "("):
    # obtiene la subexpresión
    x = obtenerSuma(listaLexemas) 
    # elimina los paréntesis
    obtenerLexema(listaLexemas, ")") 
    return x
  else:
    x = listaLexemas[0]
    if type(x) != type(0):
      return None
    listaLexemas[0:1] = []
   return Arbol (x, None, None)  
\end{pythoncode}

\section{Adivinar el animal}

\begin{pythoncode}
def animal():
  # Un solo nodo
  raiz = Arbol("pajaro")
      # Hasta que el usuario salga
  while True:
    print()
    if not si("Esta pensando en un animal? "):
    break
    # Recorrer el arbol
    arbol = raiz
    while arbol.obtenerizquierdo() != None:
      pregunta = arbol.obtenercarga() + "? "
      if si(pregunta):
      arbol = arbol.obtenerderecho()
      else:
      arbol = arbol.obtenerizquierdo()
    # conjetura
    conjetura = arbol.obtenercarga()
    pregunta = "¿Es un" + conjetura + "? "
    if si(pregunta):
      print("¡Soy el mejor!")
      continue
    # obtener mas informacion
    pregunta = "¿Cual es el nombre el animal? "
    animal = input(pregunta)
    pregunta = "¿Que pregunta permitiria distinguir 
        un %s de un %s? "
    q = input(pregunta % (animal,conjetura))
    # agrega un nuevo nodo arbol
    arbol.asignarcarga(q)
    pregunta = "¿Si el animal fuera %s 
        la respuesta sería? "
    if si(pregunta % animal):
      arbol.asignarizquierdo(Arbol(conjetura))
      arbol.asignarderecho(Arbol(animal))
    else:
      arbol.asignarizquierdo(Arbol(animal))
      arbol.asignarderecho(Arbol(conjetura))

def si(preg):
  from string import lower
  r = lower(input(preg))
  return (r[0] == 's')
\end{pythoncode}

\section{Clase \texttt{Fraccion}}

\texttt{}
\begin{pythoncode}
class Fraccion:
  def __init__(self,numerador,denominador=1):
    self.numerador = numerador
    self.denominador = denominador

  def __str__(self):
    return "%d/%d" % (self.numerador, self.denominador)
  
  def __mul__(self, otro):
    if type(otro) == type(5):
      otro = Fraccion(otro)
    return Fraccion(self.numerador * otro.numerador,
            self.denominador * otro.denominador)
  
  __rmul__ = __mul__ 

  #suma de fracciones
  def __add__(self, otro):
    if type(otro) == type(5):
      otro = Fraccion(otro)
    return Fraccion(self.numerador * otro.denominador +
            self.denominador * otro.numerador,
            self.denominador * otro.denominador)
  
  __radd__ = __add__
  
  def __init__(self, numerador, denominador=1):      
    g=MDC(numerador,denominador)
    self.numerador=numerador / g
    self.denominador=denominador / g
  
  def __cmp__(self, otro):
    dif = (self.numerador * otro.denominador -
       otro.numerador * self.denominador)
    return dif

def MDC (m,n):
      if m%n==0:
      return n
      else:
      return MDC(n,m%n)
\end{pythoncode}


\clearemptydoublepage % programas completos

\chapter{Lecturas adicionales recomendadas}

¿Así que, hacia adonde ir desde aquí? Hay muchas direcciones para
avanzar, extender su conocimiento de Python específicamente y sobre
la ciencia de la computación en general.

Los ejemplos en este libro han sido deliberadamente sencillos, por
esto no han mostrado las capacidades más excitantes de Python. Aquí
hay una pequeña muestra de las extensiones de Python y sugerencias
de proyectos que las utilizan.
\begin{itemize}
\item La programación de interfaces gráficas de usuario, tiene muchos mas
elementos de los que vimos en el último capítulo.
\item La programación en la Web integra a Python con Internet. Por ejemplo,
usted puede construir programas cliente que abran y lean páginas remotas
(casi) tan fácilmente como abren un archivo en disco. También hay
módulos en Python que permiten acceder a archivos remotos por medio
de ftp, y otros que posibilitan enviar y recibir correo electrónico.
Python también se usa extensivamente para desarrollar servidores web
que presten servicios.
\item Las bases de datos son como superarchivos, donde la información se
almacena en esquemas predefinidos y las relaciones entre los datos
permiten navegar por ellos de diferentes formas. Python tiene varios
módulos que facilitan conectar programas a varios motores de bases
de datos, de código abierto y comerciales.
\item La programación con hilos permite ejecutar diferentes hilos de ejecución
dentro de un mismo programa. Si usted ha tenido la experiencia de
desplazarse al inicio de una página web mientras el navegador continúa
cargando el resto de ella, entonces tiene una noción de lo que los
hilos pueden lograr.
\item Cuando la preocupación es la velocidad, se pueden escribir extensiones
a Python en un lenguaje compilado como C o C++. Estas extensiones
forman la base de muchos módulos en la biblioteca de Python. El mecanismo
para enlazar funciones y datos es algo complejo. La herramienta SWIG
(Simplified Wrapper and Interface Generator) simplifica mucho estas
tareas. 
\end{itemize}

\section{Libros y sitios web relacionados con Python}

Aquí están las recomendaciones de los autores sobre sitios web:
\begin{itemize}
\item La página web de \texttt{www.python.org} es el lugar para empezar
cualquier búsqueda de material relacionado con Python. Encontrará
ayuda, documentación, enlaces a otros sitios, SIG (Special Interest
Groups), y listas de correo a las que se puede unir.
\item EL proyecto Open Book \texttt{www.ibiblio.com/obp} no sólo contiene
este libro en línea, también los libros similares para Java y C++
de Allen Downey. Además, está {\em Lessons in Electric Circuits}
de Tony R. Kuphaldt, {\em Getting down with ...} un conjunto de
tutoriales (que cubren varios tópicos en ciencias de la computación)
que han sido escritos y editados por estudiantes universitarios; {\em
Python for Fun}, un conjunto de casos de estudio en Python escrito
por Chris Meyers, y {\em The Linux Cookbook} de Michael Stultz,
con 300 páginas de consejos y sugerencias.
\item Finalmente si usted Googlea la cadena ``python -snake -monty'' obtendrá
unos $337000000$ resultados.
\end{itemize}
%\adjustpage{-1}%\pagebreak

Aquí hay algunos libros que contienen más material sobre el lenguaje
Python:
\begin{itemize}
\item {\em Core Python Programming}, de Wesley Chun, es un gran libro
de 750 páginas, aproximadamente. La primera parte cubre las características
básicas. La segunda introduce adecuadamente muchos tópicos más avanzados,
incluyendo muchos de los que mencionamos anteriormente.
\item {\em Python Essential Reference}, de David M. Beazley, es un pequeño
libro, pero contiene mucha información sobre el lenguaje y la biblioteca
estándar. También provee un excelente índice.
\item {\em Python Pocket Reference}, de Mark Lutz, realmente cabe en
su bolsillo. Aunque no es tan comprensivo como {\em Python Essential
Reference}; es una buena referencia para las funciones y módulos
más usados. Mark Lutz también es el autor de {\em Programming Python},
uno de los primeros (y más grandes) libros sobre Python que no está
dirigido a novatos. Su libro posterior {\em Learning Python} es
más pequeño y accesible.
\item {\em Python Programming on Win32}, de Mark Hammond y Andy Robinson,
se ``debe tener'' si pretende construir aplicaciones para el sistema
operativo Windows. Entre otras cosas cubre la integración entre Python
y COM, construye una pequeña aplicación con wxPython, e incluso realiza
guiones que agregan funcionalidad a aplicaciones como Word y Excel.
\end{itemize}

\section{Libros generales de ciencias de la computación recomendados}

Las siguientes sugerencias de lectura incluyen muchos de los libros
favoritos de los autores. Tratan sobre buenas prácticas de programación
y las ciencias de la computación en general.
\begin{itemize}
\item {\em The Practice of Programming} de Kernighan y Pike no sólo cubre
el diseño y la codificación de algoritmos y estructuras de datos,
sino que también trata la depuración, las pruebas y la optimización
de los programas. La mayoría de los ejemplos está escrita en C++ y
Java, no hay ninguno en Python.
\item {\em The Elements of Java Style}, editado por Al Vermeulen, es
otro libro pequeño que discute algunos de los puntos mas sutiles de
la buena programación, como el uso de buenas convenciones para los
nombres, comentarios, incluso el uso de los espacios en blanco y la
indentación (algo que no es problema en Python). El libro también
cubre la programación por contrato que usa aserciones para atrapar
errores mediante el chequeo de pre y postcondiciones, y la programación
multihilo.
\item {\em Programming Pearls}, de Jon Bentley, es un libro clásico.
Comprende varios casos de estudio que aparecieron originalmente en
la columna del autor en las {\em Communications of the ACM}. Los
estudios examinan los compromisos que hay que tomar cuando se programa
y por qué tan a menudo es una mala idea apresurarse con la primera
idea que se tiene para desarrollar un programa. Este libro es uno
poco más viejo que los otros (1986), así que los ejemplos están escritos
en lenguajes más viejos. Hay muchos problemas para resolver, algunos
traen pistas y otros su solución. Este libro fue muy popular, incluso
hay un segundo volumen.
\item {\em The New Turing Omnibus}, de A.K Dewdney, hace una amable introducción
a 66 tópicos en ciencias de la computación, que van desde la computación
paralela hasta los virus de computador, desde escanografías hasta
algoritmos genéticos. Todos son cortos e interesantes. Un libro anterior
de Dewdney {\em The Armchair Universe} es una colección de artículos
de su columna {\em Computer Recreations} en la revista {\em Scientific
American (Investigación y Ciencia)}, estos libros son una rica fuente
de ideas para emprender proyectos.
\item {\em Turtles, Termites and Traffic Jams}, de Mitchel Resnick, trata
sobre el poder de la descentralización y cómo el comportamiento complejo
puede emerger de la simple actividad coordinada de una multitud de
agentes. Introduce el lenguaje StarLogo que permite escribir programas
multiagentes. Los programas examinados demuestran el comportamiento
complejo agregado, que a menudo es contraintuitivo. Muchos de estos
programas fueron escritos por estudiantes de colegio y universidad.
Programas similares pueden escribirse en Python usando hilos y gráficos
simples.
\item {\em Gödel, Escher y Bach}, de Douglas Hofstadter. Simplemente,
si usted ha encontrado magia en la recursión, también la encontrará
en éste best seller. Uno de los temas que trata Hofstadter es el de
los ``ciclos extraños'', en los que los patrones evolucionan y ascienden
hasta que se encuentran a sí mismos otra vez. La tesis de Hofstadter
es que esos ``ciclos extraños'' son una parte esencial de lo que
separa lo animado de lo inanimado. Él muestra patrones como éstos
en la música de Bach, los cuadros de Escher y el teorema de incompletitud
de Gödel.
\end{itemize}


\clearemptydoublepage % lecturas recomendadas

\chapter{Licencia de documentación libre de GNU}

Versión 1.2, Noviembre 2002 \\

This is an unofficial translation of the GNU Free Documentation License
into Spanish. It was not published by the Free Software Foundation,
and does not legally state the distribution terms for documentation
that uses the GNU FDL – only the original English text of the GNU
FDL does that. However, we hope that this translation will help Spanish
speakers understand the GNU FDL better.

Esta es una traducción no oficial de la GNU Free Document License
a Español (Castellano). No ha sido publicada por la Free Software
Foundation y no establece legalmente los términos de distribución
para trabajos que usen la GFDL (sólo el texto de la versión original
en Inglés de la GFDL lo hace). Sin embargo, esperamos que esta traducción
ayude a los hispanohablantes a entender mejor la GFDL. La versión
original de la GFDL está disponible en la Free Software Foundation{[}1{]}.

Esta traducción está basada en una la versión 1.1 de Igor Támara y
Pablo Reyes. Sin embargo la responsabilidad de su interpretación es
de Joaquín Seoane.

Copyright (C) 2000, 2001, 2002 Free Software Foundation, Inc. 59 Temple
Place, Suite 330, Boston, MA 02111-1307 USA. Se permite la copia y
distribución de copias literales de este documento de licencia, pero
no se permiten cambios{[}1{]}.

%\rule{\linewidth}{1pt}

\section*{Preámbulo}

El propósito de esta licencia es permitir que un manual, libro de
texto, u otro documento escrito sea libre en el sentido de libertad:
asegurar a todo el mundo la libertad efectiva de copiarlo y redistribuirlo,
con o sin modificaciones, de manera comercial o no. En segundo término,
esta licencia proporciona al autor y al editor{[}2{]} una manera de
obtener reconocimiento por su trabajo, sin que se le considere responsable
de las modificaciones realizadas por otros.

Esta licencia es de tipo copyleft, lo que significa que los trabajos
derivados del documento deben a su vez ser libres en el mismo sentido.
Complementa la Licencia Pública General de GNU, que es una licencia
tipo copyleft diseñada para el software libre.

Hemos diseñado esta licencia para usarla en manuales de software libre,
ya que el software libre necesita documentación libre: un programa
libre debe venir con manuales que ofrezcan la mismas libertades que
el software. Pero esta licencia no se limita a manuales de software;
puede usarse para cualquier texto, sin tener en cuenta su temática
o si se publica como libro impreso o no. Recomendamos esta licencia
principalmente para trabajos cuyo fin sea instructivo o de referencia.

\section{Aplicabilidad y definiciones}

Esta licencia se aplica a cualquier manual u otro trabajo, en cualquier
soporte, que contenga una nota del propietario de los derechos de
autor que indique que puede ser distribuido bajo los términos de esta
licencia. Tal nota garantiza en cualquier lugar del mundo, sin pago
de derechos y sin límite de tiempo, el uso de dicho trabajo según
las condiciones aquí estipuladas. En adelante la palabra Documento
se referirá a cualquiera de dichos manuales o trabajos. Cualquier
persona es un licenciatario y será referido como Usted. Usted acepta
la licencia si copia. modifica o distribuye el trabajo de cualquier
modo que requiera permiso según la ley de propiedad intelectual.

Una Versión Modificada del Documento significa cualquier trabajo que
contenga el Documento o una porción del mismo, ya sea una copia literal
o con modificaciones y/o traducciones a otro idioma.

Una Sección Secundaria es un apéndice con título o una sección preliminar
del Documento que trata exclusivamente de la relación entre los autores
o editores y el tema general del Documento (o temas relacionados)
pero que no contiene nada que entre directamente en dicho tema general
(por ejemplo, si el Documento es en parte un texto de matemáticas,
una Sección Secundaria puede no explicar nada de matemáticas). La
relación puede ser una conexión histórica con el tema o temas relacionados,
o una opinión legal, comercial, filosófica, ética o política acerca
de ellos.

Las Secciones Invariantes son ciertas Secciones Secundarias cuyos
títulos son designados como Secciones Invariantes en la nota que indica
que el documento es liberado bajo esta Licencia. Si una sección no
entra en la definición de Secundaria, no puede designarse como Invariante.
El documento puede no tener Secciones Invariantes. Si el Documento
no identifica las Secciones Invariantes, es que no las tiene.

Los Textos de Cubierta son ciertos pasajes cortos de texto que se
listan como Textos de Cubierta Delantera o Textos de Cubierta Trasera
en la nota que indica que el documento es liberado bajo esta Licencia.
Un Texto de Cubierta Delantera puede tener como mucho 5 palabras,
y uno de Cubierta Trasera puede tener hasta 25 palabras.

Una copia Transparente del Documento, significa una copia para lectura
en máquina, representada en un formato cuya especificación está disponible
al público en general, apto para que los contenidos puedan ser vistos
y editados directamente con editores de texto genéricos o (para imágenes
compuestas por puntos) con programas genéricos de manipulación de
imágenes o (para dibujos) con algún editor de dibujos ampliamente
disponible, y que sea adecuado como entrada para formateadores de
texto o para su traducción automática a formatos adecuados para formateadores
de texto. Una copia hecha en un formato definido como Transparente,
pero cuyo marcaje o ausencia de él haya sido diseñado para impedir
o dificultar modificaciones posteriores por parte de los lectores
no es Transparente. Un formato de imagen no es Transparente si se
usa para una cantidad de texto sustancial. Una copia que no es Transparente
se denomina Opaca.

Como ejemplos de formatos adecuados para copias Transparentes están
ASCII puro sin marcaje, formato de entrada de Texinfo, formato de
entrada de LaTeX, SGML o XML usando una DTD disponible públicamente,
y HTML, PostScript o PDF simples, que sigan los estándares y diseñados
para que los modifiquen personas. Ejemplos de formatos de imagen transparentes
son PNG, XCF y JPG. Los formatos Opacos incluyen formatos propietarios
que pueden ser leídos y editados únicamente en procesadores de palabras
propietarios, SGML o XML para los cuáles las DTD y/o herramientas
de procesamiento no estén ampliamente disponibles, y HTML, PostScript
o PDF generados por algunos procesadores de palabras sólo como salida.

La Portada significa, en un libro impreso, la página de título, más
las páginas siguientes que sean necesarias para mantener legiblemente
el material que esta Licencia requiere en la portada. Para trabajos
en formatos que no tienen página de portada como tal, Portada significa
el texto cercano a la aparición más prominente del título del trabajo,
precediendo el comienzo del cuerpo del texto.

Una sección Titulada XYZ significa una parte del Documento cuyo título
es precisamente XYZ o contiene XYZ entre paréntesis, a continuación
de texto que traduce XYZ a otro idioma (aquí XYZ se refiere a nombres
de sección específicos mencionados más abajo, como Agradecimientos,
Dedicatorias , Aprobaciones o Historia. Conservar el Título de tal
sección cuando se modifica el Documento significa que permanece una
sección Titulada XYZ según esta definición .

El Documento puede incluir Limitaciones de Garantía cercanas a la
nota donde se declara que al Documento se le aplica esta Licencia.
Se considera que estas Limitaciones de Garantía están incluidas, por
referencia, en la Licencia, pero sólo en cuanto a limitaciones de
garantía: cualquier otra implicación que estas Limitaciones de Garantía
puedan tener es nula y no tiene efecto en el significado de esta Licencia.

%\rule{\linewidth}{1pt}

\section{Copia literal}

Usted puede copiar y distribuir el Documento en cualquier soporte,
sea en forma comercial o no, siempre y cuando esta Licencia, las notas
de copyright y la nota que indica que esta Licencia se aplica al Documento
se reproduzcan en todas las copias y que usted no añada ninguna otra
condición a las expuestas en esta Licencia. Usted no puede usar medidas
técnicas para obstruir o controlar la lectura o copia posterior de
las copias que usted haga o distribuya. Sin embargo, usted puede aceptar
compensación a cambio de las copias. Si distribuye un número suficientemente
grande de copias también deberá seguir las condiciones de la sección
3.

Usted también puede prestar copias, bajo las mismas condiciones establecidas
anteriormente, y puede exhibir copias públicamente.

\section{Copiado en cantidad}

Si publica copias impresas del Documento (o copias en soportes que
tengan normalmente cubiertas impresas) que sobrepasen las 100, y la
nota de licencia del Documento exige Textos de Cubierta, debe incluir
las copias con cubiertas que lleven en forma clara y legible todos
esos Textos de Cubierta: Textos de Cubierta Delantera en la cubierta
delantera y Textos de Cubierta Trasera en la cubierta trasera. Ambas
cubiertas deben identificarlo a Usted clara y legiblemente como editor
de tales copias. La cubierta debe mostrar el título completo con todas
las palabras igualmente prominentes y visibles. Además puede añadir
otro material en las cubiertas. Las copias con cambios limitados a
las cubiertas, siempre que conserven el título del Documento y satisfagan
estas condiciones, pueden considerarse como copias literales.

Si los textos requeridos para la cubierta son muy voluminosos para
que ajusten legiblemente, debe colocar los primeros (tantos como sea
razonable colocar) en la verdadera cubierta y situar el resto en páginas
adyacentes.

Si Usted publica o distribuye copias Opacas del Documento cuya cantidad
exceda las 100, debe incluir una copia Transparente, que pueda ser
leída por una máquina, con cada copia Opaca, o bien mostrar, en cada
copia Opaca, una dirección de red donde cualquier usuario de la misma
tenga acceso por medio de protocolos públicos y estandarizados a una
copia Transparente del Documento completa, sin material adicional.
Si usted hace uso de la última opción, deberá tomar las medidas necesarias,
cuando comience la distribución de las copias Opacas en cantidad,
para asegurar que esta copia Transparente permanecerá accesible en
el sitio establecido por lo menos un año después de la última vez
que distribuya una copia Opaca de esa edición al público (directamente
o a través de sus agentes o distribuidores).

Se solicita, aunque no es requisito, que se ponga en contacto con
los autores del Documento antes de redistribuir gran número de copias,
para darles la oportunidad de que le proporcionen una versión actualizada
del Documento.

%\rule{\linewidth}{1pt}

\section{Modificaciones}

Puede copiar y distribuir una Versión Modificada del Documento bajo
las condiciones de las secciones 2 y 3 anteriores, siempre que usted
libere la Versión Modificada bajo esta misma Licencia, con la Versión
Modificada haciendo el rol del Documento, por lo tanto dando licencia
de distribución y modificación de la Versión Modificada a quienquiera
posea una copia de la misma. Además, debe hacer lo siguiente en la
Versión Modificada:
\begin{itemize}
\item Usar en la Portada (y en las cubiertas, si hay alguna) un título distinto
al del Documento y de sus versiones anteriores (que deberían, si hay
alguna, estar listadas en la sección de Historia del Documento). Puede
usar el mismo título de versiones anteriores al original siempre y
cuando quien las publicó originalmente otorgue permiso.
\item Listar en la Portada, como autores, una o más personas o entidades
responsables de la autoría de las modificaciones de la Versión Modificada,
junto con por lo menos cinco de los autores principales del Documento
(todos sus autores principales, si hay menos de cinco), a menos que
le eximan de tal requisito.
\item Mostrar en la Portada como editor el nombre del editor de la Versión
Modificada.
\item Conservar todas las notas de copyright del Documento.
\item Añadir una nota de copyright apropiada a sus modificaciones, adyacente
a las otras notas de copyright.
\item Incluir, inmediatamente después de las notas de copyright, una nota
de licencia dando el permiso para usar la Versión Modificada bajo
los términos de esta Licencia, como se muestra en la Adenda al final
de este documento.
\item Conservar en esa nota de licencia el listado completo de las Secciones
Invariantes y de los Textos de Cubierta que sean requeridos en la
nota de Licencia del Documento original.
\item Incluir una copia sin modificación de esta Licencia.
\item Conservar la sección Titulada Historia, conservar su Título y añadirle
un elemento que declare al menos el título, el año, los nuevos autores
y el editor de la Versión Modificada, tal como figuran en la Portada.
Si no hay una sección Titulada Historia en el Documento, crear una
estableciendo el título, el año, los autores y el editor del Documento,
tal como figuran en su Portada, añadiendo además un elemento describiendo
la Versión Modificada, como se estableció en la oración anterior.
\item Conservar la dirección en red, si la hay, dada en el Documento para
el acceso público a una copia Transparente del mismo, así como las
otras direcciones de red dadas en el Documento para versiones anteriores
en las que estuviese basado. Pueden ubicarse en la sección Historia.
Se puede omitir la ubicación en red de un trabajo que haya sido publicado
por lo menos cuatro años antes que el Documento mismo, o si el editor
original de dicha versión da permiso.
\item En cualquier sección Titulada Agradecimientos o Dedicatorias, Conservar
el Título de la sección y conservar en ella toda la sustancia y el
tono de los agradecimientos y/o dedicatorias incluidas por cada contribuyente.
\item Conservar todas las Secciones Invariantes del Documento, sin alterar
su texto ni sus títulos. Números de sección o el equivalente no son
considerados parte de los títulos de la sección.
\item Borrar cualquier sección titulada Aprobaciones. Tales secciones no
pueden estar incluidas en las Versiones Modificadas.
\item No cambiar el título de ninguna sección existente a Aprobaciones ni
a uno que entre en conflicto con el de alguna Sección Invariante.
\item Conservar todas las Limitaciones de Garantía.
\end{itemize}
Si la Versión Modificada incluye secciones o apéndices nuevos que
califiquen como Secciones Secundarias y contienen material no copiado
del Documento, puede opcionalmente designar algunas o todas esas secciones
como invariantes. Para hacerlo, añada sus títulos a la lista de Secciones
Invariantes en la nota de licencia de la Versión Modificada. Tales
títulos deben ser distintos de cualquier otro título de sección.

Puede añadir una sección titulada Aprobaciones, siempre que contenga
únicamente aprobaciones de su Versión Modificada por otras fuentes
–por ejemplo, observaciones de peritos o que el texto ha sido aprobado
por una organización como la definición oficial de un estándar.

Puede añadir un pasaje de hasta cinco palabras como Texto de Cubierta
Delantera y un pasaje de hasta 25 palabras como Texto de Cubierta
Trasera en la Versión Modificada. Una entidad solo puede añadir (o
hacer que se añada) un pasaje al Texto de Cubierta Delantera y uno
al de Cubierta Trasera. Si el Documento ya incluye un textos de cubiertas
añadidos previamente por usted o por la misma entidad que usted representa,
usted no puede añadir otro; pero puede reemplazar el anterior, con
permiso explícito del editor que agregó el texto anterior.

Con esta Licencia ni los autores ni los editores del Documento dan
permiso para usar sus nombres para publicidad ni para asegurar o implicar
aprobación de cualquier Versión Modificada.

\section{Combinación de documentos}

Usted puede combinar el Documento con otros documentos liberados bajo
esta Licencia, bajo los términos definidos en la sección 4 anterior
para versiones modificadas, siempre que incluya en la combinación
todas las Secciones Invariantes de todos los documentos originales,
sin modificar, listadas todas como Secciones Invariantes del trabajo
combinado en su nota de licencia. Así mismo debe incluir la Limitación
de Garantía.

El trabajo combinado necesita contener solamente una copia de esta
Licencia, y puede reemplazar varias Secciones Invariantes idénticas
por una sola copia. Si hay varias Secciones Invariantes con el mismo
nombre pero con contenidos diferentes, haga el título de cada una
de estas secciones único añadiéndole al final del mismo, entre paréntesis,
el nombre del autor o editor original de esa sección, si es conocido,
o si no, un número único. Haga el mismo ajuste a los títulos de sección
en la lista de Secciones Invariantes de la nota de licencia del trabajo
combinado.

En la combinación, debe combinar cualquier sección Titulada Historia
de los documentos originales, formando una sección Titulada Historia;
de la misma forma combine cualquier sección Titulada Agradecimientos,
y cualquier sección Titulada Dedicatorias. Debe borrar todas las secciones
tituladas Aprobaciones.

\section{Colecciones de documentos}

Puede hacer una colección que conste del Documento y de otros documentos
liberados bajo esta Licencia, y reemplazar las copias individuales
de esta Licencia en todos los documentos por una sola copia que esté
incluida en la colección, siempre que siga las reglas de esta Licencia
para cada copia literal de cada uno de los documentos en cualquiera
de los demás aspectos.

Puede extraer un solo documento de una de tales colecciones y distribuirlo
individualmente bajo esta Licencia, siempre que inserte una copia
de esta Licencia en el documento extraído, y siga esta Licencia en
todos los demás aspectos relativos a la copia literal de dicho documento.

\section{Agregación con trabajos independientes}

Una recopilación que conste del Documento o sus derivados y de otros
documentos o trabajos separados e independientes, en cualquier soporte
de almacenamiento o distribución, se denomina un agregado si el copyright
resultante de la compilación no se usa para limitar los derechos de
los usuarios de la misma más allá de lo que los de los trabajos individuales
permiten. Cuando el Documento se incluye en un agregado, esta Licencia
no se aplica a otros trabajos del agregado que no sean en sí mismos
derivados del Documento.

Si el requisito de la sección 3 sobre el Texto de Cubierta es aplicable
a estas copias del Documento y el Documento es menor que la mitad
del agregado entero, los Textos de Cubierta del Documento pueden colocarse
en cubiertas que enmarquen solamente el Documento dentro del agregado,
o el equivalente electrónico de las cubiertas si el documento está
en forma electrónica. En caso contrario deben aparecer en cubiertas
impresas enmarcando todo el agregado.

\section{Traducción}

La Traducción es considerada como un tipo de modificación, por lo
que usted puede distribuir traducciones del Documento bajo los términos
de la sección 4. El reemplazo de las Secciones Invariantes con traducciones
requiere permiso especial de los dueños de derecho de autor, pero
usted puede añadir traducciones de algunas o todas las Secciones Invariantes
a las versiones originales de las mismas. Puede incluir una traducción
de esta Licencia, de todas las notas de licencia del documento, así
como de las Limitaciones de Garantía, siempre que incluya también
la versión en Inglés de esta Licencia y las versiones originales de
las notas de licencia y Limitaciones de Garantía. En caso de desacuerdo
entre la traducción y la versión original en Inglés de esta Licencia,
la nota de licencia o la limitación de garantía, la versión original
en Inglés prevalecerá.

Si una sección del Documento está Titulada Agradecimientos, Dedicatorias
o Historia el requisito (sección 4) de Conservar su Título (Sección
1) requerirá, típicamente, cambiar su título.

\section{Terminación}

Usted no puede copiar, modificar, sublicenciar o distribuir el Documento
salvo por lo permitido expresamente por esta Licencia. Cualquier otro
intento de copia, modificación, sublicenciamiento o distribución del
Documento es nulo, y dará por terminados automáticamente sus derechos
bajo esa Licencia. Sin embargo, los terceros que hayan recibido copias,
o derechos, de usted bajo esta Licencia no verán terminadas sus licencias,
siempre que permanezcan en total conformidad con ella.

\section{Revisiones futuras de esta licencia}

De vez en cuando la Free Software Foundation puede publicar versiones
nuevas y revisadas de la Licencia de Documentación Libre GNU. Tales
versiones nuevas serán similares en espíritu a la presente versión,
pero pueden diferir en detalles para solucionar nuevos problemas o
intereses. Vea \url{http://www.gnu.org/copyleft/}.

Cada versión de la Licencia tiene un número de versión que la distingue.
Si el Documento especifica que se aplica una versión numerada en particular
de esta licencia o cualquier versión posterior, usted tiene la opción
de seguir los términos y condiciones de la versión especificada o
cualquiera posterior que haya sido publicada (no como borrador) por
la Free Software Foundation. Si el Documento no especifica un número
de versión de esta Licencia, puede escoger cualquier versión que haya
sido publicada (no como borrador) por la Free Software Foundation.

\section{ADENDA: Cómo usar esta Licencia en sus documentos}

Para usar esta licencia en un documento que usted haya escrito, incluya
una copia de la Licencia en el documento y ponga el siguiente copyright
y nota de licencia justo después de la página de título:
\begin{quote}
Copyright (c) AÑO SU NOMBRE. Se concede permiso para copiar, distribuir
y/o modificar este documento bajo los términos de la Licencia de Documentación
Libre de GNU, Versión 1.2 o cualquier otra versión posterior publicada
por la Free Software Foundation; sin Secciones Invariantes ni Textos
de Cubierta Delantera ni Textos de Cubierta Trasera. Una copia de
la licencia está incluida en la sección titulada GNU Free Documentation
License. 
\end{quote}
Si tiene Secciones Invariantes, Textos de Cubierta Delantera y Textos
de Cubierta Trasera, reemplace la frase sin ... Trasera por esto:
\begin{quote}
siendo las Secciones Invariantes LISTE SUS TÍTULOS, siendo los Textos
de Cubierta Delantera LISTAR, y siendo sus Textos de Cubierta Trasera
LISTAR. 
\end{quote}
Si tiene Secciones Invariantes sin Textos de Cubierta o cualquier
otra combinación de los tres, mezcle ambas alternativas para adaptarse
a la situación.

Si su documento contiene ejemplos de código de programa no triviales,
recomendamos liberar estos ejemplos en paralelo bajo la licencia de
software libre que usted elija, como la Licencia Pública General de
GNU (GNU General Public License), para permitir su uso en software
libre.

\rule{1\linewidth}{1pt}

Notas

{[}1{]} Ésta es la traducción del Copyright de la Licencia, no es
el Copyright de esta traducción no autorizada.

{[}2{]} La licencia original dice publisher, que es, estrictamente,
quien publica, diferente de editor, que es más bien quien prepara
un texto para publicar. En castellano editor se usa para ambas cosas.

{[}3{]} En sentido estricto esta licencia parece exigir que los títulos
sean exactamente Acknowledgements, Dedications, Endorsements e History,
en inglés.

\clearemptydoublepage
% GNU free doc license (spanish)
% fdl.tex
%This file is a chapter.  It must be included in a larger 
%document to work properly.

\chapter{GNU Free Documentation License}

Version 1.1, March 2000\\

Copyright $\copyright$ 2000 Free Software Foundation, Inc.\\
 59 Temple Place, Suite 330, Boston, MA 02111-1307 USA\\
 Everyone is permitted to copy and distribute verbatim copies of this
license document, but changing it is not allowed.

\section*{Preamble}

The purpose of this License is to make a manual, textbook, or other
written document ``free'' in the sense of freedom: to assure everyone
the effective freedom to copy and redistribute it, with or without
modifying it, either commercially or noncommercially. Secondarily,
this License preserves for the author and publisher a way to get credit
for their work, while not being considered responsible for modifications
made by others.

This License is a kind of ``copyleft,'' which means that derivative
works of the document must themselves be free in the same sense. It
complements the GNU General Public License, which is a copyleft license
designed for free software.

We have designed this License in order to use it for manuals for free
software, because free software needs free documentation: a free program
should come with manuals providing the same freedoms that the software
does. But this License is not limited to software manuals; it can
be used for any textual work, regardless of subject matter or whether
it is published as a printed book. We recommend this License principally
for works whose purpose is instruction or reference.

\section{Applicability and Definitions}

This License applies to any manual or other work that contains a notice
placed by the copyright holder saying it can be distributed under
the terms of this License. The ``Document,'' below, refers to any
such manual or work. Any member of the public is a licensee, and is
addressed as ``you.''

A ``Modified Version'' of the Document means any work containing
the Document or a portion of it, either copied verbatim, or with modifications
and/or translated into another language.

A ``Secondary Section'' is a named appendix or a front-matter section
of the Document that deals exclusively with the relationship of the
publishers or authors of the Document to the Document's overall subject
(or to related matters) and contains nothing that could fall directly
within that overall subject. (For example, if the Document is in part
a textbook of mathematics, a Secondary Section may not explain any
mathematics.) The relationship could be a matter of historical connection
with the subject or with related matters, or of legal, commercial,
philosophical, ethical, or political position regarding them.

The ``Invariant Sections'' are certain Secondary Sections whose
titles are designated, as being those of Invariant Sections, in the
notice that says that the Document is released under this License.

The ``Cover Texts'' are certain short passages of text that are
listed, as Front-Cover Texts or Back-Cover Texts, in the notice that
says that the Document is released under this License.

A ``Transparent'' copy of the Document means a machine-readable
copy, represented in a format whose specification is available to
the general public, whose contents can be viewed and edited directly
and straightforwardly with generic text editors or (for images composed
of pixels) generic paint programs or (for drawings) some widely available
drawing editor, and that is suitable for input to text formatters
or for automatic translation to a variety of formats suitable for
input to text formatters. A copy made in an otherwise Transparent
file format whose markup has been designed to thwart or discourage
subsequent modification by readers is not Transparent. A copy that
is not ``Transparent'' is called ``Opaque.''

Examples of suitable formats for Transparent copies include plain
ASCII without markup, Texinfo input format, \LaTeX{}~input format,
SGML or XML using a publicly available DTD, and standard-conforming
simple HTML designed for human modification. Opaque formats include
PostScript, PDF, proprietary formats that can be read and edited only
by proprietary word processors, SGML or XML for which the DTD and/or
processing tools are not generally available, and the machine-generated
HTML produced by some word processors for output purposes only.

The ``Title Page'' means, for a printed book, the title page itself,
plus such following pages as are needed to hold, legibly, the material
this License requires to appear in the title page. For works in formats
which do not have any title page as such, ``Title Page'' means the
text near the most prominent appearance of the work's title, preceding
the beginning of the body of the text.

\section{Verbatim Copying}

You may copy and distribute the Document in any medium, either commercially
or noncommercially, provided that this License, the copyright notices,
and the license notice saying this License applies to the Document
are reproduced in all copies, and that you add no other conditions
whatsoever to those of this License. You may not use technical measures
to obstruct or control the reading or further copying of the copies
you make or distribute. However, you may accept compensation in exchange
for copies. If you distribute a large enough number of copies you
must also follow the conditions in Section 3.

You may also lend copies, under the same conditions stated above,
and you may publicly display copies.

\section{Copying in Quantity}

If you publish printed copies of the Document numbering more than
100, and the Document's license notice requires Cover Texts, you must
enclose the copies in covers that carry, clearly and legibly, all
these Cover Texts: Front-Cover Texts on the front cover, and Back-Cover
Texts on the back cover. Both covers must also clearly and legibly
identify you as the publisher of these copies. The front cover must
present the full title with all words of the title equally prominent
and visible. You may add other material on the covers in addition.
Copying with changes limited to the covers, as long as they preserve
the title of the Document and satisfy these conditions, can be treated
as verbatim copying in other respects.

If the required texts for either cover are too voluminous to fit legibly,
you should put the first ones listed (as many as fit reasonably) on
the actual cover, and continue the rest onto adjacent pages.

If you publish or distribute Opaque copies of the Document numbering
more than 100, you must either include a machine-readable Transparent
copy along with each Opaque copy, or state in or with each Opaque
copy a publicly accessible computer-network location containing a
complete Transparent copy of the Document, free of added material,
which the general network-using public has access to download anonymously
at no charge using public-standard network protocols. If you use the
latter option, you must take reasonably prudent steps, when you begin
distribution of Opaque copies in quantity, to ensure that this Transparent
copy will remain thus accessible at the stated location until at least
one year after the last time you distribute an Opaque copy (directly
or through your agents or retailers) of that edition to the public.

It is requested, but not required, that you contact the authors of
the Document well before redistributing any large number of copies,
to give them a chance to provide you with an updated version of the
Document.

\section{Modifications}

You may copy and distribute a Modified Version of the Document under
the conditions of Sections 2 and 3 above, provided that you release
the Modified Version under precisely this License, with the Modified
Version filling the role of the Document, thus licensing distribution
and modification of the Modified Version to whoever possesses a copy
of it. In addition, you must do these things in the Modified Version:
\begin{itemize}
\item Use in the Title Page (and on the covers, if any) a title distinct
from that of the Document, and from those of previous versions (which
should, if there were any, be listed in the History section of the
Document). You may use the same title as a previous version if the
original publisher of that version gives permission. 
\item List on the Title Page, as authors, one or more persons or entities
responsible for authorship of the modifications in the Modified Version,
together with at least five of the principal authors of the Document
(all of its principal authors, if it has less than five). 
\item State on the Title page the name of the publisher of the Modified
Version, as the publisher. 
\item Preserve all the copyright notices of the Document. 
\item Add an appropriate copyright notice for your modifications adjacent
to the other copyright notices. 
\item Include, immediately after the copyright notices, a license notice
giving the public permission to use the Modified Version under the
terms of this License, in the form shown in the Addendum below. 
\item Preserve in that license notice the full lists of Invariant Sections
and required Cover Texts given in the Document's license notice. 
\item Include an unaltered copy of this License. 
\item Preserve the section entitled ``History,'' and its title, and add
to it an item stating at least the title, year, new authors, and publisher
of the Modified Version as given on the Title Page. If there is no
section entitled ``History'' in the Document, create one stating
the title, year, authors, and publisher of the Document as given on
its Title Page, then add an item describing the Modified Version as
stated in the previous sentence. 
\item Preserve the network location, if any, given in the Document for public
access to a Transparent copy of the Document, and likewise the network
locations given in the Document for previous versions it was based
on. These may be placed in the ``History'' section. You may omit
a network location for a work that was published at least four years
before the Document itself, or if the original publisher of the version
it refers to gives permission. 
\item In any section entitled ``Acknowledgements'' or ``Dedications,''
preserve the section's title, and preserve in the section all the
substance and tone of each of the contributor acknowledgements and/or
dedications given therein. 
\item Preserve all the Invariant Sections of the Document, unaltered in
their text and in their titles. Section numbers or the equivalent
are not considered part of the section titles. 
\item Delete any section entitled ``Endorsements.'' Such a section may
not be included in the Modified Version. 
\item Do not retitle any existing section as ``Endorsements'' or to conflict
in title with any Invariant Section.
\end{itemize}
If the Modified Version includes new front-matter sections or appendices
that qualify as Secondary Sections and contain no material copied
from the Document, you may at your option designate some or all of
these sections as invariant. To do this, add their titles to the list
of Invariant Sections in the Modified Version's license notice. These
titles must be distinct from any other section titles.

You may add a section entitled ``Endorsements,'' provided it contains
nothing but endorsements of your Modified Version by various parties—for
example, statements of peer review or that the text has been approved
by an organization as the authoritative definition of a standard.

You may add a passage of up to five words as a Front-Cover Text, and
a passage of up to 25 words as a Back-Cover Text, to the end of the
list of Cover Texts in the Modified Version. Only one passage of Front-Cover
Text and one of Back-Cover Text may be added by (or through arrangements
made by) any one entity. If the Document already includes a cover
text for the same cover, previously added by you or by arrangement
made by the same entity you are acting on behalf of, you may not add
another; but you may replace the old one, on explicit permission from
the previous publisher that added the old one.

The author(s) and publisher(s) of the Document do not by this License
give permission to use their names for publicity for or to assert
or imply endorsement of any Modified Version.

\section{Combining Documents}

You may combine the Document with other documents released under this
License, under the terms defined in Section 4 above for modified versions,
provided that you include in the combination all of the Invariant
Sections of all of the original documents, unmodified, and list them
all as Invariant Sections of your combined work in its license notice.

The combined work need only contain one copy of this License, and
multiple identical Invariant Sections may be replaced with a single
copy. If there are multiple Invariant Sections with the same name
but different contents, make the title of each such section unique
by adding at the end of it, in parentheses, the name of the original
author or publisher of that section if known, or else a unique number.
Make the same adjustment to the section titles in the list of Invariant
Sections in the license notice of the combined work.

In the combination, you must combine any sections entitled ``History''
in the various original documents, forming one section entitled ``History'';
likewise combine any sections entitled ``Acknowledgements,'' and
any sections entitled ``Dedications.'' You must delete all sections
entitled ``Endorsements.''

\section{Collections of Documents}

You may make a collection consisting of the Document and other documents
released under this License, and replace the individual copies of
this License in the various documents with a single copy that is included
in the collection, provided that you follow the rules of this License
for verbatim copying of each of the documents in all other respects.

You may extract a single document from such a collection, and distribute
it individually under this License, provided you insert a copy of
this License into the extracted document, and follow this License
in all other respects regarding verbatim copying of that document.

\section{Aggregation with Independent Works}

A compilation of the Document or its derivatives with other separate
and independent documents or works, in or on a volume of a storage
or distribution medium, does not as a whole count as a Modified Version
of the Document, provided no compilation copyright is claimed for
the compilation. Such a compilation is called an ``aggregate,''
and this License does not apply to the other self-contained works
thus compiled with the Document, on account of their being thus compiled,
if they are not themselves derivative works of the Document.

If the Cover Text requirement of Section 3 is applicable to these
copies of the Document, then if the Document is less than one quarter
of the entire aggregate, the Document's Cover Texts may be placed
on covers that surround only the Document within the aggregate. Otherwise
they must appear on covers around the whole aggregate.

\section{Translation}

Translation is considered a kind of modification, so you may distribute
translations of the Document under the terms of Section 4. Replacing
Invariant Sections with translations requires special permission from
their copyright holders, but you may include translations of some
or all Invariant Sections in addition to the original versions of
these Invariant Sections. You may include a translation of this License
provided that you also include the original English version of this
License. In case of a disagreement between the translation and the
original English version of this License, the original English version
will prevail.

\section{Termination}

You may not copy, modify, sublicense, or distribute the Document except
as expressly provided for under this License. Any other attempt to
copy, modify, sublicense, or distribute the Document is void, and
will automatically terminate your rights under this License. However,
parties who have received copies, or rights, from you under this License
will not have their licenses terminated so long as such parties remain
in full compliance.

\section{Future Revisions of This License}

The Free Software Foundation may publish new, revised versions of
the GNU Free Documentation License from time to time. Such new versions
will be similar in spirit to the present version, but may differ in
detail to address new problems or concerns. See \url{http:///www.gnu.org/copyleft/}.

Each version of the License is given a distinguishing version number.
If the Document specifies that a particular numbered version of this
License ``or any later version'' applies to it, you have the option
of following the terms and conditions either of that specified version
or of any later version that has been published (not as a draft) by
the Free Software Foundation. If the Document does not specify a version
number of this License, you may choose any version ever published
(not as a draft) by the Free Software Foundation.

\section{Addendum: How to Use This License for Your Documents}

To use this License in a document you have written, include a copy
of the License in the document and put the following copyright and
license notices just after the title page:
\begin{quote}
Copyright $\copyright$ YEAR YOUR NAME. Permission is granted to copy,
distribute and/or modify this document under the terms of the GNU
Free Documentation License, Version 1.1 or any later version published
by the Free Software Foundation; with the Invariant Sections being
LIST THEIR TITLES, with the Front-Cover Texts being LIST, and with
the Back-Cover Texts being LIST. A copy of the license is included
in the section entitled ``GNU Free Documentation License.''
\end{quote}
If you have no Invariant Sections, write ``with no Invariant Sections''
instead of saying which ones are invariant. If you have no Front-Cover
Texts, write ``no Front-Cover Texts'' instead of ``Front-Cover
Texts being LIST''; likewise for Back-Cover Texts.

If your document contains nontrivial examples of program code, we
recommend releasing these examples in parallel under your choice of
free software license, such as the GNU General Public License, to
permit their use in free software.

\clearemptydoublepage
% GNU free doc license

\printindex{}\clearemptydoublepage \clearemptydoublepage 
\end{document}
