% !TEX TS-program = xelatex

% LaTeX source for the spanish traslation of the textbook ``How to think like a 
% computer scientist''
% Copyright (c)  2001,2002  Allen B. Downey.
% Traslation completed by
% Andrés Becerra Sandoval
% abecerra@cic.puj.edu.co


% Permission is granted to copy, distribute and/or modify this
% document under the terms of the GNU Free Documentation License,
% Version 1.1  or any later version published by the Free Software
% Foundation; with the Invariant Sections being "Contributor List",
% with no Front-Cover Texts, and with no Back-Cover Texts. A copy of
% the license is included in the section entitled "GNU Free
% Documentation License".

% This distribution includes a file named fdl.tex that contains the text
% of the GNU Free Documentation License.  If it is missing, you can obtain
% it from www.gnu.org or by writing to the Free Software Foundation,
% Inc., 59 Temple Place - Suite 330, Boston, MA 02111-1307, USA.
%

\documentclass[a4paper, 11pt]{book}
%\usepackage[spanish]{babel}   % Not needed using polyglossia in lecv.tex
\usepackage{fancyhdr}
%\usepackage{graphicx}         % Not needed, already loaded in lecv.tex
\usepackage{makeidx}
\usepackage{ucs}
%\usepackage[utf8x]{inputenc}  % Not needed with XeLaTeX

\label{MY_PACKAGES}
\usepackage{fontspec}
	\setmainfont{Myriad Pro}

\usepackage{polyglossia}           % Change this depending on your language
	\setdefaultlanguage{spanish}
	\setotherlanguage{english}

\usepackage{cancel}                % To cross out over text.

%\usepackage{datetime}
%	\newdateformat{serialdate}{\THEYEAR\twodigit{\THEMONTH}\twodigit{\THEDAY}}
%	
%\newtimeformat{serialtime}{\twodigit{\THEHOUR}\twodigit{\THEMINUTE}\twodigit{\THESECOND}}

%\usepackage{showframe}            % Show margins (deactivate xcolor package)
\usepackage[
	usenames,
	dvipsnames,
	svgnames,
	table]{xcolor}                 % Incluimos el uso de colores.

\usepackage{setspace}              % Nos permite controlar el interlineado.
\usepackage{multirow}              % Para poder juntar celdas en las tablas.
%\usepackage{lipsum}

%\usepackage{background}
%\backgroundsetup{
%	scale=1,
%	angle=0,
%	opacity=1,  %% adjust
%	contents={
%		\ifnum \value{page} > 1	
%			\includegraphics[width=\paperwidth,height=\paperheight]{./img/ARCHIVO1.png}
%		\else
%			\includegraphics[width=\paperwidth,height=\paperheight]{./img/ARCHIVO2.png}
%		\fi
%	}
%}

\usepackage{graphicx}

%\label{PAGE_LAYOUT}
%% https://en.wikibooks.org/wiki/LaTeX/Page_Layout#Page_dimensions
%\setlength\oddsidemargin{-0.04cm}
%\setlength\evensidemargin{-0.04cm}
%\setlength\topmargin{0cm}
%\setlength\headheight{0.5cm}
%\setlength\headsep{0.5cm}
%\setlength\footskip{1cm}
%\setlength\textwidth{16cm}                      % ancho para apunte
%\setlength\textheight{23cm}                     % largo para apunte
%\usepackage{geometry}
%\geometry{
%	a4paper,
%	total={210mm,297mm},
%	left=30mm,
%	right=30mm,
%	top=40mm,
%	bottom=30mm,
%}

\label{HYPERREF}
\usepackage{hyperref}              % Incluimos la posibilidad de agregar 
%hipervínculos.
	\hypersetup{                   % Opciones del paquete de hipervínculos.
%	    bookmarks=true,            % show bookmarks bar?
	    unicode=true,              % non-Latin characters in Acrobat’s bookmarks
	    pdftoolbar=true,           % show Acrobat’s toolbar?
	    pdfmenubar=true,           % show Acrobat’s menu?
	    pdffitwindow=false,        % window fit to page when opened
	    pdfstartview={FitH},       % fits the width of the page to the window
	    pdftitle={Aprende a pensar como un programador con Python},         % 
	    %title
	    pdfauthor={Allen Downey, Jeffrey Elkner, Chris Meyers | Andrés Becerra 
	    Sandoval | Leandro E. Colombo Viña},  % author
	    pdfsubject={Adaptación de la traducción de la 2° Edición},  % subject 
	    %of the document
%	    pdfcreator={\LaTeX},       % creator of the document
%	    pdfproducer={},            % producer of the document
	    pdfkeywords={} {},         % list of keywords
	    pdfnewwindow=true,         % links in new window
	    colorlinks=true,           % false: boxed links; true: colored links
	    linkcolor=red,             % color of internal links (change box color 
%with
	                               % linkbordercolor)
	    citecolor=green,           % color of links to bibliography
	    filecolor=magenta,         % color of file links
	    urlcolor=cyan              % color of external links
}

\label{MY_COMMANDS}
\newcommand{\HRule}{\rule{\linewidth}{0.5mm}} % Nueva línea horizontal.

\label{MY_COLORS}
\definecolor{gray15p}{rgb}{0.85,0.85,0.85}
\definecolor{gray05p}{rgb}{0.95,0.95,0.95}

\label{LISTINGS}
\usepackage{listings}                        % Permite mostrar código de forma 
%mas linda
	\lstset{
		backgroundcolor=\color{gray05p},     % choose the background color;
		basicstyle=\ttfamily,                % the size of the fonts that are 
%used for the code
		breakatwhitespace=false,             % sets if automatic breaks should 
%only happen at
		                                     % whitespace
		breaklines=true,                     % sets automatic line breaking
		captionpos=b,                        % sets the caption-position to 
%bottom
		commentstyle=\color{green},          % comment style
		deletekeywords={},                   % if you want to delete keywords 
%from the given
		                                     % language
		escapeinside={\%*}{*)},              % if you want to add LaTeX within 
%your code
		extendedchars=true,                  % lets you use non-ASCII 
%characters; for 8-bits
		                                     % encodings only, does not work 
%with UTF-8
		frame=single,                        % adds a frame around the code
		frameround=tttt,
		keepspaces=true,                     % keeps spaces in text, useful for 
%keeping
		                                     % indentation of code (possibly 
%needs 
		                                     % columns=flexible)
		keywordstyle=\color{blue}\textbf,    % keyword style
		language=Python,                     % the language of the code
		morecomment=[l]{\#}                  % Agrega el "comentario" 
		classoffset=1,
		morekeywords={},                     % if you want to add more keywords 
%to the set
		keywordstyle=\color{red}\textbf,
		classoffset=0,
		numbers=left,                        % where to put the line-numbers; 
%possible values
		                                     % are (none, left, right)
		numbersep=5pt,                       % how far the line-numbers are 
%from the code
		numberstyle=\tiny\color{gray},       % the style that is used for the 
%line-numbers
		rulecolor=\color{black},             % if not set, the frame-color may 
%be changed on
		                                     % line-breaks within not-black 
%text (e.g. comments
		                                     % (green here))
		showspaces=false,                    % show spaces everywhere adding 
%particular
		                                     % underscores; it overrides 
%'showstringspaces'
		showstringspaces=false,              % underline spaces within strings 
%only
		showtabs=false,                      % show tabs within strings adding 
%particular
		                                     % underscores
		stepnumber=1,                        % the step between two 
%line-numbers. If it's 1, 
		                                     % each line will be numbered
		stringstyle=\color{magenta},         % string literal style
		tabsize=2,                           % sets default tabsize to 2 spaces
		title=\lstname,                      % show the filename of files 
%included with
		                                     % \lstinputlisting; also try 
%caption instead 
		                                     % of title
	}
	
\usepackage[final]{pdfpages} % Include PDF directly into your document.
%                                    This will higlight source code like Code::Blocks IDE.
%                                    Install this theme copying codeblocks.py to
%                                    "/usr/local/lib/python2.7/dist-packages/pygments/styles/"

\usepackage{upquote}               % Show "realistic" quotes in verbatim.
                                   % To use single straight quote in minted with pygmentize 1.6
\usepackage{minted}                % Usamos minted para el resaltado de sintaxis de los códigos.
\usemintedstyle{codeblocks}        % Elegimos el perfil de "codeblocks".

\label{PYTHON_CODE}
% Con newminted{python} definimos un nuevo ambiente:
%	\begin{pythoncode}
%#!/usr/bin/env python
%# coding: utf-8
%	
%print("Hello World!")
%	\end{pythoncode}

\newminted{python}{
	linenos=true,
	numbersep=5pt,                 % Setea la separación de los números de línea en 5pt. 
	frame=single,
	framesep=2mm,                  % Setea la separación del frame a 2mm.
%	gobble=2,                      % Borra los 2 espacios adicionales agregados.
%	mathescape,                    % Permite insertar secuencias de escape matemáticas de LaTeX.
	rulecolor=\color{gray15p},     % color de la línea del recuadro.
	framerule=0.4mm,               % ancho de línea de recuadro
	bgcolor=gray05p,               % color de fondo del recuadro.
	fontsize=\small,
}

\label{PYTHON_FILE}
% Con \newmintedfile[pythonfile]{python} definimos un nuevo comando:
%\pythonfile{./src/test.py}

\newmintedfile[pythonfile]{python}{
	linenos=true,
	numbersep=5pt,                 % Setea la separación de los números de línea en 5pt. 
	frame=single,
	framesep=2mm,                  % Setea la separación del frame a 2mm.
%	gobble=2,                      % Borra los 2 espacios adicionales agregados.
%	mathescape,                    % Permite insertar secuencias de escape matemáticas de LaTeX.
	rulecolor=\color{gray15p},     % color de la línea del recuadro.
	framerule=0.4mm,               % ancho de línea de recuadro
	bgcolor=gray05p,               % color de fondo del recuadro.
	fontsize=\small,
}

\label{C_CODE}
% Con \newminted{c} definimos un nuevo ambiente:
%	\begin{ccode}
%#include <stdio.h>
%
%int main (int argc, char** argv)
%{
%    printf("Hola Mundo!\n");
%
%    return 0;
%}
%	\end{ccode}
	
\newminted{c}{
	linenos=true,
	numbersep=5pt,                 % Setea la separación de los números de línea en 5pt. 
	frame=single,
	framesep=2mm,                  % Setea la separación del frame a 2mm.
%	gobble=2,                      % Borra los 2 espacios adicionales agregados.
%	mathescape,                    % Permite insertar secuencias de escape matemáticas de LaTeX.
	rulecolor=\color{gray15p},     % color de la línea del recuadro.
	framerule=0.4mm,               % ancho de línea de recuadro
	bgcolor=gray05p,               % color de fondo del recuadro.
	fontsize=\small,
}

\label{C_FILE}
%Con \newmintedfile[cfile]{c} definimos un nuevo comando:
%\cfile{./src/test.c}

\newmintedfile[cfile]{c}{
	linenos=true,
	numbersep=5pt,                 % Setea la separación de los números de línea en 5pt. 
	frame=single,
	framesep=2mm,                  % Setea la separación del frame a 2mm.
%	gobble=2,                      % Borra los 2 espacios adicionales agregados.
%	mathescape,                    % Permite insertar secuencias de escape matemáticas de LaTeX.
	rulecolor=\color{gray15p},     % color de la línea del recuadro.
	framerule=0.4mm,               % ancho de línea de recuadro
	bgcolor=gray05p,               % color de fondo del recuadro.
	fontsize=\small,
}

\label{PYTHON_CONSOLE}
% Con newminted{pycon} definimos un nuevo ambiente:
%	\begin{pyconcode}
%>>> print("Hello World!")
%Hello World!
%>>>
%	\end{pyconcode}

\newminted{pycon}{
	linenos=true,
	numbersep=5pt,                 % Setea la separación de los números de línea en 5pt. 
	frame=single,
	framesep=2mm,                  % Setea la separación del frame a 2mm.
%	gobble=2,                      % Borra los 2 espacios adicionales agregados.
%	mathescape,                    % Permite insertar secuencias de escape matemáticas de LaTeX.
	rulecolor=\color{gray15p},     % color de la línea del recuadro.
	framerule=0.4mm,               % ancho de línea de recuadro
	bgcolor=gray05p,               % color de fondo del recuadro.
	fontsize=\small,
}

\label{PYTHON_CONSOLE_FILE}
% Con \newmintedfile[pyconfile]{python} definimos un nuevo comando:
%\pyconfile{./src/test.py}

\newmintedfile[pyconfile]{pycon}{
	linenos=true,
	numbersep=5pt,                 % Setea la separación de los números de línea en 5pt. 
	frame=single,
	framesep=2mm,                  % Setea la separación del frame a 2mm.
%	gobble=2,                      % Borra los 2 espacios adicionales agregados.
%	mathescape,                    % Permite insertar secuencias de escape matemáticas de LaTeX.
	rulecolor=\color{gray15p},     % color de la línea del recuadro.
	framerule=0.4mm,               % ancho de línea de recuadro
	bgcolor=gray05p,               % color de fondo del recuadro.
	fontsize=\small,
}

\renewcommand\listingscaption{Code}
\addto\captionsspanish{
	\renewcommand{\listingscaption}{Código}
}

%\label{MY_VARS}
%\def\AutorPrefijo{Prof.}
%\def\AutorNombre{Leandro E.}
%\def\AutorApellido{Colombo Viña}
%\def\AutorNombreCompleto{\AutorPrefijo\ \AutorNombre\ \AutorApellido}
%\def\AutorURL{http://leo.bitson.com.ar/}
%\def\AutorNombreLink{\href{\AutorURL}{\AutorNombreCompleto}}
%\def\AutorCargo{Profesor en Disciplinas Industriales - Técnico Superior en 
%Informática Aplicada}
%
%\def\Title{Aprenda a pensar como un programador con Python}
%\def\PDFSubject{Adaptación de la traducción de la 2da edición}
%
%\def\Fecha{13 de febrero de 2016}
%\def\Lugar{Ciudad Autónoma de Buenos Aires}
%\pssilent

%%%%%%%%%%%%%%%%%%%%%%%%%%%%%%%%%%%%%%
%         INCLUDE ONLY 
%\includeonly{chap20}
%FIX
%\setcounter{chapter}{20}

%%%%%%%%%%%%%%%%%%%%%%%%%%%%%%%%%%%%%%

\sloppy
\setlength{\topmargin}{-0.375in}
\setlength{\oddsidemargin}{0.0in}
\setlength{\evensidemargin}{0.0in}

% Uncomment these to center on 8.5 x 11
\setlength{\topmargin}{0.625in}
\setlength{\oddsidemargin}{0.875in}
\setlength{\evensidemargin}{0.875in}

\setlength{\headsep}{3ex}
\setlength{\textheight}{7.2in}

\setlength{\parindent}{0.0in}
\setlength{\parskip}{1.7ex plus 0.5ex minus 0.5ex}
\renewcommand{\baselinestretch}{1.02}

% see LaTeX Companion page 62
\setlength{\topsep}{-0.0\parskip}
\setlength{\partopsep}{-0.5\parskip}
\setlength{\itemindent}{0.0in}
\setlength{\listparindent}{0.0in}

% see LaTeX Companion page 26
% these are copied from /usr/local/teTeX/share/texmf/tex/latex/base/book.cls
% all I changed is afterskip

\makeatletter
\renewcommand{\section}{\@startsection 
    {section} {1} {0mm}%
    {-3.5ex \@plus -1ex \@minus -.2ex}%
    {0.7ex \@plus.2ex}%
    {\normalfont\Large\bfseries}}
\renewcommand\subsection{\@startsection {subsection}{2}{0mm}%
    {-3.25ex\@plus -1ex \@minus -.2ex}%
    {0.3ex \@plus .2ex}%
    {\normalfont\large\bfseries}}
\renewcommand\subsubsection{\@startsection {subsubsection}{3}{0mm}%
    {-3.25ex\@plus -1ex \@minus -.2ex}%
    {0.3ex \@plus .2ex}%
    {\normalfont\normalsize\bfseries}}
\makeatother

\newcommand{\beforefig}{\vspace{1.3\parskip}}
\newcommand{\afterfig}{\vspace{-0.2\parskip}}

\newcommand{\beforeverb}{\vspace{0.6\parskip}}
\newcommand{\afterverb}{\vspace{0.6\parskip}}

\newcommand{\adjustpage}[1]{\enlargethispage{#1\baselineskip}}
\newcommand{\clearemptydoublepage}{\newpage{\pagestyle{empty}\cleardoublepage}}
\newcommand{\blankpage}{\pagestyle{empty}\vspace*{1in}\newpage}

\pagestyle{fancyplain}

\renewcommand{\chaptermark}[1]{\markboth{#1}{}}
\renewcommand{\sectionmark}[1]{\markright{\thesection\ #1}{}}

\lhead[\fancyplain{}{\bfseries\thepage}]%
      {\fancyplain{}{\bfseries\rightmark}}
\rhead[\fancyplain{}{\bfseries\leftmark}]%
      {\fancyplain{}{\bfseries\thepage}}
\cfoot{}

% turn off the rule under the header
%\setlength{\headrulewidth}{0pt}

% the following is a brute-force way to prevent the headers
% from getting transformed into all-caps
\renewcommand\MakeUppercase{}
\makeindex

\begin{document}

\frontmatter

%-half title--------------------------------------------------
\thispagestyle{empty}

\begin{flushright}
\vspace*{2.5in}
{\huge Como Pensar como un Científico de la Computación}
\vspace{0.25in}
{\LARGE con Python}
\vfill
\end{flushright}

%--verso------------------------------------------------------
\clearemptydoublepage
%\pagebreak
%\thispagestyle{empty}
%\vspace*{6in}

%--title page--------------------------------------------------
\pagebreak
\thispagestyle{empty}

\begin{flushright}
\vspace*{2.5in}
{\huge Como Pensar como un Científico de la Computación}
\vspace{0.25in}
{\LARGE con Python}
\vspace{0.5in}
\author{Allen Downey, Jeffrey Elkner y Chris Meyers}

{\small
Allen Downey\\
Jeffrey Elkner\\
Chris Meyers\\
}

\vspace{.25in}

{\small
Traducido y adaptado por\\
Andrés Becerra Sandoval \\
}

\vspace{0.5in}
{\Large Pontificia Universidad Javeriana} \\
{\small Santiago de Cali, Colombia}
%\includegraphics{illustrations/logo1.eps,width=1in}
\vfill
\end{flushright}


%--copyright--------------------------------------------------
\pagebreak
\thispagestyle{empty}
Copyright \copyright ~2002 Allen Downey, Jeffrey Elkner, y Chris Meyers.
\vspace{0.25in}
%Corregido por Shannon Turlington y Lisa Cutler.
%Diseño de la cubierta por Rebecca Gimenez.

\vspace{0.25in}

\begin{flushleft}
Pontificia Universidad Javeriana \\
Calle 18 No. 118-250 \\
A.A. No.  26239 \\
Cali, Colombia \\
%Green Tea Press       \\
%1 Grove St.           \\
%P.O. Box 812901       \\
%Wellesley, MA 02482   \\
\end{flushleft}

\vspace{0.25in}

Se concede permiso para copiar, distribuir, y/o modificar este documento bajo
los terminos de la GNU Free Documentation License, Versión 1.1 o cualquier
versión posterior publicada por la Free Software Foundation; manteniendo 
sin variaciones las secciones ``Prólogo,'' ``Prefacio,'' y ``Lista de contribuidores,''
sin texto de cubierta, y sin texto de contracubierta. Una copia de la licencia
está incluida en el apéndice titulado ``GNU Free Documentation License'' y una 
traducción de ésta al español en el apéndice titulado Licencia de Documentación Libre de GNU

La GNU Free Documentation License está disponible a través de \texttt{www.gnu.org}
o escribiendo a la Free Software Foundation, Inc., 59 Temple Place,
Suite 330, Boston, MA 02111-1307, USA.

La forma original de este libro es código fuente \LaTeX\  y compilarlo
tiene el efecto de generar un libro de texto en una 
repesentacion independiente del dispositivo que puede ser convertida a otros 
formatos e imprimirse.

El código  fuente \LaTeX\  para este libro y mas información sobre este proyecto
se encuentra en:

\begin{verbatim}
      http://www.thinkpython.com
\end{verbatim}

Este libro ha sido preparado utilizando \LaTeX\ y las figuras
se han realizado con xfig.  Todos estos son programas de código
abierto, gratuito.

\vspace{0.25in}

Historia de la impresión:

%\begin{description}
%\item[Abril 2002:] Primera edición.
%\end{description}

%ISBN 0-9716775-0-6

\begin{verbatim}

\end{verbatim}


%-----------------------------------------------------------------

\chapter{Prólogo}

Por David Beazley

Como educador, investigador y autor de libros, estoy encantado de
ver la terminación de este texto. Python es un lenguaje de programación
divertido y extremadamente fácil de usar que ha ganado renombre constantemente
en los años recientes. Desarrollado hace diez años por Guido van Rossum,
la sintaxis simple de Python y su ``sabor'' se derivan, en gran
parte del ABC, un lenguaje de programación para enseñanza que se desarrolló
en los 1980s. Sin embargo, Python también fue creado para resolver
problemas reales y tiene una amplia gama de características que se
encuentran en lenguajes de programación como C++, Java, Modula-3,
y Scheme. Debido a esto, uno de las características notables de Python
es la atracción que ejerce sobre programadores profesionales, científicos,
investigadores, artistas y educadores.

A pesar de ésta atracción que ejerce en muchas comunidades diversas,
usted puede todavía preguntarse ``¿porque Python?'' o ``¿porque
enseñar programación con Python?'' Responder éstas preguntas no es
una tarea fácil— especialmente cuando la opinión popular está del
lado masoquista de usar alternativas como C++ y Java. Sin embargo,
pienso que la respuesta más directa es que la programación en Python
es simplemente más divertida y más productiva.

Cuando enseño cursos de informática, yo quiero cubrir conceptos importantes,
hacer el material interesante y enganchar a los estudiantes. Desafortunadamente,
hay una tendencia en la que los cursos de programación introductorios
dedican demasiada atención hacia la abstracción matemática y a hacer
que los estudiantes se frustren con problemas molestos relacionados
con la sintaxis, la compilación y la presencia de reglas arcanas en
los lenguajes. Aunque la abstracción y el formalismo son importantes
para los ingenieros de software y para los estudiantes de ciencias
de la computación, usar este enfoque hace la informática muy aburrida.
Cuando enseño un curso no quiero tener un grupo de estudiantes sin
inspiración. Quisiera verlos intentando resolver problemas interesantes,
explorando ideas diferentes, intentando enfoques no convencionales,
rompiendo reglas y aprendiendo de sus errores. En el proceso no quiero
perder la mitad del semestre tratando de resolver problemas sintácticos
oscuros, interpretando mensajes de error del compilador incomprensibles,
o descifrando cuál de las muchas maneras en que un programa puede
generar un error grave de memoria se está presentando.

Una de las razones del por qué me gusta Python es que proporciona
un equilibrio muy bueno entre lo práctico y lo conceptual. Puesto
que se interpreta Python, los principiantes pueden empezar a hacer
cosas interesantes casi de inmediato sin perderse en problemas de
compilación y enlace. Además, Python viene con una biblioteca grande
de módulos, que pueden ser usados en dominios que van desde programación
en la web hasta aplicaciones gráficas. Tener un foco práctico es una
gran manera de enganchar a los estudiantes y permite que emprendan
proyectos significativos. Sin embargo, Python también puede servir
como una excelente base para introducir conceptos importantes de la
informática. Puesto que Python soporta completamente procedimientos
y clases, los estudiantes pueden ser introducidos gradualmente a temas
como la abstracción procedimental, las estructuras de datos y la programación
orientada a objetos—lo que se puede aplicar después a cursos posteriores
en Java o C++. Python proporciona, incluso, varias características
de los lenguajes de programación funcionales y puede usarse para introducir
conceptos que se pueden explorar con más detalle en cursos con Scheme
y Lisp.

Leyendo, el prefacio de Jeffrey, estoy sorprendido por sus comentarios
de que Python le permita ver un ``más alto nivel de éxito y un nivel
bajo de frustración'' y que puede ``avanzar mas rápido con mejores
resultados.'' Aunque estos comentarios se refieren a sus cursos introductorios,
a veces uso Python por estas mismas razones en los cursos de informática
avanzada en la Universidad de Chicago. En estos cursos enfrento constantemente
la tarea desalentadora de cubrir un montón de material difícil durante
nueve semanas. Aunque es totalmente posible para mi infligir mucho
dolor y sufrimiento usando un lenguaje como C++, he encontrado a menudo
que este enfoque es improductivo—especialmente cuando el curso se
trata de un asunto sin relación directa con la ``programación.''
He encontrado que usar Python me permite enfocar el tema del curso
y dejar a los estudiantes desarrollar proyectos substanciales.

Aunque Python siga siendo un lenguaje joven y en desarrollo, creo
que tiene un futuro brillante en la educación. Este libro es un paso
importante en esa dirección.

\vspace{0.25in}
 
\begin{flushleft}
David Beazley \\
Universidad de Chicago, Autor de {\em Python Essential Reference} 
\par
\end{flushleft}
 	\clearemptydoublepage

\chapter{Prefacio}

Por Jeff Elkner

Este libro debe su existencia a la colaboración hecha posible por
Internet y el movimiento de software libre. Sus tres autores—un profesor
de colegio, un profesor de secundaria y un programador profesional—tienen
todavía que verse cara a cara, pero han podido trabajar juntos y han
sido ayudados por maravillosas personas, quienes han donado su tiempo
y energía para ayudar a hacer ver mejor este libro.

Nosotros pensamos que este libro es un testamento a los beneficios
y futuras posibilidades de esta clase de colaboración, el marco que
se ha puesto en marcha por Richard Stallman y el movimiento de software
libre.

\section*{Cómo y porqué vine a utilizar Python}

En 1999, el examen del College Board's Advanced Placement (AP) de
Informática se hizo en C++ por primera vez. Como en muchas escuelas
de Estados Unidos, la decisión para cambiar el lenguaje tenía un impacto
directo en el plan de estudios de informática en la escuela secundaria
de Yorktown en Arlington, Virginia, donde yo enseño. Hasta este punto,
Pascal era el lenguaje de instrucción en nuestros cursos del primer
año y del AP. Conservando la práctica usual de dar a los estudiantes
dos años de exposición al mismo lenguaje, tomamos la decisión de cambiar
a C++ en el curso del primer año durante el periodo escolar 1997-98
de modo que siguiéramos el cambio del College Board's para el curso
del AP el año siguiente.

Dos años después, estoy convencido de que C++ no era una buena opción
para introducir la informática a los estudiantes. Aunque es un lenguaje
de programación de gran alcance, también es extremadamente difícil
de aprender y de enseñar. Me encontré constantemente peleando con
la sintaxis difícil de C++ y sus múltiples maneras de hacer las cosas,
y estaba perdiendo muchos estudiantes, innecesariamente, como resultado.
Convencido de que tenía que haber una mejor opción para nuestras clases
de primer año, fui en busca de una alternativa a C++.

Necesitaba un lenguaje que pudiera correr en las máquinas en nuestro
laboratorio Linux, también en las plataformas de Windows y Macintosh,
que muchos de los estudiantes tienen en casa. Quería que fuese un
lenguaje de código abierto, para que los estudiantes lo pudieran usar
en casa sin pagar por una licencia. Quería un lenguaje usado por programadores
profesionales, y que tuviera una comunidad activa alrededor de él.
Tenía que soportar la programación procedimental y orientada a objetos.
Y más importante, tenía que ser fácil de aprender y de enseñar. Cuando
investigué las opciones con estas metas en mente, Python saltó como
el mejor candidato para la tarea.

Pedí a uno de los estudiantes más talentosos de Yorktown, Matt Ahrens,
que le diera a Python una oportunidad. En dos meses él no sólo aprendió
el lenguaje, sino que escribió una aplicación llamada pyTicket que
permitió a nuestro personal atender peticiones de soporte tecnológico
vía web. Sabia que Matt no podría terminar una aplicación de esa escala
en tan poco tiempo con C++, y esta observación, combinada con el gravamen
positivo de Matt sobre Python, sugirió que este lenguaje era la solución
que buscaba.

\section*{Encontrando un libro de texto}

Al decidir utilizar Python en mis dos clases de informática introductoria
para el año siguiente, el problema más acuciante era la carencia de
un libro.

El contenido libre vino al rescate. A principios del año, Richard
Stallman me presentó a Allen Downey. Los dos habíamos escrito a Richard
expresando interés en desarrollar un contenido gratis y educativo.
Allen ya había escrito un libro de texto para el primer año de informática,
{\em Como pensar como un científico de la computación}. Cuando
leí este libro, inmediatamente quise usarlo en mi clase. Era el texto
más claro y mas provechoso de introducción a la informática que había
visto. Acentúa los procesos del pensamiento implicados en la programación
más bien que las características de un lenguaje particular. Leerlo
me hizo sentir un mejor profesor inmediatamente. {\em Como pensar
como un científico de la computación con Java} no solo es un libro
excelente, sino que también había sido publicado bajo la licencia
publica GNU, lo que significa que podría ser utilizado libremente
y ser modificado para resolver otras necesidades. Una vez que decidí
utilizar Python, se me ocurrió que podía traducir la versión original
del libro de Allen (en Java) al nuevo lenguaje (Python). Aunque no
podía escribir un libro de texto solo, tener el libro de Allen me
facilitó la tarea, y al mismo tiempo demostró que el modelo cooperativo
usado en el desarrollo de software también podía funcionar para el
contenido educativo.

Trabajar en este libro, por los dos últimos años, ha sido una recompensa
para mis estudiantes y para mí; y mis estudiantes tuvieron una gran
participación en el proceso. Puesto que podía realizar cambios inmediatos,
siempre que alguien encontrara un error de deletreo o un paso difícil,
yo les animaba a que buscaran errores en el libro, dándoles un punto
cada vez que hicieran una sugerencia que resultara en un cambio en
el texto. Eso tenía la ventaja doble de animarles a que leyeran el
texto más cuidadosamente y de conseguir la corrección del texto por
sus lectores críticos más importantes, los estudiantes usándolo para
aprender informática.

Para la segunda parte del libro, enfocada en la programación orientada
a objetos, sabía que alguien con más experiencia en programación que
yo era necesario para hacer el trabajo correctamente. El libro estuvo
incompleto la mayoría del año hasta que la comunidad de software abierto
me proporcionó de nuevo los medios necesarios para su terminación.

Recibí un correo electrónico de Chris Meyers, expresando su interés
en el libro. Chris es un programador profesional que empezó enseñando
un curso de programación el año anterior, usando Python en el Lane
Community College en Eugene, Oregon. La perspectiva de enseñar el
curso llevó a Chris al libro, y él comenzó a ayudarme inmediatamente.
Antes del fin de año escolar él había creado un proyecto complementario
en nuestro Sitio Web \url{http://www.ibiblio.org/obp}, titulado {\em
Python for Fun} y estaba trabajando con algunos de mis estudiantes
más avanzados como profesor principal, guiándolos mas allá de donde
yo podía llevarlos.

\section*{Introduciendo la programación con Python}

El proceso de uso y traducción de {\em Como pensar como un científico
de la computación}, por los últimos dos años, ha confirmado la conveniencia
de Python para enseñar a estudiantes principiantes. Python simplifica
bastante los ejemplos de programación y hace que las ideas importantes
sean más fáciles de enseñar.

El primer ejemplo del texto ilustra este punto. Es el tradicional
``hola, mundo'', programa que en la versión C++ del libro se ve
así:
\begin{verbatim}
   #include <iostream.h>

   void main()
   {
     cout << "Hola, mundo." << endl;
   }
\end{verbatim}
en la versión Python es:
\begin{verbatim}
    print("Hola, Mundo!")
\end{verbatim}
Aunque este es un ejemplo trivial, las ventajas de Python salen a
la luz. El curso de Informática I, en Yorktown, no tiene prerrequisitos,
es por eso que muchos de los estudiantes, que ven este ejemplo, están
mirando a su primer programa. Algunos de ellos están un poco nerviosos,
porque han oído que la programación de computadores es difícil de
aprender. La versión C++ siempre me ha forzado a escoger entre dos
opciones que no me satisfacen: explicar el \texttt{\#include}, \texttt{void
main()}, y las sentencias \{, y \} y arriesgar a confundir o intimidar
a algunos de los estudiantes al principio, o decirles, ``No te preocupes
por todo eso ahora; lo retomaré más tarde,'' y tomar el mismo riesgo.
Los objetivos educativos en este momento del curso son introducir
a los estudiantes la idea de sentencia y permitirles escribir su primer
programa. Python tiene exactamente lo que necesito para lograr esto,
y nada más.

Comparar el texto explicativo de este programa en cada versión del
libro ilustra más de lo que esto significa para los estudiantes principiantes.
Hay trece párrafos de explicación de ``Hola, mundo!'' en la versión
C++; en la versión Python, solo hay dos. Aún mas importante, los 11
párrafos que faltan no hablan de ``grandes ideas'' en la programación
de computadores, sino de minucias de la sintaxis de C++. Encontré
la misma situación al repasar todo el libro. Párrafos enteros desaparecían
en la versión Python del texto, porque su sencilla sintaxis los hacía
innecesarios.

Usar un lenguaje de muy alto nivel, como Python, le permite a un profesor
posponer los detalles de bajo nivel de la máquina hasta que los estudiantes
tengan el bagaje que necesitan para entenderlos. Permite ``poner
cosas primero'' pedagógicamente. Unos de los mejores ejemplos de
esto es la manera en la que Python maneja las variables. En C++ una
variable es un nombre para un lugar que almacena una cosa. Las variables
tienen que ser declaradas con tipos, al menos parcialmente, porque
el tamaño del lugar al cual se refieren tiene que ser predeterminado.
Así, la idea de una variable se liga con el hardware de la máquina.
El concepto poderoso y fundamental de variable ya es difícil para
los estudiantes principiantes (de informática y álgebra). Bytes y
direcciones de memoria no ayudan para nada. En Python una variable
es un nombre que se refiere a una cosa. Este es un concepto más intuitivo
para los estudiantes principiantes y está más cerca del significado
de ``variable'' que aprendieron en los cursos de matemática del
colegio. Yo me demoré menos tiempo ayudándolos con el concepto de
variable y en su uso este año, que en el pasado.

Otro ejemplo de cómo Python ayuda en la enseñanza y aprendizaje de
la programación es su sintaxis para las funciones. Mis estudiantes
siempre han tenido una gran dificultad comprendiendo las funciones.
El problema principal se centra alrededor de la diferencia entre una
definición de función y un llamado de función, y la distinción relacionada
entre un parámetro y un argumento. Python viene al rescate con una
bella sintaxis. Una definición de función empieza con la palabra clave
\texttt{def}, y simplemente digo a mis estudiantes: ``cuando definas
una función, empieza con \texttt{def}, seguido del nombre de la función
que estás definiendo, cuando llames una función, simplemente llama
(digita) su nombre.'' Los parámetros van con las definiciones y los
argumentos van con los llamados. No hay tipos de retorno, tipos para
los parámetros, o pasos de parámetro por referencia y valor, y ahora
yo puedo enseñar funciones en la mitad de tiempo que antes, con una
mejor comprensión.

Usar Python ha mejorado la eficacia de nuestro programa de informática
para todos los estudiantes. Veo un nivel general de éxito más alto
y un nivel más bajo de frustración, de lo que ya había experimentado
durante los dos años que enseñé C++. Avanzo más rápido y con mejores
resultados. Más estudiantes terminan el curso con la habilidad de
crear programas significativos; esto genera una actitud positiva hacia
la experiencia de la programación.

\section*{Construyendo una comunidad}

He recibido correos electrónicos de todas partes del mundo, de personas
que están usando este libro para aprender o enseñar programación.
Una comunidad de usuarios ha comenzado a emerger, y muchas personas
han contribuido al proyecto mandando materiales a través del sitio
Web complementario: \\
 \\
\url{ http://www.thinkpython.com }\\
 \\

Con la publicación del libro, en forma impresa, espero que continúe
y se acelere el crecimiento de esta comunidad de usuarios.

La emergencia de esta comunidad y la posibilidad que sugiere para
otras experiencias de colaboración similar entre educadores han sido
las partes más excitantes de trabajar en este proyecto, para mí. Trabajando
juntos, nosotros podemos aumentar la calidad del material disponible
para nuestro uso y ahorrar tiempo valioso.

Yo les invito a formar parte de nuestra comunidad y espero escuchar
de ustedes. Por favor escriba a los autores a \texttt{\url{feedback@thinkpython.com}}.

\vspace{0.25in}
 
\begin{flushleft}
Jeffrey Elkner\\
 Escuela Secundaria Yortown\\
 Arlington, Virginia.\\
 
\par\end{flushleft}


  	\clearemptydoublepage
% LaTeX source for textbook ``How to think like a computer scientist''
% Copyright (c)  2001  Allen B. Downey, Jeffrey Elkner, and John Dewey.

% Permission is granted to copy, distribute and/or modify this
% document under the terms of the GNU Free Documentation License,
% Version 1.1  or any later version published by the Free Software
% Foundation; with the Invariant Sections being "Contributor List",
% with no Front-Cover Texts, and with no Back-Cover Texts. A copy of
% the license is included in the section entitled "GNU Free
% Documentation License".

% This distribution includes a file named fdl.tex that contains the text
% of the GNU Free Documentation License.  If it is missing, you can obtain
% it from www.gnu.org or by writing to the Free Software Foundation,
% Inc., 59 Temple Place - Suite 330, Boston, MA 02111-1307, USA.

\chapter{Lista de los colaboradores}

Este libro vino a la luz debido a una colaboración que no sería posible sin 
la licencia de documentación libre de la GNU (Free Documentation License).  
Quisiéramos agradecer a la Free Software Foundation por desarrollar esta 
licencia y, por supuesto, por ponerla a nuestra disposición.

Nosotros queremos agradecer a los mas de 100 juiciosos y reflexivos lectores 
que nos han enviado sugerencias y correcciones durante los años pasados.  
En el espíritu del  software libre, decidimos expresar nuestro agradecimiento 
en la forma de una lista de colaboradores.  Desafortunadamente, esta lista no 
está completa, pero estamos haciendo nuestro mejor esfuerzo para mantenerla 
actualizada.

Si tiene la oportunidad de leer la lista, tenga en cuenta que cada persona 
mencionada aquí le ha ahorrado a usted y a todos los lectores subsecuentes  la 
confusión debida a un error técnico o debido a una explicación confusa, solo 
por enviarnos una nota.

Después de tantas correcciones, todavía pueden haber errores en este libro.  
Si ve uno, esperamos que tome un minuto para contactarnos.  El correo 
electrónico es 
\href{mailto:feedback@thinkpython.com}{feedback@thinkpython.com}. Si hacemos 
un cambio debido a su sugerencias,usted aparecerá en la siguiente versión de la 
lista de colaboradores (a menos que usted pida ser omitido). Gracias!

\begin{itemize}
	\item Lloyd Hugh Allen remitió una corrección a la Sección 8.4.
	%He can be reached at: \texttt{lha2@columbia.edu}
	\item Yvon Boulianne corrigió un error semántico en el Capítulo 5.
	%She can be reached at: \texttt{mystic@monuniverse.net}
	\item Fred Bremmer hizo una corrección en la Sección 2.1.
	%He can be reached at:  \texttt{Fred.Bremmer@ubc.cu}
	\item Jonah Cohen escribió los guiones en Perl para convertir la fuente
	LaTeX, de este libro, a un maravilloso HTML.
	%His Web page is \texttt{jonah.ticalc.org}
	%and his email is \texttt{JonahCohen@aol.com}
	\item Michael Conlon remitió una corrección de gramática en el Capítulo 3
	una mejora de estilo en el Capítulo 2, e inició la discusión de 
	los aspectos técnicos de los intérpretes.
	%Michael can be reached at: \texttt{michael.conlon@sru.edu}
	\item Benoit Girard envió una corrección a un extraño error en la Sección 
	5.6.
	%Benoit can be reached at:
	%\texttt{benoit.girard@gouv.qc.ca}
	\item Courtney Gleason y Katherine Smith escribieron \texttt{horsebet.py}, 
	que
	se usaba como un caso de estudio en una versión anterior de este libro.  Su 
	programa se puede encontrar en su website.
	%Courtney can be reached at: {\tt
	%orion1558@aol.com}
	\item Lee Harr sometió más correcciones de las que tenemos espacio para 
	enumerar aquí, 
	y, por supuesto, debería ser listado como uno de los editores principales 
	del 
	texto.
	%He can be reached at: \texttt{missive@linuxfreemail.com}
	\item James Kaylin es un estudiante usando el texto. Él ha enviado numerosas
	correcciones.
	%James can be reached by email at: \texttt{Jamarf@aol.com}
	\item David Kershaw arregló la función errónea \texttt{imprimaDoble} en 
	la Sección 3.10.
	%He can be reached at: \texttt{david\_kershaw@merck.com}
	\item Eddie Lam ha enviado numerosas correcciones a los Capítulos 1, 2, y 
	3.  Él
	corrigió el Makefile para que creara un índice, la primera vez
	que se compilaba el documento, y  nos ayudó a instalar un sistema
	de control de versiones.
	%Eddie can be reached at:
	%\texttt{nautilus@yoyo.cc.monash.edu.au}
	\item Man-Yong Lee envió una corrección al código de ejemplo en la
	Sección 2.4.  
	%He can be reaced at: \texttt{yong@linuxkorea.co.kr}
	\item David Mayo notó que la palabra ``inconscientemente"
	debe cambiarse por  ``subconscientemente".
	%David can be reached at:\texttt{bdbear44@netscape.net}
	\item Chris McAloon envió varias correcciones a las Secciones 3.9 y 3.10.
	%He can be reached at: \texttt{cmcaloon@ou.edu}
	\item Matthew J. Moelter ha sido un contribuidor de mucho tiempo quien 
	remitió 
	numerosas correcciones y sugerencias al libro.  
	%He can be reached at:
	%\texttt{mmoelter@calpoly.edu}
	\item Simon Dicon Montford reportó una definición de función que faltaba y 
	varios errores en el Capítulo 3.  Él también encontró errores en la
	función \texttt{incrementar} del Capítulo 13.
	%He can be reached at: \texttt{dicon@bigfoot.com}
	\item John Ouzts corrigió la definición de ``valor de retorno'' en el
	Capítulo 3.
	%He can be reached at: \texttt{jouzts@bigfoot.com}
	\item Kevin Parks envió  sugerencias valiosas para mejorar la distribución 
	del libro.
	%He can be reached at: \texttt{cpsoct@lycos.com}
	\item David Pool envió la corrección de un error en el glosario del 
	Capítulo 1 y palabras de estímulo.
	%He can be reached at: \texttt{pooldavid@hotmail.com}
	\item Michael Schmitt envió una corrección al capítulo de archivos y 
	excepciones.
	%He can be reached at: \texttt{ipv6\_128@yahoo.com}
	\item Robin Shaw notó un error en la Sección 13.1, donde la función
	imprimirHora se usaba en un ejemplo sin estar definida.
	%Robin can be reached at: \texttt{randj@iowatelecom.net}
	\item Paul Sleigh encontró un error en el Capítulo 7 y otro en los
	guiones de Perl de Jonah Cohen que generan HTML a partir de LaTeX.
	%He can be reached at: \texttt{bat@atdot.dotat.org}
	%\item Christopher Smith is a computer science teacher at the Blake
	%School in Minnesota who teaches Python to his beginning students.
	%He can be reached at: \texttt{csmith@blakeschool.org or 
	%smiles@saysomething.com}
	\item Craig T. Snydal está probando el texto en un curso en  Drew
	University. El ha aportado varias sugerencias valiosas y correcciones.
	%and can be reached at: \texttt{csnydal@drew.edu}
	\item Ian Thomas y sus estudiantes están usando el texto en un curso
	de programación. Ellos son los primeros en probar los capítulos de la
	segunda mitad del libro y han enviado numerosas correcciones y 
	sugerencias.
	%Ian can be reached at: \texttt{ithomas@sd70.bc.ca}
	\item Keith Verheyden envió una corrección al Capítulo 3.
	%He can be reached at: \texttt{kverheyd@glam.ac.uk}
	\item Peter Winstanley descubrió un viejo error en nuestro Latín,
	en el capítulo 3.
	%He can be reached at:\texttt{Peter.Winstanley@scotland.gsi.gov.uk} 
	\item Chris Wrobel hizo correcciones al código en el capítulo sobre 
	archivos, E/S y excepciones. 
	%He can be reached at: \texttt{ferz980@yahoo.com}
	\item Moshe Zadka hizo contribuciones inestimables a este proyecto.  Además
	de escribir el primer bosquejo del capítulo sobre Diccionarios, también 
	proporcionó una dirección continua en los primeros años del libro.
	%He can be reached at: \texttt{moshez@math.huji.ac.il}
	\item Christoph Zwerschke envió varias correcciones y sugerencias 
	pedagógicas, y explicó la diferencia entre {\em gleich} y {\em selbe}.
	\item James Mayer nos envió una ciénaga entera de  errores tipográficos 
	y de deletreo, incluyendo dos en la lista de colaboradores
	% james.mayer@acm.org
	\item Hayden McAfee descubrió una inconsistencia potencialmente confusa 
	entre dos ejemplos.
	\item Angel Arnal hace parte de un equipo internacional de traductores que
	trabajan en la versión española del texto. Él también ha encontrado varios 
	errores en la versión inglesa.
\end{itemize}
  	\clearemptydoublepage

\chapter{Traducción al español}

Al comienzo de junio de 2007 tomé la iniciativa de traducir el texto
``How to think like a Computer Scientist, with Python'' al español.
Rápidamente me dí cuenta de que ya había un trabajo inicial de traducción
empezado por:
\begin{itemize}
\item Angel Arnal 
\item I Juanes 
\item Litza Amurrio 
\item Efrain Andia
\end{itemize}
Ellos habían traducido los capítulos 1,2,10,11, y 12, así como el
prefacio, la introducción y la lista de colaboradores. Tomé su valioso
trabajo como punto de partida, adapté los capítulos, traduje las secciones
faltantes del libro y añadí un primer capítulo adicional sobre solución
de problemas.

Aunque el libro traduce la primera edición del original, todo se ha
corregido para que sea compatible con Python 2.7, por ejemplo se usan
booleanos en vez de enteros en los condicionales y ciclos.

Para realizar este trabajo ha sido invaluable la colaboración de familiares,
colegas, amigos y estudiantes que han señalado errores, expresiones
confusas y han aportado toda clase de sugerencias constructivas. Mi
agradecimiento va para los traductores antes mencionados y para los
estudiantes de Biología que tomaron el curso de Informática en la
Pontificia Universidad Javeriana (Cali-Colombia), durante el semestre
2014-1:
\begin{itemize}
\item Estefanía Lopez 
\item Gisela Chaves 
\item Marlyn Zuluaga 
\item Francisco Sanchez 
\item María del Mar Lopez 
\item Diana Ramirez 
\item Guillermo Perez 
\item María Alejandra Gutierrez 
\item Sara Rodriguez 
\item Claudia Escobar
\item Yisveire Fontalvo
\end{itemize}
\vspace{0.25in}
 

Para la segunda edición todo el código fuente se cambió para ejecutarse
con Python 3 y se añadieron 2 capítulos interludios y un posludio
como capítulo final.
\begin{flushleft}
Andrés Becerra Sandoval \\
 Universidad Santiago de Cali \\
 andres.becerra00@usc.edu.co \\
\par\end{flushleft}
	\clearemptydoublepage

% The following lines add a little extra space to the column
% in which the Section numbers appear in the table of contents
\makeatletter
\renewcommand{\l@section}{\@dottedtocline{1}{1.5em}{3.0em}}
\makeatother
\setcounter{tocdepth}{1}

\tableofcontents
\clearemptydoublepage

\mainmatter

%% Texto original LaTeX del libro ``Aprenda a Pensar Como un
% Científico Informático''
% Copyright (c)  2001  Allen B. Downey, Jeffrey Elkner y John Dewey.
% Se da permiso para copiar, distribuir o modificar este documento
% bajo los términos de la Licencia de Documentación Libre GNU, Version
% 1.1 o cualquier version posterior publicada por la Free Software
% Foundation. Las Secciones Invariables son "Lista de Contribuyentes",
% sin los Textos de Portada y sin los Textos de Cubierta Posterior. Una
% copia de la licencia esta incluida en la section titulada "Licencia de
% Documentación Libre GNU".
% Esta distribución incluye un archivo llamado fdl.tex que contiene el
% texto de la Licencia de Documentación Libre GNU. Si no se encuentra,
% se lo puede obtener de www.gnu.org o puede escribir a la Free Software
% Foundation, Inc., 59 Temple Place - Suite 330, Boston, MA 02111-1307, USA.

\chapter{Solución de problemas}
\index{solución de problemas}

\section{Solución de acertijos}
\index{acertijo}

Todos nos hemos topado con acertijos como el siguiente. Disponga 
los dígitos del 1 al 9 en el recuadro siguiente, de manera que la suma de cada fila, cada columna y las dos diagonales dé el mismo resultado:
\begin{center}
	\begin{tabular}{|c|c|c|}
		\hline 
		 &  & \tabularnewline
		\hline 
		 &  & \tabularnewline
		\hline 
		 &  & \tabularnewline
		\hline
	\end{tabular}
\end{center}

Este acertijo  se denomina construcción de un { \bf cuadrado mágico}. Un 
acertijo, normalmente, es un enigma o adivinanza que se propone como 
pasatiempo. Otros ejemplos de acertijo son un crucigrama, una sopa de letras y 
un sudoku.

Los acertijos pueden tener varias soluciones, por ejemplo, la siguiente es una solución propuesta al acertijo anterior:

\begin{center}
	\begin{tabular}{|c|c|c|}
		\hline 
		1 & 2 & 3\tabularnewline
		\hline
		4 & 5 & 6\tabularnewline
		\hline 
		7 & 8 & 9 \tabularnewline
		\hline
	\end{tabular}
\end{center}

Usted puede notar que esta solución candidata no es correcta. Si tomamos la suma por filas, obtenemos valores distintos:

\begin{itemize}
	\item En la fila 1: \texttt{1 + 2 + 3 = 6}
	\item En la fila 2: \texttt{4 + 5 + 6 = 15}
	\item En la fila 3: \texttt{7 + 8 + 9 = 24}
\end{itemize}

Si tomamos las columnas, tampoco obtenemos el mismo resultado en 
cada suma:

\begin{itemize}
	\item En la columna 1: \texttt{1 + 4 + 7 = 12}
	\item En la columna 2: \texttt{2 + 5 + 8 = 15}
	\item En la columna 3: \texttt{3 + 6 + 9 = 18}
\end{itemize}

A pesar de que las diagonales sí suman lo mismo:

\begin{itemize}
	\item En la diagonal 1: \texttt{1 + 5 + 9 = 15}
	\item En la diagonal 2: \texttt{7 + 5 + 3 = 15}
\end{itemize}

Tómese un par de minutos para resolver este acertijo, es decir, para construir 
un cuadrado mágico y regrese a la lectura cuando obtenga la solución. 

Ahora, responda para sí mismo las siguientes preguntas:

\begin{itemize}
	\item ¿Cuál es la solución que encontró?
	\item ¿Es correcta su solución?
	\item ¿Cómo le demuestra a alguien que su solución es correcta?
	\item ¿Cuál fue el proceso de solución que llevo a cabo en su mente?
	\item ¿Cómo le explicaría a alguien el proceso de solución que llevó a cabo?
	\item ¿Puede poner por escrito el proceso de solución que llevó	a cabo?
\end{itemize}

El reflexionar seriamente sobre estas preguntas es muy importante,
tenga la seguridad de que esta actividad será muy importante para 
continuar con la lectura.

Vamos a ir contestando las preguntas desde una solución particular,
y desde un proceso de solución particular, el de los autores. Su solución y su 
proceso de solución son igualmente valiosos, el nuestro solo es otra 
alternativa; es más, puede que hayamos descubierto la misma:

\begin{center}
	\begin{tabular}{|c|c|c|}
		\hline 
		4 & 9 & 2\tabularnewline
		\hline
		3 & 5 & 7\tabularnewline
		\hline 
		8 & 1 & 6 \tabularnewline
		\hline
	\end{tabular}
\end{center}

Esta solución es correcta, porque la suma por filas, columnas y de
las dos diagonales da el mismo valor, 15. Ahora, para demostrarle
a alguien este hecho podemos revisar las sumas por filas, columnas y 
diagonales detalladamente.

% \begin{itemize}
% 
% \item En la fila 1: 4+9+2=15
% \item En la fila 2: 3+5+7=15
% \item En la fila 3: 8+1+6=15
% \item En la columna 1: 4+3+8=15
% \item En la columna 2: 9+5+1=15
% \item En la columna 3: 2+7+6=15
% \item En la diagonal 1: 4+5+6=15
% \item En la diagonal 2: 8+5+2=15
% \end{itemize}

El proceso de solución que llevamos a cabo fue el siguiente:

\begin{itemize}
	\item Sospechábamos que el 5 debía estar en la casilla central, ya que
	es el número medio de los 9: 1 2 3 4 {\bf 5} 6 7 8 9. 
	
	\item Observamos un patrón interesante de la primera solución propuesta: 
	las diagonales sumaban igual, 15: 
	\begin{center}
		\begin{tabular}{|c|c|c|}
			\hline 
			{\bf 1} & 2 & {\bf 3}\tabularnewline
			\hline
			4 & {\bf 5} & 6\tabularnewline
			\hline 
			{\bf 7} & 8 & {\bf 9} \tabularnewline
			\hline
		\end{tabular}
	\end{center}
	
	\item  La observación anterior, 1+5+9=7+5+3, también permite deducir
	otro hecho interesante. Como el 5 está en las dos sumas, podemos 
	deducir que 1+9=7+3, y esto es 10. 
	
	\item Una posible estrategia consiste en colocar parejas de números
	que sumen 10, dejando al 5 ``emparedado'', por ejemplo, poner la 
	pareja 6,4:
	
	\begin{center}
		\begin{tabular}{|c|c|c|}
			\hline 
			  & {\bf 6} & \tabularnewline
			\hline
			  & {\bf 5} & \tabularnewline
			\hline 
			  & {\bf 4} &  \tabularnewline
			\hline
		\end{tabular}
	\end{center}
	
	\item Para agilizar ésta estrategia es conveniente enumerar todas
	las parejas de números entre 1 y 9 que suman 10, excluyendo al 5:
	(1,9),(2,8),(3,7),(4,6)
	
	\item Ahora, podemos probar colocando éstas parejas en filas, columnas
	y diagonales. 
	
	\item Un primer ensayo: 	
	\begin{center}
		\begin{tabular}{|c|c|c|}
			\hline 
			 7 & {\bf 6} & 2\tabularnewline
			\hline
			  & {\bf 5} & \tabularnewline
			\hline 
			 8 & {\bf 4} & 3 \tabularnewline
			\hline
		\end{tabular}
	\end{center}
	Aquí encontramos que es imposible armar el cuadrado, pues no hay como
	situar el 9 ni el 1. Esto sugiere que la pareja (6,4) no va en la 
	columna central, debemos cambiarla.

	\item Después de varios ensayos, moviendo la pareja (6,4), y 
	reacomodando los otros números logramos llegar a la solución 
	correcta.
\end{itemize}

\section{El método de solución}
\label{sec:metodosolucion}
\index{método de solución de problemas}

Mas allá de la construcción de cuadrados mágicos, lo que el ejemplo
anterior pretende ilustrar es la importancia que tiene el usar un 
método ordenado de solución de problemas, acertijos en este caso.

Observe que la solución anterior fue conseguida a través de varios
pasos sencillos, aunque el encadenamiento de todos éstos hasta 
producir un resultado correcto pueda parecer algo complejo.

Lea cualquier historia de Sherlock Holmes y obtendrá la misma
impresión. Cada caso, tomado por el famoso detective, presenta un
enigma difícil, que al final es resuelto por un encadenamiento
de averiguaciones, deducciones, conjeturas y pruebas sencillas. Aunque
la solución completa de cada caso demanda un proceso complejo de
razonamiento, cada paso intermedio es sencillo y está guiado por
preguntas sencillas y puntuales.

Las aventuras de Sherlock Holmes quizás constituyen una de las 
mejores referencias bibliográficas en el proceso de solución de problemas. 
Son como los capítulos de C.S.I\footnote{http://www.cbs.com/primetime/csi/}, 
puestos por escrito. Para Holmes, y quizás para Grissom, existen varios 
principios que se deben seguir:

\begin{itemize}
	\item Obtener las soluciones y quedarse satisfecho con ellas. El peor 
	inconveniente para consolidar un método de solución de problemas 
	poderoso consiste en que cuando nos topemos con una solución completa de 
	un solo golpe nos quedemos sin reflexionar cómo llegamos a ella. Así, nunca
	aprenderemos estrategias valiosas para resolver el siguiente problema. 
	Holmes decía que nunca adivinaba, primero
	recolectaba la mayor información posible sobre los problemas
	que tenía a mano, de forma que estos datos le ``sugirieran'' alguna
	solución.
	
	\item Todo empieza por la observación, un proceso minucioso de recolección
	de datos sobre el problema a mano. Ningún dato puede ignorarse de entrada,
	hasta que uno tenga una comprensión profunda sobre el problema.
	
	\item Hay que prepararse adecuadamente en una variedad de dominios. Para 
	Holmes esto iba desde tiro con pistola, esgrima, boxeo, análisis de 
	huellas, entre otros.
	Para cada problema que intentemos resolver habrá un dominio en el cual 
	debemos aprender lo más que podamos, esto facilitará enormemente el proceso 
	de solución.
	
	\item Prestar atención a los detalles ``inusuales''. Por ejemplo, en 
	nuestro cuadrado  mágico fue algo inusual que las diagonales de la primera 
	alternativa de solución, la más ingenua posible, sumaran lo mismo. 
	
	\item Para  deducir más hechos, a partir de los datos recolectados y los 
	detalles inusuales, hay que usar todas las armas del razonamiento: el poder 
	de la deducción (derivar nuevos hechos a partir de los datos), la inducción 
	(generalizar a partir de casos), la refutación (el proceso de probar la 
	falsedad de alguna conjetura), el pensamiento analógico (encontrando 
	relaciones,metáforas, analogías) y, por último, pero no menos importante, 
	el uso del sentido común.
	
	\item Después del análisis de datos, uno siempre debe proponer una 
	alternativa de solución, así sea simple e ingenua, y proceder intentando 
	probarla y {\bf refutarla} al mismo tiempo. Vale la pena recalcar esto: no 
	importa que tan ingenua, sencilla e incompleta es una alternativa de 
	solución, con tal de que nos permita seguir	indagando. Esto es como la 
	primera frase que se le dice a una chica (o chico, dado el caso) que uno 
	quiere conocer; no importa qué frase sea, no importa que tan trivial sea, 
	con tal de que permita iniciar una conversación.
	
	\item El proceso de búsqueda de soluciones es como una conversación que se 
	inicia, aunque en este caso el interlocutor no es una persona, sino el 
	problema que tenemos a mano. 
	Con una primera alternativa de solución---no importa lo sencilla e 
	incompleta--- podemos formularnos una pregunta interesante: ¿resuelve esta 
	alternativa el problema?
	
	\item Lo importante de responder la pregunta anterior no es la obtención de 
	un {\bf No} como respuesta; pues esto es lo que sucede la mayoría de las 
	veces. Lo importante viene cuando nos formulamos esta segunda pregunta ¿Con 
	lo que sabemos del problema hasta ahora {\bf por qué mi alternativa no es 
	capaz de resolverlo?} 
	
	\item La respuesta a la pregunta anterior puede ser: todavía no sé lo 
	suficiente sobre el problema para entender por qué mi alternativa de 
	solución no lo resuelve; esto es una señal de alerta para recolectar más 
	datos y estudiar mas el dominio	del problema.
	
	\item Una respuesta más constructiva a la pregunta anterior puede ser: mi 
	alternativa de solución no resuelve el problema porque no considera algunos 
	hechos importantes,	y no considera algunas restricciones que debe cumplir 
	una solución. Un ejemplo de	este tipo de respuesta lo da nuestro ensayo 
	de colocar la pareja (6,4) emparedando al 5 en el problema del cuadrado 
	mágico:
	
	\begin{center}
		\begin{tabular}{|c|c|c|}
			\hline 
			 7 & {\bf 6} & 2\tabularnewline
			\hline
			  & {\bf 5} & \tabularnewline
			\hline 
			 8 & {\bf 4} & 3 \tabularnewline
			\hline
		\end{tabular}
	\end{center}
	
	Cuando notamos que la pareja (6,4) no puede colocarse en la columna central 
	del cuadrado, intentamos otra alternativa de solución, colocando estos
	números en filas o en las esquinas. Lo importante de este tipo de respuesta
	es que nos va a permitir avanzar a {\em otra} alternativa de solución, casi
	siempre más compleja y más cercana a la solución.
	
	\item No poner obstáculos a la creatividad. Es muy difícil lograr esto 
	porque la mente humana siempre busca límites para respetar; así, hay que 
	realizar un esfuerzo consciente para eliminar todo límite o restricción que 
	nuestra mente va creando. Una estrategia interesante es el uso y 
	fortalecimiento del pensamiento lateral.
	
	\item Perseverar. El motivo más común de fracaso en la solución de 
	acertijos y problemas es el abandono. No existe una receta mágica para 
	resolver problemas, lo único que uno puede hacer es seguir un método y 
	perseverar, perseverar sin importar cuantas alternativas de solución 
	incorrectas se hayan generado.
	Esta es la clave para el éxito. Holmes decía que eran muchísimos más los 
	casos que no había podido resolver, quizás Grissom reconocería lo mismo. Lo 
	importante entonces es perseverar ante cada nuevo caso, esta es la única 
	actitud razonable para enfrentar y resolver problemas.
\end{itemize}

\section{Reflexión sobre este método de solución}

El proceso de solución de problemas que hemos presentado es muy sencillo. Los 
grandes profesionales que tienen mucho éxito en su campo (científicos, 
humanistas, ingenieros, médicos, empresarios, etc.) tienen métodos de solución 
de problemas mucho más avanzados que el que hemos presentado aquí. Con esto 
pretendemos ilustrar un punto: lo importante no es el método propuesto, porque 
muy probablemente no le servirá para resolver todos los problemas o acertijos 
que enfrente; lo importante es:

\begin{itemize}
	\item Contar con un método de solución de problemas. Si no hay método, 
	podemos solucionar problemas, claro está, pero cada vez que lo hagamos será 
	por golpes de suerte o inspiración --que no siempre nos acompaña, 
	desafortunadamente. 
	
	\item Desarrollar, a medida que transcurra el tiempo y adquiera más 
	conocimientos sobre su profesión y la vida, un método propio de solución de 
	problemas. Este desarrollo personal puede tomar como base el método 
	expuesto aquí, o algún otro que encuentre en la literatura o a través de la 
	interacción con otras personas.
\end{itemize}


\section{Acertijos propuestos}

Para que empiece inmediatamente a construir su método personal de solución de
problemas, tome cada uno de los siguientes acertijos, siga el proceso de 
solución recomendado y documéntelo a manera de entrenamiento. Si usted descubre 
estrategias generales que no han sido consideradas aquí, compártalas con sus
compañeros, profesores y, mejor aún, con los autores del libro.

\begin{enumerate}
	\item Considere un tablero de ajedrez de 4x4 y 4 damas del mismo color. Su 
	misión es colocar las 4 damas en el tablero sin que éstas se ataquen entre 
	sí. Recuerde que una dama ataca a otra ficha si se encuentran en la misma 
	fila, columna o diagonal.
	
	\item Partiendo de la igualdad $a=b$, encuentre cuál es el problema en el 
	siguiente razonamiento:
	
	\begin{eqnarray*}
	a & = & b\\
	a^{2} & = & ba\\
	a^{2}-b^{2} & = & ba-b^{2}\\
	(a-b)(a+b) & = & b(a-b)\\
	a+b & = & b\\
	a & = & 2b\\
	\frac{a}{b} & = & 2\\
	1 & = & 2
	\end{eqnarray*}
	
	Tenga en cuenta que en el último paso se volvió a usar el hecho de que 
	$a=b$, en la forma $\frac{a}{b}=1$.
	
	\item Encuentre el menor número entero positivo que pueda descomponerse 
	como la suma de los cubos de dos números enteros positivos de dos maneras 
	distintas. Esto es,	encontrar el mínimo $A$ tal que $A=b^3+c^3$ y 
	$A=d^3+e^3$, con $A,b,c,d$ y $e$ números positivos, mayores a cero, y 
	distintos entre si.
\end{enumerate}


\section{Mas allá de los acertijos: problemas computacionales}
\index{problema computacional}

Un problema computacional es parecido a un acertijo; se presenta una situación
problemática y uno debe diseñar alguna solución. En los problemas 
computacionales la solución consiste en una {\bf descripción general de 
procesos}; esto es, un problema computacional tiene como solución la 
descripción de un conjunto de pasos que se podrían llevar a cabo de manera 
general para lograr un objetivo.

Un ejemplo que ilustra esto es la multiplicación. Todos sabemos multiplicar 
números de dos cifras, por ejemplo:

%\newpage
 \[
 \begin{array}{cccc}
           \ &\ &3&4\\
           \ & \times&2&1 \\ \hline
           \ &\ &3&4 \\
            +&6&8&\ \\  \hline
           \ &7&1&4 \\
 \end{array}
\]

Pero el problema computacional asociado a la multiplicación de números de dos
cifras consiste en hallar la descripción general de todos los procesos posibles
de multiplicación de parejas de números de dos cifras. Este problema ha sido
resuelto, desde hace varios milenios, por diferentes civilizaciones humanas,
siguiendo métodos alternativos. Un método de solución moderno podría describirse
así:

Tome los dos números de dos cifras, P y Q. Suponga que las cifras de P son 
$p_1$ y $p_2$, esto es, $P=p_1p_2$. Igualmente, suponga que $Q=q_1q_2$. La 
descripción {\bf general} de todas las multiplicaciones de dos cifras puede 
hacerse así:

\[
  \begin{array}{cccc}
           \ &\ &p_1&p_2\\
           \ & \times&q_1&q_2 \\ \hline
           \ &\ &q_2p_1&q_2p_2 \\
            +&q_1p_1&q_1p_2&\ \\  \hline
           \ &q_1p_1&q_2p_1+q_1p_2&q_2p_2 \\
  \end{array}
\]

\begin{itemize}
	\item Tome la cifra $q_2$ y multiplíquela por las cifras de  $P$ (con ayuda 
	de una tabla de multiplicación). 
	Ubique los resultados debajo de cada cifra de $P$ correspondiente. 
	
	\item Tome la cifra $q_1$ y multiplíquela por las cifras de  $P$ (con ayuda 
	de una tabla de multiplicación). 
	Ubique los resultados debajo de las cifras que se generaron en el paso 
	anterior, aunque desplazadas una columna hacia la izquierda.
	
	\item Si en alguno de los pasos anteriores el resultado llega a 10 o se 
	pasa de 10, ubique las unidades únicamente 
	y lleve un acarreo, en decenas o centenas) para la columna de la izquierda.
	
	\item Sume los dos resultados parciales, obteniendo el resultado final.
\end{itemize}

Usted puede estar quejándose en este momento, ¿para qué hay que complicar tanto
nuestro viejo y conocido proceso de multiplicación? Bueno, hay varias razones
para  esto:

\begin{itemize}
	\item Una descripción impersonal como ésta puede ser leída y ejecutada por 
	cualquier persona ---o computador, como veremos mas adelante---.
	
	\item Sólo creando descripciones generales de procesos se pueden analizar 
	para demostrar que funcionan correctamente.
	
	\item Queríamos sorprenderlo, tomando algo tan conocido como la suma y 
	dándole
	una presentación que, quizás, nunca había visto. Este cambio de perspectiva
	es una invitación a que abra su mente a pensar en descripciones generales 
	de procesos.
\end{itemize}

Precisando un poco, un problema computacional es la descripción general de una
situación en la que se presentan unos datos de entrada y una salida deseada
que se quiere calcular. Por ejemplo, en el problema computacional de la 
multiplicación de números de dos cifras, los datos de entrada son los números 
a multiplicar; la salida es el producto de los dos números. Existen más 
problemas computacionales como el de ordenar un conjunto de números y el 
problema de encontrar una palabra en un párrafo de texto. Como ejercicio 
defina para estos problemas cuales son los datos de entrada 
y la salida deseada.

La solución de un problema computacional es una descripción general del  
conjunto de pasos que se deben llevar a cabo con las entradas del problema para 
producir los datos de salida deseados.  Solucionar problemas computacionales no 
es muy diferente de solucionar acertijos, las dos actividades producen la misma 
clase de retos intelectuales, y el método de solución de la sección 
\ref{sec:metodosolucion} es aplicable en los dos casos.
Lo único que hay que tener en cuenta es que la solución de un problema es una 
{\bf descripción general o programa, como veremos más adelante}, que se refiere 
a las entradas y salidas de una manera más técnica de lo que estamos 
acostumbrados. Un ejemplo de esto lo constituyen los nombres $p_1,p_2,q_1$ y 
$q_2$ que usamos en la descripción general de la multiplicación de números de 
dos cifras. 

Aunque la solución de problemas es una actividad compleja, es muy interesante,
estimulante e intelectualmente gratificante; incluso cuando no llegamos a 
solucionar los problemas completamente. En el libro vamos a enfocarnos
en la solución de problemas computacionales por medio de programas, y, aunque
solo vamos a explorar este tipo de problemas, usted verá que las estrategias
de solución, los conceptos que aprenderá y la actitud de científico de la
computación que adquirirá serán valiosas herramientas para resolver todo tipo
de problemas de la vida real.


\section{Glosario}

\begin{description}
	\item[Acertijo:] enigma o adivinanza que se propone como pasatiempo.
	
	\item[Solución de problemas:]  el proceso de formular un problema,
	hallar la solución y expresar la solución.
	
	\item[Método de solución de problemas:] un conjunto de pasos, estrategias y 
	técnicas
	organizados que permiten solucionar problemas de una manera ordenada.
	
	\item[Problema:] una situación o circunstancia en la que se dificulta 
	lograr un
	fin. 
	
	\item[Problema computacional:] una situación general con una especificación 
	de los 
	datos de entrada y los datos de  salida deseados.
	
	\item[Solución a un problema:] conjunto de pasos y estrategias que permiten 
	lograr
	un fin determinado en una situación problemática, cumpliendo ciertas 
	restricciones.
	
	\item[Solución a un problema computacional:] descripción general de los 
	pasos
	que toman cualquier entrada en un problema computacional y la transforman
	en una salida deseada.
	
	\item[Restricción:] una condición que tiene que cumplirse en un problema 
	dado.
	
	\index{acertijo}
	\index{solución de problemas}
	\index{método de solución de problemas}
	\index{método!de solución de problemas}
	\index{problema}
	\index{solución a un problema}
	\index{problema!solución}
	\index{restricción}
\end{description}

\section{Ejercicios}

Intente resolver los siguientes problemas computacionales, proponiendo 
soluciones {\bf generales e impersonales}:

\begin{enumerate}
	\item Describa cómo ordenar tres números a, b y c.
	\item Describa cómo encontrar el menor elemento en un conjunto de números.
	\item Describa cómo encontrar una palabra dentro de un texto más largo.
\end{enumerate}
	\clearemptydoublepage  % solucion de problemas
% Texto original LaTeX del libro ``Aprenda a Pensar Como un
% Científico Informático''
% Copyright (c)  2001  Allen B. Downey, Jeffrey Elkner y John Dewey.
% Se da permiso para copiar, distribuir o modificar este documento
% bajo los términos de la Licencia de Documentación Libre GNU, Version
% 1.1 o cualquier version posterior publicada por la Free Software
% Foundation. Las Secciones Invariables son "Lista de Contribuyentes",
% sin los Textos de Portada y sin los Textos de Cubierta Posterior. Una
% copia de la licencia esta incluida en la section titulada "Licencia de
% Documentación Libre GNU".
% Esta distribución incluye un archivo llamado fdl.tex que contiene el
% texto de la Licencia de Documentación Libre GNU. Si no se encuentra,
% se lo puede obtener de www.gnu.org o puede escribir a la Free Software
% Foundation, Inc., 59 Temple Place - Suite 330, Boston, MA 02111-1307, USA.


\chapter{El camino hacia el programa}

El objetivo de este libro es el de enseñar al estudiante a pensar
como lo hacen los científicos informáticos. Esta manera de
pensar combina las mejores características de la matemática, la
ingeniería y las ciencias naturales. Como los matemáticos, los
científicos informáticos usan lenguajes formales para diseñar
ideas (específicamente, cómputos). Como los ingenieros, ellos
diseñan cosas, construyendo sistemas mediante el ensamble de componentes y 
evaluando las ventajas y desventajas de cada una de las alternativas de
construcción. Como los científicos, ellos observan el comportamiento de sistemas
complejos, forman hipótesis, y prueban sus predicciones.

La habilidad más importante del científico informático es {\bf
la solución de problemas}. La solución de problemas incluye poder
formular problemas, pensar en soluciones de manera
creativa y expresar soluciones clara y precisamente. Como se
verá, el proceso de aprender a programar es la oportunidad
perfecta para desarrollar la habilidad de resolver problemas.
Por esa razón este capítulo se llama ``El camino hacia el programa''.

A cierto nivel, usted aprenderá a programar, lo cual es una
habilidad muy útil por sí misma. A otro nivel, usted utilizará la
programación para obtener algún resultado. Ese resultado se verá
más claramente durante el proceso.

\section{El lenguaje de programación Python}
\index{lenguaje de programación}
\index{lenguaje!programación}

El lenguaje de programación que aprenderá es Python. Python es un
ejemplo de {\bf lenguaje de alto nivel}; otros ejemplos de
lenguajes de alto nivel son C, C++, Perl y Java.

Como se puede deducir de la nomenclatura ``lenguaje de alto
nivel,'' también existen {\bf lenguajes de bajo nivel}, que
también se denominan lenguajes de máquina o lenguajes ensambladores. 
A propósito, las computadoras sólo ejecutan programas escritos en lenguajes de bajo nivel. Los programas de alto nivel
tienen que ser traducidos antes de ser ejecutados. Esta traducción
lleva tiempo, lo cual es una pequeña desventaja de los lenguajes
de alto nivel.

\index{portátil}
\index{lenguaje de alto nivel}
\index{lenguaje de bajo nivel}
\index{lenguaje!alto nivel}
\index{lenguaje!bajo nivel}

Aun así, las ventajas son enormes. En primer lugar, la programación
en lenguajes de alto nivel es mucho más fácil; escribir programas
en un lenguaje de alto nivel toma menos tiempo ya que los programas son
más cortos, más fáciles de leer, y es más probable que  queden correctos. 
En segundo lugar, los lenguajes de alto
nivel son {\bf portables}, lo que significa que los programas
escritos con estos pueden ser
ejecutados en tipos diferentes de computadoras sin modificación
alguna o con pocas modificaciones. Programas escritos en lenguajes
de bajo nivel sólo pueden ser ejecutados en un tipo de computadora
y deben ser reescritos para ser ejecutados en otra.

Debido a estas ventajas, casi todo programa se escribe en un
lenguaje de alto nivel. Los lenguajes de bajo nivel son sólo usados
para unas pocas aplicaciones especiales.

\index{compilar}
\index{interpretar}

Hay dos tipos de programas que traducen lenguajes de alto nivel a
lenguajes de bajo nivel: {\bf intérpretes} y {\bf
compiladores}. Una intérprete lee un programa de alto nivel y lo
ejecuta, lo que significa que lleva a cabo lo que indica el
programa. Traduce el programa poco a poco, leyendo y ejecutando
cada comando.

\vspace{0.1in}
\centerline{\includegraphics[scale=0.7]{illustrations/interpret.eps}}
\vspace{0.1in}

Un compilador lee el programa y lo traduce todo al mismo tiempo,
antes de ejecutar alguno de los programas. A menudo se compila un
programa como un paso aparte, y luego se ejecuta el código
compilado. En este caso, al programa de alto nivel se lo llama el
{\bf código fuente}, y al programa traducido es llamado el {\bf
código de objeto} o el {\bf código ejecutable}.

\vspace{0.1in}
\centerline{\includegraphics[scale=0.7]{illustrations/compile.eps}}
\vspace{0.1in}

A Python se lo considera un lenguaje interpretado, porque sus programas
son ejecutados por un intérprete. Existen dos maneras
de usar el intérprete: modo de comando y modo de guión. En modo
de comando se escriben sentencias en el lenguaje Python y
el intérprete muestra el resultado.

\begin{pyconcode}
leo@localhost: $ python
Python 3.5.1 (default, Jan  8 2016, 00:31:22) 
[GCC 4.9.2] on linux
Type "help", "copyright", "credits" or "license" for more 
information.
>>> 1 + 1
2
>>> exit()
leo@localhost: $
\end{pyconcode} 

% % % % % % % % % % % % % % % % % % % % % % % % % % % %
% Hacer explicación al margen sobre línea de comandos,
% SHELL, prompt, etc...
% % % % % % % % % % % % % % % % % % % % % % % % % % % %

La primera línea de este ejemplo es el comando que pone en marcha
al intérprete de Python. Las tres líneas siguientes son mensajes
del intérprete. La quinta línea comienza con \texttt{>>>}, lo que
indica que el intérprete de Python está listo para recibir comandos.
Escribimos \texttt{1 + 1} y el intérprete contestó \texttt{2}.

Alternativamente, se puede escribir el programa en un archivo de texto y
usar el intérprete para ejecutar el contenido de dicho
archivo. El archivo, en este caso, se denomina un {\bf \textit{script} (guión)};
por ejemplo, en un editor de texto se puede crear un archivo 
\texttt{unomasuno.py} que contenga esta línea:

\begin{pyconcode}
print(1 + 1)
\end{pyconcode}

Por convención, los archivos que contienen programas de Python 
tienen nombres que terminan con \texttt{.py}.

Para ejecutar el programa, se le tiene que indicar el nombre del
guión a la interpretadora.

\begin{pyconcode}
leo@localhost: $ python unomasuno.py
2
leo@localhost: $
\end{pyconcode}

En otros entornos de desarrollo, los detalles de la ejecución de
programas diferirán. Además, la mayoría de programas son más
interesantes que el anterior.

La mayoría de ejemplos en este libro son ejecutados en la línea de
comandos. La línea de comandos es más conveniente para el desarrollo de
programas y para pruebas rápidas, porque las instrucciones de Python 
se pueden pasar a la máquina para ser ejecutadas inmediatamente. 
Una vez que el programa está completo, se lo puede
archivar en un \textit{script} para ejecutarlo o modificarlo en el futuro.

\section{¿Qué es un programa?}

Un programa es una secuencia de instrucciones que especifican cómo
ejecutar un cómputo. El cómputo puede ser 
matemático, cómo solucionar un sistema de ecuaciones o determinar
las raíces de un polinomio, pero también puede ser 
simbólico, cómo buscar y reemplazar el texto de un
documento o (aunque parezca raro) compilar un programa.

\index{instrucción}

Las instrucciones (comandos, órdenes) tienen una apariencia
diferente en lenguajes de programación diferentes, pero existen
algunas funciones básicas que se presentan en casi todo lenguaje:

\begin{description}
	\item[Entrada:] recibir datos del teclado, o de un archivo o de otro
	aparato.
	\item[Salida:] mostrar datos en el monitor o enviar datos a un
	archivo u otro aparato.
	\item[Matemáticas:] ejecutar operaciones básicas, como la adición y la 
	multiplicación.
	\item[Operación condicional:] probar la veracidad de alguna
	condición y ejecutar una secuencia de instrucciones apropiada.
	\item[Repetición] ejecutar alguna acción repetidas veces,
	usualmente con alguna variación.
\end{description}

Aunque sea difícil de creer, todos los programas en existencia son
formulados exclusivamente con tales instrucciones. Así, una manera
de describir la programación es: el proceso de romper una tarea en
tareas cada vez más pequeñas hasta que éstas sean lo
suficientemente sencillas como para ser ejecutadas con una secuencia
de estas instrucciones básicas.

Quizás esta descripción es un poco ambigua. No se preocupe.
Explicaremos ésto con más detalle en el tema de {\bf algoritmos}.

\section{¿Qué es la depuración (debugging)?}
\index{depuración (debugging)}
\index{error (bug)}

La programación es un proceso complejo, y a veces este proceso
lleva a {\bf errores indefinidos}, también llamados {\bf defectos}
o {\bf errores de programación} (en inglés `bugs'), y el proceso de
buscarlos y corregirlos es llamado {\bf depuración} (en inglés
`debugging').

Hay tres tipos de errores que pueden ocurrir en un programa. Es
muy útil distinguirlos para encontrarlos más rápido.

\subsection{Errores sintácticos}
\index{error sintáctico}
\index{error!sintaxis}

Python sólo puede ejecutar un programa si está sintácticamente
bien escrito. Al contrario, es decir, si el programa
tiene algún error de sintaxis, el proceso falla y devuelve un
mensaje de error. La palabra {\bf sintáctica} se refiere a la
estructura de cualquier programa y a las reglas de esa estructura.
\index{sintáctica} Por ejemplo, en español, la primera letra de
toda oración debe ser mayúscula y el fin de toda oración debe llevar un punto. 
esta oración tiene un error sintáctico. Esta oración también

Para la mayoría de lectores, unos pocos errores no impiden la
comprensión de los grafitis en la calle que, a menudo, rompen muchas
reglas de sintaxis. Sin embargo Python no es así. Si hay aunque
sea un error sintáctico en el programa, Python mostrará un mensaje
de error y abortará su ejecución. Al principio usted
pasará mucho tiempo buscando errores sintácticos, pero con el
tiempo no cometerá tantos errores y los encontrará rápidamente.

\subsection{Errores en tiempo de ejecución}
\label{runtime}
\index{error en tiempo de ejecución}
\index{error!en tiempo de ejecución}
\index{excepción}
\index{lenguaje seguro}
\index{seguro!lenguaje}

El segundo tipo de error es el de tiempo de ejecución. Este
error aparece sólo cuando se ejecuta un programa. Estos errores
también se llaman {\bf excepciones}, porque indican que algo
excepcional ha ocurrido.

Con los programas que vamos a escribir al principio, los errores de
tiempo de ejecución ocurrirán con poca frecuencia.

\subsection{Errores semánticos}
\index{semántica}
\index{error semántico}
\index{semántico!error}

El tercer tipo de error es el {\bf semántico}.
Si hay un error de lógica en su programa, éste será
ejecutado sin ningún mensaje de error, pero el resultado no
será el deseado. El programa ejecutará la lógica que usted
le dijo que ejecutara.

A veces ocurre que el programa escrito no es el que se
tenía en mente. El sentido o significado del programa no es
correcto. Es difícil hallar errores de lógica. Eso requiere
trabajar al revés, comenzando a analizar la salida para
encontrar al problema.

\subsection{Depuración experimental}

Una de las técnicas más importantes que usted aprenderá es la
depuración. Aunque a veces es frustrante, la depuración es una
de las partes de la programación más estimulantes, interesantes e 
intelectualmente exigentes.


La depuración es una actividad parecida a la tarea de un
investigador: se tienen que estudiar las pistas para inferir los
procesos y eventos que han generado los resultados encontrados.

La depuración también es una ciencia experimental. Una vez
que se tiene conciencia de un error, se modifica el programa y se intenta
nuevamente. Si la hipótesis fue la correcta se pueden predecir los
resultados de la modificación y estaremos más cerca a un programa
correcto. Si la hipótesis fue la errónea tendrá que idearse otra
hipótesis. Como dijo Sherlock Holmes: ``Cuando se ha descartado lo
imposible, lo que queda, no importa cuan inverosímil, debe ser la
verdad'' (A. Conan Doyle, {\em The Sign of Four})

\index{Holmes, Sherlock}
\index{Doyle, Arthur Conan}

Para algunas personas, la programación y la depuración son lo
mismo: la programación es el proceso de depurar un programa
gradualmente, hasta que el programa tenga el resultado deseado.
Esto quiere decir que el programa debe ser, desde un principio, un
programa que funcione, aunque su función sea solo mínima. El
programa es depurado mientras crece y se desarrolla.

Por ejemplo, aunque el sistema operativo Linux contenga miles de
líneas de instrucciones, Linus Torvalds lo comenzó como un
programa para explorar el microprocesador Intel 80836. Según Larry
Greenfield: ``Uno de los proyectos tempranos de Linus fue un
programa que intercambiaría la impresión de AAAA con BBBB. Este
programa se convirtió en Linux'' (de {\em The Linux Users' Guide}
Versión Beta 1).

\index{Linux}

Otros capítulos tratarán más  el tema de la depuración y otras técnicas
de programación.

\section{Lenguajes formales y lenguajes naturales}
\index{lenguaje formal}
\index{lenguaje natural}
\index{formal!lenguaje}
\index{natural!lenguaje}

Los {\bf lenguajes naturales} son los hablados por seres
humanos, como el español, el inglés y el francés. Éstos no han sido
diseñados artificialmente (aunque se trate de imponer cierto orden en
ellos), pues se han desarrollado naturalmente.

Los {\bf Lenguajes formales}  son diseñados por
humanos y tienen aplicaciones específicas. La notación
matemática, por ejemplo, es un lenguaje formal, ya que se presta a
la representación de las relaciones entre números y símbolos. Los
químicos utilizan un lenguaje formal para representar la
estructura química de las moléculas. Es necesario notar que:

\begin{quote}
{\bf Los lenguajes de programación son formales y han sido
desarrollados para expresar cómputos.}
\end{quote}

Los lenguajes formales casi siempre tienen reglas sintácticas 
estrictas.
Por ejemplo, $3+3=6$ es una expresión matemática correcta, pero
$3=+6\$$ no lo es. De la misma manera, $H_2O$ es una nomenclatura
química correcta, pero $_2Zz$ no lo es.

Existen dos clases de reglas sintácticas, en cuanto a unidades y
estructura. Las unidades son los elementos básicos de un lenguaje,
como lo son las palabras, los números y los elementos químicos.
Por ejemplo, en \texttt{3=+6\$}, \texttt{\$} no es una unidad matemática
aceptada. Similarmente, $_2Zz$ no es formal porque no hay ningún
elemento químico con la abreviación $Zz$.

La segunda clase de error sintáctico está relacionado con la estructura
de un elemento; mejor dicho, el orden de las unidades. La estructura de
la sentencia \texttt{3=+6\$} no es aceptada porque no se puede escribir el
símbolo de igualdad seguido de un símbolo más. Similarmente, las 
fórmulas moleculares tienen que mostrar el número de subíndice después del 
elemento, no antes.

Al leer una oración, sea en un lenguaje natural o una sentencia
en un lenguaje técnico, se debe discernir la estructura de la
oración. En un lenguaje natural este proceso, llamado {\bf
análisis sintáctico}, ocurre subconscientemente.

\index{analizar sintácticamente}

Por ejemplo cuando se escucha una oración simple como ``el otro zapato
se cayó'', se puede
distinguir el sustantivo ``el otro zapato'' y el predicado ``se cayó''. Cuando se ha analizado la
oración sintácticamente, se puede deducir el significado, o la
semántica, de la oración. Si usted sabe lo que es un zapato y el significado
de caer, comprenderá el significado de la oración.

Aunque existen muchas cosas en común entre los lenguajes naturales y 
los formales---por ejemplo las unidades, la estructura,
la sintáctica y la semántica--- también existen muchas diferencias.

\index{Ambigüedad}
\index{Redundancia}
\index{Literalidad}

\begin{description}
	\item[Ambigüedad:] los lenguajes naturales tienen muchísimas 
	ambigüedades que se superan usando claves contextuales e información
	adicional.
	Los lenguajes formales son diseñados para estar completamente libres de
	ambigüedades o, tanto como sea posible, lo que quiere decir que 
	cualquier sentencia tiene sólo un significado sin importar el contexto
	en el que se encuentra.
	\item[Redundancia:] para reducir la ambigüedad y los malentendidos,
	los lenguajes naturales utilizan bastante redundancia. Como resultado
	tienen una abundancia de posibilidades para expresarse. Los lenguajes 
	formales son menos redundantes y mas concisos.
	\item[Literalidad:] los lenguajes naturales tienen muchas metáforas
	y frases comunes. El significado de un dicho, por ejemplo: ``Estirar
	la pata'', es diferente al significado de sus sustantivos y
	verbos. En este ejemplo, la oración no tiene nada que ver con una
	pata y significa 'morirse'. En los lenguajes formales solo existe el 
	significado literal.
\end{description}

Los que aprenden a hablar un lenguaje natural---es decir todo el 
mundo---muchas
veces tienen dificultad en adaptarse a los lenguajes formales. A veces
la diferencia entre los lenguajes formales y los naturales es 
comparable
a la diferencia entre la prosa y la poesía:

\index{Poesía}
\index{Prosa}

\begin{description}
	\item[Poesía:] se utiliza una palabra por su cualidad auditiva
	tanto como por su significado. El poema, en su totalidad, produce
	un efecto o reacción emocional. La ambigüedad no es sólo común, sino
	utilizada a propósito.
	\item[Prosa:] el significado literal de la palabra es más importante
	y la estructura contribuye más al significado. La prosa se presta
	más al análisis que la poesía, pero todavía contiene ambigüedad.
	\item[Programa:] el significado de un programa es inequívoco y
	literal, y es entendido en su totalidad analizando las unidades y
	la estructura.
\end{description}

He aquí unas sugerencias para la lectura de un programa (y de otros
lenguajes formales). Primero, recuerde que los lenguajes formales
son mucho más densos que los lenguajes naturales y, por consecuencia,
toma mas tiempo dominarlos. Además, la estructura es muy
importante, entonces no es una buena idea leerlo de pies a cabeza,
de izquierda a derecha. En lugar de ésto, aprenda a separar las diferentes
partes en su mente, identificar las unidades e interpretar la
estructura. Finalmente, ponga atención a los detalles. La fallas
de puntuación y la ortografía afectarán negativamente la ejecución de 
sus programas.

\section{El primer programa}
\label{hello}
\label{hello world}

% % % % % % % % % % % % % % % % % % % % % % % % % % %
% Nota al margen de la historia del HOLA MUNDO
% % % % % % % % % % % % % % % % % % % % % % % % % % %

Tradicionalmente el primer programa en un lenguaje nuevo se llama
``Hola todo el mundo!'' (en inglés, Hello world!) porque sólo muestra las
palabras ``Hola todo el mundo'' . En el lenguaje Python es así:

\beforeverb
\begin{pythoncode}
print("Hola todo el mundo!")
\end{pythoncode}
\afterverb
%

Este es un ejemplo de la función  \texttt{print}, la cual no
imprime nada en papel, más bien muestra un valor. En este caso,
el resultado es mostrar en pantalla las palabras:

\beforeverb
\begin{verbatim}
Hola todo el mundo!
\end{verbatim}
\afterverb
%
Las comillas señalan el comienzo y el final del valor; no
aparecen en el resultado.

\index{sentencia print}
\index{print!sentencia}

Hay gente que evalúa la calidad de un lenguaje de programación por
la simplicidad del programa ``Hola todo el mundo!''. Si seguimos
ese criterio, Python cumple con esta meta.

\section{Glosario}

\begin{description}
	\item[Solución de problemas:]  el proceso de formular un problema,
	hallar la solución y expresarla.
	
	\item[Lenguaje de alto nivel:]  un lenguaje como Python que es diseñado
	para ser fácil de leer y escribir por la gente.
	
	\item[Lenguaje de bajo nivel:]  un lenguaje de programación que es
	diseñado para ser fácil de ejecutar para una computadora; también
	se lo llama ``lenguaje de máquina'' o ``lenguaje ensamblador''.
	
	\item[Portabilidad:]  la cualidad de un programa que puede ser 
	ejecutado
	en más de un tipo de computadora.
	
	\item[Interpretar:]  ejecutar un programa escrito en un lenguaje de
	alto nivel traduciéndolo línea por línea.
	
	\item[Compilar:]  traducir un programa escrito en un lenguaje
	de alto nivel a un lenguaje de bajo nivel de una vez,
	en preparación para la ejecución posterior.
	
	\item[Código fuente:]  un programa escrito en un lenguaje de alto
	nivel antes de ser compilado.
	
	\item[Código objeto:]  la salida del compilador una vez que
	el programa ha sido traducido.
	
	\item[Programa ejecutable:]  otro nombre para el código de objeto
	que está listo para ser ejecutado.
	
	\item[Guión (script):] un programa archivado (que va a ser interpretado).
	
	\item[Programa:] un grupo de instrucciones que especifica un cómputo.
	
	\item[Algoritmo:]  un proceso general para resolver una
	clase completa de problemas.
	
	\item[Error (bug):]  un error en un programa.
	
	\item[Depuración:]  el proceso de hallazgo y eliminación de los
	tres tipos de errores de programación.
	
	\item[Sintaxis:]  la estructura de un programa.
	
	\item[Error sintáctico:]  un error estructural que hace que un
	programa sea imposible de analizar sintácticamente (e imposible de
	interpretar).
	
	\item[Error en tiempo de ejecución:]  un error que no ocurre hasta
	que el programa ha comenzado a ejecutar e impide que el programa
	continúe.
	
	\item[Excepción:]  otro nombre para un error en tiempo de ejecución.
	
	\item[Error semántico:]   un error en un programa que hace que ejecute
	algo que no era lo deseado.
	
	\item[Semántica:]  el significado de un programa.
	
	\item[Lenguaje natural:]  cualquier lenguaje hablado que evolucionó
	de forma natural.
	
	\item[Lenguaje formal:]  cualquier lenguaje diseñado que tiene un
	propósito específico, como la representación de ideas matemáticas
	o programas de computadoras; todos los lenguajes de programación
	son lenguajes formales.
	
	\item[Unidad:]  uno de los elementos básicos de la estructura
	sintáctica de un programa, análogo a una palabra en un lenguaje
	natural.
	
	\item[Análisis sintáctico:]  la revisión de un programa y el
	análisis de su estructura sintáctica.
	
	\item[Sentencia print:]  una instrucción que causa que el
	intérprete de Python muestre un valor en el monitor.
	
	
	\index{programa}
	\index{solución de problemas}
	\index{lenguaje de alto nivel}
	\index{lenguaje de bajo nivel}
	\index{portabilidad}
	\index{interpretar}
	\index{compilar}
	\index{código de fuente}
	\index{código de objeto}
	\index{código ejecutable}
	\index{algoritmo}
	\index{error(bug)}
	\index{depuración}
	\index{sintaxis}
	\index{semántica}
	\index{error sintáctico}
	\index{error en tiempo de ejecución}
	\index{excepción}
	\index{error semántico}
	\index{lenguaje formal}
	\index{lenguaje natural}
	\index{análisis sintáctico}
	\index{unidad}
	\index{guión}
	\index{sentencia print}
	\index{print!sentencia}
\end{description}


\section{Ejercicios}

En los ejercicios 1, 2, 3 y 4 escriba una oración en español:

\begin{enumerate}

\item Con estructura válida pero compuesta de unidades irreconocibles.

	\item Con unidades aceptables pero con estructura no válida.
	
	\item Semánticamente comprensible pero sintácticamente incorrecta. 
	
	\item Sintácticamente correcta pero que contenga errores semánticos.
	
	\item Inicie la terminal de Python. Escriba \texttt{1 + 2} y luego presione 
	la tecla \texttt{Enter}. 
	Python evalúa esta expresión, presenta el resultado, y enseguida muestra
	otro intérprete.
	 
	Considerando que el símbolo \texttt{*} es el signo de multiplicación y el
	doble símbolo \texttt{**} es el signo de potenciación, realice dos
	ejercicios adicionales escribiendo diferentes expresiones 
	y reportando lo mostrado por el intérprete de Python. 
	
	\item ¿Qué sucede si utiliza el signo de división (\texttt{/})? 
	¿Son los resultados obtenidos los esperados? Explique.
	
	\item Escriba \texttt{1 2} y presione la tecla \texttt{Enter}. Python trata
	de evaluar esta expresión, pero no puede, porque la expresión es 
	sintácticamente incorrecta. Así, Python responde 
	con el siguiente mensaje de error:

	\begin{pyconcode}
File "<stdin>", line 1
 1 2
   ^
SyntaxError: invalid syntax
	\end{pyconcode}

	Muchas veces Python indica la ubicación del error de sintaxis, sin embargo, 
	no siempre es precisa, por lo que no proporciona suficiente información
	sobre cuál es el problema. De esta manera, el mejor antídoto es que usted
	aprenda la sintaxis de Python. 
	En este caso, Python protesta porque no encuentra signo de operación alguno
	entre los números.

	Escriba una entrada que produzca un mensaje de error cuando se 
	introduzca en el intérprete de Python. Explique por qué no tiene una 
	sintaxis válida.

	\item Escriba \verb+print('hola')+. Python ejecuta esta sentencia que 
	muestra las letras \texttt{h}-\texttt{o}-\texttt{l}-\texttt{a}. 
	
	Nótese que las comillas simples en los extremos de la cadena no son parte 
	de la salida mostrada. Ahora escriba \verb+print('"hola"')+ y describa y
	explique el resultado.
	
	\item Escriba \verb+print(queso)+ sin comillas. ¿Qué sucede?
	
	
	\item Escriba \verb+'Esta es una prueba...'+ en el intérprete de Python y 
	presione la tecla \texttt{Enter}. 
	Observe lo que pasa. 
	
	\item Ahora cree un script de Python con el nombre \texttt{prueba1.py} que
	contenga lo siguiente {\small (asegúrese de guardar el archivo antes de 
	intentar ejecutarlo)}:
	\texttt{'Esta es una prueba...'}
	
	¿Qué pasa cuando ejecuta este script? 
	
	\item Ahora cambie el contenido del script a: 
	\verb+print('Esta es una prueba...')+ y ejecutelo de nuevo.
	
	¿Qué pasó esta vez?
	
	Cuando se escribe una expresión en el intérprete de Python, ésta es 
	evaluada y el resultado es mostrado en la línea siguiente. 
	\texttt{'Esta es una prueba...'} es una expresión, que se 
	evalúa a \texttt{'Esta es una prueba...'} (de la misma manera que
	\texttt{42} se evalúa a \texttt{42}). Sin embargo, la evaluación de
	expresiones en un guión no se envía a la salida del programa, por lo que 
	es necesario mostrarla explícitamente.
\end{enumerate}

	\clearemptydoublepage  % el camino del programa
% LaTeX source for textbook ``How to think like a computer scientist''
% Copyright (c)  2001  Allen B. Downey, Jeffrey Elkner, and Chris Meyers.

% Permission is granted to copy, distribute and/or modify this
% document under the terms of the GNU Free Documentation License,
% Version 1.1  or any later version published by the Free Software
% Foundation; with the Invariant Sections being ``Contributor List'',
% with no Front-Cover Texts, and with no Back-Cover Texts. A copy of
% the license is included in the section entitled ``GNU Free
% Documentation License''.

% This distribution includes a file named fdl.tex that contains the text
% of the GNU Free Documentation License.  If it is missing, you can obtain
% it from www.gnu.org or by writing to the Free Software Foundation,
% Inc., 59 Temple Place - Suite 330, Boston, MA 02111-1307, USA.
\chapter{Variables, expresiones y sentencias}

\section{Valores y tipos}
\index{valor}
\index{tipo}
\index{cadena}


Un {\bf valor} es una de las cosas fundamentales---como una letra o un 
número---que una progama manipula. Los valores que hemos visto hasta ahora
son \texttt{2} (el resultado cuando añadimos \texttt{1 + 1}, y
{\verb+"Hola todo el Mundo!"+}.

Los valores pertenecen a diferentes {\bf tipos}:
\texttt{2} es un entero, y {\verb+"Hola, Mundo!"+} es una {\bf cadena},
llamada así porque contiene una ``cadena'' de letras.
Usted (y el intérprete) pueden identificar cadenas porque  están
encerradas entre comillas.

La sentencia de impresión también trabaja con enteros.

\beforeverb
\begin{pyconcode}
>>> print(4)
4
\end{pyconcode}
\afterverb
%

Si no está seguro del tipo que un valor tiene, el intérprete le puede decir.

\beforeverb
\begin{pyconcode}
>>> type("Hola, Mundo!")
<class 'str'>
>>> type(17)
<class 'int'>
\end{pyconcode}
\afterverb
%

Sin despertar ninguna sorpresa, las cadenas pertenecen al tipo \texttt{string
(cadena)} y los enteros pertenecen al tipo \texttt{int}. Menos obvio, los
números con cifras decimales pertenecen a un tipo llamado \texttt{float},
porque éstos se representan en un formato denominado {\bf punto flotante}.

\index{tipo}
\index{cadena}
\index{tipo!cadena}
\index{int}
\index{tipo!int}
\index{float}
\index{tipo!float}

\beforeverb
\begin{pyconcode}
>>> type(3.2)
<class 'float'>
\end{pyconcode}
\afterverb
%

¿Qué ocurre con valores como {\verb+"17"+} y {\verb+"3.2"+}?
Parecen números, pero están encerrados entre comillas como las cadenas.

\beforeverb
\begin{pyconcode}
>>> type("17")
<class 'str'>
>>> type("3.2")
<class 'str'>
\end{pyconcode}
\afterverb
%

Ellos son cadenas.

Cuando usted digita un número grande, podría estar tentado a usar
comas para separar grupos de tres dígitos, como en \texttt{1,000,000}.  
Esto no es un número entero legal en Python, pero esto si es legal:

\beforeverb
\begin{pyconcode}
>>> print(1,000,000)
1 0 0
\end{pyconcode}
\afterverb
%

¡Bueno, eso no es lo que esperábamos!.  Resulta que  {\tt 1,000,000} es una 
tupla, algo que encontraremos en el Capítulo \ref{tuplechap}.  De momento, 
recuerde no poner comas en sus números enteros.


\section{Variables}
\index{variable}
\index{asignación}
\index{sentencia!asignación}

Una de las características más poderosas en un lenguaje de programación
es la capacidad de manipular {\bf variables}. Una variable es un nombre
que se refiere a un valor.

La {\bf sentencia de asignación} crea nuevas variables y les da valores:

\beforeverb
\begin{pyconcode}
>>> mensaje = "¿Qué Onda?"
>>> n = 17
>>> pi = 3.14159
\end{pyconcode}
\afterverb
%

Este ejemplo hace tres asignaciones: la primera asigna la cadena
{\verb+"¿Qué Onda?"+} a una nueva variable denominada \texttt{mensaje}, 
la segunda le asigna el entero \texttt{17} a \texttt{n} y la tercera le
asigna el número de punto flotante \texttt{3.14159} a \texttt{pi}.

\index{diagrama de estados}

Una manera común de representar variables en el papel es escribir el nombre
de la variable con una flecha apuntando a su valor. Esta clase de dibujo se
denomina  {\bf diagrama de estados} porque muestra el estado de cada una de 
las variables (piense en los valores como el estado mental de las variables).
Este diagrama muestra el resultado de las sentencias de asignación anteriores:

\beforefig
	\centerline{\includegraphics{illustrations/state2.eps}}
\afterfig

La función \texttt{print} también funciona con variables.

\beforeverb
\begin{pyconcode}
>>> print(mensaje)
Que Onda?
>>> print(n)
17
>>> print(pi)
3.14159
\end{pyconcode}
\afterverb
%

En cada caso el resultado es el valor de la variable.
Las variables también tienen tipos; nuevamente, le podemos preguntar 
al intérprete cuales son.

\beforeverb
\begin{pyconcode}
>>> type(mensaje)
<class 'str'>
>>> type(n)
<class 'int'>
>>> type(pi)
<class 'float'>
\end{pyconcode}
\afterverb
%

El tipo de una variable es el mismo del valor al que se refiere.


\section{Nombres de variables y palabras reservadas}
\index{palabra reservada}
\index{palabra!reservada}

Los programadores, generalmente, escogen nombres significativos para
sus variables ---que especifiquen para qué se usa la variable.

%%%%%%%%%%%%%%%%%%%%%%%%%%%%%%%%%%%%%%%%%%%%%%%%%%%%%%%%%%%%%%%%%%%%
% Comentar en nota al margen la dificultad de nombrar bien las cosas
%%%%%%%%%%%%%%%%%%%%%%%%%%%%%%%%%%%%%%%%%%%%%%%%%%%%%%%%%%%%%%%%%%%%

Estos nombres pueden ser arbitrariamente largos. Pueden contener
letras y números, pero tienen que empezar con una letra. Aunque 
es legal usar letras mayúsculas, por convención no lo hacemos. Si
usted lo hace, recuerde que la capitalización importa, \texttt{Pedro}
y \texttt{pedro} son variables diferentes.

El carácter subrayado (\texttt{\_}) puede aparecer en un nombre. A menudo
se usa en nombres con múltiples palabras, tales como 
\texttt{mi\_nombre} ó \texttt{precio\_del\_café\_en\_china}.

\index{carácter subrayado}

Si usted le da un nombre ilegal a una variable obtendrá un error sintáctico:

\adjustpage{-2}
%\pagebreak
\beforeverb
\begin{pyconcode}
>>> 76trombones = "gran desfile"
  File "<stdin>", line 1
    76trombones = "gran desfile"
              ^
SyntaxError: invalid syntax
>>> mas& = 1000000
  File "<stdin>", line 1
    mas& = 1000000
       \^
SyntaxError: invalid syntax
>>> class  = "arquitectura de computadoras"
  File "<stdin>", line 1
    class  = "arquitectura de computadoras"
           \^
SyntaxError: invalid syntax
>>> &otro = 504050
  File "<stdin>", line 1
    &otro = 504050
    \^
SyntaxError: invalid syntax
\end{pyconcode}
\afterverb
%

\texttt{76trombones} es ilegal porque no empieza con una letra.

\texttt{mas\&} y \texttt{\&otro} es ilegal porque contiene un carácter ilegal, 
el símbolo \texttt{\&}. 

¿Qué sucede con \texttt{class}?

Resulta que \texttt{class} es una de las {\bf palabras reservadas (keywords)}
de Python.
Las palabras reservadas definen las reglas del lenguaje y su estructura, y no 
pueden ser usadas como nombres de variables.

\index{palabra reservada}

Python tiene veintiocho palabras reservadas:

\begin{center}
\beforeverb
\begin{verbatim}
and       continue  else      for       import    not       
assert    def       except    from      in        or        
break     del       exec      global    is        pass      
class     elif      finally   if        lambda    print     
raise     return    try       while
\end{verbatim}
\afterverb
\end{center}%

Usted puede mantener esta lista a mano. Si el intérprete se queja 
por alguno de sus nombres de variables, y usted no sabe por qué,
búsquelo en esta lista.

\section{Sentencias}

Una sentencia es una instrucción que el intérprete de Python puede
ejecutar. Hemos visto dos clases de sentencias: la asignación y \texttt{print}.

Cuando usted digita una sentencia en la línea de comandos, Python la
ejecuta y despliega el resultado, si hay alguno. El resultado de un 
\texttt{print} es un valor. Las asignaciones no producen un resultado.

Un \textit{script} usualmente contiene una secuencia de sentencias. Si hay más 
de una, los resultados aparecen uno a uno a medida que las sentencias
se ejecutan.

Por ejemplo, el script:

\beforeverb
\begin{pythoncode}
print(1)
x = 2
print(x)
\end{pythoncode}
\afterverb
%

produce la salida

\beforeverb
\begin{pyconcode}
1
2
\end{pyconcode}
\afterverb
%

Observe nuevamente que la sentencia de asignación no produce salida.

\section{Evaluando expresiones}

Una expresión es una combinación de valores, variables y operadores.
Si usted digita una expresión en la línea de comandos, el intérprete
la {\bf evalúa} y despliega su resultado:

\beforeverb
\begin{pyconcode}
>>> 1 + 1
2
\end{pyconcode}
\afterverb
%

Un valor, por si mismo, se considera como una expresión, lo mismo ocurre para
las variables.

\beforeverb
\begin{pyconcode}
>>> 17
17
>>> x
2
\end{pyconcode}
\afterverb
%

Aunque es un poco confuso, evaluar una expresión no es lo mismo que imprimir o
desplegar un valor.

\beforeverb
\begin{pyconcode}
>>> mensaje = "Como le va, Doc?"
>>> mensaje
"Como le va, Doc?"
>>> print(mensaje)
Como le va, Doc?
\end{pyconcode}
\afterverb
%

Cuando  Python  muestra el valor de una expresión que
ha evaluado, utiliza el mismo formato que se usaría para entrar un valor. 
En el caso de las cadenas, esto implica que se incluyen las
comillas. Cuando se usa la función \texttt{print}, el efecto es distinto 
como usted ya lo ha evidenciado.

En un script, una expresión, por sí misma, es una sentencia legal, pero no 
realiza nada. El script:

\beforeverb
\begin{pythoncode}
17
3.2
"Hola, Mundo!"
1 + 1
\end{pythoncode}
\afterverb
%

no produce ninguna salida. ¿Cómo cambiaría el programa 
de manera que despliegue los valores de las cuatro expresiones?

\section{Operadores y operandos}
\index{operador}
\index{operando}
\index{expresión}

Los {\bf operadores} son símbolos especiales que representan cómputos,
como la suma y la multiplicación. Los valores que el operador usa se denominan
{\bf operandos}.

Los siguientes son expresiones válidas en Python, cuyo significado es más o
menos claro:
\adjustpage{2}
\beforeverb
\begin{verbatim}
20+32       hora-1   hora*60+minuto   
minuto/60   5**2     (5+9)*(15-7)
\end{verbatim}
\afterverb
%

Los  símbolos \texttt{+}, \texttt{-}, y \texttt{/}, y los paréntesis para
agrupar, significan en Python lo mismo que en la matemática.
El asterisco (\texttt{*}) es el símbolo para la multiplicación, y \texttt{**}
es el  símbolo para la exponenciación.

Cuando el nombre de una variable aparece en lugar de un operando, se
reemplaza por su valor antes de calcular la operación

La suma, resta, multiplicación y exponenciación realizan lo que usted
esperaría, pero la división podría sorprenderlo.  La siguiente
operación tiene un resultado inesperado:

\beforeverb
\begin{pyconcode}
>>> minuto = 59
>>> minuto/60
0
\end{pyconcode}
\afterverb
%

El valor de \texttt{minuto} es 59, y 59 dividido por 60 es 0.98333,
no 0. La razón para esta discrepancia radica en que Python está realizando
{\bf división entera}.

\index{división entera}

Cuando los dos operandos son enteros el resultado también debe ser un entero;
y, por convención, la división entera siempre redondea {\em hacia abajo},
incluso en casos donde el siguiente entero está muy cerca.

Una solución posible a este problema consiste en calcular un porcentaje, en
lugar de una fracción:

\beforeverb
\begin{pyconcode}
>>> minuto*100/60
98
\end{pyconcode}
\afterverb
%

De nuevo, el resultado se redondea; pero, al menos ahora, el resultado estará
mas aproximado.  Otra alternativa es usar la división en punto
flotante, lo que haremos en el Capítulo \ref{floatchap}.

\section{Orden de las operaciones}
\index{orden de las operaciones}
\index{reglas de precedencia}

Cuando hay más de un operador en una expresión, el orden de
evaluación depende de las {\bf reglas de precedencia}. Python sigue
las mismas reglas de precedencia a las que estamos acostumbrados para sus
operadores matemáticos.
El acrónimo {\bf PEMDAS} es útil para recordar el
orden de las operaciones:

\begin{itemize}
	\item Los {\bf P}aréntesis tienen la precedencia más alta y pueden
	usarse para forzar la evaluación de una expresión de la manera que usted
	desee. Ya que las expresiones en paréntesis se evalúan primero, 
	\texttt{2 * (3 - 1)} es 4, y \texttt{(1 + 1) ** (5 - 2)} es 8. 
	Usted también puede usar paréntesis para
	que una expresión quede más legible, como en \texttt{(minuto * 100)/60},
	aunque esto no cambie el resultado.
	\item La {\bf E}xponenciación tiene la siguiente precedencia más alta, así 
	que
	\texttt{2 ** 1 + 1} es 3 y no 4, y \texttt{3 * 1 ** 3} es 3 y no 27.
	\item La {\bf M}ultiplicación y la {\bf D}ivisión tienen la misma 
	precedencia,
	aunque es más alta que la de la {\bf A}dición y la {\bf S}ubtracción, que
	también tienen la misma precedencia. Así que \texttt{2 * 3 - 1} da 5 en 
	lugar
	de 4, y \texttt{2/3 - 1} es \texttt{-1}, no \texttt{1} (recuerde que en
	división entera, \texttt{2/3 = 0}).
	\item Los operadores con la misma precedencia se evalúan de izquierda
	a derecha.  Recordando que \texttt{minuto = 59}, en la expresión
	\texttt{minuto*100/60}; la multiplicación
	se hace primero, resultando \texttt{5900/60}, lo que a su vez da 
	\texttt{98}.
	Si las operaciones se hubieran evaluado de derecha a izquierda, el resultado
	sería \texttt{59/1}, que es \texttt{59}, y no es lo correcto.
\end{itemize}


\section{Operaciones sobre cadenas}
\index{operación sobre cadenas}

En general, usted no puede calcular operaciones matemáticas sobre cadenas,
incluso si las cadenas lucen como números. Las siguientes operaciones son 
ilegales (asumiendo que \texttt{mensaje} tiene el tipo \texttt{cadena}):

\beforeverb
\begin{verbatim}
 mensaje-1   "Hola"/123   mensaje*"Hola"   "15"+2
\end{verbatim}
\afterverb
%

Sin embargo, el operador \texttt{+} funciona con cadenas, aunque no calcula lo
que usted esperaría.  Para las cadenas, el operador \texttt{+}
representa la {\bf concatenación}, que significa unir los dos operandos 
enlazándolos en el orden en que aparecen. Por ejemplo:

\index{concatenación}

\beforeverb
\begin{pythoncode}
fruta = "banano"
bien_cocida = " pan con nueces"
print(fruta + bien_cocida)
\end{pythoncode}
\afterverb
%

La salida de este programa es  \texttt{banano pan con nueces}. El espacio
antes de la palabra \texttt{pan} es parte de la cadena y sirve para
producir el espacio entre las cadenas concatenadas.

El operador \texttt{*} también funciona con las cadenas; hace una repetición.
Por ejemplo, \texttt{'Fun'*3} es \texttt{'FunFunFun'}.  Uno de los operandos tiene que ser una cadena, el otro tiene que ser un entero.

Estas interpretaciones de \texttt{+} y \texttt{*} tienen sentido por la analogía
 con la suma y la  multiplicación.  Así como \texttt{4*3} es
equivalente a \texttt{4+4+4}, esperamos que \texttt{"Fun"*3} sea lo mismo que
{\verb/"Fun"+"Fun"+"Fun"/}, y lo és.  Sin embargo, las operaciones de concatenación y repetición 
sobre cadenas tienen una diferencia
significativa con las operaciones de suma y multiplicación.
¿Puede usted pensar en una propiedad que la suma y la multiplicación tengan
y que la concatenación y repetición no?

\section{Composición}
\index{composición}

Hasta aquí hemos considerado a los elementos de un programa---variables,
expresiones y sentencias---aisladamente, sin especificar cómo combinarlos.

Una de las características mas útiles de los lenguajes de programación es
su capacidad de tomar pequeños bloques para  {\bf componer} con ellos.  Por
ejemplo, ya que sabemos cómo sumar números y cómo imprimirlos; podemos hacer
las dos cosas al mismo tiempo:

\beforeverb
\begin{pyconcode}
>>>  print(17 + 3)
20
\end{pyconcode}
\afterverb
%

De hecho, la suma tiene que calcularse antes que la impresión, así que las
acciones no están ocurriendo realmente al mismo tiempo. El punto es que  
cualquier expresión que tenga números, cadenas y variables puede ser usada
en una sentencia de impresión (\texttt{print}).  Usted ha visto un ejemplo 
de esto:

\beforeverb
\begin{pythoncode}
print("Número de minutos desde media noche: ", hora * 60 + minuto)
\end{pythoncode}
\afterverb
%

Usted también puede poner  expresiones arbitrarias en el lado derecho de
una sentencia de asignación:

\beforeverb
\begin{pythoncode}
porcentaje = (minuto * 100) / 60
\end{pythoncode}
\afterverb
%

Esto no parece nada impresionante ahora, pero vamos a ver otros ejemplos
en los que la composición hace posible expresar cálculos complejos organizada
y concisamente.

Advertencia: hay restricciones sobre los lugares en los que 
se pueden usar las expresiones. Por ejemplo, el lado izquierdo de una 
asignación tiene que ser un nombre de {\em variable}, no una expresión.  Así 
que esto es ilegal: 
\texttt{minuto + 1 = hora}.

\section{Comentarios}
\index{comentario}

A medida que los programas se hacen más grandes y complejos, se vuelven más
difíciles de leer. Los lenguajes formales son densos; y, a menudo, es difícil
mirar una sección de código y saber qué hace, o por qué lo hace.

Por esta razón, es una muy buena idea añadir notas a sus programas para
explicar, en lenguaje natural, lo que hacen. Estas notas se denominan
{\bf comentarios }y se marcan con el símbolo \texttt{\#}:

\beforeverb
\begin{pythoncode}
# calcula el porcentaje de la hora que ha pasado
porcentaje = (minuto * 100) / 60
\end{pythoncode}
\afterverb
%

En este caso, el comentario aparece en una línea completa. También pueden ir 
comentarios al final de una línea:

\beforeverb
\begin{pythoncode}
# precaucion: division entera
porcentaje = (minute * 100) / 60   
\end{pythoncode}
\afterverb
%

Todo lo que sigue desde el \texttt{\#} hasta el fin de la línea se ignora---no 
tiene efecto en el programa. El mensaje es para el programador que escribe
el programa o para algún programador que podría usar este código en el futuro.
En este caso, le recuerda al lector el sorprendente comportamiento de la 
división entera en Python.

\section{Glosario}

\begin{description}
	\item[Valor:]  un número o una cadena (u otra cosa que se introduzca más 
	adelante) que puede ser almacenado en una variable o calculado en una 
	expresión.
	\item[Tipo:]  conjunto de valores.  El tipo del valor determina cómo se 
	puede usar en expresiones.  Hasta aquí, los tipos que usted ha visto son 
	enteros (tipo \texttt{int}), números de punto flotante (tipo \texttt{float})
	y cadenas (tipo \texttt{string}).
	\item[Punto flotante:] formato para representar números con parte decimal.
	\item[Variable:]  nombre que se refiere a un valor.
	\item[Sentencia:]  sección de código que representa un comando o acción. 
	Hasta aquí las sentencias que usted ha visto son la de asignación y 
	la de impresión.
	\item[Asignación:]  corresponde a la sentencia que pone un valor en una 
	variable.
	\item[Diagrama de estados:]  es la representación gráfica de un conjunto de
	variables y los valores a los que se refieren.
	\item[Palabra reservada:]  es una palabra usada por el compilador
	para analizar sintácticamente un programa; usted no puede usar palabras 
	reservadas 
	como \texttt{if}, \texttt{def}, y \texttt{while} como nombres de variables.
	\item[Operador:]  símbolo especial que representa un simple cálculo como
	una suma, multiplicación o concatenación de cadenas.
	\item[Operando:]  uno de los valores sobre el cual actúa un operador.
	\item[Expresión:]  combinación de variables, operadores y valores que 
	representa un único valor de resultado.
	\item[Evaluar:]  simplificar una expresión ejecutando varias operaciones a 
	fin
	de retornar un valor único.
	\item[División entera:]  operación que divide un entero por otro
	y retorna un entero. La división entera retorna el número de veces que 
	el denominador cabe en el numerador y descarta el residuo.
	\item[Reglas de precedencia:] reglas que gobiernan el orden en que
	las expresiones que tienen múltiples operadores y operandos se evalúan.
	\item[Concatenar:]  unir dos operandos en el orden en que aparecen.
	\item[Composición:]  es la capacidad de combinar simples expresiones y 
	sentencias
	dentro de sentencias y expresiones compuestas para representar cálculos 
	complejos
	concisamente.
	\item[Comentario:]  información que se incluye en un programa para 
	otro programador (o lector del código fuente) que no tiene efecto en la 
	ejecución.
	
	\index{valor}
	\index{punto flotante}
	\index{variable}
	\index{tipo}
	\index{palabra reservada}
	\index{sentencia}
	\index{asignación}
	\index{comentario}
	\index{diagrama de estados}
	\index{expresión}
	\index{operador}
	\index{operando}
	\index{división entera}
	\index{reglas de precedencia}
	\index{precedencia}
	\index{concatenación}
	\index{composición}
\end{description}

\section{Ejercicios}

\begin{enumerate}
	\item Registre qué sucede cuando usa la función \texttt{print} en 
	combinación con una sentencia de asignación, por ejemplo 
	\verb+print(n = 7)+.
	\item ¿Qué sucede cuando se usa la función \texttt{print} con una 
	expresión,  por ejemplo	\verb# print(8+5) #  ?
	\item ¿Qué sucede cuando se ejecuta esto?\\
	\verb+ print(5.2, "esto", 4 - 2, "aquello", 5/2.0)+
	\item Tome la siguiente oración: \textit{Sólo trabajo y nada de juegos 
	hacen de Juan un niño aburrido}. Almacene cada palabra en variables 
	separadas, después muestre la oración en una sola línea usando la función 
	\texttt{print}.
	\item Incluya paréntesis a la expresión \texttt{6 * 1 - 2} para cambiar su 
	resultado de \texttt{4} a \texttt{-6}.
	\item Inserte una línea de comentario en un línea previa a una de código 
	funcional, y registre qué es lo que sucede cuando corre de nuevo el 
	programa.
%	\item La diferencia entre la función input y la función raw\_input es que 
%la función input evalúa la cadena introducida y la función raw\_input no lo 
%hace. Escriba lo siguiente en el intérprete de Python, registre qué sucede y 
%expliquelo:
%	\begin{pyconcode}
%	>>> x = input()
%	3.14
%	>>> type(x)
%	\end{pyconcode}
%	\begin{pyconcode}
%	>>> x = raw_input()
%	3.14
%	>>> type(x)
%	\end{pyconcode}
	\item Escriba una expresión que calcule la nota definitiva de su curso de 
	programación.
\end{enumerate}
	\clearemptydoublepage  % variables, expresiones, sentencias
% LaTeX source for textbook ``How to think like a computer scientist''
% Copyright (c)  2001  Allen B. Downey, Jeffrey Elkner, and Chris Meyers.

% Permission is granted to copy, distribute and/or modify this
% document under the terms of the GNU Free Documentation License,
% Version 1.1  or any later version published by the Free Software
% Foundation; with the Invariant Sections being "Contributor List",
% with no Front-Cover Texts, and with no Back-Cover Texts. A copy of
% the license is included in the section entitled "GNU Free
% Documentation License".

% This distribution includes a file named fdl.tex that contains the text
% of the GNU Free Documentation License.  If it is missing, you can obtain
% it from www.gnu.org or by writing to the Free Software Foundation,
% Inc., 59 Temple Place - Suite 330, Boston, MA 02111-1307, USA.

\chapter{Funciones}
\label{floatchap}

\section{Llamadas a funciones}
\label{functionchap}
\index{llamada a función}
\index{llamada!función}

Usted ya ha visto un ejemplo de una {\bf llamada a función}:

\beforeverb
\begin{pyconcode}
>>> type("32")
<class 'str'>
\end{pyconcode}
\afterverb
%

El nombre de la función es  \texttt{type}, y despliega el tipo de un 
valor o variable. El valor o variable, que se denomina el 
{\bf argumento} de la función, tiene que encerrarse entre paréntesis.
Es usual decir que una función ``toma'' un argumento y ``retorna'' un resultado.
El resultado se denomina el  {\bf valor de retorno}.

\index{argumento}
\index{valor de retorno}

En lugar de imprimir el valor de retorno, podemos asignarlo a una variable:

\beforeverb
\begin{pyconcode}
>>> betty = type("32")
>>> print(betty)
<class 'str'>
\end{pyconcode}
\afterverb
%

Otro  ejemplo es la función \texttt{id} que toma un valor o una variable y
retorna un entero que actúa como un identificador único:

\beforeverb
\begin{pyconcode}
>>> id(3)
134882108
>>> betty = 3
>>> id(betty)
134882108
\end{pyconcode}
\afterverb
%

Cada valor tiene un \texttt{id} que es un número único relacionado con el 
lugar en la memoria en el que está almacenado. El  \texttt{id} de una
variable es el \texttt{id} del valor al que la variable se refiere.

\section{Conversión de tipos}
\index{conversión!tipo}
\index{conversión}

Python proporciona una colección de funciones que convierten valores de un
tipo a otro. La función  \texttt{int} toma cualquier valor y lo 
convierte a un entero, si es posible, de lo contrario \textit{se queja}:

\beforeverb
\begin{pyconcode}
>>> int("32")
32
>>> int("Hola")
ValueError: invalid literal for int(): Hola
\end{pyconcode}
\afterverb
%

\texttt{int} también puede convertir valores de punto flotante a enteros, 
pero hay que tener en cuenta que va a eliminar la parte decimal:

\beforeverb
\begin{pyconcode}
>>> int(3.99999)
3
>>> int(-2.3)
-2
\end{pyconcode}
\afterverb
%

La función \texttt{float} convierte enteros y cadenas a números de 
punto flotante:

\beforeverb
\begin{pyconcode}
>>> float(32)
32.0
>>> float("3.14159")
3.14159
\end{pyconcode}
\afterverb
%

Finalmente, la función \texttt{str} convierte al tipo cadena
(\texttt{string}):

\beforeverb
\begin{pyconcode}
>>> str(32)
'32'
>>> str(3.14149)
'3.14149'
\end{pyconcode}
\afterverb
%

Puede parecer extraño el hecho de que Python distinga el valor entero \texttt{1}
del valor en punto flotante \texttt{1.0}. Pueden representar el mismo número
pero tienen diferentes tipos. La razón para esto es que su representación 
interna en la memoria de la computadora es distinta.

\section{Coerción de tipos}
\index{coerción de tipos}
\index{coerción!tipo}
\index{división entera}
\index{división!entera}

Ahora que podemos convertir entre tipos, tenemos otra forma de 
esquivar a la división entera. Retomando el ejemplo del capítulo
anterior, suponga que deseamos calcular la fracción de una hora
que ha transcurrido. La expresión más obvia \texttt{minuto/60}, 
hace división entera, así que el resultado siempre es 0, incluso
cuando han transcurrido 59 minutos.

Una solución es convertir \texttt{minuto} a punto flotante
para realizar la  división  en punto flotante:

\beforeverb
\begin{pyconcode}
>>> minuto = 59
>>> float(minuto)/60.0
0.983333333333
\end{pyconcode}
\afterverb
%

Otra alternativa es sacar provecho de las reglas de conversión
automática de tipos, que se denominan  {\bf coerción de tipos}.
Para los operadores matemáticos, si algún operando es un número
\texttt{flotante}, el otro se convierte automáticamente a \texttt{flotante}:

\beforeverb
\begin{pyconcode}
>>> minuto = 59
>>> minuto / 60.0
0.983333333333
\end{pyconcode}
\afterverb
%

Así que haciendo el denominador flotante, forzamos a Python a realizar
división en punto flotante.

\section{Funciones matemáticas}
\index{función matemática}
\index{función!matemática}

En matemática usted probablemente ha visto funciones como el \texttt{seno} y el 
\texttt{logaritmo}, y ha aprendido a evaluar expresiones como {\tt
sen(pi/2)} y \texttt{log(1/x)}.  Primero, se evalúa la expresión entre 
paréntesis (el argumento).  Por ejemplo, \texttt{pi/2} es aproximadamente 
1.571, y \texttt{1/x} es 0.1 (si \texttt{x} tiene el valor 10.0).

Entonces, se evalúa la función, ya sea mirando el resultado en una tabla
o calculando varias operaciones. El \texttt{seno} de 1.571
es 1, y el \texttt{logaritmo} de 0.1 es -1 (asumiendo que \texttt{log}
indica el logaritmo en base 10).

Este proceso puede aplicarse repetidamente para evaluar expresiones
más complicadas como  \texttt{log(1/sen(pi/2))}. Primero se evalúa el argumento
de la función más interna, luego la función, y se continúa así.

Python tiene un módulo matemático que proporciona la mayoría de las funciones
matemáticas. Un módulo es un archivo que contiene una colección de
funciones relacionadas.

\index{módulo}

Antes de que podamos usar funciones de un módulo, tenemos que importarlas:

\beforeverb
\begin{pyconcode}
>>> import math
\end{pyconcode}
\afterverb
%

Para llamar a una de las funciones, tenemos que especificar el nombre
del módulo y el nombre de la función, separados por un punto. Este 
formato se denomina {\bf notación punto}.

\index{notación punto}

\beforeverb
\begin{pyconcode}
>>> decibel = math.log10(17.0)
>>> angulo = 1.5
>>> altura = math.sin(angulo)
\end{pyconcode}
\afterverb
%

La primera sentencia le asigna a \texttt{decibel} el logaritmo de 17, en base
\texttt{10}. También hay una función llamada \texttt{log} que usa la 
base logarítmica \texttt{e}. 

La tercera sentencia encuentra el seno del valor  de la  variable {\tt
angulo}. \texttt{sin} y las otras funciones  trigonométricas (\texttt{cos},
\texttt{tan}, etc.)  reciben sus argumentos en radianes. Para convertir
de grados a radianes hay que dividir por 360 y multiplicar por \texttt{2*pi}.  
Por ejemplo, para encontrar el seno de 45 grados, primero calculamos
el ángulo en radianes y luego tomamos el seno:

\beforeverb
\begin{pyconcode}
>>> grados = 45
>>> angulo = grados * 2 * math.pi / 360.0
>>> math.sin(angulo)
\end{pyconcode}
\afterverb
%

La constante \texttt{pi} también hace parte del módulo matemático. Si usted
recuerda geometría puede verificar el resultado comparándolo con la
raíz cuadrada de 2 dividida por 2:

\beforeverb
\begin{pyconcode}
>>> math.sqrt(2) / 2.0
0.707106781187
\end{pyconcode}
\afterverb
%

\section{Composición}
\index{composición}
\index{función!composición}

Así como las funciones matemáticas, las funciones de Python pueden
componerse, de forma que una expresión sea parte de otra. Por ejemplo, usted
puede usar cualquier expresión como argumento a una función:

\beforeverb
\begin{pyconcode}
>>> x = math.cos(angulo + math.pi/2)
\end{pyconcode}
\afterverb
%

Esta sentencia toma el valor de \texttt{pi}, lo divide por 2, y suma este
resultado al valor de  \texttt{angulo}.  Después, la suma se le pasa como 
argumento a la función coseno (\texttt{cos}).

También se puede tomar el resultado de una función y pasarlo como 
argumento a otra:

\beforeverb
\begin{pyconcode}
>>> x = math.exp(math.log(10.0))
\end{pyconcode}
\afterverb
%

Esta sentencia halla el logaritmo en base \texttt{e} de 10 y luego eleva 
\texttt{e} a dicho resultado. El resultado se asigna a  \texttt{x}.


\section{Agregando nuevas funciones}

Hasta aquí solo hemos usado las funciones que vienen con Python, pero también 
es posible agregar nuevas funciones. Crear nuevas funciones para resolver 
nuestros problemas particulares es una de las capacidades mas importantes de un 
lenguaje de programación de propósito general.

En el contexto de la programación, una {\bf función} es una secuencia de 
sentencias que ejecuta una operación deseada y tiene un nombre.  Esta operación 
se especifica en una {\bf definición de función}.  Las funciones que hemos 
usado hasta ahora ya han sido definidas para nosotros. Esto es bueno, porque 
nos permite usarlas sin preocuparnos de los detalles de sus definiciones.

\index{función}
\index{función definición}
\index{definición!función}

La sintaxis para una definición de función es:

\beforeverb
\begin{pythoncode}
def NOMBRE( LISTA DE PARAMETROS ):
  SENTENCIAS
\end{pythoncode}
\afterverb
%

Usted puede inventar los nombres que desee para sus funciones con tal de que no 
use una palabra reservada. La lista de parámetros especifica que información, 
si es que la hay, se debe proporcionar a fin de usar la nueva función.

Se puede incluir cualquier número de sentencias dentro de la función, pero
tienen que sangrarse o indentarse a partir de la margen izquierda. En los 
ejemplos de este libro usaremos un sangrado de dos espacios.

Las primeras funciones que vamos a escribir no tienen parámetros, así que la
sintaxis luce así:

\beforeverb
\begin{pythoncode}
def nueva_linea():
  print()
\end{pythoncode}
\afterverb
%

Esta función se llama \texttt{nueva\_linea}.  Los paréntesis vacíos indican que 
no tiene parámetros. Contiene solamente una sentencia, que produce como salida
una línea vacía. Eso es lo que ocurre cuando se usa el comando \texttt{print}
sin argumentos.

La sintaxis para llamar la nueva función es la misma que para las funciones
predefinidas en Python:

\beforeverb
\begin{pythoncode}
print("Primera Linea.")
nueva_linea()
print("Segunda Linea.")
\end{pythoncode}
\afterverb
%

La salida para este programa es:

\beforeverb
\begin{pyconcode}
Primera Linea.

Segunda Linea.
\end{pyconcode}
\afterverb
%

Note el espacio extra entre las dos líneas. ¿Qué pasa si deseamos 
más espacio entre las líneas? Podemos llamar la misma función 
repetidamente:

\beforeverb
\begin{pythoncode}
print("Primera Linea.")
nueva_linea()
nueva_linea()
nueva_linea()
print("Segunda Linea.")
\end{pythoncode}
\afterverb
%

O podemos escribir una nueva función llamada \texttt{tres\_lineas} que imprima
tres líneas:

\beforeverb
\begin{pythoncode}
def tres_lineas():
  nueva_linea()
  nueva_linea()
  nueva_linea()

print "Primera Linea."
tres_lineas()
print "Segunda Linea."
\end{pythoncode}
\afterverb
%

Esta función contiene tres sentencias, y todas están sangradas por dos espacios.
Como la próxima sentencia (print "Primera Linea") no está sangrada, Python la
interpreta afuera de la función.

Hay que enfatizar dos hechos sobre este programa:

\begin{enumerate}
	\item Usted puede llamar la misma función repetidamente.  De hecho, es una 
	práctica muy común y útil.
	
	\item Usted puede llamar una función dentro de otra función; en este caso
	\texttt{tres\_lineas} llama a \texttt{nueva\_linea}.
\end{enumerate}

Hasta este punto, puede que no parezca claro porque hay que tomarse la molestia
de crear todas estas funciones. De hecho, hay muchas razones, y este ejemplo 
muestra dos:

\begin{itemize}
	\item Crear una nueva función le da a usted la oportunidad de nombrar un
	grupo de sentencias. Las funciones pueden simplificar un programa 
	escondiendo un cálculo complejo detrás de un comando único que usa palabras 
	en lenguaje	natural, en lugar de un código arcano.
	
	\item Crear una nueva función puede recortar el tamaño de un programa 
	eliminando el código repetitivo. Por ejemplo, una forma más corta de 
	imprimir nueve líneas consecutivas consiste en llamar la función 
	\texttt{tres\_lineas} tres veces.
\end{itemize}

\section{Definiciones y uso}

Uniendo los fragmentos de la sección 3.6, el programa completo luce así:

\beforeverb
\begin{pythoncode}
def nueva_linea():
  print

def tres_lineas():
  nueva_linea()
  nueva_linea()
  nueva_linea()

print("Primera Linea.")
tres_lineas()
print("Segunda Linea.")
\end{pythoncode}
\afterverb
%

Este programa contiene dos definiciones de funciones: \texttt{nueva\_linea} y
\texttt{tres\_lineas}. Las definiciones de funciones se ejecutan como 
las otras sentencias, pero su efecto es crear nuevas funciones. Las
sentencias, dentro de la función, no se ejecutan hasta que la función
sea llamada, y la definición no genera salida.

Como usted puede imaginar, se tiene que crear una función antes de
ejecutarla. En otras palabras, la definición de función tiene que
ejecutarse antes de llamarla por primera vez.

\section{Flujo de ejecución}
\index{flujo de ejecución}

Con el objetivo de asegurar que una función se defina antes de su 
primer uso usted tiene que saber el orden en el que las sentencias
se ejecutan, lo que denominamos {\bf flujo de ejecución}.

La ejecución siempre empieza con la primera sentencia del programa. Las 
sentencias se ejecutan una a una, desde arriba hacia abajo.

Las definiciones de funciones no alteran el flujo de ejecución del programa, 
recuerde que las sentencias que están adentro de las funciones no se ejecutan 
hasta que éstas sean llamadas. Aunque no es muy común, usted puede definir una 
función adentro de otra.
En este caso, la definición interna no se ejecuta hasta que la otra función se llame.

Las llamadas a función son como un desvío en el flujo de ejecución. En lugar de
continuar con la siguiente sentencia, el flujo salta a la primera línea de la 
función llamada, ejecuta todas las sentencias internas, y regresa para 
continuar donde estaba previamente.

Esto suena sencillo, hasta que tenemos en cuenta que una función puede llamar a 
otra. Mientras  se está ejecutando una función, el programa puede ejecutar las 
sentencias en otra función. Pero, mientras se está ejecutando la nueva función, 
¡el programa puede tener que ejecutar \textit{otra} función!.

Afortunadamente, Python lleva la pista de donde está fielmente, así que cada vez
que una función termina, el programa continúa su ejecución en el punto donde se 
la llamó. Cuando llega al fin del programa, la ejecución termina.

¿Cual es la moraleja de esta sórdida historia? Cuando lea un programa, no lo 
haga de arriba hacia abajo. En lugar de ésto, siga el flujo de ejecución.

\section{Parámetros y argumentos}
\label{parameters}
\index{parámetro}
\index{función!parámetro}
\index{argumento}
\index{función!argumento}

Algunas de las funciones primitivas que usted ha usado requieren argumentos,
los valores que controlan el trabajo de la función. Por ejemplo, si usted
quiere encontrar el seno de un número, tiene que indicar cual es el número.
Así que, \texttt{sin} toma un valor numérico como argumento.

Algunas funciones toman más de un argumento. Por ejemplo  \texttt{pow} 
(potencia) toma dos argumentos, la base y el exponente.  Dentro de una función,
los valores que se pasan se asignan a variables llamadas  {\bf parámetros}.

Aquí hay un ejemplo de una función definida por el programador que toma un 
parámetro:

\beforeverb
\begin{pythoncode}
def imprima_doble(pedro):
  print(pedro, pedro)
\end{pythoncode}
\afterverb
%

Esta función toma un argumento y lo asigna a un parámetro llamado 
\texttt{pedro}. El valor del parámetro (en este momento no tenemos idea de lo 
que será) se imprime dos veces, y después, se imprime una línea vacía.
El nombre \texttt{pedro} se escogió para sugerir que el nombre que se le asigna 
a un parámetro queda a su libertad; pero, en general, usted desea escoger algo 
mas ilustrativo que \texttt{pedro}.

La función  \texttt{imprima\_doble} funciona para cualquier tipo que pueda
imprimirse:

\beforeverb
\begin{pyconcode}
>>> imprima_doble('Spam')
Spam Spam
>>> imprima_doble(5)
5 5
>>> imprima_doble(3.14159)
3.14159 3.14159
\end{pyconcode}
\afterverb
%

En el primer llamado de función el argumento es una cadena. En el segundo es 
un entero. En el tercero es un flotante (\texttt{float}).

Las mismas reglas de composición que se aplican a las funciones primitivas,
se aplican a las definidas por el programador, así que podemos
usar cualquier clase de expresión como un argumento para  
\texttt{imprima\_doble}:

\beforeverb
\begin{pyconcode}
>>> imprima_doble('Spam'*4)
SpamSpamSpamSpam SpamSpamSpamSpam
>>> imprima_doble(math.cos(math.pi))
-1.0 -1.0
\end{pyconcode}
\afterverb
%

Como de costumbre, la expresión se evalúa antes de que la función se ejecute
así que \texttt{imprima\_doble} retorna \texttt{SpamSpamSpamSpam 
SpamSpamSpamSpam} en lugar de \texttt{'Spam'*4 'Spam'*4}.

También podemos usar una variable como argumento:

\beforeverb
\begin{pyconcode}
>>> m = 'Oh, mundo cruel.'
>>> imprima_doble(m)
Oh, mundo cruel. Oh, mundo cruel.
\end{pyconcode}
\afterverb
%

Observe algo muy importante, el nombre de la variable que pasamos como
argumento  (\texttt{m}) no tiene nada que ver con el nombre del parámetro 
(\texttt{pedro}).  No importa como se nombraba el valor originalmente (en el 
lugar donde se hace el llamado); en la función  \texttt{imprima\_doble}, la 
seguimos llamando de la misma manera \texttt{pedro}.

\section{Las variables y los parámetros son locales}
\index{variable local}
\index{variable!local}

Cuando usted crea una {\bf variable local} en una función, solamente 
existe dentro de ella, y no se puede usar por fuera. Por ejemplo:

\beforeverb
\begin{pythoncode}
def concatenar_doble(parte1, parte2):
  cat = parte1 + parte2
  imprima_doble(cat)
\end{pythoncode}
\afterverb
%

Esta función toma dos argumentos, los concatena, y luego imprime el resultado
dos veces.
Podemos llamar a la función con dos cadenas:

\beforeverb
\begin{pyconcode}
>>> cantar1 = "Pie Jesu domine, "
>>> cantar2 = "Dona eis requiem."
>>> concatenar_doble(cantar1, cantar2)
Pie Jesu domine, Dona eis requiem. Pie Jesu domine, Dona 
eis requiem.
\end{pyconcode}
\afterverb
%

Cuando \texttt{concatenar\_doble} termina, la variable \texttt{cat} se destruye.
Si intentaramos imprimirla obtendríamos un error:

\beforeverb
\begin{pyconcode}
>>> print(cat)
NameError: cat
\end{pyconcode}
\afterverb
%

Los parámetros también son locales.
Por ejemplo, afuera de la función \texttt{imprima\_doble}, no existe
algo como \texttt{pedro}. Si usted intenta usarlo Python se quejará.

\section{Diagramas de pila}
\label{stackdiagram}
\index{diagrama de pila}
\index{marco de función}
\index{marco}

Para llevar pista de los lugares en que pueden usarse las variables
es útil dibujar un {\bf diagrama de pila}. Como los diagramas de estados,
los diagramas de pila muestran el valor de cada variable y además muestran
a que función pertenece cada una.

Cada función se representa por un {\bf marco}.  Un marco es una caja
con el nombre de una función al lado y los parámetros y variables adentro.
El diagrama de pila para el ejemplo anterior luce así:

\adjustpage{-4}
\beforefig
\centerline{\includegraphics{illustrations/stack.eps}}
\afterfig

El orden de la pila muestra el flujo  de ejecución.  \texttt{imprima\_doble}
fue llamada por \texttt{concatenar\_doble}, y \texttt{concatenar\_doble} fue 
llamada por {\tt\_\_main\_\_}, que es un nombre especial para la función más 
superior (la principal, que tiene todo programa). Cuando usted crea una 
variable afuera de cualquier función, pertenece a {\tt\_\_main\_\_}.

Cada parámetro se refiere al mismo valor que su argumento correspondiente. Así 
que \texttt{parte1} tiene el mismo valor que \texttt{cantar1}, \texttt{parte2} 
tiene el mismo valor que \texttt{cantar2}, y \texttt{pedro} tiene el mismo 
valor que  \texttt{cat}.

Si hay un error durante una llamada de función, Python imprime el nombre de 
ésta, el nombre de la función que la llamó, y así sucesivamente hasta llegar a 
\texttt{\_\_main\_\_}.

Por ejemplo, si intentamos acceder a  \texttt{cat} desde {\tt
imprima\_doble}, obtenemos un \texttt{error de nombre (NameError)}:

\beforeverb
\begin{pyconcode}
Traceback (innermost last):
  File "test.py", line 13, in __main__
    concatenar_doble(cantar1, cantar2)
  File "test.py", line 5, in concatenar_doble
    imprima_doble(cat)
  File "test.py", line 9, in imprima_doble
    print cat
NameError: cat
\end{pyconcode}
\afterverb
%

Esta lista de funciones se denomina un {\bf trazado inverso}. Nos informa
en qué archivo de programa ocurrió el error, en qué línea, y qué funciones
se estaban ejecutando en ese momento. También muestra la línea de código que
causó el error.

\index{trazado inverso}

Note la similaridad entre el trazado inverso y el diagrama de pila. Esto no 
es una coincidencia.

\section{Funciones con resultados}

Usted ya puede haber notado que algunas de las funciones que estamos usando,
como las matemáticas, entregan resultados. Otras funciones, como 
\texttt{nueva\_linea}, ejecutan una acción pero no entregan un resultado. Esto 
genera algunas preguntas:

\begin{enumerate}
	\item ¿Qué pasa si usted llama a una función y no hace nada con el 
	resultado (no lo 
	asigna a una variable o no lo usa como parte de una expresión mas grande)?
	
	\item ¿Qué pasa si usted usa una función sin un resultado como parte de 
	una expresión, tal como  \texttt{nueva\_linea() + 7}?
	
	\item ¿Se pueden escribir funciones que entreguen resultados, o estamos 
	limitados
	a funciones tan simples como \texttt{nueva\_linea} y 
	\texttt{imprima\_doble}?
\end{enumerate}

La respuesta a la tercera pregunta es afirmativa y lo lograremos en el capítulo \ref{funcReturn}.

\section{Glosario}

\begin{description}
	\item[Llamada a función:]  sentencia que ejecuta una función. Consiste en el
	nombre de la función seguido por una lista de argumentos encerrados entre
	paréntesis.
	
	\item[Argumento:]  valor que se le da a una función cuando se la está 
	llamando.
	Este valor se le asigna al parámetro correspondiente en la función.
	
	\item[Valor de retorno:]  es el resultado de una función. Si una llamada a 
	función
	se usa como una expresión, el valor de ésta es el valor de retorno de la 
	función.
	
	\item[Conversión de tipo:] sentencia explícita que toma un valor
	de un tipo y calcula el valor correspondiente de otro tipo.
	
	\item[Coerción de tipos:]  conversión de tipo que se hace automáticamente
	de acuerdo a las reglas de coerción del lenguaje de programación.
	
	\item[Módulo:]  archivo que contiene una colección de funciones y clases 
	relacionadas.
	
	\item[Notación punto:] sintaxis para llamar una función que se encuentra en 
	otro
	módulo, especificando el nombre módulo seguido por un punto y el nombre de 
	la función
	(sin dejar espacios intermedios).
	
	\item[Función:]  es la secuencia de sentencias que ejecuta alguna operación 
	útil y que tiene un nombre definido. 
	                Las funciones pueden tomar o no tomar parámetros y pueden 
	                entregar o no entregar un resultado.
	
	\item[Definición de función:]  sentencia que crea una nueva función
	especificando su nombre, parámetros y las sentencias que ejecuta.
	
	\item[Flujo de ejecución:]  orden en el que las sentencias se ejecutan 
	cuando un programa corre.
	
	\item[Parámetro:]  nombre usado dentro de una función para referirse al 
	valor
	que se pasa como argumento.
	
	\item[Variable local:]  variable definida dentro de una función.  Una
	variable local solo puede usarse dentro de su función.
	
	\item[Diagrama de pila:]  es la representación gráfica de una pila de  
	funciones,
	sus variables, y los valores a los que se refieren.
	
	\item[Marco:]  una caja en un diagrama de pila que representa un llamado de 
	función.
	Contiene las variables locales y los parámetros de la función.
	
	\item[Trazado inverso:]  lista de las funciones que se estaban ejecutando y 
	que se
	imprime cuando ocurre un error en tiempo de ejecución.
	
	\index{llamada a función}
	\index{valor de retorno}
	\index{argumento}
	\index{coerción}
	\index{módulo}
	\index{notación punto}
	\index{función}
	\index{definición de función}
	\index{flujo de ejecución}
	\index{parámetro}
	\index{variable local}
	\index{diagrama de pila}
	\index{marco de función}
	\index{marco}
	\index{trazado inverso}
\end{description}

\section{Ejercicios}

\begin{enumerate}
	\item Con un editor de texto cree un guión de Python que se llame 
	pruebame3.py. Escriba en este 
	archivo una función que se llame \verb+nueve_lineas+ que use la función 
	\verb+tres_lineas+ para mostrar nueve líneas 
	en blanco. Enseguida agregue una función que se llame 
	\verb+limpia_pantalla+ que muestre veinticinco líneas en 
	blanco. La última instrucción en su programa debe ser una llamada a 
	\verb+limpia_pantalla+.
	
	\item Mueva la última instrucción del archivo pruebame3.py al inicio del 
	programa, de forma tal que la llamada a la 
	función \verb+limpia_pantalla+ esté antes que la definición de función. 
	Ejecute el programa y registre qué mensaje de 
	error obtiene. ¿Puede establecer una regla sobre las definciones de 
	funciones y las llamadas a función que describa 
	la posición relativa entre ellas en el programa?
	
	\item Escriba una función que imprima la distancia que hay entre dos puntos 
	ubicados sobre el eje X de un plano cartesiano conociendo sus coordenadas 
	horizontales.
	
	\item Escriba una función que imprima la distancia que hay entre dos puntos 
	ubicados sobre el eje Y de un plano cartesiano conociendo sus coordenadas 
	verticales.
	
	\item Escriba una función que imprima la distancia que hay entre dos puntos 
	en un plano coordenado, recordando el teorema de Pitágoras.
	
	\item Tome la solución del último ejercicio del capítulo anterior y 
	conviértala en una función que calcule la nota definitiva de su curso de 
	programación.
 \end{enumerate}
	\clearemptydoublepage  % funciones
% LaTeX source for textbook ``How to think like a computer scientist''
% Copyright (c)  2001  Allen B. Downey, Jeffrey Elkner, and Chris Meyers.

% Permission is granted to copy, distribute and/or modify this
% document under the terms of the GNU Free Documentation License,
% Version 1.1  or any later version published by the Free Software
% Foundation; with the Invariant Sections being "Contributor List",
% with no Front-Cover Texts, and with no Back-Cover Texts. A copy of
% the license is included in the section entitled "GNU Free
% Documentation License".

% This distribution includes a file named fdl.tex that contains the text
% of the GNU Free Documentation License.  If it is missing, you can obtain
% it from www.gnu.org or by writing to the Free Software Foundation,
% Inc., 59 Temple Place - Suite 330, Boston, MA 02111-1307, USA.
\chapter{Condicionales y recursión}

\section{El operador residuo}
\index{operador residuo}
\index{operador!residuo}

El {\bf operador residuo} trabaja con enteros (y expresiones enteras)
calculando el residuo del primer operando cuando se divide por el segundo.
En Python este operador es un signo porcentaje ({\tt\%}). 
La sintaxis es la misma que para los otros operadores:

\beforeverb
\begin{pyconcode}
>>> cociente = 7 / 3
>>> print(cociente)
2
>>> residuo = 7 % 3
>>> print(residuo)
1
\end{pyconcode}
\afterverb
%

Así que 7 dividido por 3 da 2 con residuo 1.

El operador residuo resulta ser sorprendentemente útil. Por ejemplo,
usted puede chequear si un número es divisible por otro ---si
\texttt{x\%y} es cero, entonces \texttt{x} es divisible por \texttt{y}.

Usted también puede extraer el dígito o dígitos más a la derecha
de un número. Por ejemplo, \texttt{x \% 10} entrega el dígito más a la derecha
de  \texttt{x} (en base 10).  Igualmente, \texttt{x \% 100}
entrega los dos últimos dígitos.

\adjustpage{1}

\section{Expresiones booleanas}
\index{expresión Booleana}
\index{expresión!booleana}
\index{operador lógico}
\index{operador!lógico}

El tipo que Python provee para almacenar valores de verdad (cierto o falso)
se denomina bool por el matemático británico George Bool. Él creó el
Álgebra Booleana, que es la base para la aritmética que se usa en 
los computadores modernos.

Sólo hay dos valores booleanos: True (cierto) y False (falso). Las mayúsculas importan, ya que true y false no son valores booleanos.

El operador \texttt{==} compara dos valores y produce una expresión  booleana:

\beforeverb
\begin{pyconcode}
>>> 5 == 5
True
>>> 5 == 6
False
\end{pyconcode}
\afterverb
%

En la primera sentencia, los dos operandos son iguales, así que la expresión
evalúa a True (cierto); en la segunda sentencia, 5 no es igual a 6, así que
obtenemos False (falso).

El operador \texttt{==} es uno de los  {\bf operadores de comparación}; los 
otros son:

\beforeverb
\begin{verbatim}
      x != y               # x no es igual y
      x > y                # x es mayor que y
      x < y                # x es menor que y
      x >= y               # x es mayor o igual a y
      x <= y               # x es menor o igual a y
\end{verbatim}
\afterverb
%

Aunque estas operaciones probablemente son familiares para usted, los
símbolos en Python difieren de los matemáticos. Un error común consiste
en usar un solo signo igual (\texttt{=}) en lugar en un doble signo igual
(\texttt{==}).  Recuerde que \texttt{=} es el operador para la asignación y que
\texttt{==} es el operador para comparación.  Tenga en cuenta que
no existen los signos \texttt{=<} o \texttt{=>}.


\section {Operadores lógicos}
\index{operadores lógicos}
\index{operador!lógicos}

Hay tres  {\bf operadores lógicos}: \texttt{and}, {\tt or} y \texttt{not}.  La 
semántica (el significado) de ellos es similar a su significado en inglés.  Por 
ejemplo, \texttt{x>0 and x<10} es cierto, sólo si  \texttt{x} es mayor a cero
{\em y} menor que 10.

\texttt{n\%2 == 0 or n\%3 == 0} es cierto si {\em alguna} de las condiciones
es cierta, esto es, si el número es divisible por 2 {\em o} por 3.

Finalmente, el operador \texttt{not} niega una expresión booleana,
así que  \texttt{not(x>y)} es cierta si \texttt{(x>y)} es falsa,
esto es, si \texttt{x} es menor o igual a \texttt{y}.

Formalmente, los operandos de los operadores lógicos deben ser
expresiones booleanas, pero Python no es muy formal. Cualquier
número diferente de cero se interpreta como ``cierto.''

\beforeverb
\begin{pyconcode}
>>>  x = 5
>>>  x and 1
1
>>>  y = 0
>>>  y and 1
0
\end{pyconcode}
\afterverb
%
En general, esto no se considera un buen estilo de programación.
Si usted desea comparar un valor con cero, procure codificarlo explícitamente.

\section{Ejecución condicional}
\label{alternative execution}
\index{ramificación condicional}
\index{ejecución condicional }

A fin de escribir programas útiles, casi siempre necesitamos la capacidad
de chequear condiciones y cambiar el comportamiento del programa en 
consecuencia. Las {\bf sentencias condicionales} nos dan este poder. La
 más simple es la sentencia \texttt{if}:

\beforeverb
\begin{pythoncode}
if x > 0:
  print("x es positivo")
\end{pythoncode}
\afterverb
%

La expresión después de la sentencia \texttt{if} se denomina
la {\bf condición}.  Si es cierta, la sentencia de abajo se
ejecuta. Si no lo es, no pasa nada.

\index{sentencia compuesta}
\index{sentencia compuesta!cabecera}
\index{sentencia compuesta!cuerpo}
\index{sentencia compuesta!bloque de sentencias}
\index{sentencia!compuesta}

Como otras sentencias compuestas, la sentencia \texttt{if}
comprende una cabecera y un bloque de sentencias:

\beforeverb
\begin{pythoncode}
CABECERA:
  PRIMERA SENTENCIA
  ...
  ULTIMA SENTENCIA
\end{pythoncode}
\afterverb
%

La cabecera comienza en una nueva línea y termina con dos puntos
seguidos (:).  Las sentencias sangradas o indentadas que
vienen a continuación se denominan el {\bf bloque}.
La primera sentencia sin sangrar marca el fin del bloque. Un bloque
de sentencias dentro de una sentencia compuesta también se denomina 
el {\bf cuerpo} de la sentencia.

\index{bloque}
\index{sentencia!bloque}
\index{cuerpo}

No hay límite en el número de sentencias que pueden aparecer en el cuerpo
de una sentencia, pero siempre tiene que haber, al menos, una.
Ocasionalmente, es útil tener un cuerpo sin sentencias (como un 
hueco para código que aún no se ha escrito). En ese caso se puede
usar la sentencia  \texttt{pass}, que no hace nada.

\index{sentencia pass}
\index{sentencia!pass}



\section{Ejecución alternativa}
\label{alternative execution}

Una segunda forma de sentencia \texttt{if} es la ejecución alternativa
en la que hay dos posibilidades y la condición determina cual de ellas
se ejecuta. La sintaxis luce así:

\beforeverb
\begin{pythoncode}
if x % 2 == 0:
  print(x, "es par")
else:
  print(x, "es impar")
\end{pythoncode}
\afterverb
%

Si el residuo de dividir \texttt{x} por  2 es 0, entonces sabemos 
que \texttt{x} es par, y el programa despliega un mensaje anunciando esto.
Si la condición es falsa, la segunda sentencia se ejecuta. Como la condición, 
que es una expresión booleana, debe ser cierta o falsa, exactamente una
de las alternativas se va a ejecutar. Estas alternativas se denominan
{\bf ramas}, porque, de hecho, son ramas en el flujo de ejecución.

\index{rama}

Yéndonos ``por las ramas'', si usted necesita chequear la paridad
(si un número es par o impar) a menudo, se podría ``envolver''
el código anterior en una función:

\beforeverb
\begin{pythoncode}
def imprimir_paridad(x):
  if x % 2 == 0:
    print(x, "es par")
  else:
    print(x, "es impar")
\end{pythoncode}
\afterverb
%

Para cualquier valor de \texttt{x}, \texttt{imprimir\_paridad} despliega
un mensaje apropiado. Cuando se llama la función, se le puede pasar
cualquier expresión entera como argumento.

\beforeverb
\begin{pyconcode}
>>> imprimir_paridad(17)
>>> imprimir_paridad(y + 1)
\end{pyconcode}
\afterverb
%

\section{Condicionales encadenados}
\index{condicional encadenados}
\index{condicional!encadenados}

Algunas veces hay más de dos posibilidades y necesitamos más de dos 
ramas. Una forma de expresar un cálculo así es un  {\bf condicional encadenado}:

\beforeverb
\begin{pythoncode}
if x < y:
  print(x, "es menor que", y)
elif x > y:
  print(x, "es mayor que", y)
else:
  print(x, "y", y, "son iguales")
\end{pythoncode}
\afterverb
%

\texttt{elif} es una abreviatura de \textit{``else if''}. De nuevo, exactamente
una de las ramas se ejecutará. No hay límite en el número de sentencias 
\texttt{elif}, pero la última rama tiene que ser una sentencia \texttt{else}:

\adjustpage{3}
\beforeverb
\begin{pythoncode}
if eleccion == 'A':
  funcionA()
elif eleccion == 'B':
  funcionB()
elif eleccion == 'C':
  funcionC()
else:
  print "Eleccion incorrecta."
\end{pythoncode}
\afterverb
%

Cada condición se chequea en orden. Si la primera es falsa,
se chequea la siguiente, y así sucesivamente. Si una de ellas
es cierta, se ejecuta la rama correspondiente y la sentencia
termina. Si hay más de una condición cierta, sólo la primera
rama que evalúa a cierto se ejecuta.


\section{Condicionales anidados}

Un condicional también se puede anidar dentro de otro. La tricotomía
anterior se puede escribir así:

\beforeverb
\begin{pythoncode}
if x == y:
  print(x, "y", y, "son iguales")
else:
  if x < y:
    print(x, "es menor que", y)
  else:
    print(x, "es mayor que", y)
\end{pythoncode}
\afterverb
%

El condicional externo contiene dos ramas: la primera contiene una sentencia de 
salida sencilla, la segunda contiene otra sentencia  \texttt{if}, que tiene dos 
ramas propias.Esas dos ramas son sentencias de impresión, aunque también 
podrían ser sentencias condicionales.

Aunque la indentación o sangrado de las sentencias sugiere la estructura, los 
condicionales anidados rápidamente se hacen difíciles de leer. En general, es 
una buena idea evitarlos cada vez que se pueda.

Los operadores lógicos proporcionan formas de simplificar las sentencias
condicionales anidadas. Por ejemplo, podemos reescribir el siguiente
código usando un solo condicional:

\beforeverb
\begin{pythoncode}
if 0 < x:
  if x < 10:
    print("x es un digito positivo.")
\end{pythoncode}
\afterverb
%

La función \texttt{print} se ejecuta solamente si
el flujo de ejecución ha pasado las dos condiciones, así que
podemos usar el operador \texttt{and}:

\beforeverb
\begin{pythoncode}
if 0 < x and x < 10:
  print("x es un digito positivo.")
\end{pythoncode}
\afterverb
%

Esta clase de condiciones es muy común, por esta razón Python proporciona
una sintaxis alternativa que es similar a la notación matemática:

\beforeverb
\begin{pythoncode}
if 0 < x < 10:
  print("x es un digito positivo")
\end{pythoncode}
\afterverb
%

Desde el punto de vista semántico ésta condición es la misma
que la expresión compuesta y que el condicional anidado.


\section{La  sentencia {\tt return} }
\index{sentencia return}
\index{sentencia!return}

La sentencia \texttt{return} permite terminar la ejecución de una función
antes de llegar al final. Una razón para usarla es reaccionar a una
condición de error:

\beforeverb
\begin{pythoncode}
import math

def imprimir_logaritmo(x):
  if x <= 0:
    print "Numeros positivos solamente. Por favor"
    return

  result = math.log(x)
  print "El logaritmo de ",  x ," es ", result
\end{pythoncode}
\afterverb
%

La función \texttt{imprimir\_logaritmo} toma un parámetro
denominado \texttt{x}.  Lo primero que hace es chequear si 
\texttt{x} es menor o igual a 0, caso en el que despliega un 
mensaje de error y luego usa a \texttt{return} para salir de la función.
El flujo de ejecución inmediatamente retorna al punto donde se había llamado
la función, y las líneas restantes de la función no se ejecutan.

Recuerde que para usar una función del módulo matemático (math)
hay que importarlo previamente.


\section{Recursión}
\label{recursion}
\index{recursión}

Hemos mencionado que es legal que una función llame a otra, y usted
ha visto varios ejemplos así. Hemos olvidado mencionar el hecho de 
que una función también puede llamarse a sí misma. Al principio no parece
algo útil, pero resulta ser una de las capacidades más interesantes y
mágicas que un programa puede tener. Por ejemplo, observe la siguiente
función:

\beforeverb
\begin{pythoncode}
def conteo(n):
  if n == 0:
    print("Despegue!")
  else:
    print n
    conteo(n-1)
\end{pythoncode}
\afterverb
%

\texttt{conteo} espera que el parámetro \texttt{n} sea un número entero 
positivo.
Si  \texttt{n} es 0, despliega la cadena, ``Despegue!''.
Si no lo es, despliega  \texttt{n} y luego llama a la función llamada
\texttt{conteo}---ella misma---pasando a  \texttt{n-1} como argumento.

Analizemos lo que sucede si llamamos a esta función así:

\beforeverb
\begin{verbatim}
>>> conteo(3)
\end{verbatim}
\afterverb
%
La ejecución de \texttt{conteo} comienza con \texttt{n = 3}, y como
\texttt{n} no es  0, despliega el valor 3, y se llama a sí misma ...

\begin{quote}
	La ejecución de \texttt{conteo} comienza con \texttt{n = 2}, y como
	\texttt{n} no es 0, despliega el valor 2, y se llama a si misma ...
	
	\begin{quote}
		La ejecución de \texttt{conteo} comienza con \texttt{n = 1}, y como
		\texttt{n} no es 0, despliega el valor 1, y se llama a sí misma ...
		
		\begin{quote}
			La ejecución de \texttt{conteo} comienza con \texttt{n = 0}, y como
			\texttt{n} es 0, despliega la cadena ``Despegue!'' y retorna 
			(finaliza).
		\end{quote}
		
		El  \texttt{conteo} que recibió \texttt{n = 1} retorna.
	\end{quote}
	
	El \texttt{conteo} que recibió \texttt{n = 2} retorna.
\end{quote}

El \texttt{conteo} que recibió \texttt{n = 3} retorna.

Y  el flujo de ejecución regresa a \texttt{\_\_main\_\_} (vaya viaje!).  Así que, la 
salida total luce así:

\beforeverb
\begin{pyconcode}
3
2
1
Despegue!
\end{pyconcode}
\afterverb
%

Como otro ejemplo, utilizaremos nuevamente las funciones \texttt{nueva\_linea} y
\texttt{tres\_lineas}:

\beforeverb
\begin{verbatim}
def nueva_linea():
  print()

def tres_lineas():
  nueva_linea()
  nueva_linea()
  nueva_linea()
\end{verbatim}
\afterverb
%

Este trabajo no sería de mucha ayuda si quisiéramos desplegar 2 líneas o 106.
Una mejor alternativa sería:

\beforeverb
\begin{pythoncode}
def n_lineas(n):
  if n > 0:
    print()
    n_lineas(n-1)
\end{pythoncode}
\afterverb
%

Esta función es similar a \texttt{conteo}; en tanto \texttt{n} sea 
mayor a 0, despliega una nueva línea y luego se llama a sí misma
para desplegar \texttt{n-1} líneas adicionales.  Así, el número total de
nuevas líneas es \texttt{1 + (n - 1)} que, si usted verifica con álgebra, 
resulta ser \texttt{n}.

El proceso por el cual una función se llama a sí misma es la {\bf recursión}, y 
se dice que estas funciones son recursivas.

\index{recursión}
\index{función!recursiva}


\section{Diagramas de pila para funciones recursivas}
\index{diagrama de pila}
\index{marco de función}
\index{marco}

En la Sección~\ref{stackdiagram}, usamos un diagrama de pila para
representar el estado de un programa durante un llamado de función.
La misma clase de diagrama puede ayudarnos a interpretar una función
recursiva.

Cada vez que una función se llama, Python crea un nuevo marco de función
que contiene los parámetros y variables locales de ésta. Para una
función recursiva, puede existir más de un marco en la pila al mismo
tiempo.

Este es el diagrama de pila para \texttt{conteo} llamado con \texttt{n = 3}:

\beforefig
\centerline{\includegraphics{illustrations/stack2.eps}}
\afterfig

Como siempre, el tope de la pila es el marco para \texttt{\_\_main\_\_}.
Está vacío porque no creamos ninguna variable en {\tt \_\_main\_\_} ni le 
pasamos parámetros.

Los cuatro marcos de \texttt{conteo} tienen diferentes valores para
el parámetro \texttt{n}. El fondo de la pila, donde  \texttt{n=0}, se denomina
el {\bf caso base }.  Como no hace una llamada recursiva, no hay mas marcos.

\index{case base}
\index{recursión!caso base}

\section{Recursión infinita}
\index{recursión infinita}
\index{recursión!infinita}
\index{error de tiempo de ejecución}
\index{error!de tiempo de ejecución}
\index{trazado inverso}

Si una función recursiva nunca alcanza un caso base va a hacer llamados 
recursivos por siempre y el programa nunca termina. Esto se conoce como {\bf 
recursión infinita}, y, generalmente, no se considera una buena idea. Aquí hay 
un programa minimalista con recursión infinita:

\beforeverb
\begin{pythoncode}
def recurrir():
  recurrir()
\end{pythoncode}
\afterverb
%

En la mayoría de ambientes de programación un programa con recursión 
infinita no corre realmente para siempre. Python reporta un mensaje de error
cuando alcanza la máxima profundidad de recursión:

\beforeverb
\begin{pyconcode}
  File "<stdin>", line 2, in recurrir
  ...
  File "<stdin>", line 2, in recurrir
RuntimeError: Maximum recursion depth exceeded
\end{pyconcode}
\afterverb
%

Este trazado inverso es un poco más grande que el que vimos en el 
capítulo anterior. Cuando se presenta el error, ¡hay más de
100 marcos de \texttt{recurrir} en la pila!.

\section{Entrada por el teclado}

Los programas que hemos escrito son un poco toscos ya que no aceptan 
entrada de un usuario. Sólo hacen la misma operación todo el tiempo.

Python proporciona funciones primitivas que obtienen entrada desde el 
teclado. La más sencilla se llama \texttt{input}. Cuando esta función
se llama el programa se detiene y espera a que el usuario digite algo.
Cuando el usuario digita la tecla \texttt{Enter}, el programa retoma
la ejecución y la función \texttt{input} retorna lo que el usuario digitó
como una cadena (\texttt{string}):

\beforeverb
\begin{verbatim}
>>> entrada = input ()
Que esta esperando?
>>> print(entrada)
Que esta esperando?
\end{verbatim}
\afterverb
%

Antes de llamar a  \texttt{input} es una muy buena idea desplegar
un mensaje diciéndole al usuario qué digitar. Este mensaje se denomina
indicador de entrada ({\bf prompt} en inglés).  
Podemos dar un argumento prompt a \texttt{input}:

\index{prompt}

\beforeverb
\begin{pyconcode}
>>> nombre = input("Cual es tu nombre? ")
Cual es tu nombre? Arturo, Rey de los Bretones!
>>> print(nombre)
Arturo, Rey de los Bretones!
\end{pyconcode}
\afterverb
%

Si esperamos que la respuesta sea un entero, tenemos que usar la función 
\texttt{int} para convertir explícitamente lo que nos devuelve \texttt{input}:

\beforeverb
\begin{pythoncode}
prompt = "¿Cual es la velocidad de una golondrina sin carga?\n"
velocidad = int(input(prompt))
\end{pythoncode}
\afterverb
%

Si el usuario digita una cadena de dígitos, éstos se convierten
a un entero que se asigna a \texttt{velocidad}.  Desafortunadamente, si el 
usuario digita un carácter que no sea un dígito, el programa se aborta:

\beforeverb
\begin{pyconcode}
>>> prompt = "¿Cual es la velocidad una golondrina sin carga?\n"
>>> velocidad = input(prompt)
¿Cual es la velocidad una golondrina sin carga?
¿Que quiere decir, una golondria Africana o Europea?
Traceback (most recent call last):
  File "<stdin>", line 1, in <module>
ValueError: invalid literal for int() with base 10: '¿Que quiere decir, una 
golondria Africana o Europea?'
\end{pyconcode}
\afterverb
%

\section{Glosario}

\begin{description}
	\item[Operador residuo:]  operador que se denota con un  signo porcentaje
	(\texttt{\%}), y trabaja sobre enteros produciendo el residuo de un número 
	al 
	dividirlo por otro.
	
	\item[Expresión booleana:]  expresión cierta o falsa.
	
	\item[Operador de comparación:] uno de los operadores que compara dos 
	valores: 
	\texttt{==}, \texttt{!=}, \texttt{>}, \texttt{<}, \texttt{>=}, y 
	\texttt{<=}.
	
	\item[Operador lógico:] uno de los operadores que combina
	expresiones booleanas: \texttt{and}, \texttt{or}, y \texttt{not}.
	
	\item[Sentencia condicional:]  sentencia que controla el flujo
	de ejecución dependiendo de alguna  condición.
	
	\item[Condición:] la expresión booleana en una sentencia condicional 
	que determina que rama se ejecuta.
	
	\item[Sentencia compuesta:]  es la sentencia que comprende una
	cabecera y un cuerpo. La cabecera termina con dos puntos seguidos (:). 
	El cuerpo se sangra o indenta con respecto a la cabecera.
	
	\item[Bloque:] grupo de sentencias consecutivas con la misma
	indentación.
	
	\item[Cuerpo:] el bloque, en una sentencia compuesta, que va 
	después de la cabecera.
	
	\item[Anidamiento:]  situación en la que hay una estructura dentro de otra,
	tal como una sentencia  condicional dentro de una rama
	de otra sentencia condicional.
	
	\item[Recursión:]  es el proceso de llamar la función que se está
	ejecutando actualmente.
	
	\item[Caso base:]  corresponde a una rama de la sentencia  condicional 
	dentro 
	de una función recursiva, que no hace un llamado recursivo.
	
	\item[Recursión infinita:]  función que se llama a sí misma
	recursivamente sin alcanzar nunca el caso base. En Python una 
	recursión infinita eventualmente causa un error en tiempo de
	ejecución.
	
	\item[Prompt (indicador de entrada):]  una pista visual que le indica al 
	usuario que
	digite alguna información.
	
	\index{operador residuo}
	\index{expresión booleana}
	\index{expresión!booleana}
	\index{sentencia condicional }
	\index{sentencia!condicional}
	\index{condición}
	\index{sentencia compuesta}
	\index{rama}
	\index{cuerpo}
	\index{bloque}
	\index{anidamiento}
	\index{recursión}
	\index{caso base}
	\index{recursión infinita}
	\index{prompt}
\end{description}

\section{Ejercicios}
\begin{enumerate}
	\item Evalue la expresión \verb+7 % 0+. Explique lo que ocurre.
	
	\item Envuelva el código que viene a continuación en una función llamada 
	\verb+comparar(x, y)+. Llame a la función comparar tres veces: 
	una en la que el primer argumento sea menor que el segundo, otra en la que 
	aquel sea mayor que éste, y 
	una tercera en la que los argumentos sean iguales.
	
	\begin{pythoncode} 
if x < y:
    print(x, "es menor que", y)
elif x > y:
    print(x, "es mayor que", y)
else:
    print(x, "y", y, "son iguales")
	\end{pythoncode}
	
	\item Copie este programa en un archivo llamado \texttt{tabladeverdad.py}:
	\begin{pythoncode}
def tabladeverdad(expresion):
    print(" p      q      %s"  % expresion)
    longitud = len(" p      q      %s"  % expresion)
    print(longitud * "=")

    for p in True, False:
        for q in True, False:
            print("%-7s %-7s %-7s" % (p, q, eval(expresion)))
	
	\end{pythoncode}
	Pruébelo con el llamado \verb+tabladeverdad("p or q")+. Ahora ejecútelo con 
	las siguientes expresiones:

	\begin{enumerate}
		\item \texttt{not(p or q)}
		\item \texttt{p and q}
		\item \texttt{not(p and q)}
		\item \texttt{not(p) or not(q)}
		\item \texttt{not(p) and not(q)}
	\end{enumerate}
	
	¿Cuales de estas expresiones tienen el mismo valor de verdad (son 
	logicamente equivalentes)?
\end{enumerate}

	\clearemptydoublepage  % condicionales y recursion 
% LaTeX source for textbook ``How to think like a computer scientist''
% Copyright (c)  2001  Allen B. Downey, Jeffrey Elkner, and Chris Meyers.

% Permission is granted to copy, distribute and/or modify this
% document under the terms of the GNU Free Documentation License,
% Version 1.1  or any later version published by the Free Software
% Foundation; with the Invariant Sections being "Contributor List",
% with no Front-Cover Texts, and with no Back-Cover Texts. A copy of
% the license is included in the section entitled "GNU Free
% Documentation License".

% This distribution includes a file named fdl.tex that contains the text
% of the GNU Free Documentation License.  If it is missing, you can obtain
% it from www.gnu.org or by writing to the Free Software Foundation,
% Inc., 59 Temple Place - Suite 330, Boston, MA 02111-1307, USA.

\chapter{Funciones fructíferas}
\label{funcReturn}

\section{Valores de retorno}
\index{valor de retorno}

Algunas de las funciones primitivas que hemos usado, como las matemáticas,
entregan resultados. El llamar a estas funciones genera un valor nuevo, que
usualmente asignamos a una variable o usamos como parte de una expresión.

\beforeverb
\begin{pythoncode}
e = math.exp(1.0)
altura = radio * math.sin(angulo)
\end{pythoncode}
\afterverb
%

Pero hasta ahora ninguna de las funciones que hemos escrito ha retornado
un valor.

En este capítulo vamos a escribir funciones que retornan valores, los
cuales denominamos {\bf funciones fructíferas}, o provechosas\footnote{En 
algunos libros de programación, las \textit{funciones} que desarrollamos en el 
capítulo anterior se denominan \textit{procedimientos} y las que veremos en 
este capítulo sí se denominan \textit{funciones}, ya que los lenguajes de 
programación usados para enseñar (como Pascal) hacían la distinción. Muchos 
lenguajes de programación vigentes (incluido Python y C) no diferencian
sintacticamente entre procedimientos y funciones, por eso usamos esta 
terminología}. El primer ejemplo es \texttt{area}, que retorna el área
de un círculo dado su radio:

\beforeverb
\begin{pythoncode}
import math

def area(radio):
  temp = math.pi * radio**2
  return temp
\end{pythoncode}
\afterverb
%

Ya nos habíamos topado con la sentencia \texttt{return} antes, pero, en una
función fructífera, la sentencia \texttt{return} incluye un {\bf valor de 
retorno}.  
Esta sentencia significa: ``Retorne inmediatamente de esta función
y use la siguiente expresión como un valor de retorno.''
La expresión proporcionada puede ser arbitrariamente compleja, así que 
podríamos escribir esta función más concisamente:

\beforeverb
\begin{pythoncode}
def area(radio):
  return math.pi * radio**2
\end{pythoncode}
\afterverb
%

Por otro lado, las {\bf  variables temporales}, como \texttt{temp}, a  menudo
permiten depurar los programas más fácilmente.

\index{variable temporal}
\index{variable!temporal}

Algunas veces es muy útil tener múltiples sentencias return, ubicadas
en ramas distintas de un condicional:

\beforeverb
\begin{pythoncode}
def valorAbsoluto(x):
  if x < 0:
    return -x
  else:
    return x
\end{pythoncode}
\afterverb
%

Ya que estas sentencias \texttt{return} están en un condicional alternativo,
sólo una será ejecutada. Tan pronto como esto suceda, la función termina
sin ejecutar las sentencias que siguen.

El código que aparece después de la sentencia \texttt{return}, o en un lugar
que el flujo de ejecución nunca puede alcanzar, se denomina {\bf código muerto}.

\index{código muerto}

En una función fructífera es una buena idea garantizar que toda ruta posible
de ejecución del programa llegue a una sentencia \texttt{return}.  Por ejemplo:

\beforeverb
\begin{pythoncode}
def valorAbsoluto(x):
  if x < 0:
    return -x
  elif x > 0:
    return x
\end{pythoncode}
\afterverb
%

Este programa no es correcto porque si \texttt{x} llega a ser 0,
ninguna condición es cierta y la función puede terminar sin alcanzar
una sentencia \texttt{return}. En este caso el valor de retorno que 
Python entrega es un valor especial denominado \texttt{None}:

\index{None}

\beforeverb
\begin{pyconcode}
>>> print valorAbsoluto(0)
None
\end{pyconcode}
\afterverb

\section{Desarrollo de programas}
\label{program development}
\index{andamiaje}

En este momento usted debería ser capaz de leer funciones completas
y deducir lo que hacen. También, si ha realizado los ejercicios,
ya ha escrito algunas funciones pequeñas. A medida en que usted
escriba funciones más grandes puede empezar a tener una dificultad
mayor, especialmente con los errores semánticos y de tiempo de
ejecución.

Para desarrollar programas cada vez más complejos, vamos a sugerir
una técnica denominada {\bf desarrollo incremental}.  El objetivo
del desarrollo incremental es evitar largas sesiones de depuración
mediante la adición y prueba de una pequeña cantidad de código 
en cada paso.

\index{desarrollo incremental }
\index{desarrollo!incremental}

Como ejemplo, suponga que usted desea hallar la distancia entre
dos puntos dados por las coordenadas  $(x_1, y_1)$ y $(x_2, y_2)$.
Por el teorema de Pitágoras, la distancia se calcula con:

\begin{equation}
distancia = \sqrt{(x_2 - x_1)^2 + (y_2 - y_1)^2}
\end{equation}
%

El primer paso es considerar cómo luciría la función \texttt{distancia}
en Python. En otras palabras, ¿cuales son las entradas (parámetros)
y cual es la salida (valor de retorno)?

En este caso, los dos puntos son las entradas, que podemos representar
usando cuatro parámetros. El valor de retorno es la distancia, que es
un valor de punto flotante.

Ya podemos escribir un borrador de la función:

\beforeverb
\begin{pythoncode}
def distancia(x1, y1, x2, y2):
  return 0.0
\end{pythoncode}
\afterverb
%

Obviamente, esta versión de la función no calcula distancias; siempre
retorna cero. Pero es correcta sintácticamente  y puede correr, 
lo que implica que la podemos probar antes de que la hagamos
más compleja.

Para probar la nueva función la llamamos con valores simples:

\beforeverb
\begin{pyconcode}
>>> distancia(1, 2, 4, 6)
0.0
\end{pyconcode}
\afterverb
%

Escogemos estos valores de forma que la distancia horizontal sea 3
y la vertical 4; de esta forma el resultado es 5 (la hipotenusa
de un triángulo con medidas 3-4-5). Cuando probamos una función
es fundamental conocer algunas respuestas correctas.

En este punto hemos confirmado que la función está bien sintácticamente,
y que podemos empezar a agregar líneas de código. Después de cada
cambio, probamos la función otra vez. Si hay un error, sabemos
dónde debe estar ---en la última línea que agregamos.

Un primer paso lógico en este cómputo es encontrar las diferencias
$x_2 - x_1$ y $y_2 - y_1$.  Almacenaremos estos valores en
variables temporales llamadas \texttt{dx} y \texttt{dy} y los imprimiremos.

\beforeverb
\begin{pythoncode}
def distancia(x1, y1, x2, y2):
  dx = x2 - x1
  dy = y2 - y1
  print "dx es", dx
  print "dy es", dy
  return 0.0
\end{pythoncode}
\afterverb
%

Si la función trabaja bien, las salidas deben ser 3 y 4. Si es así,
sabemos que la función está obteniendo los parámetros correctos y
calculando el primer paso correctamente. Si no ocurre ésto, entonces
hay unas pocas líneas para chequear.

Ahora calculamos la suma de los cuadrados de \texttt{dx} y \texttt{dy}:

\beforeverb
\begin{pythoncode}
def distancia(x1, y1, x2, y2):
  dx = x2 - x1
  dy = y2 - y1
  discuadrado = dx**2 + dy**2
  print "discuadrado es: ", discuadrado
  return 0.0
\end{pythoncode}
\afterverb
%

Note que hemos eliminado las sentencias \texttt{print} que teníamos en el paso
anterior. Este código se denomina {\bf andamiaje} porque es útil para
construir el programa pero no hace parte del producto final.

De nuevo, corremos el programa y chequeamos la salida (que debe ser 25).

Finalmente, si importamos el módulo math, podemos usar la función
\texttt{sqrt} para calcular y retornar el resultado:

\beforeverb
\begin{pythoncode}
def distancia(x1, y1, x2, y2):
  dx = x2 - x1
  dy = y2 - y1
  discuadrado = dx**2 + dy**2
  resultado = math.sqrt(discuadrado)
  return resultado
\end{pythoncode}
\afterverb
%

Si esto funciona bien, usted ha terminado.  Si no, se podría
imprimir el valor de  \texttt{resultado} antes de la sentencia return.

Recapitulando, para empezar, usted debería agregar solamente una línea o dos cada vez.

A medida que gane más experiencia  podrá escribir y depurar trozos
mayores. De cualquier forma el proceso de desarrollo incremental
puede evitarle mucho tiempo de depuración.

Los aspectos claves del proceso son:

\begin{enumerate}
	\item Empezar con un programa correcto y hacer pequeños cambios 
	incrementales.
	Si en cualquier punto hay un error, usted sabrá exactamente donde está.
	
	\item Use variables temporales para almacenar valores intermedios de manera
	que se puedan imprimir y chequear.
	
	\item Ya que el programa esté corriendo, usted puede remover parte del 
	andamiaje o consolidar múltiples sentencias en expresiones compuestas, 
	pero sólo si ésto no dificulta la lectura del programa.
\end{enumerate}


\section{Composición}
\index{composición}
\index{función!composición}

Como usted esperaría, se puede llamar una función fructífera
desde otra. Esta capacidad es la {\bf composición}.

Como ejemplo vamos a escribir una función que toma dos puntos:
el centro de  un círculo y un punto en el perímetro, y que 
calcule el área total del círculo.

Asuma que el punto central está almacenado en las variables \texttt{xc} y
\texttt{yc}, y que el punto perimetral está en \texttt{xp} y \texttt{yp}. El
primer paso es encontrar el radio del círculo, que es la distancia
entre los dos puntos. Afortunadamente, hay una función, {\tt
distancia}, que hace eso:

\beforeverb
\begin{pythoncode}
radio = distancia(xc, yc, xp, yp)
\end{pythoncode}
\afterverb
%

El segundo paso es encontrar el área de un círculo con dicho radio y 
retornarla:

\beforeverb
\begin{pythoncode}
resultado = area(radio)
return resultado
\end{pythoncode}
\afterverb
%

Envolviendo todo en una función obtenemos:

\beforeverb
\begin{pythoncode}
def area2(xc, yc, xp, yp):
  radio = distancia(xc, yc, xp, yp)
  resultado = area(radio)
  return resultado
\end{pythoncode}
\afterverb
%

Llamamos a esta función \texttt{area2} para distinguirla de la función {\tt
area} definida previamente.  Solo puede haber una función con un nombre
dado dentro de un módulo.

Las variables temporales \texttt{radio} y \texttt{area} son útiles para
desarrollar y depurar, pero una vez que el programa está funcionando
podemos hacer la función más concisa componiendo las llamadas a
funciones:

\beforeverb
\begin{pythoncode}
def area2(xc, yc, xp, yp):
  return area(distancia(xc, yc, xp, yp))
\end{pythoncode}
\afterverb
%


\section{Funciones booleanas}
\label{boolean}
\index{función booleana}
\index{función booleana}

Las funciones que pueden retornar un valor booleano son convenientes para
ocultar chequeos complicados adentro de funciones. Por ejemplo:

\beforeverb
\begin{pythoncode}
def esDivisible(x, y):
  if x % y == 0:
    return True       #  es cierto
  else:
    return False      # es falso
\end{pythoncode}
\afterverb
%

El nombre de esta función es \texttt{esDivisible}.  Es muy usual nombrar
las funciones booleanas con palabras o frases que suenan como preguntas de sí o 
no (que tienen como respuesta un sí o un no).  {\tt esDivisible} retorna  
\texttt{True} ó \texttt{False} para indicar si x es divisible exactamente por y.

Podemos hacerla más concisa tomando ventaja del hecho de que una
condición dentro de una sentencia \texttt{if} es una expresión booleana. 
Podemos retornarla directamente, evitando completamente el \texttt{if}:

\beforeverb
\begin{pythoncode}
def esDivisible(x, y):
  return x % y == 0
\end{pythoncode}
\afterverb
%

Esta sesión muestra la nueva función en acción:

\beforeverb
\begin{pyconcode}
>>>   esDivisible(6, 4)
False
>>>   esDivisible(6, 3)
True
\end{pyconcode}
\afterverb
%

Las funciones booleanas se usan a menudo en las sentencias condicionales:

\beforeverb
\begin{pythoncode}
if esDivisible(x, y):
  print "x es divisible por y"
else:
  print "x no es divisible por y"
\end{pythoncode}
\afterverb
%

Puede parecer tentador escribir algo como:

\beforeverb
\begin{pythoncode}
if esDivisible(x, y) == True:
\end{pythoncode}
\afterverb
%

Pero la comparación extra es innecesaria.



\section{Más recursión}
\index{recursión}
\index{lenguaje completo}
\index{lenguaje!completo}
\index{Turing, Alan}
\index{Turing, Tésis de }

Hasta aquí, usted sólo ha aprendido un pequeño subconjunto de Python,
pero podría interesarle saber que este subconjunto es un lenguaje
de programación {\em completo}, lo que quiere decir que cualquier 
cosa que pueda ser calculada puede ser expresada en este subconjunto.
Cualquier programa escrito alguna vez puede ser reescrito usando solamente
las características que usted ha aprendido hasta ahora (de hecho, 
necesitaría algunos comandos mas para manejar dispositivos como
el teclado, el ratón, los discos, etc., pero eso sería todo).

Demostrar esta afirmación no es un ejercicio trivial y fue logrado por 
Alan Turing, uno de los primeros científicos de la computación (algunos
dirían que el era un matemático, pero la mayoría de los científicos
pioneros de la computación  eran matemáticos). Esto se conoce como la Tesis
de Turing. Si usted toma un curso de Teoría de la Computación tendrá
la oportunidad de ver la demostración.

\adjustpage{2}

Para darle una idea de lo que puede hacer con las herramientas que ha
aprendido, vamos a evaluar unas pocas funciones matemáticas definidas
recursivamente. 

Una definición recursiva es similar a una circular, ya que éstas 
contienen una referencia al concepto que se pretende definir. 
Una definición circular verdadera no es muy útil:

\begin{description}

\item[frabjuoso:] un adjetivo usado para describir algo que es frabjuoso.

\end{description}

\index{frabjuoso}
\index{definición circular}
\index{definición!circular}

Si usted viera dicha definición en el diccionario, quedaría confundido.
Por otro lado, si encontrara la definición de la función factorial 
hallaría algo como esto:

\vspace{-0.35in}
\begin{eqnarray*}
&&  0! = 1 \\
&&  n! = n (n-1)!
\end{eqnarray*}
\vspace{-0.25in}

Esta definición dice que el factorial de 0 es 1, y que el factorial
de cualquier otro valor, $n$, es $n$ multiplicado por el factorial de $n-1$.

Así que $3!$ es 3 veces $2!$, que  es 2 veces $1!$, que es 1 vez
$0!$. Juntando todo esto, $3!$ es igual a 3 veces 2 veces 1 vez 1,
lo que da 6.

\index{función factorial}
\index{función!factorial}

Si usted puede escribir una definición recursiva de algo, usualmente
podrá escribir un programa para evaluarlo. El primer paso es decidir
cuales son los parámetros para esta función. Con un poco de esfuerzo
usted concluiría que \texttt{factorial} recibe un único parámetro:

\beforeverb
\begin{pythoncode}
def factorial(n):
\end{pythoncode}
\afterverb
%
Si el argumento es 0, todo lo que hacemos es retornar 1:

\beforeverb
\begin{pythoncode}
def factorial(n):
  if n == 0:
    return 1
\end{pythoncode}
\afterverb
%

Sino, y ésta es la parte interesante, tenemos que hacer una
llamada recursiva para encontrar el factorial de $n-1$ y, entonces,  
multiplicarlo por $n$:

\beforeverb
\begin{pythoncode}
def factorial(n):
  if n == 0:
    return 1
  else:
    recur = factorial(n-1)
    da = n * recur
    return da
\end{pythoncode}
\afterverb
%

El flujo de ejecución de este programa es similar al flujo de {\tt
conteo} en la Sección~\ref{recursion}.  Si llamamos a \texttt{factorial} con 
el valor 3:

\adjustpage{1}

Como  3 no es 0, tomamos la segunda rama y calculamos el factorial
de \texttt{n-1}...

\begin{quote}
Como  2 no es 0, tomamos la segunda rama y calculamos el factorial de
\texttt{n-1}...


  \begin{quote}
  Como  1 no es 0, tomamos la segunda rama y calculamos el factorial
  de \texttt{n-1}...


    \begin{quote}
    Como  0 {\em es} 0, tomamos la primera rama y retornamos 1
    sin hacer más llamados recursivos.
    \end{quote}


  El valor de retorno  (1) se multiplica por $n$, que es 1, y el 
  resultado se retorna.
  \end{quote}


El valor de retorno (1) se multiplica por $n$, que es 2, y el 
resultado se retorna.
\end{quote}


El valor de retorno (2) se multiplica por $n$, que es 3, y el resultado, 6,
se convierte en el valor de retorno del llamado de función que empezó todo
el proceso.

Así queda el diagrama de pila para esta secuencia de llamados de función:

\vspace{0.1in}
\beforefig
\centerline{\includegraphics{illustrations/stack3.eps}}
\afterfig
\vspace{0.1in}

Los valores de retorno mostrados se pasan hacia arriba a través de la pila.
En cada marco, el valor de retorno es el valor de  \texttt{da},
 que es el producto de \texttt{n} y \texttt{recur}.

Observe que en el último marco, las variables locales \texttt{recur} y \texttt{da} 
no existen porque la rama que las crea no se ejecutó.


\section{El salto de fe}
\index{recursión}
\index{salto de fe}

Seguir el flujo de ejecución es una forma de leer programas, pero
rápidamente puede tornarse algo laberíntico. Una alternativa es
lo que denominamos hacer el ``salto de fe.'' Cuando usted llega a un 
llamado de función, en lugar de seguir el flujo de ejecución, se
{\em asume} que la función trabaja correctamente y retorna el 
valor apropiado.

De hecho, usted ya está haciendo el salto de fe cuando usa
las funciones primitivas. Cuando llama a  \texttt{math.cos} ó a \texttt{math.exp},
no está examinando las implementaciones de estas funciones.
Usted sólo asume que están correctas porque los que escribieron el
módulo math son buenos programadores.

Lo mismo se cumple para una de sus propias funciones. Por ejemplo,
en la  Sección~\ref{boolean}, escribimos una función llamada \texttt{esDivisible}
que determina si un número es divisible por otro.  Una vez que
nos hemos convencido de que esta función es correcta ---probándola
y examinando el código---podemos usarla sin mirar el código nuevamente.

\adjustpage{-1}

Lo mismo vale para los programas recursivos. Cuando usted llega a una
llamada recursiva, en lugar de seguir el flujo de ejecución, debería
asumir que el llamado recursivo funciona (retorna el resultado correcto)
y luego preguntarse, ``Asumiendo que puedo encontrar el factorial de $n-1$, 
¿puedo calcular el factorial de $n$?''  En este caso, es claro que
se puede lograr, multiplicándolo por $n$.

Por supuesto que es un poco raro asumir que la función trabaja correctamente
cuando ni siquiera hemos terminado de escribirla, ¡por eso es que denominamos
a esto el salto de fe!.


\section{Un ejemplo más}
\label{one more example}

En el ejemplo anterior usábamos variables  temporales para desplegar
los pasos y depurar el código más fácilmente, pero podríamos
ahorrar unas cuantas líneas:

\beforeverb
\begin{pythoncode}
def factorial(n):
  if n == 0:
    return 1
  else:
    return n * factorial(n-1)
\end{pythoncode}
\afterverb
%
Desde ahora, vamos a usar esta forma más compacta, pero le recomendamos
que use la forma más explícita mientras desarrolla las funciones.
Cuando estén terminadas y funcionando, con un poco de inspiración se pueden compactar.

\index{La función de Fibonacci}

Después de  \texttt{factorial}, el ejemplo más común de  función
matemática, definida recursivamente, es la serie de  \texttt{fibonacci}, que
tiene la siguiente definición:

\vspace{-0.25in}
\begin{eqnarray*}
&& fibonacci(0) = 1 \\
&& fibonacci(1) = 1 \\
&& fibonacci(n) = fibonacci(n-1) + fibonacci(n-2);
\end{eqnarray*}
%
Traducida a Python, luce así:

\beforeverb
\begin{pythoncode}
def fibonacci (n):
  if n == 0 or n == 1:
    return 1
  else:
    return fibonacci(n-1) + fibonacci(n-2)
\end{pythoncode}
\afterverb
%
Si usted intenta seguir el flujo de ejecución de fibonacci, incluso
para valores pequeños de  $n$, le va a doler la cabeza. 
Pero, si seguimos el salto de fe, si asumimos que los dos
llamados recursivos funcionan correctamente, es claro que
el resultado correcto es la suma de éstos dos.

\adjustpage{-1}

\section{Chequeo de tipos}
\index{chequeo de tipos}
\index{chequeo de errores}
\index{función factorial}

¿Qué pasa si llamamos a \texttt{factorial} y le pasamos a 1.5 como argumento?

\beforeverb
\begin{pyconcode}
>>> factorial (1.5)
RuntimeError: Maximum recursion depth exceeded
\end{pyconcode}
\afterverb
%

Parece recursión  infinita. ¿Cómo puede darse?  Hay un caso 
base ---cuando \texttt{n == 0}.  El problema reside en que
los valores de \texttt{n} se {\em saltan} al caso base .

\index{recursión infinita}
\index{recursión!infinita}

En la primera llamada recursiva el valor de  \texttt{n} es 0.5.
En la siguiente es -0.5.  Desde allí se hace cada vez más 
pequeño, pero nunca será 0.

Tenemos dos opciones, podemos intentar generalizar la función
 \texttt{factorial} para que trabaje con números de punto flotante, o
podemos chequear el tipo del parámetro que llega. La primera opción
se denomina en matemática la función gama y está fuera del 
alcance de este libro. Optaremos por la segunda.

\index{función gama}

Podemos usar la función  \texttt{type} para comparar el tipo
del parámetro al tipo de un valor entero conocido (como 1).  
Mientras estamos en eso también aseguraremos que el parámetro
sea positivo:

\beforeverb
\begin{pythoncode}
def factorial (n):
  if type(n) != type(1):
    print "Factorial solo esta definido para enteros."
    return -1
  elif n < 0:
    print "Factorial solo esta definido para positivos"
    return -1
  elif n == 0:
    return 1
  else:
    return n * factorial(n-1)
\end{pythoncode}
\afterverb
%

Ahora tenemos tres casos base. El primero atrapa a los valores
que no son enteros, el segundo a los enteros negativos.
En ambos casos el programa imprime un mensaje de error y retorna
un valor especial, -1, para indicar que algo falló:

\beforeverb
\begin{pyconcode}
>>> factorial("pedro")
Factorial solo esta definido para enteros.
-1
>>> factorial(-2)
Factorial solo esta definido para positivos.
-1
\end{pyconcode}
\afterverb
%

Si pasamos los dos chequeos, tenemos la garantía de que $n$ es 
un número entero  positivo, y podemos probar que la recursión termina.

Este programa demuestra el uso de un patrón denominado {\bf guarda}.
Los primeros dos condicionales actúan como guardas, protegiendo al 
código interno de los valores que pueden causar un error. Las 
guardas hacen posible demostrar que el código es correcto.

\section{Pruebas unitarias con doctest}

Con funciones fructíferas podemos realizar pruebas unitarias. Por ejemplo, 
la función área de un cuadrado puede adornarse con un bloque 
de comentarios con triple comillas, que explica su propósito:

\beforeverb
\begin{pythoncode}
def area(lado):
    """ Calcula el area de un cuadrado
        Parámetros:
            radio: número
    """
    return lado**2
\end{pythoncode}
\afterverb

Si al bloque le agregamos una línea probando el llamado de la función, 
seguida del valor de retorno que debe entregar:

\beforeverb
\begin{pythoncode}
def area(lado):
    """ Calcula el area de un cuadrado
        Parámetros:
            radio: número
        Pruebas:
        >>> area(1)
        1        
    """
    return lado**2
\end{pythoncode}
\afterverb

Logramos obtener una función que se puede probar en un caso particular. El 
módulo doctest de Python permite ejecutar automáticamente los casos de 
prueba que tengamos en las funciones agregando al final del guión
su importación y el llamado de la función testmod(), como se ilustra a
continuación con la función area, ahora con cuatro casos de prueba:

\beforeverb
\begin{pythoncode}
def area(lado):
    """ Calcula el area de un cuadrado
        Parámetros:
            radio: número
        Pruebas:
        >>> area(1)
        1
        >>> area(2)
        4
        >>> area(4)
        16
        >>> area(10)
        100
        
    """
    return lado**2

if __name__ == '__main__':
    import doctest
    doctest.testmod()
\end{pythoncode}
\afterverb

Si se ejecuta el guión se ejecutarán todas las pruebas unitarias de todas
las funciones, esto nos permite atrapar errores rapidamente y corregirlos. 
En Unix/Linux, al ejecutar \verb+python -m doctest -v guión.py+ se logran 
ejecutar los casos de prueba y visualizar detalladamente en la pantalla.

\section{Glosario}

\begin{description}

\item[Función fructífera:] función que retorna un resultado.

\item[Valor de retorno:]  el valor que entrega como resultado un llamado de
función.

\item[Variable temporal:]  variable usada para almacenar un 
valor intermedio en un cálculo complejo.

\item[Código muerto:]  parte de un programa que nunca puede ser ejecutada, 
a menudo porque aparece después de una sentencia  \texttt{return}.

\item[\texttt{None}:]  valor especial en Python retornado por las funciones
que no tienen una sentencia return, o que tienen una sentencia return
sin un argumento.

\item[Desarrollo incremental:]  un plan de desarrollo de programas 
que evita la depuración, agregando y probando solo pequeñas
porciones de código en cada momento.

\item[Andamiaje:]  código que se usa durante el desarrollo de programas, pero
no hace parte de la solución final.

\item[Guarda:]  una condición que chequea y controla circunstancias que
pueden causar errores.

\index{variable temporal}
\index{variable!temporal}
\index{valor de retorno}
\index{código muerto}
\index{None}
\index{desarrollo incremental}
\index{andamiaje}
\index{guarda}

\end{description}

\section{Ejercicios}

\begin{enumerate}

 \item Escriba la función \verb+comparar(a,b)+ que devuelva 1 si $a<b$, 0 si $a=b$, y -1 si $a>b$
 
 \item Tome la solución del último ejercicio del capítulo anterior y conviértala en una función que retorne la nota 
 definitiva de su curso de programación.
 
 \item Calcule en una función el área de un disco, teniendo como entrada el radio menor y el radio mayor.
 
 \item Escriba la función \verb+pendiente(x1, y1, x2, y2)+ que calcule la pendiente de una línea que pasa por los puntos 
 $(x_1, y_1)$ y $(x_2, y_2)$. 
 
 \item Convierta las funciones de los capítulos pasados, y que se puedan transformar, a fructíferas.
 
 \item Convierta las funciones que obtuvo en el punto anterior agregando guardas para protegerlas de las situaciones
 en que reciben argumentos de un tipo de dato que no pueden manipular.
 
 \item Agregue pruebas unitarias a las funciones que obtuvo en el punto anterior.
 
\end{enumerate}	\clearemptydoublepage  % funciones fructíferas
\include{chap06}	\clearemptydoublepage  % iteracion
\include{chap07}	\clearemptydoublepage  % cadenas
\include{chap08}	\clearemptydoublepage  % listas
\include{chap09}	\clearemptydoublepage  % tuplas
\include{chap10}	\clearemptydoublepage  % diccionarios 
\include{chap11}	\clearemptydoublepage  % archivos y excepciones
\include{chap12}	\clearemptydoublepage  % clases y objetos
\include{chap13}	\clearemptydoublepage  % clases y funciones
\include{chap14}	\clearemptydoublepage  % clases y metodos
%FIX %\setcounter{page}{168}
\include{chap15}	\clearemptydoublepage  % conjuntos de objetos
\include{chap16}	\clearemptydoublepage  % herencia
\include{chap17}	\clearemptydoublepage  % listas enlazadas
\include{chap18}	\clearemptydoublepage  % pilas
\include{chap19}	\clearemptydoublepage  % colas (normales y de prioridad)
\include{chap20}	\clearemptydoublepage  % arboles


\appendix
\include{app01}	\clearemptydoublepage          % depuracion
\include{app02}	\clearemptydoublepage          % creando un nuevo tipo de dato
\include{app03}	\clearemptydoublepage	       % programas completos
\include{app04}	\clearemptydoublepage  	       % lecturas recomendadas

\chapter{Licencia de documentación libre de GNU}

Versión 1.2, Noviembre 2002 \\

This is an unofficial translation of the GNU Free Documentation License
into Spanish. It was not published by the Free Software Foundation,
and does not legally state the distribution terms for documentation
that uses the GNU FDL – only the original English text of the GNU
FDL does that. However, we hope that this translation will help Spanish
speakers understand the GNU FDL better.

Esta es una traducción no oficial de la GNU Free Document License
a Español (Castellano). No ha sido publicada por la Free Software
Foundation y no establece legalmente los términos de distribución
para trabajos que usen la GFDL (sólo el texto de la versión original
en Inglés de la GFDL lo hace). Sin embargo, esperamos que esta traducción
ayude a los hispanohablantes a entender mejor la GFDL. La versión
original de la GFDL está disponible en la Free Software Foundation{[}1{]}.

Esta traducción está basada en una la versión 1.1 de Igor Támara y
Pablo Reyes. Sin embargo la responsabilidad de su interpretación es
de Joaquín Seoane.

Copyright (C) 2000, 2001, 2002 Free Software Foundation, Inc. 59 Temple
Place, Suite 330, Boston, MA 02111-1307 USA. Se permite la copia y
distribución de copias literales de este documento de licencia, pero
no se permiten cambios{[}1{]}.

%\rule{\linewidth}{1pt}

\section*{Preámbulo}

El propósito de esta licencia es permitir que un manual, libro de
texto, u otro documento escrito sea libre en el sentido de libertad:
asegurar a todo el mundo la libertad efectiva de copiarlo y redistribuirlo,
con o sin modificaciones, de manera comercial o no. En segundo término,
esta licencia proporciona al autor y al editor{[}2{]} una manera de
obtener reconocimiento por su trabajo, sin que se le considere responsable
de las modificaciones realizadas por otros.

Esta licencia es de tipo copyleft, lo que significa que los trabajos
derivados del documento deben a su vez ser libres en el mismo sentido.
Complementa la Licencia Pública General de GNU, que es una licencia
tipo copyleft diseñada para el software libre.

Hemos diseñado esta licencia para usarla en manuales de software libre,
ya que el software libre necesita documentación libre: un programa
libre debe venir con manuales que ofrezcan la mismas libertades que
el software. Pero esta licencia no se limita a manuales de software;
puede usarse para cualquier texto, sin tener en cuenta su temática
o si se publica como libro impreso o no. Recomendamos esta licencia
principalmente para trabajos cuyo fin sea instructivo o de referencia.

\section{Aplicabilidad y definiciones}

Esta licencia se aplica a cualquier manual u otro trabajo, en cualquier
soporte, que contenga una nota del propietario de los derechos de
autor que indique que puede ser distribuido bajo los términos de esta
licencia. Tal nota garantiza en cualquier lugar del mundo, sin pago
de derechos y sin límite de tiempo, el uso de dicho trabajo según
las condiciones aquí estipuladas. En adelante la palabra Documento
se referirá a cualquiera de dichos manuales o trabajos. Cualquier
persona es un licenciatario y será referido como Usted. Usted acepta
la licencia si copia. modifica o distribuye el trabajo de cualquier
modo que requiera permiso según la ley de propiedad intelectual.

Una Versión Modificada del Documento significa cualquier trabajo que
contenga el Documento o una porción del mismo, ya sea una copia literal
o con modificaciones y/o traducciones a otro idioma.

Una Sección Secundaria es un apéndice con título o una sección preliminar
del Documento que trata exclusivamente de la relación entre los autores
o editores y el tema general del Documento (o temas relacionados)
pero que no contiene nada que entre directamente en dicho tema general
(por ejemplo, si el Documento es en parte un texto de matemáticas,
una Sección Secundaria puede no explicar nada de matemáticas). La
relación puede ser una conexión histórica con el tema o temas relacionados,
o una opinión legal, comercial, filosófica, ética o política acerca
de ellos.

Las Secciones Invariantes son ciertas Secciones Secundarias cuyos
títulos son designados como Secciones Invariantes en la nota que indica
que el documento es liberado bajo esta Licencia. Si una sección no
entra en la definición de Secundaria, no puede designarse como Invariante.
El documento puede no tener Secciones Invariantes. Si el Documento
no identifica las Secciones Invariantes, es que no las tiene.

Los Textos de Cubierta son ciertos pasajes cortos de texto que se
listan como Textos de Cubierta Delantera o Textos de Cubierta Trasera
en la nota que indica que el documento es liberado bajo esta Licencia.
Un Texto de Cubierta Delantera puede tener como mucho 5 palabras,
y uno de Cubierta Trasera puede tener hasta 25 palabras.

Una copia Transparente del Documento, significa una copia para lectura
en máquina, representada en un formato cuya especificación está disponible
al público en general, apto para que los contenidos puedan ser vistos
y editados directamente con editores de texto genéricos o (para imágenes
compuestas por puntos) con programas genéricos de manipulación de
imágenes o (para dibujos) con algún editor de dibujos ampliamente
disponible, y que sea adecuado como entrada para formateadores de
texto o para su traducción automática a formatos adecuados para formateadores
de texto. Una copia hecha en un formato definido como Transparente,
pero cuyo marcaje o ausencia de él haya sido diseñado para impedir
o dificultar modificaciones posteriores por parte de los lectores
no es Transparente. Un formato de imagen no es Transparente si se
usa para una cantidad de texto sustancial. Una copia que no es Transparente
se denomina Opaca.

Como ejemplos de formatos adecuados para copias Transparentes están
ASCII puro sin marcaje, formato de entrada de Texinfo, formato de
entrada de LaTeX, SGML o XML usando una DTD disponible públicamente,
y HTML, PostScript o PDF simples, que sigan los estándares y diseñados
para que los modifiquen personas. Ejemplos de formatos de imagen transparentes
son PNG, XCF y JPG. Los formatos Opacos incluyen formatos propietarios
que pueden ser leídos y editados únicamente en procesadores de palabras
propietarios, SGML o XML para los cuáles las DTD y/o herramientas
de procesamiento no estén ampliamente disponibles, y HTML, PostScript
o PDF generados por algunos procesadores de palabras sólo como salida.

La Portada significa, en un libro impreso, la página de título, más
las páginas siguientes que sean necesarias para mantener legiblemente
el material que esta Licencia requiere en la portada. Para trabajos
en formatos que no tienen página de portada como tal, Portada significa
el texto cercano a la aparición más prominente del título del trabajo,
precediendo el comienzo del cuerpo del texto.

Una sección Titulada XYZ significa una parte del Documento cuyo título
es precisamente XYZ o contiene XYZ entre paréntesis, a continuación
de texto que traduce XYZ a otro idioma (aquí XYZ se refiere a nombres
de sección específicos mencionados más abajo, como Agradecimientos,
Dedicatorias , Aprobaciones o Historia. Conservar el Título de tal
sección cuando se modifica el Documento significa que permanece una
sección Titulada XYZ según esta definición .

El Documento puede incluir Limitaciones de Garantía cercanas a la
nota donde se declara que al Documento se le aplica esta Licencia.
Se considera que estas Limitaciones de Garantía están incluidas, por
referencia, en la Licencia, pero sólo en cuanto a limitaciones de
garantía: cualquier otra implicación que estas Limitaciones de Garantía
puedan tener es nula y no tiene efecto en el significado de esta Licencia.

%\rule{\linewidth}{1pt}

\section{Copia literal}

Usted puede copiar y distribuir el Documento en cualquier soporte,
sea en forma comercial o no, siempre y cuando esta Licencia, las notas
de copyright y la nota que indica que esta Licencia se aplica al Documento
se reproduzcan en todas las copias y que usted no añada ninguna otra
condición a las expuestas en esta Licencia. Usted no puede usar medidas
técnicas para obstruir o controlar la lectura o copia posterior de
las copias que usted haga o distribuya. Sin embargo, usted puede aceptar
compensación a cambio de las copias. Si distribuye un número suficientemente
grande de copias también deberá seguir las condiciones de la sección
3.

Usted también puede prestar copias, bajo las mismas condiciones establecidas
anteriormente, y puede exhibir copias públicamente.

\section{Copiado en cantidad}

Si publica copias impresas del Documento (o copias en soportes que
tengan normalmente cubiertas impresas) que sobrepasen las 100, y la
nota de licencia del Documento exige Textos de Cubierta, debe incluir
las copias con cubiertas que lleven en forma clara y legible todos
esos Textos de Cubierta: Textos de Cubierta Delantera en la cubierta
delantera y Textos de Cubierta Trasera en la cubierta trasera. Ambas
cubiertas deben identificarlo a Usted clara y legiblemente como editor
de tales copias. La cubierta debe mostrar el título completo con todas
las palabras igualmente prominentes y visibles. Además puede añadir
otro material en las cubiertas. Las copias con cambios limitados a
las cubiertas, siempre que conserven el título del Documento y satisfagan
estas condiciones, pueden considerarse como copias literales.

Si los textos requeridos para la cubierta son muy voluminosos para
que ajusten legiblemente, debe colocar los primeros (tantos como sea
razonable colocar) en la verdadera cubierta y situar el resto en páginas
adyacentes.

Si Usted publica o distribuye copias Opacas del Documento cuya cantidad
exceda las 100, debe incluir una copia Transparente, que pueda ser
leída por una máquina, con cada copia Opaca, o bien mostrar, en cada
copia Opaca, una dirección de red donde cualquier usuario de la misma
tenga acceso por medio de protocolos públicos y estandarizados a una
copia Transparente del Documento completa, sin material adicional.
Si usted hace uso de la última opción, deberá tomar las medidas necesarias,
cuando comience la distribución de las copias Opacas en cantidad,
para asegurar que esta copia Transparente permanecerá accesible en
el sitio establecido por lo menos un año después de la última vez
que distribuya una copia Opaca de esa edición al público (directamente
o a través de sus agentes o distribuidores).

Se solicita, aunque no es requisito, que se ponga en contacto con
los autores del Documento antes de redistribuir gran número de copias,
para darles la oportunidad de que le proporcionen una versión actualizada
del Documento.

%\rule{\linewidth}{1pt}

\section{Modificaciones}

Puede copiar y distribuir una Versión Modificada del Documento bajo
las condiciones de las secciones 2 y 3 anteriores, siempre que usted
libere la Versión Modificada bajo esta misma Licencia, con la Versión
Modificada haciendo el rol del Documento, por lo tanto dando licencia
de distribución y modificación de la Versión Modificada a quienquiera
posea una copia de la misma. Además, debe hacer lo siguiente en la
Versión Modificada:
\begin{itemize}
\item Usar en la Portada (y en las cubiertas, si hay alguna) un título distinto
al del Documento y de sus versiones anteriores (que deberían, si hay
alguna, estar listadas en la sección de Historia del Documento). Puede
usar el mismo título de versiones anteriores al original siempre y
cuando quien las publicó originalmente otorgue permiso.
\item Listar en la Portada, como autores, una o más personas o entidades
responsables de la autoría de las modificaciones de la Versión Modificada,
junto con por lo menos cinco de los autores principales del Documento
(todos sus autores principales, si hay menos de cinco), a menos que
le eximan de tal requisito.
\item Mostrar en la Portada como editor el nombre del editor de la Versión
Modificada.
\item Conservar todas las notas de copyright del Documento.
\item Añadir una nota de copyright apropiada a sus modificaciones, adyacente
a las otras notas de copyright.
\item Incluir, inmediatamente después de las notas de copyright, una nota
de licencia dando el permiso para usar la Versión Modificada bajo
los términos de esta Licencia, como se muestra en la Adenda al final
de este documento.
\item Conservar en esa nota de licencia el listado completo de las Secciones
Invariantes y de los Textos de Cubierta que sean requeridos en la
nota de Licencia del Documento original.
\item Incluir una copia sin modificación de esta Licencia.
\item Conservar la sección Titulada Historia, conservar su Título y añadirle
un elemento que declare al menos el título, el año, los nuevos autores
y el editor de la Versión Modificada, tal como figuran en la Portada.
Si no hay una sección Titulada Historia en el Documento, crear una
estableciendo el título, el año, los autores y el editor del Documento,
tal como figuran en su Portada, añadiendo además un elemento describiendo
la Versión Modificada, como se estableció en la oración anterior.
\item Conservar la dirección en red, si la hay, dada en el Documento para
el acceso público a una copia Transparente del mismo, así como las
otras direcciones de red dadas en el Documento para versiones anteriores
en las que estuviese basado. Pueden ubicarse en la sección Historia.
Se puede omitir la ubicación en red de un trabajo que haya sido publicado
por lo menos cuatro años antes que el Documento mismo, o si el editor
original de dicha versión da permiso.
\item En cualquier sección Titulada Agradecimientos o Dedicatorias, Conservar
el Título de la sección y conservar en ella toda la sustancia y el
tono de los agradecimientos y/o dedicatorias incluidas por cada contribuyente.
\item Conservar todas las Secciones Invariantes del Documento, sin alterar
su texto ni sus títulos. Números de sección o el equivalente no son
considerados parte de los títulos de la sección.
\item Borrar cualquier sección titulada Aprobaciones. Tales secciones no
pueden estar incluidas en las Versiones Modificadas.
\item No cambiar el título de ninguna sección existente a Aprobaciones ni
a uno que entre en conflicto con el de alguna Sección Invariante.
\item Conservar todas las Limitaciones de Garantía.
\end{itemize}
Si la Versión Modificada incluye secciones o apéndices nuevos que
califiquen como Secciones Secundarias y contienen material no copiado
del Documento, puede opcionalmente designar algunas o todas esas secciones
como invariantes. Para hacerlo, añada sus títulos a la lista de Secciones
Invariantes en la nota de licencia de la Versión Modificada. Tales
títulos deben ser distintos de cualquier otro título de sección.

Puede añadir una sección titulada Aprobaciones, siempre que contenga
únicamente aprobaciones de su Versión Modificada por otras fuentes
–por ejemplo, observaciones de peritos o que el texto ha sido aprobado
por una organización como la definición oficial de un estándar.

Puede añadir un pasaje de hasta cinco palabras como Texto de Cubierta
Delantera y un pasaje de hasta 25 palabras como Texto de Cubierta
Trasera en la Versión Modificada. Una entidad solo puede añadir (o
hacer que se añada) un pasaje al Texto de Cubierta Delantera y uno
al de Cubierta Trasera. Si el Documento ya incluye un textos de cubiertas
añadidos previamente por usted o por la misma entidad que usted representa,
usted no puede añadir otro; pero puede reemplazar el anterior, con
permiso explícito del editor que agregó el texto anterior.

Con esta Licencia ni los autores ni los editores del Documento dan
permiso para usar sus nombres para publicidad ni para asegurar o implicar
aprobación de cualquier Versión Modificada.

\section{Combinación de documentos}

Usted puede combinar el Documento con otros documentos liberados bajo
esta Licencia, bajo los términos definidos en la sección 4 anterior
para versiones modificadas, siempre que incluya en la combinación
todas las Secciones Invariantes de todos los documentos originales,
sin modificar, listadas todas como Secciones Invariantes del trabajo
combinado en su nota de licencia. Así mismo debe incluir la Limitación
de Garantía.

El trabajo combinado necesita contener solamente una copia de esta
Licencia, y puede reemplazar varias Secciones Invariantes idénticas
por una sola copia. Si hay varias Secciones Invariantes con el mismo
nombre pero con contenidos diferentes, haga el título de cada una
de estas secciones único añadiéndole al final del mismo, entre paréntesis,
el nombre del autor o editor original de esa sección, si es conocido,
o si no, un número único. Haga el mismo ajuste a los títulos de sección
en la lista de Secciones Invariantes de la nota de licencia del trabajo
combinado.

En la combinación, debe combinar cualquier sección Titulada Historia
de los documentos originales, formando una sección Titulada Historia;
de la misma forma combine cualquier sección Titulada Agradecimientos,
y cualquier sección Titulada Dedicatorias. Debe borrar todas las secciones
tituladas Aprobaciones.

\section{Colecciones de documentos}

Puede hacer una colección que conste del Documento y de otros documentos
liberados bajo esta Licencia, y reemplazar las copias individuales
de esta Licencia en todos los documentos por una sola copia que esté
incluida en la colección, siempre que siga las reglas de esta Licencia
para cada copia literal de cada uno de los documentos en cualquiera
de los demás aspectos.

Puede extraer un solo documento de una de tales colecciones y distribuirlo
individualmente bajo esta Licencia, siempre que inserte una copia
de esta Licencia en el documento extraído, y siga esta Licencia en
todos los demás aspectos relativos a la copia literal de dicho documento.

\section{Agregación con trabajos independientes}

Una recopilación que conste del Documento o sus derivados y de otros
documentos o trabajos separados e independientes, en cualquier soporte
de almacenamiento o distribución, se denomina un agregado si el copyright
resultante de la compilación no se usa para limitar los derechos de
los usuarios de la misma más allá de lo que los de los trabajos individuales
permiten. Cuando el Documento se incluye en un agregado, esta Licencia
no se aplica a otros trabajos del agregado que no sean en sí mismos
derivados del Documento.

Si el requisito de la sección 3 sobre el Texto de Cubierta es aplicable
a estas copias del Documento y el Documento es menor que la mitad
del agregado entero, los Textos de Cubierta del Documento pueden colocarse
en cubiertas que enmarquen solamente el Documento dentro del agregado,
o el equivalente electrónico de las cubiertas si el documento está
en forma electrónica. En caso contrario deben aparecer en cubiertas
impresas enmarcando todo el agregado.

\section{Traducción}

La Traducción es considerada como un tipo de modificación, por lo
que usted puede distribuir traducciones del Documento bajo los términos
de la sección 4. El reemplazo de las Secciones Invariantes con traducciones
requiere permiso especial de los dueños de derecho de autor, pero
usted puede añadir traducciones de algunas o todas las Secciones Invariantes
a las versiones originales de las mismas. Puede incluir una traducción
de esta Licencia, de todas las notas de licencia del documento, así
como de las Limitaciones de Garantía, siempre que incluya también
la versión en Inglés de esta Licencia y las versiones originales de
las notas de licencia y Limitaciones de Garantía. En caso de desacuerdo
entre la traducción y la versión original en Inglés de esta Licencia,
la nota de licencia o la limitación de garantía, la versión original
en Inglés prevalecerá.

Si una sección del Documento está Titulada Agradecimientos, Dedicatorias
o Historia el requisito (sección 4) de Conservar su Título (Sección
1) requerirá, típicamente, cambiar su título.

\section{Terminación}

Usted no puede copiar, modificar, sublicenciar o distribuir el Documento
salvo por lo permitido expresamente por esta Licencia. Cualquier otro
intento de copia, modificación, sublicenciamiento o distribución del
Documento es nulo, y dará por terminados automáticamente sus derechos
bajo esa Licencia. Sin embargo, los terceros que hayan recibido copias,
o derechos, de usted bajo esta Licencia no verán terminadas sus licencias,
siempre que permanezcan en total conformidad con ella.

\section{Revisiones futuras de esta licencia}

De vez en cuando la Free Software Foundation puede publicar versiones
nuevas y revisadas de la Licencia de Documentación Libre GNU. Tales
versiones nuevas serán similares en espíritu a la presente versión,
pero pueden diferir en detalles para solucionar nuevos problemas o
intereses. Vea \url{http://www.gnu.org/copyleft/}.

Cada versión de la Licencia tiene un número de versión que la distingue.
Si el Documento especifica que se aplica una versión numerada en particular
de esta licencia o cualquier versión posterior, usted tiene la opción
de seguir los términos y condiciones de la versión especificada o
cualquiera posterior que haya sido publicada (no como borrador) por
la Free Software Foundation. Si el Documento no especifica un número
de versión de esta Licencia, puede escoger cualquier versión que haya
sido publicada (no como borrador) por la Free Software Foundation.

\section{ADENDA: Cómo usar esta Licencia en sus documentos}

Para usar esta licencia en un documento que usted haya escrito, incluya
una copia de la Licencia en el documento y ponga el siguiente copyright
y nota de licencia justo después de la página de título:
\begin{quote}
Copyright (c) AÑO SU NOMBRE. Se concede permiso para copiar, distribuir
y/o modificar este documento bajo los términos de la Licencia de Documentación
Libre de GNU, Versión 1.2 o cualquier otra versión posterior publicada
por la Free Software Foundation; sin Secciones Invariantes ni Textos
de Cubierta Delantera ni Textos de Cubierta Trasera. Una copia de
la licencia está incluida en la sección titulada GNU Free Documentation
License. 
\end{quote}
Si tiene Secciones Invariantes, Textos de Cubierta Delantera y Textos
de Cubierta Trasera, reemplace la frase sin ... Trasera por esto:
\begin{quote}
siendo las Secciones Invariantes LISTE SUS TÍTULOS, siendo los Textos
de Cubierta Delantera LISTAR, y siendo sus Textos de Cubierta Trasera
LISTAR. 
\end{quote}
Si tiene Secciones Invariantes sin Textos de Cubierta o cualquier
otra combinación de los tres, mezcle ambas alternativas para adaptarse
a la situación.

Si su documento contiene ejemplos de código de programa no triviales,
recomendamos liberar estos ejemplos en paralelo bajo la licencia de
software libre que usted elija, como la Licencia Pública General de
GNU (GNU General Public License), para permitir su uso en software
libre.

\rule{1\linewidth}{1pt}

Notas

{[}1{]} Ésta es la traducción del Copyright de la Licencia, no es
el Copyright de esta traducción no autorizada.

{[}2{]} La licencia original dice publisher, que es, estrictamente,
quien publica, diferente de editor, que es más bien quien prepara
un texto para publicar. En castellano editor se usa para ambas cosas.

{[}3{]} En sentido estricto esta licencia parece exigir que los títulos
sean exactamente Acknowledgements, Dedications, Endorsements e History,
en inglés.
\clearemptydoublepage          % GNU free doc license (spanish)
% fdl.tex
%This file is a chapter.  It must be included in a larger 
%document to work properly.

\chapter{GNU Free Documentation License}

Version 1.1, March 2000\\

Copyright $\copyright$ 2000 Free Software Foundation, Inc.\\
 59 Temple Place, Suite 330, Boston, MA 02111-1307 USA\\
 Everyone is permitted to copy and distribute verbatim copies of this
license document, but changing it is not allowed.

\section*{Preamble}

The purpose of this License is to make a manual, textbook, or other
written document ``free'' in the sense of freedom: to assure everyone
the effective freedom to copy and redistribute it, with or without
modifying it, either commercially or noncommercially. Secondarily,
this License preserves for the author and publisher a way to get credit
for their work, while not being considered responsible for modifications
made by others.

This License is a kind of ``copyleft,'' which means that derivative
works of the document must themselves be free in the same sense. It
complements the GNU General Public License, which is a copyleft license
designed for free software.

We have designed this License in order to use it for manuals for free
software, because free software needs free documentation: a free program
should come with manuals providing the same freedoms that the software
does. But this License is not limited to software manuals; it can
be used for any textual work, regardless of subject matter or whether
it is published as a printed book. We recommend this License principally
for works whose purpose is instruction or reference.

\section{Applicability and Definitions}

This License applies to any manual or other work that contains a notice
placed by the copyright holder saying it can be distributed under
the terms of this License. The ``Document,'' below, refers to any
such manual or work. Any member of the public is a licensee, and is
addressed as ``you.''

A ``Modified Version'' of the Document means any work containing
the Document or a portion of it, either copied verbatim, or with modifications
and/or translated into another language.

A ``Secondary Section'' is a named appendix or a front-matter section
of the Document that deals exclusively with the relationship of the
publishers or authors of the Document to the Document's overall subject
(or to related matters) and contains nothing that could fall directly
within that overall subject. (For example, if the Document is in part
a textbook of mathematics, a Secondary Section may not explain any
mathematics.) The relationship could be a matter of historical connection
with the subject or with related matters, or of legal, commercial,
philosophical, ethical, or political position regarding them.

The ``Invariant Sections'' are certain Secondary Sections whose
titles are designated, as being those of Invariant Sections, in the
notice that says that the Document is released under this License.

The ``Cover Texts'' are certain short passages of text that are
listed, as Front-Cover Texts or Back-Cover Texts, in the notice that
says that the Document is released under this License.

A ``Transparent'' copy of the Document means a machine-readable
copy, represented in a format whose specification is available to
the general public, whose contents can be viewed and edited directly
and straightforwardly with generic text editors or (for images composed
of pixels) generic paint programs or (for drawings) some widely available
drawing editor, and that is suitable for input to text formatters
or for automatic translation to a variety of formats suitable for
input to text formatters. A copy made in an otherwise Transparent
file format whose markup has been designed to thwart or discourage
subsequent modification by readers is not Transparent. A copy that
is not ``Transparent'' is called ``Opaque.''

Examples of suitable formats for Transparent copies include plain
ASCII without markup, Texinfo input format, \LaTeX{}~input format,
SGML or XML using a publicly available DTD, and standard-conforming
simple HTML designed for human modification. Opaque formats include
PostScript, PDF, proprietary formats that can be read and edited only
by proprietary word processors, SGML or XML for which the DTD and/or
processing tools are not generally available, and the machine-generated
HTML produced by some word processors for output purposes only.

The ``Title Page'' means, for a printed book, the title page itself,
plus such following pages as are needed to hold, legibly, the material
this License requires to appear in the title page. For works in formats
which do not have any title page as such, ``Title Page'' means the
text near the most prominent appearance of the work's title, preceding
the beginning of the body of the text.

\section{Verbatim Copying}

You may copy and distribute the Document in any medium, either commercially
or noncommercially, provided that this License, the copyright notices,
and the license notice saying this License applies to the Document
are reproduced in all copies, and that you add no other conditions
whatsoever to those of this License. You may not use technical measures
to obstruct or control the reading or further copying of the copies
you make or distribute. However, you may accept compensation in exchange
for copies. If you distribute a large enough number of copies you
must also follow the conditions in Section 3.

You may also lend copies, under the same conditions stated above,
and you may publicly display copies.

\section{Copying in Quantity}

If you publish printed copies of the Document numbering more than
100, and the Document's license notice requires Cover Texts, you must
enclose the copies in covers that carry, clearly and legibly, all
these Cover Texts: Front-Cover Texts on the front cover, and Back-Cover
Texts on the back cover. Both covers must also clearly and legibly
identify you as the publisher of these copies. The front cover must
present the full title with all words of the title equally prominent
and visible. You may add other material on the covers in addition.
Copying with changes limited to the covers, as long as they preserve
the title of the Document and satisfy these conditions, can be treated
as verbatim copying in other respects.

If the required texts for either cover are too voluminous to fit legibly,
you should put the first ones listed (as many as fit reasonably) on
the actual cover, and continue the rest onto adjacent pages.

If you publish or distribute Opaque copies of the Document numbering
more than 100, you must either include a machine-readable Transparent
copy along with each Opaque copy, or state in or with each Opaque
copy a publicly accessible computer-network location containing a
complete Transparent copy of the Document, free of added material,
which the general network-using public has access to download anonymously
at no charge using public-standard network protocols. If you use the
latter option, you must take reasonably prudent steps, when you begin
distribution of Opaque copies in quantity, to ensure that this Transparent
copy will remain thus accessible at the stated location until at least
one year after the last time you distribute an Opaque copy (directly
or through your agents or retailers) of that edition to the public.

It is requested, but not required, that you contact the authors of
the Document well before redistributing any large number of copies,
to give them a chance to provide you with an updated version of the
Document.

\section{Modifications}

You may copy and distribute a Modified Version of the Document under
the conditions of Sections 2 and 3 above, provided that you release
the Modified Version under precisely this License, with the Modified
Version filling the role of the Document, thus licensing distribution
and modification of the Modified Version to whoever possesses a copy
of it. In addition, you must do these things in the Modified Version:
\begin{itemize}
\item Use in the Title Page (and on the covers, if any) a title distinct
from that of the Document, and from those of previous versions (which
should, if there were any, be listed in the History section of the
Document). You may use the same title as a previous version if the
original publisher of that version gives permission. 
\item List on the Title Page, as authors, one or more persons or entities
responsible for authorship of the modifications in the Modified Version,
together with at least five of the principal authors of the Document
(all of its principal authors, if it has less than five). 
\item State on the Title page the name of the publisher of the Modified
Version, as the publisher. 
\item Preserve all the copyright notices of the Document. 
\item Add an appropriate copyright notice for your modifications adjacent
to the other copyright notices. 
\item Include, immediately after the copyright notices, a license notice
giving the public permission to use the Modified Version under the
terms of this License, in the form shown in the Addendum below. 
\item Preserve in that license notice the full lists of Invariant Sections
and required Cover Texts given in the Document's license notice. 
\item Include an unaltered copy of this License. 
\item Preserve the section entitled ``History,'' and its title, and add
to it an item stating at least the title, year, new authors, and publisher
of the Modified Version as given on the Title Page. If there is no
section entitled ``History'' in the Document, create one stating
the title, year, authors, and publisher of the Document as given on
its Title Page, then add an item describing the Modified Version as
stated in the previous sentence. 
\item Preserve the network location, if any, given in the Document for public
access to a Transparent copy of the Document, and likewise the network
locations given in the Document for previous versions it was based
on. These may be placed in the ``History'' section. You may omit
a network location for a work that was published at least four years
before the Document itself, or if the original publisher of the version
it refers to gives permission. 
\item In any section entitled ``Acknowledgements'' or ``Dedications,''
preserve the section's title, and preserve in the section all the
substance and tone of each of the contributor acknowledgements and/or
dedications given therein. 
\item Preserve all the Invariant Sections of the Document, unaltered in
their text and in their titles. Section numbers or the equivalent
are not considered part of the section titles. 
\item Delete any section entitled ``Endorsements.'' Such a section may
not be included in the Modified Version. 
\item Do not retitle any existing section as ``Endorsements'' or to conflict
in title with any Invariant Section.
\end{itemize}
If the Modified Version includes new front-matter sections or appendices
that qualify as Secondary Sections and contain no material copied
from the Document, you may at your option designate some or all of
these sections as invariant. To do this, add their titles to the list
of Invariant Sections in the Modified Version's license notice. These
titles must be distinct from any other section titles.

You may add a section entitled ``Endorsements,'' provided it contains
nothing but endorsements of your Modified Version by various parties—for
example, statements of peer review or that the text has been approved
by an organization as the authoritative definition of a standard.

You may add a passage of up to five words as a Front-Cover Text, and
a passage of up to 25 words as a Back-Cover Text, to the end of the
list of Cover Texts in the Modified Version. Only one passage of Front-Cover
Text and one of Back-Cover Text may be added by (or through arrangements
made by) any one entity. If the Document already includes a cover
text for the same cover, previously added by you or by arrangement
made by the same entity you are acting on behalf of, you may not add
another; but you may replace the old one, on explicit permission from
the previous publisher that added the old one.

The author(s) and publisher(s) of the Document do not by this License
give permission to use their names for publicity for or to assert
or imply endorsement of any Modified Version.

\section{Combining Documents}

You may combine the Document with other documents released under this
License, under the terms defined in Section 4 above for modified versions,
provided that you include in the combination all of the Invariant
Sections of all of the original documents, unmodified, and list them
all as Invariant Sections of your combined work in its license notice.

The combined work need only contain one copy of this License, and
multiple identical Invariant Sections may be replaced with a single
copy. If there are multiple Invariant Sections with the same name
but different contents, make the title of each such section unique
by adding at the end of it, in parentheses, the name of the original
author or publisher of that section if known, or else a unique number.
Make the same adjustment to the section titles in the list of Invariant
Sections in the license notice of the combined work.

In the combination, you must combine any sections entitled ``History''
in the various original documents, forming one section entitled ``History'';
likewise combine any sections entitled ``Acknowledgements,'' and
any sections entitled ``Dedications.'' You must delete all sections
entitled ``Endorsements.''

\section{Collections of Documents}

You may make a collection consisting of the Document and other documents
released under this License, and replace the individual copies of
this License in the various documents with a single copy that is included
in the collection, provided that you follow the rules of this License
for verbatim copying of each of the documents in all other respects.

You may extract a single document from such a collection, and distribute
it individually under this License, provided you insert a copy of
this License into the extracted document, and follow this License
in all other respects regarding verbatim copying of that document.

\section{Aggregation with Independent Works}

A compilation of the Document or its derivatives with other separate
and independent documents or works, in or on a volume of a storage
or distribution medium, does not as a whole count as a Modified Version
of the Document, provided no compilation copyright is claimed for
the compilation. Such a compilation is called an ``aggregate,''
and this License does not apply to the other self-contained works
thus compiled with the Document, on account of their being thus compiled,
if they are not themselves derivative works of the Document.

If the Cover Text requirement of Section 3 is applicable to these
copies of the Document, then if the Document is less than one quarter
of the entire aggregate, the Document's Cover Texts may be placed
on covers that surround only the Document within the aggregate. Otherwise
they must appear on covers around the whole aggregate.

\section{Translation}

Translation is considered a kind of modification, so you may distribute
translations of the Document under the terms of Section 4. Replacing
Invariant Sections with translations requires special permission from
their copyright holders, but you may include translations of some
or all Invariant Sections in addition to the original versions of
these Invariant Sections. You may include a translation of this License
provided that you also include the original English version of this
License. In case of a disagreement between the translation and the
original English version of this License, the original English version
will prevail.

\section{Termination}

You may not copy, modify, sublicense, or distribute the Document except
as expressly provided for under this License. Any other attempt to
copy, modify, sublicense, or distribute the Document is void, and
will automatically terminate your rights under this License. However,
parties who have received copies, or rights, from you under this License
will not have their licenses terminated so long as such parties remain
in full compliance.

\section{Future Revisions of This License}

The Free Software Foundation may publish new, revised versions of
the GNU Free Documentation License from time to time. Such new versions
will be similar in spirit to the present version, but may differ in
detail to address new problems or concerns. See \url{http:///www.gnu.org/copyleft/}.

Each version of the License is given a distinguishing version number.
If the Document specifies that a particular numbered version of this
License ``or any later version'' applies to it, you have the option
of following the terms and conditions either of that specified version
or of any later version that has been published (not as a draft) by
the Free Software Foundation. If the Document does not specify a version
number of this License, you may choose any version ever published
(not as a draft) by the Free Software Foundation.

\section{Addendum: How to Use This License for Your Documents}

To use this License in a document you have written, include a copy
of the License in the document and put the following copyright and
license notices just after the title page:
\begin{quote}
Copyright $\copyright$ YEAR YOUR NAME. Permission is granted to copy,
distribute and/or modify this document under the terms of the GNU
Free Documentation License, Version 1.1 or any later version published
by the Free Software Foundation; with the Invariant Sections being
LIST THEIR TITLES, with the Front-Cover Texts being LIST, and with
the Back-Cover Texts being LIST. A copy of the license is included
in the section entitled ``GNU Free Documentation License.''
\end{quote}
If you have no Invariant Sections, write ``with no Invariant Sections''
instead of saying which ones are invariant. If you have no Front-Cover
Texts, write ``no Front-Cover Texts'' instead of ``Front-Cover
Texts being LIST''; likewise for Back-Cover Texts.

If your document contains nontrivial examples of program code, we
recommend releasing these examples in parallel under your choice of
free software license, such as the GNU General Public License, to
permit their use in free software.
	\clearemptydoublepage          % GNU free doc license

\printindex				       % indice alfabetico

\clearemptydoublepage
\clearemptydoublepage
\end{document}
