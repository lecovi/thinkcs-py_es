%\usepackage{fontspec}
%	\setmainfont{Myriad Pro}
\usepackage{palatino}

\usepackage{polyglossia}           % Change this depending on your language
	\setdefaultlanguage{spanish}
	\setotherlanguage{english}

\usepackage{cancel}                % To cross out over text.

%\usepackage{datetime}
%	\newdateformat{serialdate}{\THEYEAR\twodigit{\THEMONTH}\twodigit{\THEDAY}}
%	
%\newtimeformat{serialtime}{\twodigit{\THEHOUR}\twodigit{\THEMINUTE}\twodigit{\THESECOND}}

%\usepackage{showframe}            % Show margins (deactivate xcolor package)
\usepackage[
	usenames,
	dvipsnames,
	svgnames,
	table]{xcolor}                 % Incluimos el uso de colores.

\usepackage{setspace}              % Nos permite controlar el interlineado.
\usepackage{multirow}              % Para poder juntar celdas en las tablas.
%\usepackage{lipsum}

%\usepackage{background}
%\backgroundsetup{
%	scale=1,
%	angle=0,
%	opacity=1,  %% adjust
%	contents={
%		\ifnum \value{page} > 1	
%			\includegraphics[width=\paperwidth,height=\paperheight]{./img/ARCHIVO1.png}
%		\else
%			\includegraphics[width=\paperwidth,height=\paperheight]{./img/ARCHIVO2.png}
%		\fi
%	}
%}

\usepackage{graphicx}

%\label{PAGE_LAYOUT}
%% https://en.wikibooks.org/wiki/LaTeX/Page_Layout#Page_dimensions
%\setlength\oddsidemargin{-0.04cm}
%\setlength\evensidemargin{-0.04cm}
%\setlength\topmargin{0cm}
%\setlength\headheight{0.5cm}
%\setlength\headsep{0.5cm}
%\setlength\footskip{1cm}
%\setlength\textwidth{16cm}                      % ancho para apunte
%\setlength\textheight{23cm}                     % largo para apunte
%\usepackage{geometry}
%\geometry{
%	a4paper,
%	total={210mm,297mm},
%	left=30mm,
%	right=30mm,
%	top=40mm,
%	bottom=30mm,
%}

\label{HYPERREF}
\usepackage{hyperref}              % Incluimos la posibilidad de agregar 
%hipervínculos.
	\hypersetup{                   % Opciones del paquete de hipervínculos.
%	    bookmarks=true,            % show bookmarks bar?
	    unicode=true,              % non-Latin characters in Acrobat’s bookmarks
	    pdftoolbar=true,           % show Acrobat’s toolbar?
	    pdfmenubar=true,           % show Acrobat’s menu?
	    pdffitwindow=false,        % window fit to page when opened
	    pdfstartview={FitH},       % fits the width of the page to the window
	    pdftitle={\Title},         % 
	    %title
	    pdfauthor={Allen Downey, Jeffrey Elkner, Chris Meyers | Andrés Becerra 
	    Sandoval | \AutorNombreCompleto},  % author
	    pdfsubject={\PDFSubject},  % subject 
	    %of the document
%	    pdfcreator={\LaTeX},       % creator of the document
%	    pdfproducer={},            % producer of the document
	    pdfkeywords={} {},         % list of keywords
	    pdfnewwindow=true,         % links in new window
	    colorlinks=true,           % false: boxed links; true: colored links
	    linkcolor=red,             % color of internal links (change box color 
%with
	                               % linkbordercolor)
	    citecolor=green,           % color of links to bibliography
	    filecolor=magenta,         % color of file links
	    urlcolor=cyan              % color of external links
}

\label{MY_COMMANDS}
\newcommand{\HRule}{\rule{\linewidth}{0.5mm}} % Nueva línea horizontal.

\label{MY_COLORS}
\definecolor{gray15p}{rgb}{0.85,0.85,0.85}
\definecolor{gray05p}{rgb}{0.95,0.95,0.95}

\label{LISTINGS}
\usepackage{listings}                        % Permite mostrar código de forma 
%mas linda
	\lstset{
		backgroundcolor=\color{gray05p},     % choose the background color;
		basicstyle=\ttfamily,                % the size of the fonts that are 
%used for the code
		breakatwhitespace=false,             % sets if automatic breaks should 
%only happen at
		                                     % whitespace
		breaklines=true,                     % sets automatic line breaking
		captionpos=b,                        % sets the caption-position to 
%bottom
		commentstyle=\color{green},          % comment style
		deletekeywords={},                   % if you want to delete keywords 
%from the given
		                                     % language
		escapeinside={\%*}{*)},              % if you want to add LaTeX within 
%your code
		extendedchars=true,                  % lets you use non-ASCII 
%characters; for 8-bits
		                                     % encodings only, does not work 
%with UTF-8
		frame=single,                        % adds a frame around the code
		frameround=tttt,
		keepspaces=true,                     % keeps spaces in text, useful for 
%keeping
		                                     % indentation of code (possibly 
%needs 
		                                     % columns=flexible)
		keywordstyle=\color{blue}\textbf,    % keyword style
		language=Python,                     % the language of the code
		morecomment=[l]{\#}                  % Agrega el "comentario" 
		classoffset=1,
		morekeywords={},                     % if you want to add more keywords 
%to the set
		keywordstyle=\color{red}\textbf,
		classoffset=0,
		numbers=left,                        % where to put the line-numbers; 
%possible values
		                                     % are (none, left, right)
		numbersep=5pt,                       % how far the line-numbers are 
%from the code
		numberstyle=\tiny\color{gray},       % the style that is used for the 
%line-numbers
		rulecolor=\color{black},             % if not set, the frame-color may 
%be changed on
		                                     % line-breaks within not-black 
%text (e.g. comments
		                                     % (green here))
		showspaces=false,                    % show spaces everywhere adding 
%particular
		                                     % underscores; it overrides 
%'showstringspaces'
		showstringspaces=false,              % underline spaces within strings 
%only
		showtabs=false,                      % show tabs within strings adding 
%particular
		                                     % underscores
		stepnumber=1,                        % the step between two 
%line-numbers. If it's 1, 
		                                     % each line will be numbered
		stringstyle=\color{magenta},         % string literal style
		tabsize=2,                           % sets default tabsize to 2 spaces
		title=\lstname,                      % show the filename of files 
%included with
		                                     % \lstinputlisting; also try 
%caption instead 
		                                     % of title
	}
	
\usepackage[final]{pdfpages} % Include PDF directly into your document.