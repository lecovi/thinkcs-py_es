
\chapter{Clases y métodos}

\section{Características de orientación a objetos}

\index{lenguaje de programación orientado a objetos} \index{programación orientada a objetos}

Python es un \textbf{lenguaje de programación orientado a objetos},
lo que quiere decir que proporciona características que soportan la
\textbf{programación orientada a objetos}.

No es fácil definir la programación orientada a objetos, pero ya hemos
notado algunos de sus elementos clave:
\begin{itemize}
\item Los programas se construyen a partir de definiciones de objetos y
definiciones de funciones; la mayoría de los cómputos se hacen con
base en objetos.
\item Cada definición de objetos corresponde a algún concepto o cosa del
mundo real, y las funciones que operan sobre esos objetos corresponden
a las maneras en que los conceptos o cosas reales interactúan.
\end{itemize}
Por ejemplo, la clase \texttt{Hora}, definida en el Capítulo~\ref{time},
corresponde a la forma en que la gente registra las horas del día
y las funciones que definimos corresponden a la clase de cosas que
la gente hace con horas. Similarmente, las clases \texttt{Punto} y
\texttt{Rectangulo} corresponden a los conocidos conceptos geométricos

Hasta aquí, no hemos aprovechado las características que Python proporciona
para soportar la programación orientada a objetos. De hecho, estas
características no son necesarias. La mayoría sólo proporciona una
sintaxis alternativa para cosas que ya hemos logrado; pero, en muchos
casos, esta forma alternativa es más concisa y comunica de una manera
mas precisa la estructura de los programas.

Por ejemplo, en el programa \texttt{Hora} no hay una conexión obvia
entre la definición de clase y las definiciones de funciones. Después
de examinarlo un poco, es evidente que todas las funciones toman como
parámetro al menos un objeto \texttt{Hora}.

Esta observación es la motivación para los \textbf{métodos}. Ya hemos
visto algunos métodos como \texttt{keys} y \texttt{values}, que llamamos
sobre diccionarios. Cada método se asocia con una clase y está pensado
para invocarse sobre instancias de dicha clase.

\index{método} \index{función} \index{instancia!objeto} \index{objeto instancia}

Los métodos son como las funciones, pero con dos diferencias:
\begin{itemize}
\item Los métodos se definen adentro de una definición de clase, a fin de
marcar explícitamente la relación entre la clase y éstos.
\item La sintaxis para llamar o invocar un método es distinta que para las
funciones.
\end{itemize}
En las siguientes secciones tomaremos las funciones de los capítulos
anteriores y las transformaremos en métodos. Esta transformación es
totalmente mecánica; se puede llevar a cabo siguiendo una secuencia
de pasos. Si usted se siente cómodo al transformar de una forma a
la otra, será capaz de escoger lo mejor de cada lado para resolver
los problemas que tenga a la mano.

\section{\texttt{imprimirHora}}

\label{printTime} \index{imprimir!objetos}

En el capítulo~\ref{time}, definimos una clase denominada \texttt{Hora}
y usted escribió una función denominada \texttt{imprimirHora}, que
lucía así:
\begin{pythoncode}
class Hora:
  pass

def imprimirHora(h):
  print(str(h.hora) + ":" + 
        str(h.minutos) + ":" + 
        str(h.segundos))
\end{pythoncode}

Para llamar esta función, le pasamos un objeto \texttt{Hora} como
parámetro:
\begin{pyconcode}
>>> horaActual = Hora()
>>> horaActual.hora = 9
>>> horaActual.minutos = 14
>>> horaActual.segundos = 30
>>> imprimirHora(horaActual)
\end{pyconcode}

Para convertir \texttt{imprimirHora} en un método todo lo que tenemos
que hacer es ponerla adentro de la definición de clase. Note como
ha cambiado la indentación.

\begin{pythoncode}
class Hora:
  def imprimirHora(h):
    print( str(h.hora) + ":" + 
          str(h.minutos) + ":" + 
          str(h.segundos))
\end{pythoncode}
 Ahora podemos llamar a \texttt{imprimirHora} usando la notación punto.

\index{notación punto}
\begin{pyconcode}
>>> horaActual.imprimirHora()
\end{pyconcode}

Como de costumbre, el objeto en el que el método se llama aparece
antes del punto y el nombre del método va a la derecha. El objeto
al cual se invoca el método se asigna al primer parámetro, así que
\texttt{horaActual} se asigna al parámetro \texttt{h}.

Por convención, el primer parámetro de un método se denomina \texttt{self}
(en inglés, eso es algo como ``sí mismo''). La razón para hacerlo
es un poco tortuosa, pero se basa en una metáfora muy útil.

La sintaxis para una llamada de función, \texttt{imprimirHora(horaActual)},
sugiere que la función es el agente activo. Dice algo como ``Hey,
\texttt{imprimirHora}! Aquí hay un objeto para que imprimas''.

En la programación orientada a objetos, los objetos son los agentes
activos. Una invocación como \texttt{horaActual.imprimirHora()} dice
algo como ``Hey, objeto \texttt{horaActual}! Por favor, imprímase
a sí mismo!''.

Este cambio de perspectiva parece ser sólo ``cortesía'', pero puede
ser útil. En los ejemplos que hemos visto no lo es. Pero, el transferir
la responsabilidad desde las funciones hacia los objetos hace posible
escribir funciones más versátiles y facilita la reutilización y el
mantenimiento de código.

\section{Otro ejemplo}

Convirtamos \texttt{incrementar} (de la Sección~\ref{increment})
en un método. Para ahorrar espacio, omitiremos los métodos que ya
definimos, pero usted debe conservarlos en su programa:
\begin{pythoncode}
class Hora:
  # Las definiciones anteriores van aquí...
  
  def incrementar(self, segundos):
    self.segundos = self.segundos + segundos

    if self.segundos >= 60:
      self.segundos = self.segundos - 60
      self.minutos = self.minutos + 1

    if self.minutos >= 60:
      self.minutos = self.minutos - 60
      self.hora = self.hora + 1

    return self
\end{pythoncode}
 La transformación es totalmente mecánica —ponemos la definición del
método adentro de la clase y cambiamos el nombre del primer parámetro.

Ahora podemos llamar a \texttt{incrementar} como método:
\begin{pythoncode}
horaActual.incrementar(500)
\end{pythoncode}

Nuevamente, el objeto con el cual se invoca el método se asigna al
primer parámetro, \texttt{self}. El segundo parámetro, \texttt{segundos}
recibe el valor \texttt{500}.

\section{Un ejemplo más complejo}

El método \texttt{despues} es un poco más complejo ya que opera sobre
dos objetos \texttt{Hora}, no sólo uno. Solamente podemos convertir
uno de los parámetros a \texttt{self}; el otro continúa igual:

\begin{pythoncode}
class Hora:
  # Las definiciones anteriores van aqui...

  def despues(self, hora2):
    if self.hora > hora2.hora:
      return True
    if self.hora < hora2.hora:
      return False

    if self.minutos > hora2.minutos:
      return True
    if self.minutos < hora2.minutos:
      return False

    if self.segundos > hora2.segundos:
      return True
    return False
\end{pythoncode}
 Llamamos a este método sobre un objeto y le pasamos el otro como
argumento:
\begin{pythoncode}
if horaComer.despues(horaActual):
  print("El pan estará listo para comer en un momento.")
\end{pythoncode}

Casi se puede leer el llamado en lenguaje natural:``Si la hora para
Comer viene después de la hora Actual, entonces ...''.

\section{Argumentos opcionales}

Hemos visto varias funciones primitivas que toman un número variable
de argumentos. Por ejemplo, \texttt{string.find} puede tomar dos,
tres o cuatro.

Es posible escribir funciones con listas de argumentos opcionales.
Por ejemplo, podemos mejorar nuestra versión de \texttt{buscar} para
que sea tan sofisticada como \texttt{string.find}.

Esta es la versión original que introdujimos en la Sección~\ref{find}:

\pagebreak{}

\begin{pythoncode}
def buscar(cad, c):
  indice = 0
  while indice < len(cad):
    if cad[indice] == c:
      return indice
    indice = indice + 1
  return -1
\end{pythoncode}
 Esta es la nueva versión, mejorada:
\begin{pythoncode}
def buscar(cad, c,ini=0):
  indice = ini
  while indice < len(cad):
    if cad[indice] == c:
      return indice
    indice = indice + 1
  return -1
\end{pythoncode}

El tercer parámetro, \texttt{ini}, es opcional, ya que tiene un valor
por defecto, \texttt{0}. Si llamamos a \texttt{buscar} con dos argumentos,
se usa el valor por defecto y la búsqueda se hace desde el principio
de la cadena:
\begin{pythoncode}
>>> buscar("apple", "p")
1
\end{pythoncode}

Si se pasa el tercer parámetro, este \textbf{sobreescribe} el valor
por defecto:
\begin{pyconcode}
>>> buscar("apple", "p", 2)
2
>>> buscar("apple", "p", 3)
-1
\end{pyconcode}

\section{El método de inicialización}

\index{método de inicialización} \index{método!de inicialización}

El \textbf{de inicialización} es un método especial que se llama cuando
se crea un objeto. El nombre de este método es \texttt{\_\_init\_\_}
(dos caracteres de subrayado, seguidos por \texttt{init}, y luego
dos caracteres de subrayado más). Un método de inicialización para
la clase \texttt{Hora} se presenta a continuación:
\begin{pythoncode}
class Hora:
  def __init__(self, hora=0, minutos=0, segundos=0):
    self.hora = hora
    self.minutos = minutos
    self.segundos = segundos
\end{pythoncode}

No hay conflicto entre el atributo \texttt{self.hora} y el parámetro
\texttt{hora}. La notación punto especifica a qué variable nos estamos
refiriendo.

\index{notación punto}

Cuando llamamos al método constructor de \texttt{Hora}, los argumentos
se pasan a \texttt{init}:
\begin{pyconcode}
>>> horaActual = Hora(9, 14, 30)
>>> horaActual.imprimirHora()
>>> 9:14:30
\end{pyconcode}

Como los parámetros son opcionales, se pueden omitir:
\begin{pyconcode}
>>> horaActual = Hora()
>>> horaActual.imprimirHora()
>>> 0:0:0
\end{pyconcode}

O podemos pasar solo un parámetro:
\begin{pyconcode}
>>> horaActual = Hora(9)
>>> horaActual.imprimirHora()
>>> 9:0:0
\end{pyconcode}

O, sólo los dos primeros:
\begin{pyconcode}
>>> horaActual = Hora(9, 14)
>>> horaActual.imprimirHora()
>>> 9:14:0
\end{pyconcode}

Finalmente, podemos proporcionar algunos parámetros, nombrándolos
explícitamente:
\begin{pyconcode}
>>> horaActual = Hora(segundos = 30, hora = 9)
>>> horaActual.imprimirHora()
>>> 9:0:30
\end{pyconcode}

\section{Reconsiderando la clase Punto}

\index{clase Punto} \index{clase!Punto}

Reescribamos la clase \texttt{Punto} de la Sección~\ref{point} en
un estilo más orientado a objetos:
\begin{pythoncode}
class Punto:
  def __init__(self, x=0, y=0):
    self.x = x
    self.y = y

  def __str__(self):
    return '(' + str(self.x) + ', ' + str(self.y) + ')'
\end{pythoncode}

El método de inicialización toma los valores $x$ y $y$ como parámetros
opcionales, el valor por defecto que tienen es 0.

El método \texttt{\_\_str\_\_}, retorna una representación de un objeto
\texttt{Punto} en forma de cadena de texto. Si una clase proporciona
un método denominado \texttt{\_\_str\_\_}, éste sobreescribe el comportamiento
por defecto de la función primitiva \texttt{str}.
\begin{pyconcode}
>>> p = Punto(3, 4)
>>> str(p)
'(3, 4)'
\end{pyconcode}

Imprimir un objeto \texttt{Punto} implícitamente invoca a \texttt{\_\_str\_\_}
o sobre éste, así que definir a \texttt{\_\_str\_\_} también cambia
el comportamiento de la sentencia \texttt{print}:
\begin{pyconcode}
>>> p = Punto(3, 4)
>>> print(p)
(3, 4)
\end{pyconcode}

Cuando escribimos una nueva clase, casi siempre empezamos escribiendo
\texttt{\_\_init\_\_}, ya que facilita la instanciación de objetos,
y \texttt{\_\_str\_\_}, que casi siempre es esencial para la depuración.

\section{Sobrecarga de operadores}

\label{operator overloading} \index{operadores sobrecarga de} \index{operadores!sobrecarga}
\index{producto punto} \index{multiplicación escalar}

Algunos lenguajes hacen posible cambiar la definición de los operadores
primitivos cuando se aplican sobre tipos definidos por el programador.
Esta característica se denomina \textbf{sobrecarga de operadores}.
Es especialmente útil para definir tipos de datos matemáticos.

Por ejemplo, para sobrecargar el operador suma, \texttt{+}, proporcionamos
un método denominado \texttt{\_\_add\_\_}:
\begin{pythoncode}
class Punto:
  # los métodos definidos previamente van aquí...

  def __add__(self, otro):
    return Punto(self.x + otro.x, self.y + otro.y)
\end{pythoncode}

Como de costumbre, el primer parámetro es el objeto con el cual se
invoca el método. El segundo parámetro se denomina con la palabra
\texttt{otro} para marcar la distinción entre éste y \texttt{self}.
Para sumar dos \texttt{Punto}s, creamos y retornamos un nuevo \texttt{Punto}
que contiene la suma de las coordenadas en el eje $x$ y la suma de
las coordenadas en el eje $y$.

Ahora, cuando aplicamos el operador \texttt{+} a dos objetos \texttt{Punto},
Python hace el llamado del método \texttt{\_\_add\_\_}:
\begin{pyconcode}
>>>   p1 = Punto(3, 4)
>>>   p2 = Punto(5, 7)
>>>   p3 = p1 + p2
>>>   print(p3)
(8, 11)
\end{pyconcode}

La expresión \texttt{p1 + p2} es equivalente a \texttt{p1.\_\_add\_\_(p2)},
pero luce mucho mas elegante.

Hay varias formas de sobrecargar el comportamiento del operador multiplicación:
definiendo un método \texttt{\_\_mul\_\_}, o \texttt{\_\_rmul\_\_},
o ambos.

Si el operando izquierdo de \texttt{{*}} es un \texttt{Punto}, Python
invoca a \texttt{\_\_mul\_\_}, asumiendo que el otro operando también
es un \texttt{Punto}. Calcula el \textbf{producto escalar} de los
dos puntos de acuerdo a las reglas del álgebra lineal:
\begin{pythoncode}
def __mul__(self, otro):
  return self.x * otro.x + self.y * otro.y
\end{pythoncode}

Si el operando izquierdo de \texttt{{*}} es un tipo primitivo y el
operando derecho es un \texttt{Punto}, Python llama a \texttt{\_\_rmul\_\_},
que ejecuta la \textbf{multiplicación escalar }:
\begin{pythoncode}
def __rmul__(self, otro):
  return Punto(otro * self.x,  otro * self.y)
\end{pythoncode}

El resultado ahora es un nuevo \texttt{Punto} cuyas coordenadas son
múltiplos de las originales. Si \texttt{otro} pertenece a un tipo
que no se puede multiplicar por un número de punto flotante, la función
\texttt{\_\_rmul\_\_} producirá un error.

Este ejemplo ilustra las dos clases de multiplicación:
\begin{pyconcode}
>>> p1 = Punto(3, 4)
>>> p2 = Punto(5, 7)
>>> print(p1 * p2)
43
>>> print(2 * p2)
(10, 14)
\end{pyconcode}
 ¿Que pasa si tratamos de evaluar \texttt{p2 {*} 2}? Ya que el primer
parámetro es un \texttt{Punto}, Python llama a \texttt{\_\_mul\_\_}
con \texttt{2} como segundo argumento. Dentro de \texttt{\_\_mul\_\_},
el programa intenta acceder al valor \texttt{x} de \texttt{otro},
lo que falla porque un número entero no tiene atributos:

\begin{pythoncode}
>>> print(p2 * 2)
AttributeError: 'int' object has no attribute 'x'
\end{pythoncode}
 Desafortunadamente, el mensaje de error es un poco opaco. Este ejemplo
demuestra una de las dificultades de la programación orientada a objetos.
Algunas veces es difícil saber qué código está ejecutándose.

Para un ejemplo completo de sobrecarga de operadores vea el capítulo
\ref{overloading}.

\section{Polimorfismo}

\index{polimorfismo}

La mayoría de los métodos que hemos escrito sólo funcionan para un
tipo de dato específico. Cuando se crea un nuevo tipo de objeto, se
escriben métodos que operan sobre ese tipo.

Pero hay ciertas operaciones que se podrían aplicar a muchos tipos,
un ejemplo de éstas son las operaciones aritméticas de las secciones
anteriores. Si muchos tipos soportan el mismo conjunto de operaciones,
usted puede escribir funciones que trabajen con cualquiera de estos
tipos.

Por ejemplo la operación \texttt{multsuma} (que se usa en el álgebra
lineal) toma tres parámetros, multiplica los primeros dos y luego
suma a esto el tercero. En Python se puede escribir así:

\begin{pythoncode}
def multsuma (x, y, z):
  return x * y + z
\end{pythoncode}
 Este método funcionará para cualesquier valores de \texttt{x} e \texttt{y}
que puedan multiplicarse, y para cualquier valor de \texttt{z} que
pueda sumarse al producto.

Podemos llamarla sobre números:

\begin{pyconcode}
>>> multsuma (3, 2, 1)
7
\end{pyconcode}
 O sobre \texttt{Punto}s:
\begin{pyconcode}
>>> p1 = Punto(3, 4)
>>> p2 = Punto(5, 7)
>>> print(multsuma (2, p1, p2))
(11, 15)
>>> print(multsuma (p1, p2, 1))
44
\end{pyconcode}

En el primer caso, el \texttt{Punto} se multiplica por un escalar
y luego se suma a otro \texttt{Punto}. En el segundo caso, el producto
punto produce un valor numérico, así que el tercer parámetro también
tiene que ser un número.

Una función como ésta, que puede tomar parámetros con tipos distintos
se denomina \textbf{polimórfica}.

Otro ejemplo es la función \texttt{derechoyAlReves}, que imprime una
lista dos veces, al derecho y al revés:

\begin{pythoncode}
def derechoyAlReves(l):
  import copy
  r = copy.copy(l)
  r.reverse()
  print(str(l) + str(r))
\end{pythoncode}
 Como el método \texttt{reverse} es una función modificadora, tenemos
que tomar la precaución de hacer una copia de la lista antes de llamarlo.
De esta forma la lista que llega como parámetro no se modifica.

Aquí hay un ejemplo que aplica \texttt{derechoyAlReves} a una lista:

\begin{pyconcode}
>>> miLista = [1, 2, 3, 4]
>>> derechoyAlReves(miLista)
[1, 2, 3, 4][4, 3, 2, 1]
\end{pyconcode}
 Por supuesto que funciona para listas, esto no es sorprendente. Lo
que sería sorprendente es que pudiéramos aplicarla a un \texttt{Punto}.

Para determinar si una función puede aplicarse a un nuevo tipo de
dato usamos la regla fundamental del polimorfismo:
\begin{quote}
\textbf{Si todas las operaciones adentro de la función pueden aplicarse
al otro tipo, la función puede aplicarse al tipo.} 
\end{quote}
Las operaciones que usa el método son \texttt{copy}, \texttt{reverse},
y \texttt{print}.

\texttt{copy} funciona para cualquier objeto, y como ya hemos escrito
un método \texttt{\_\_str\_\_} para los \texttt{Punto}s, lo único
que nos falta es el método \texttt{reverse} dentro de la clase \texttt{Punto}:

\begin{pythoncode}
def reverse(self):
  self.x , self.y = self.y, self.x
\end{pythoncode}
 Entonces podemos aplicar \texttt{derechoyAlReves} a objetos \texttt{Punto}:

\begin{pyconcode}
>>>   p = Punto(3, 4)
>>>   derechoyAlReves(p)
(3, 4)(4, 3)
\end{pyconcode}
 El mejor tipo de polimorfismo es el que no se pretendía lograr, aquel
en el que se descubre que una función escrita puede aplicarse a un
tipo para el cual no se había planeado hacerlo.

\section{Glosario}
\begin{description}
\item [{Lenguaje orientado a objetos:}] lenguaje que tiene características,
como las clases definidas por el usuario y la herencia, que facilitan
la programación orientada a objetos.
\item [{Programación orientada a objetos:}] estilo de programación en
el que los datos y las operaciones que los manipulan se organizan
en clases y métodos.
\item [{Método:}] función que se define adentro de una clase y se llama
sobre instancias de ésta.
\item [{Sobreescribir:}] reemplazar un valor preexistente. Por ejemplo,
se puede reemplazar un parámetro por defecto con un argumento particular
y un método ya definido, proporcionando un nuevo método con el mismo
nombre.
\item [{Método de inicialización:}] método especial que se llama automáticamente
cuando se crea un nuevo objeto. Inicializa los atributos del objeto.
\item [{Sobrecarga de operadores:}] extender el significado de los operadores
primitivos (\texttt{+}, \texttt{-}, \texttt{{*}}, \texttt{>}, \texttt{<},
etc.) de forma que acepten tipos definidos por el usuario.
\item [{Producto punto:}] operación del álgebra lineal que multiplica
dos \texttt{Punto}s y produce un valor numérico.
\item [{Multiplicación escalar:}] operación del álgebra lineal que multiplica
cada una de las coordenadas de un \texttt{Punto} por un valor numérico.
\item [{Polimórfica:}] función que puede operar sobre varios tipos de datos.
Si todas las operaciones que se llaman dentro de la función se le
pueden aplicar al tipo de dato, entonces la función puede aplicársela
al tipo.

\index{lenguaje de programación orientado a objetos} \index{método}
\index{método de inicialización} \index{sobreescribir} \index{sobrecarga}
\index{sobrecarga de operadores} \index{producto punto} \index{multiplicación escalar}
\index{polimórfica}
\end{description}

\section{Ejercicios}
\begin{enumerate}
\item Convierta \texttt{convertirASegundos} (de la Sección~\ref{convert})
a un método de la clase \texttt{Hora}.
\item Añada un cuarto parámetro \texttt{fin} a la función \texttt{buscar}
que especifique hasta donde continuar la búsqueda.

Advertencia: Este ejercicio tiene una cascarita. El valor por defecto
de \texttt{fin} debería ser \texttt{len(cad)}, pero esto no funciona.
Los valores por defecto se evalúan en el momento de definición de
las funciones, no cuando se llaman. Cuando se define \texttt{buscar},
\texttt{cad} no existe todavía, así que no se puede obtener su longitud.
\item Agregue un método \texttt{\_\_sub\_\_(self, otro)} que sobrecargue
el operador resta de la clase \texttt{Punto}, y pruébelo.
\end{enumerate}

