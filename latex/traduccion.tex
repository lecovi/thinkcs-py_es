% LaTeX source for textbook ``How to think like a computer scientist''
% Copyright (c)  2001  Allen B. Downey, Jeffrey Elkner, and John Dewey.

% Permission is granted to copy, distribute and/or modify this
% document under the terms of the GNU Free Documentation License,
% Version 1.1  or any later version published by the Free Software
% Foundation; with the Invariant Sections being "Contributor List",
% with no Front-Cover Texts, and with no Back-Cover Texts. A copy of
% the license is included in the section entitled "GNU Free
% Documentation License".

% This distribution includes a file named fdl.tex that contains the text
% of the GNU Free Documentation License.  If it is missing, you can obtain
% it from www.gnu.org or by writing to the Free Software Foundation,
% Inc., 59 Temple Place - Suite 330, Boston, MA 02111-1307, USA.

\chapter{Traducción al español}


Al comienzo de junio de 2007 tomé la iniciativa de traducir el texto 
``How to think like a Computer Scientist, with Python'' al español. Rápidamente
me dí cuenta de que ya había un trabajo inicial de traducción empezado por:

\begin{itemize}

\item Angel Arnal
\item I Juanes
\item Litza Amurrio
\item Efrain Andia

\end{itemize}

Ellos habían traducido los capítulos 1, 2, 10, 11, y 12, así como el prefacio,
la introducción y la lista de colaboradores. Tomé su valioso trabajo como
punto de partida, adapté los capítulos, traduje las secciones faltantes 
del libro y añadí un primer capítulo adicional sobre solución de problemas.

Aunque el libro traduce la primera edición del original,
todo se ha corregido para que sea compatible con Python 3.5,
por ejemplo se usan booleanos en vez de enteros en los 
condicionales y ciclos. 

Para realizar este trabajo ha sido invaluable la colaboración de familiares, 
colegas, amigos y estudiantes que han señalado errores, expresiones confusas y han 
aportado toda clase de sugerencias constructivas. Mi agradecimiento va para los
traductores antes mencionados y para los estudiantes de Biología que tomaron 
el curso de Informática en la Pontificia Universidad Javeriana (Cali-Colombia), 
durante el semestre 2014-1:

\begin{itemize}
   \item Estefanía Lopez 
   \item Gisela Chaves 
   \item Marlyn Zuluaga 
   \item Francisco Sanchez
   \item María del Mar Lopez
   \item Diana Ramirez
   \item Guillermo Perez
   \item María Alejandra Gutierrez
   \item Sara Rodriguez
   \item Claudia Escobar 
\end{itemize}
  

\vspace{0.25in}
\begin{flushleft}
Andrés Becerra Sandoval \\
Pontificia Universidad Javeriana - Seccional Cali \\
abecerra@cic.puj.edu.co \\

\end{flushleft}


