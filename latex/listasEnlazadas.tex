
\chapter{Listas enlazadas}

\label{lista} \index{lista}

\section{Referencias incrustadas}

\index{referencia} \index{referencia incrustada} \index{referencia!incrustada}
\index{lista enlazada} \index{lista!enlazada} \index{nodo} \index{carga}

Hemos visto ejemplos de atributos (denominados \textbf{referencias
incrustadas}) que se refieren a otros objetos en la sección \ref{embedded}.
Una estructura de datos muy común (la \textbf{lista enlazada}), toma
ventaja de esta posibilidad.

Las listas enlazadas están hechas de \textbf{nodos}, que contienen
una referencia al siguiente nodo en la lista. Además, cada nodo contiene
una información denominada la \textbf{carga}.

Una lista enlazada se considera como una \textbf{estructura de datos
recursiva} si damos la siguiente definición.
\begin{quote}
Una lista enlazada es: 

\begin{itemize}
\item la lista vacía, representada por el valor \texttt{None}, o
\item un nodo que contiene una carga y una referencia a una lista enlazada.
\end{itemize}
\end{quote}
\index{estructura de datos recursiva} \index{estructura de datos!recursiva}

Las estructuras de datos recursivas se implementan naturalmente con
métodos recursivos.

\section{La clase \texttt{Nodo} }

\index{clase Nodo} \index{clase!Nodo}

Empezaremos con los métodos básicos de inicialización y el \texttt{\_\_str\_\_}
para que podamos crear y desplegar objetos:

\beforeverb 
\begin{pythoncode}
class Nodo:
  def __init__(self, carga=None, siguiente=None):
    self.carga = carga
    self.siguiente  = siguiente

  def __str__(self):
    return str(self.carga)
\end{pythoncode}
\afterverb Los parámetros para el método de inicialización son opcionales.
Por defecto la carga y el enlace \texttt{siguiente}, reciben el valor
\texttt{None}.

La representación textual de un nodo es la representación de la carga.
Como cualquier valor puede ser pasado a la función \texttt{str} ,
podemos almacenar cualquier tipo de valor en la lista.

Para probar la implementación, podemos crear un \texttt{Nodo} e imprimirlo:

\beforeverb 
\begin{pyconcode}
>>> nodo = Nodo("test")
>>> print(nodo)
test
\end{pyconcode}
\afterverb Para hacerlo más interesante, vamos a pensar en una lista
con varios nodos:

\beforeverb 
\begin{pyconcode}
>>> nodo1 = Nodo(1)
>>> nodo2 = Nodo(2)
>>> nodo3 = Nodo(3)
\end{pyconcode}
\afterverb Este código crea tres nodos, pero todavía no tenemos una
lista porque estos no estan \textbf{enlazados}. El diagrama de estados
luce así:

\beforefig \centerline{\includegraphics[scale=0.7]{illustrations/link1}}
\afterfig

Para enlazar los nodos, tenemos que lograr que el primer nodo se refiera
al segundo, y que el segundo se refiera al tercero:

\beforeverb 
\begin{pyconcode}
>>> nodo1.siguiente = nodo2
>>> nodo2.siguiente = nodo3
\end{pyconcode}
\afterverb La referencia del tercer nodo es \texttt{None}, lo que
indica que es el último nodo de la lista. Ahora el diagrama de estados
luce así:

\beforefig \centerline{\includegraphics[scale=0.9]{illustrations/link2}}
\afterfig

Ahora usted sabe cómo crear nodos y enlazarlos para crear listas.
Lo que todavía no está claro, es el por qué hacerlo.

\section{Listas como colecciones}

\index{colección}

Las listas son útiles porque proporcionan una forma de ensamblar múltiples
objetos en una entidad única, a veces llamada \textbf{colección}.
En el ejemplo, el primer nodo de la lista sirve como referencia a
toda la lista.

\index{lista!imprimir} \index{lista!como parámetro}

Para pasar la lista como parámetro, sólo tenemos que pasar una referencia
al primer nodo. Por ejemplo, la función \texttt{imprimirLista} toma
un solo nodo como argumento. Empieza con la cabeza de la lista, imprime
cada nodo hasta llegar al final:

\beforeverb 
\begin{pythoncode}
def imprimirLista(nodo):
  while nodo:
    print(nodo),
    nodo = nodo.siguiente
  print
\end{pythoncode}
\afterverb Para llamar este método, pasamos una referencia al primer
nodo:

\beforeverb 
\begin{pyconcode}
>>> imprimirLista(nodo1)
1 2 3
\end{pyconcode}
\afterverb Dentro de \texttt{imprimirLista} tenemos una referencia
al primer nodo de la lista, pero no hay variable que se refiera a
los otros nodos. Tenemos que usar el valor \texttt{siguiente} de cada
nodo para obtener el siguiente nodo.

Para recorrer una lista enlazada, es muy común usar una variable de
ciclo como \texttt{nodo} para que se refiera a cada uno de los nodos
en cada momento.

\index{variable de ciclo} \index{lista!recorrido} \index{recorrido}

Este diagrama muestra el valor de \texttt{lista} y los valores que
\texttt{nodo} toma:

\beforefig \centerline{\includegraphics{illustrations/link3}}
\afterfig
\begin{quote}
{\em Por convención, las listas se imprimen entre corchetes y los
elementos se separan por medio de comas, como en el ejemplo \texttt{{[}1,
2, 3{]}}. Como ejercicio modifique \texttt{imprimirLista} de forma
que muestre la salida en este formato.} 
\end{quote}

\section{Listas y recursión}

\label{listrecursion} \index{lista!recorrido recursivo} \index{recorrido}

Es natural implementar muchas operaciones sobre listas por medio de
métodos recursivos. Por ejemplo, el siguiente algoritmo recursivo
imprime una lista al revés:
\begin{enumerate}
\item Separe la lista en dos partes: el primer nodo (la cabeza de la lista);
y el resto.
\item Imprima el resto al revés.
\item Imprima la cabeza.
\end{enumerate}
Por supuesto, el paso 2, el llamado recursivo asume que ya tenemos
una forma de imprimir una lista al revés. Si asumimos que esto es
así —el salto de fe—entonces podemos convencernos de que el algoritmo
trabaja correctamente.

\index{salto de fe} \index{listas!imprimiendo al revés}

Todo lo que necesitamos es un caso base y una forma de demostrar que
para cualquier lista, eventualmente llegaremos al caso base. Dada
la definición recursiva de una lista, un caso base natural es la lista
vacía, representada por \texttt{None}:

\beforeverb 
\begin{pythoncode}
def imprimirAlReves(lista):
  if lista == None: 
    return
  cabeza = lista
  resto = lista.siguiente
  imprimirAlReves(resto)
  print(cabeza),
\end{pythoncode}
\afterverb La primera línea resuelve el caso base. Las siguientes
separan la \texttt{cabeza} y el \texttt{resto}. Las últimas dos líneas
imprimen la lista. La coma al final de la última línea evita que Python
introduzca una nueva línea después de cada nodo.

Ahora llamamos a este método:

\beforeverb 
\begin{pyconcode}
>>> imprimirAlReves(nodo1)
3 2 1
\end{pyconcode}
\afterverb El efecto es una impresión la lista, al revés.

Una pregunta natural que usted se puede estar formulando es, ¿por
qué razón \texttt{imprimirAlReves} e \texttt{imprimirLista} son funciones
y no métodos en la clase \texttt{Nodo}? La razón es que queremos usar
a \texttt{None} para representar la lista vacía y no se puede llamar
un método sobre \texttt{None} en Python. Esta limitación hace un poco
engorroso escribir el código para manipulación de listas siguiendo
la programación orientada a objetos.

¿Podemos demostrar que \texttt{imprimirAlReves} va a terminar siempre?
En otras palabras, ¿llegará siempre al caso base? De hecho, la respuesta
es negativa, algunas listas causarán un error de ejecución.

\section{Listas infinitas }

\index{lista infinita} \index{lista!infinita} \index{ciclos!en listas}
\index{lista!ciclo}

No hay manera de evitar que un nodo se refiera a un nodo anterior
en la lista hacia ``atrás''. Incluso, puede referirse a sí mismo.
Por ejemplo, la siguiente figura muestra una lista con dos nodos,
uno de los cuales se refiere a sí mismo:

\beforefig \centerline{\includegraphics{illustrations/link4}}
\afterfig

Si llamamos a \texttt{imprimirLista} sobre esta lista, iteraría para
siempre. Si llamamos a \texttt{imprimirAlReves}, se haría recursión
hasta causar un error en tiempo de ejecución. Este comportamiento
hace a las listas circulares muy difíciles de manipular.

Sin embargo, a veces son muy útiles. Por ejemplo, podemos representar
un número como una lista de dígitos y usar una lista infinita para
representar una fracción periódica.

Así que no es posible demostrar que \texttt{imprimirLista} e \texttt{imprimirAlReves}
terminen. Lo mejor que podemos hacer es probar la sentencia, ``Si
la lista no tiene referencias hacia atrás, los métodos terminarán.''.
Esto es una \textbf{precondición}. Impone una restricción sobre los
parámetros y describe el comportamiento del método si ésta se cumple.
Más adelante veremos otros ejemplos.

\index{precondición}

\section{El teorema de la ambigüedad fundamental}

\index{ambigüedad!teorema fundamental} \index{teorema!fundamental de la ambigüedad}

Una parte de \texttt{imprimirAlReves} puede haber suscitado su curiosidad:

\beforeverb 
\begin{pythoncode}
    cabeza = lista
    resto = lista.siguiente
\end{pythoncode}
\afterverb Después de la primera asignación \texttt{cabeza} y \texttt{lista}
tienen el mismo tipo y el mismo valor. ¿Por qué creamos una nueva
variable?

La respuesta yace en que las dos variables tienen roles distintos.
\texttt{cabeza} es una referencia a un nodo y lista es una referencia
a toda la lista. Estos ``roles'' están en la mente del programador
y le ayudan a mantener la coherencia de los programas.

\index{variable!roles} \index{rol!variable}

En general, no podemos decir inmediatamente qué rol juega una variable
en un programa. Esta ambigüedad puede ser útil, pero también dificulta
la lectura. Los nombres de las variables pueden usarse para documentar
la forma en que esperamos que se use una variable, y, a menudo, podemos
crear variables adicionates como \texttt{nodo} y \texttt{lista} para
eliminar ambigüedades.

Podríamos haber escrito \texttt{imprimirAlReves} de una manera más
concisa sin las variables \texttt{cabeza} y \texttt{resto}, pero esto
también dificulta su lectura:

\beforeverb 
\begin{pythoncode}
def imprimirAlReves(lista) :
  if lista == None : 
     return
  imprimirAlReves(lista.siguiente)
  print(lista),
\end{pythoncode}
\afterverb Cuando leamos el código, tenemos que recordar que \texttt{imprimirAlReves}
trata a su argumento como una colección y \texttt{print} como a un
solo nodo.

El \textbf{teorema de la ambigüedad fundamental} describe la ambigüedad
inherente en la referencia a un nodo:
\begin{quote}
\textbf{Una variable que se refiera a un nodo puede tratar el nodo
como un objeto único o como el acceso a la lista de nodos} 
\end{quote}

\section{Modificando listas}

\index{lista!modificando} \index{modificando listas}

Hay varias formas de modificar una lista enlazada. La obvia consiste
en cambiar la carga de uno de sus nodos. Las mas interesantes son
las que agregan, eliminan o reordenan los nodos.

Como ejemplo, escribamos un método que elimine el segundo nodo en
la lista y retorne una referencia al nodo eliminado

\beforeverb 
\begin{pythoncode}
def eliminarSegundo(lista):
  if lista == None: 
     return
  primero = lista
  segundo = lista.siguiente
  # hacemos que el primer nodo se refiera al tercero
  primero.siguiente = segundo.siguiente
  # desconectamos el segundo nodo de la lista
  segundo.siguiente = None
  return segundo
\end{pythoncode}
\afterverb Aquí también estamos usando variables temporales para
aumentar la legibilidad. Aquí hay un ejemplo de uso del método:

\beforeverb 
\begin{pyconcode}
>>> imprimirLista(nodo1)
1 2 3
>>> borrado = eliminarSegundo(nodo1)
>>> imprimirLista(borrado)
2
>>> imprimirLista(nodo1)
1 3
\end{pyconcode}
\afterverb Este diagrama de estado muestra el efecto de la operación:

\beforefig \centerline{\includegraphics{illustrations/link5}}
\afterfig

¿Qué pasa si usted llama este método con una lista que contiene un
solo elemento (un \textbf{singleton})? ¿Qué pasa si se llama con la
lista vacía como argumento? ¿Hay precondiciones para este método?
Si las hay, corríjalo de forma que maneje de manera razonable las
violaciones a la precondición.

\index{singleton}

\section{Funciones facilitadoras (wrappers) y auxiliares (helpers)}

\index{método facilitador} \index{método!facilitador} \index{función facilitadora}
\index{función!facilitadora} \index{método auxiliar} \index{método!auxiliar}

Es bastante útil dividir las operaciones de listas en dos métodos.
Con la impresión al revés podemos ilustrarlo, para desplegar \texttt{{[}3,
2, 1{]}} en pantalla podemos llamar el método \texttt{imprimirAlReves}
que desplegará \texttt{3, 2}, y llamar otro método para imprimir los
corchetes y el primer nodo. Nombrémosla así:

\beforeverb 
\begin{pythoncode}
def imprimirAlRevesBien(lista):
  print("["),
  if lista != None:
    cabeza = lista
    resto = lista.siguiente
    imprimirAlReves(resto)
    print(cabeza),
  print("]"),
\end{pythoncode}
\afterverb Es conveniente chequear que estos métodos funcionen bien
para casos especiales como la lista vacía o una lista con un solo
elemento (singleton).

\index{singleton}

Cuando usamos este método en algún programa, llamamos directamente
a la función \texttt{imprimirAlRevesBien} para que llame a \texttt{imprimirAlReves}.
En este sentido, \texttt{imprimirAlRevesBien} es una función \textbf{facilitadora},
que utiliza a la otra, \texttt{imprimirAlReves} como función \textbf{auxiliar}.

\section{La clase \texttt{ListaEnlazada}}

\index{ListaEnlazada} \index{clase!ListaEnlazada}

Hay problemas más sutiles en nuestra implementación de listas que
vamos a ilustrar desde los efectos a las causas, a partir de una implementación
alternativa exploraremos los problemas que resuelve.

Primero, crearemos una clase nueva llamada \texttt{ListaEnlazada}.
Tiene como atributos un entero con el número de elementos de la lista
y una referencia al primer nodo. Las instancias de \texttt{ListaEnlazada}
sirven como mecanismo de control de listas compuestas por instancias
de la clase \texttt{Nodo}:

\beforeverb 
\begin{pythoncode}
class ListaEnlazada :
  def __init__(self) :
    self.numElementos = 0
    self.cabeza   = None
\end{pythoncode}
\afterverb Lo bueno de la clase \texttt{ListaEnlazada} es que proporciona
un lugar natural para definir las funciones facilitadores como \texttt{imprimirAlRevesBien}
como métodos:

\beforeverb 
\begin{pythoncode}
class ListaEnlazada:
  ...
  def imprimirAlReves(self):
    print("["),
    if self.cabeza != None:
      self.cabeza.imprimirAlReves()
    print("]"),

class Nodo:
  ...
  def imprimirAlReves(self):
    if self.siguiente != None:
      resto = self.siguiente
      resto.imprimirAlReves()
    print(self.carga),
\end{pythoncode}
\afterverb Aunque inicialmente pueda parecer un poco confuso, vamos
a renombrar a la función \texttt{imprimirAlRevesBien}. Ahora vamos
a implementar dos métodos con el mismo nombre \texttt{imprimirAlReves}:
uno en la clase \texttt{Nodo} (el auxiliar); y uno en la clase \texttt{ListaEnlazada}
(el facilitador). Cuando el facilitador llama al otro método, \texttt{self.cabeza.imprimirAlReves},
está invocando al auxiliar, porque \texttt{self.cabeza} es una instancia
de la clase \texttt{Nodo}.

Otro beneficio de la clase \texttt{ListaEnlazada} es que facilita
agregar o eliminar el primer elemento de una lista. Por ejemplo, \texttt{agregarAlPrincipio}
es un método de la clase \texttt{ListaEnlazada} que toma una carga
como argumento y la pone en un nuevo nodo al principio de la lista:

\beforeverb 
\begin{pythoncode}
class ListaEnlazada:
  ...
  def agregarAlPrincipio(self, carga):
    nodo = Nodo(carga)
    nodo.siguiente = self.cabeza
    self.cabeza = nodo
    self.numElementos = self.numElementos + 1
\end{pythoncode}
\afterverb Como de costumbre, usted debe revisar este código para
verificar qué sucede con los casos especiales. Por ejemplo, ¿qué pasa
si se llama cuando la lista está vacía?

\section{Invariantes}

\index{Invariante} \index{Invariante de objetos} \index{lista!bien formada}

Algunas listas están ``bien formadas``. Por ejemplo,
si una lista contiene un ciclo, causará problemas graves a nuestros
métodos, así que deseamos evitar a toda costa que las listas tengan
ciclos. Otro requerimiento de las listas es que el número almacenado
en el atributo \texttt{numElementos} de la clase \texttt{ListaEnlazada}
sea igual al número de elementos en la lista.

Estos requerimientos se denominan \textbf{Invariantes} porque, idealmente,
deberían ser ciertos para todo objeto de la clase en todo momento.
Es una muy buena práctica especificar los Invariantes para los objetos
porque permite comprobar de manera mas sencilla la corrección del
código, revisar la integridad de las estructuras de datos y detectar
errores.

Algo que puede confundir acerca de los invariantes es que hay ciertos
momentos en que son violados. Por ejemplo, en el medio de \texttt{agregarAlPrincipio},
después de que hemos agregado el nodo, pero antes de incrementar el
atributo \texttt{numElementos}, el Invariante se viola. Esta clase
de violación es aceptable, de hecho, casi siempre es imposible modificar
un objeto sin violar un Invariante, al menos momentáneamente. Normalmente,
requerimos que cada método que viole un invariante, lo establezca
nuevamente.

Si hay una parte significativa de código en la que el Invariante se
viola, es importante documentarlo claramente, de forma que no se ejecuten
operaciones que dependan del Invariante.

\index{documentar}

\section{Glosario}

\index{referencia incrustada} \index{referencia!incrustada} \index{estructura de datos recursiva}
\index{estructura de datos!recursiva} \index{lista enlazada} \index{lista!enlazada}
\index{nodo} \index{dato} \index{enlace} \index{precondición}
\index{invariante} \index{facilitador} \index{método auxiliar}
\index{teorema fundamental de la ambigüedad} \index{singleton}
\begin{description}
\item [{Referencia incrustada:}] referencia almacenada en un atributo
de un objeto.
\item [{Lista enlazada:}] es la estructura de datos que implementa una
colección por medio de una secuencia de nodos enlazados.
\item [{Nodo:}] elemento de la lista, usualmente implementado como un objeto
que contiene una referencia hacia otro objeto del mismo tipo.
\item [{Carga:}] dato contenido en un nodo.
\item [{Enlace:}] referencia incrustada, usada para enlazar un objeto con
otro.
\item [{Precondición:}] condición lógica (o aserción) que debe ser cierta
para que un método funcione correctamente.
\item [{Teorema fundamental de la ambigüedad:}] la referencia a un nodo
de una lista puede interpretarse hacia un nodo determinado o como
la referencia a toda la lista de nodos.
\item [{Singleton:}] lista enlazada con un solo nodo.
\item [{Facilitador:}] método que actúa como intermediario entre alguien
que llama un método y un método auxiliar. Se crean normalmente para
facilitar los llamados y hacerlos menos propensos a errores.
\item [{Método auxiliar:}] es un método que el programador no llama directamente,
sino que es usado por otro método para realizar parte de una operación.
\item [{Invariante:}] aserción que debe ser cierta para un objeto en todo
momento (excepto cuando el objeto está siendo modificado).
\end{description}

